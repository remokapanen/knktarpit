% Options for packages loaded elsewhere
\PassOptionsToPackage{unicode}{hyperref}
\PassOptionsToPackage{hyphens}{url}
\documentclass[
]{book}
\usepackage{xcolor}
\usepackage{amsmath,amssymb}
\setcounter{secnumdepth}{5}
\usepackage{iftex}
\ifPDFTeX
  \usepackage[T1]{fontenc}
  \usepackage[utf8]{inputenc}
  \usepackage{textcomp} % provide euro and other symbols
\else % if luatex or xetex
  \usepackage{unicode-math} % this also loads fontspec
  \defaultfontfeatures{Scale=MatchLowercase}
  \defaultfontfeatures[\rmfamily]{Ligatures=TeX,Scale=1}
\fi
\usepackage{lmodern}
\ifPDFTeX\else
  % xetex/luatex font selection
\fi
% Use upquote if available, for straight quotes in verbatim environments
\IfFileExists{upquote.sty}{\usepackage{upquote}}{}
\IfFileExists{microtype.sty}{% use microtype if available
  \usepackage[]{microtype}
  \UseMicrotypeSet[protrusion]{basicmath} % disable protrusion for tt fonts
}{}
\makeatletter
\@ifundefined{KOMAClassName}{% if non-KOMA class
  \IfFileExists{parskip.sty}{%
    \usepackage{parskip}
  }{% else
    \setlength{\parindent}{0pt}
    \setlength{\parskip}{6pt plus 2pt minus 1pt}}
}{% if KOMA class
  \KOMAoptions{parskip=half}}
\makeatother
\usepackage{longtable,booktabs,array}
\usepackage{calc} % for calculating minipage widths
% Correct order of tables after \paragraph or \subparagraph
\usepackage{etoolbox}
\makeatletter
\patchcmd\longtable{\par}{\if@noskipsec\mbox{}\fi\par}{}{}
\makeatother
% Allow footnotes in longtable head/foot
\IfFileExists{footnotehyper.sty}{\usepackage{footnotehyper}}{\usepackage{footnote}}
\makesavenoteenv{longtable}
\usepackage{graphicx}
\makeatletter
\newsavebox\pandoc@box
\newcommand*\pandocbounded[1]{% scales image to fit in text height/width
  \sbox\pandoc@box{#1}%
  \Gscale@div\@tempa{\textheight}{\dimexpr\ht\pandoc@box+\dp\pandoc@box\relax}%
  \Gscale@div\@tempb{\linewidth}{\wd\pandoc@box}%
  \ifdim\@tempb\p@<\@tempa\p@\let\@tempa\@tempb\fi% select the smaller of both
  \ifdim\@tempa\p@<\p@\scalebox{\@tempa}{\usebox\pandoc@box}%
  \else\usebox{\pandoc@box}%
  \fi%
}
% Set default figure placement to htbp
\def\fps@figure{htbp}
\makeatother
\setlength{\emergencystretch}{3em} % prevent overfull lines
\providecommand{\tightlist}{%
  \setlength{\itemsep}{0pt}\setlength{\parskip}{0pt}}
\usepackage[]{natbib}
\bibliographystyle{plainnat}
\usepackage{booktabs}
\usepackage{bookmark}
\IfFileExists{xurl.sty}{\usepackage{xurl}}{} % add URL line breaks if available
\urlstyle{same}
\hypersetup{
  pdftitle={Kirurgiatärpit},
  hidelinks,
  pdfcreator={LaTeX via pandoc}}

\title{Kirurgiatärpit}
\author{}
\date{\vspace{-2.5em}2025-11-29}

\usepackage{amsthm}
\newtheorem{theorem}{Theorem}[chapter]
\newtheorem{lemma}{Lemma}[chapter]
\newtheorem{corollary}{Corollary}[chapter]
\newtheorem{proposition}{Proposition}[chapter]
\newtheorem{conjecture}{Conjecture}[chapter]
\theoremstyle{definition}
\newtheorem{definition}{Definition}[chapter]
\theoremstyle{definition}
\newtheorem{example}{Example}[chapter]
\theoremstyle{definition}
\newtheorem{exercise}{Exercise}[chapter]
\theoremstyle{definition}
\newtheorem{hypothesis}{Hypothesis}[chapter]
\theoremstyle{remark}
\newtheorem*{remark}{Remark}
\newtheorem*{solution}{Solution}
\begin{document}
\maketitle

{
\setcounter{tocdepth}{1}
\tableofcontents
}
\chapter{About}\label{about}

Tenttien formaatti on muuttunut niiden keräämisen alkuvuodesta lähtien huomattavasti.

\begin{itemize}
\tightlist
\item
  Vuosina 2013-2020 kysymykset olivat avonaisempia ja monivalintakysymyksiä ei pääasiassa ollut.
\item
  Vuodesta 2021 lähtien taas tentit ovat koostuneet vain monivalintakysymyksistä.

  \begin{itemize}
  \tightlist
  \item
    Monivalintakysymysten kysymyksenasettelun ja vaihtoehtojen muistamisessa sekä vastausten tarjonnassa on kuitenkin ollut 2022 eteenpäin ongelmia, joten tässä gitbookissa olen formatoinut wikin tärpit niin, että niissä käydään myös vastaukset läpi ja jos tärpissä on puutteellisuuksia vaihtoehtojen tai kysymyksenasettelun suhteen, niin olen yrittänyt korvata näitä kirjoittamalla aiheesta tärkeimmät.
  \end{itemize}
\end{itemize}

Toivottavasti tästä rebuildista on sinulle apua kerratessasi tenttiin.

\chapter{2021 (Invictus)}\label{invictus}

\section{Potilastapaus}\label{potilastapaus}

Vastaanotollesi terveyskeskuksen akuuttiajalle tulee potilas, joka on potenut akuuttia vatsakipua 2 päivää ja jonka anamneesi ja kliininen taudinkuva sopivat hyvin akuuttiin appendisiittiin. Laboratoriovastauksista huomaat kuitenkin, että sekä leukosyyttiarvo että CRP ovat molemmat viitealueella. Mielestäsi appendisiitti on

\begin{itemize}
\tightlist
\item
  \begin{enumerate}
  \def\labelenumi{\alph{enumi}.}
  \tightlist
  \item
    silti hyvin todennäköinen
  \end{enumerate}
\item
  \begin{enumerate}
  \def\labelenumi{\alph{enumi}.}
  \setcounter{enumi}{1}
  \tightlist
  \item
    silti varsin todennäköinen
  \end{enumerate}
\item
  \begin{enumerate}
  \def\labelenumi{\alph{enumi}.}
  \setcounter{enumi}{2}
  \tightlist
  \item
    epätodennäköinen
  \end{enumerate}
\item
  \begin{enumerate}
  \def\labelenumi{\alph{enumi}.}
  \setcounter{enumi}{3}
  \tightlist
  \item
    mahdoton
  \end{enumerate}
\end{itemize}

\begin{solution}
\leavevmode

Vastaus

\begin{verbatim}
  c
  
\end{verbatim}

Normaali leukosyyttiluku ja CRP-pitoisuus eivät täysin poissulje appendisiittia, mutta appendisiitti on aika epätodennäköinen sellaisessa tilanteessa. N. 1/100 aikuispotilaasta on leukosyytit ja CRP normaalit vaikka todettaisiin appendisiitti ja n.~6/100 lapsista leuk ja CRP normaalit, koska lasten immunologia hieman erilainen kuin aikuisilla.

\end{solution}

\section{Potilastapaus}\label{potilastapaus-1}

60-vuotiaalla normaalikokoisella naisella on suunnitteilla happirikastinhoito vaikean COPD:n vuoksi, FEV1 1,0 litraa. Potilaalla todetaan oikeassa keuhkon ylälohkossa 1,5 cm maksimihalkaisijaltaan oleva tuumori, joka varmistuu sytologiassa keuhkon levyepiteelikarsinoomaksi (eli ei-pienisoluinen). Potilaan KEUHKOSYÖVÄN hoito on (valitse paras vaihtoehto)?

\begin{itemize}
\tightlist
\item
  \begin{enumerate}
  \def\labelenumi{\alph{enumi}.}
  \tightlist
  \item
    Inhaloitava kortikosteroidi
  \end{enumerate}
\item
  \begin{enumerate}
  \def\labelenumi{\alph{enumi}.}
  \setcounter{enumi}{1}
  \tightlist
  \item
    Stereotaktinen sädehoito
  \end{enumerate}
\item
  \begin{enumerate}
  \def\labelenumi{\alph{enumi}.}
  \setcounter{enumi}{2}
  \tightlist
  \item
    Keuhkonsiirto
  \end{enumerate}
\item
  \begin{enumerate}
  \def\labelenumi{\alph{enumi}.}
  \setcounter{enumi}{3}
  \tightlist
  \item
    Oikean ylälohkon poisto ja onkologiset lääkehoidot
  \end{enumerate}
\end{itemize}

\begin{solution}
\leavevmode

Vastaus

\begin{verbatim}
  b
\end{verbatim}

Ei-pienisoluisen keuhkosyövän ensisijainen hoito on leikkaus, joko keuhkolohkon tai keuhkon poisto. Radikaalileikkaus on mahdollinen n.~20--25 \%:lle. Kirurgisen hoidon edellytyksiin kuuluu, että resektion jälkeen jäljelle jäävä FEV1 on oltava \textgreater1,0l. Potilas ei siis ole leikkauskuntoinen, koska leikkausriski on liian suuri FEV1:n ollessa ennen leikkausta 1,0 litraa.

a: Inhaloitava kortikosteroidi ei ole syövän hoitokeino.

b: Paikallisen keuhkosyövän leikkaushoitoon soveltumattomille voidaan käyttää korkea-annoksista kohdennettua, ns. stereotaktista, tuumorin sädehoitoa.

c ja d: Leikkaushoitoja, eivät sovellu tälle potilaalle, koska korkean leikkausriskin potilas.

\end{solution}

\section{Potilastapaus}\label{potilastapaus-2}

Miespotilaalla, jolla on akuutisti ilmentynyt tiheävirtsaisuus ja virtsankirvellys, sekä virtsassa seuraavat löydökset: U-Kemseul: eryt +++, U-sakka: eryt 4/nk, U-BaktVi E. coli yli 10E5. Hoidat hänet seuraavasti:

\begin{itemize}
\tightlist
\item
  \begin{enumerate}
  \def\labelenumi{\alph{enumi}.}
  \tightlist
  \item
    Teen lähetteen erikoissairaanhoitoon päivystyksellisesti jatkoselvittelyihin.
  \end{enumerate}
\item
  \begin{enumerate}
  \def\labelenumi{\alph{enumi}.}
  \setcounter{enumi}{1}
  \tightlist
  \item
    Aloitat antibiootiksi Ditrim duplo 1×2 viikon ajaksi ja laitan potilaalle kestokatetrin.
  \end{enumerate}
\item
  \begin{enumerate}
  \def\labelenumi{\alph{enumi}.}
  \setcounter{enumi}{2}
  \tightlist
  \item
    Aloitat antibiootiksi Ditrim duplo 1×2 viikon ajaksi.
  \end{enumerate}
\item
  \begin{enumerate}
  \def\labelenumi{\alph{enumi}.}
  \setcounter{enumi}{3}
  \tightlist
  \item
    Aloitat antibiootiksi Ditrim duplo 1×2 viikon ajaksi ja kontrolloit U-Kemseul, U-sakka tutkimukset.
  \end{enumerate}
\end{itemize}

\begin{solution}
\leavevmode

Vastaus

\begin{verbatim}
  d
\end{verbatim}

a: Pelkät virtsatieinfektiot eivät yleensä ole päivystyslähetteiden aihe. Potilaalla ei ole viitteitä tilan etenemisestä sepsikseen asti, joten potilasta voi hoitaa avoterveydenhuollossa.

b: Ei ole tarvetta kestokatetrille, potilas pystyy käydä itse vessassa ja kyseessä ei ole virtsaumpi.

c: Pelkkä antibioottikuuri ilman mitään muita toimenpiteitä voisi toimia, jos kyseessä olisi komplisoitumaton kystiitti. Komplisoitumaton kystiitti tarkoittaa satunnaista virtsatieinfektiota naisella, jolla ei ole riskitekijöitä. Täten äkillinen kystiitti voidaan 18-65-vuotiaalla naisella todeta puhelinhaastattelulla ja jos ei todeta vaikeahoitoisen VTI:n riskitekijöitä eikä hälyttäviä oireita / esitietoja -\textgreater{} hoito voidaan toteuttaa ilman laboratoriotutkimuksia. Hoitona yleensä lyhyt 3vrk kestävä antibioottihoito (esim. trimetopriimi, nitrofurantoiini tai pivmesillinaami). Komplisoitumatonta kystiittiä ei tarvitse seurata.

d: Miehillä virtsatieinfektio on aina komplisoitunut, koska komplisoituneita virtsatieinfektioita ovat sellaiset, joissa potilailla on tekijöitä, jotka ennustavat tavanmukaisen lääkehoidon epäonnistumista. Komplisoituneen kystiitin hoitoon käytetään tavallisimmin trimetopriimia (160 mg 1 x 2) tai sulfa-trimetopriimia (esim. juuri Ditrim duplo 160/800 mg 1 x 2), mutta lopullinen hoitovalinta tehdään herkkyysmäärityksen mukaan; hoitoaika on tavallisimmin 7vrk. Eturauhasen ja kivespussin elinten tunnustelu on tarpeen, koska miesten VTI:ssä on usein mukana prostatiitti tai epididymiitti.

Miesten infektiot vaativat urologisia jatkoselvittelyitä ja hoidon jälkeen on selvitettävä ja hoidettava samalla sen taustalla olevat syyt (esim. eturauhasen liikakasvu, krooninen eturauhastulehdus). Paras vaihtoehto tähän sopien on d, koska sentään siinä on jotain kontrollointia ja todennäköisesti myös olisi jatkoselvittelyä.

\end{solution}

\section{Toisen asteen pinnallinen palovamma}\label{toisen-asteen-pinnallinen-palovamma}

\begin{itemize}
\tightlist
\item
  \begin{enumerate}
  \def\labelenumi{\alph{enumi}.}
  \tightlist
  \item
    ulottuu vain epidermikseen
  \end{enumerate}
\item
  \begin{enumerate}
  \def\labelenumi{\alph{enumi}.}
  \setcounter{enumi}{1}
  \tightlist
  \item
    tarvitsee usein ihonsiirron
  \end{enumerate}
\item
  \begin{enumerate}
  \def\labelenumi{\alph{enumi}.}
  \setcounter{enumi}{2}
  \tightlist
  \item
    on yleensä liekin aiheuttama
  \end{enumerate}
\item
  \begin{enumerate}
  \def\labelenumi{\alph{enumi}.}
  \setcounter{enumi}{3}
  \tightlist
  \item
    paranee kahdessa viikossa arpeutumatta.
  \end{enumerate}
\end{itemize}

\begin{solution}
\leavevmode

Vastaus

\begin{verbatim}
  d
\end{verbatim}

a: Vaurio rajoittuu epidermikseen ensimmäisen asteen palovammoissa. Iho on pinnaltaan kuiva, punoittava ja kosketusarka. Ihossa on turvotusta muttei rakkuloita. Nämä vammat paranevat 3--7 vuorokauden sisällä arpia jättämättä. Ensimmäisen asteen palovammaa ei lasketa mukaan potilaan palovammaprosenttiin palovamman laajuutta arvioitaessa.

b: Syvissä toisen ja kolmannen asteen vammoissa spontaani paranemistaipumus on heikko ja päätöksenteko leikkauksesta on usein helppoa, mutta keskisyvissä dermaalisissa toisen asteen vammoissa tarvitaan joskus 2--3 viikkoa seuranta-aikaa, jolloin odotetaan vamman lopullista rajautumista.

Leikatut palovamma-alueet tulee aina peittää. Pienet palovammat voidaan leikata pois ja haava sulkea suoraan tai sitten käyttäen esimerkiksi paikallista iho-subkutiskielekettä. Tämä kuitenkin soveltuu käytettäväksi vain harvoin. Yleisin palovammahaavan peittoon käytetty menetelmä onkin autografti eli potilaan oma iho.

c: Pinnallinen 2. asteen palovamma on useimmiten kuuman nesteen aiheuttama

d: Toisen asteen eli dermaaliset palovammat rajoittuvat nimensä mukaisesti dermikseen eli verinahkaan. Pinnallisille dermaalisille vammoille ovat tyypillisiä ihon pintaan (epidermis ja pinnallinen kerros dermiksestä) syntyvät rakkulat, jotka usein syntyvät muutamien tuntien kuluessa vammautumisesta. Haavan pohja on vaaleanpunainen, kiiltäväpintainen, painettaessa siinä on nähtävissä nopea vitaalireaktio (capillary refill) ja vamma on hyvin kivulias. Nämä vammat paranevat paikallishoidolla 10--14 päivän kuluessa arpia jättämättä. Syvemmät vammat taas mahdollisesti voivat vaatia operatiivista hoitoa.

Palovamma syvenee 48--72 tuntia vamman jälkeen. Tämän takia vamman syvyysarvio tulee tehdä uudelleen 2--3 vuorokauden kuluttua vammasta lopullisen syvyysarvion ja hoitosuunnitelman tekoa varten.

\pandocbounded{\includegraphics[keepaspectratio]{images/palovammaluokatleo.png}}
\pandocbounded{\includegraphics[keepaspectratio]{images/palovammaluokat.png}}
\pandocbounded{\includegraphics[keepaspectratio]{images/pinnallinendermaalinen.png}}

\end{solution}

\section{Mitä tarkoittaa Breslow melanooman PAD-lausunnon yhteydessä?}\label{mituxe4-tarkoittaa-breslow-melanooman-pad-lausunnon-yhteydessuxe4}

\begin{itemize}
\tightlist
\item
  \begin{enumerate}
  \def\labelenumi{\alph{enumi}.}
  \tightlist
  \item
    Melanooman vertikaalista paksuutta (millimetreissä)
  \end{enumerate}
\item
  \begin{enumerate}
  \def\labelenumi{\alph{enumi}.}
  \setcounter{enumi}{1}
  \tightlist
  \item
    Melanooman halkaisijaa ihon pinnalla (millimetreissä)
  \end{enumerate}
\item
  \begin{enumerate}
  \def\labelenumi{\alph{enumi}.}
  \setcounter{enumi}{2}
  \tightlist
  \item
    Melanooman kasvutapaa
  \end{enumerate}
\item
  \begin{enumerate}
  \def\labelenumi{\alph{enumi}.}
  \setcounter{enumi}{3}
  \tightlist
  \item
    Ihon kerrosta, jonne melanooma ulottuu (asteikolla I-V)
  \end{enumerate}
\end{itemize}

\begin{solution}
\leavevmode

Vastaus

\begin{verbatim}
  a
\end{verbatim}

Breslow'n mitta (invaasion syvyys) on tärkein prognostinen tekijä, joka ennustaa melanooman metastasointia. Se mitataan epidermiksen granulaarisolukerroksen pinnasta syvimpiin melanoomasoluihin (Breslow'n mitta lasketaan vain invasiiviselle tuumoreille -\textgreater{} in situ-melanoomalle Breslow on 0 (yleensä ei ilmoiteta PAD-lausunnossa)). Melanooman TNM-luokituksessa T:n pääluokka määritetään Breslow'n mitan mukaan: \textless1mm (T1), 1-2mm (T2), 2-4mm (T3), \textgreater4mm (T4)

b: Melanoomia tutkittaessa ollaan kylläkin kiinnostuneita muutoksen halkaisijasta, mutta se ei ole sama kuin Breslow'n mitta. Ihon pigmenttimuutos on epäilyttävä melanooman suhteen, jos sen läpimitta on \textgreater6mm (yksi osa ABCDE-säännöstöstä).

c: Melanooman kasvutapa on karkeimmin jaettavissa kahteen: radiaalinen (horisontaalinen ihon pintaa pitkin) ja vertikaalinen. Vertikaalinen kasvutapa eli invaasio syvemmälle ihoon altistaa metastaaseille. Tietyille melanooman kliinis-patologisille luokille on tyypillistä aikaisempi vertikaalinen kasvu (esim. nodulaarinen melanooma) ja toisille pidempi radiaalisen kasvun vaihe (esim. pinnallisesti leviävä tai lentigo maligna).

d: Clarkin luokat (I-V) kuvaavat invaasion syvyyttä ihon kerrosten mukaan, mutta eivät ole ennusteellisesti yhtä luotettavia kuin Breslow'n mitta

\pandocbounded{\includegraphics[keepaspectratio]{images/breslow.png}}
\pandocbounded{\includegraphics[keepaspectratio]{images/clark.png}}

\end{solution}

\section{Ompeleiden poistoaika}\label{ompeleiden-poistoaika}

Olet poistanut 75-vuotiaalta naispotilaaltasi poskesta 1 cm kokoisen suspektin iholuomen diagnostisin marginaalein PAD-tutkimukseen. Milloin potilas käy ompeleiden poistossa?

\begin{itemize}
\tightlist
\item
  \begin{enumerate}
  \def\labelenumi{\alph{enumi}.}
  \tightlist
  \item
    10 vuorokauden kuluttua
  \end{enumerate}
\item
  \begin{enumerate}
  \def\labelenumi{\alph{enumi}.}
  \setcounter{enumi}{1}
  \tightlist
  \item
    21 vuorokauden kuluttua
  \end{enumerate}
\item
  \begin{enumerate}
  \def\labelenumi{\alph{enumi}.}
  \setcounter{enumi}{2}
  \tightlist
  \item
    7 vuorokauden kuluttua
  \end{enumerate}
\item
  \begin{enumerate}
  \def\labelenumi{\alph{enumi}.}
  \setcounter{enumi}{3}
  \tightlist
  \item
    14 vuorokauden kuluttua
  \end{enumerate}
\end{itemize}

\begin{solution}
\leavevmode

Vastaus

\begin{verbatim}
  c
\end{verbatim}

Ompeleiden kesto riippuu haavasta: koko, syvyys, kiristys. Riippuu myös potilaasta: ikä, aktiivisuus. Yleisesti ottaen: mitä paksumpi iho, sitä kauemmin ompeleita pidetään.

Yleisohjeita ompeleiden kestoon on: Kasvot 5-7vrk, vartalo 7-14 vrk (selkä 2 vko, vatsa/rinta 1 vko), raajat 10-14 vrk, haavaruptuuralle alttiit alueet (esim. sääri) ad 2 vk, jalkapohja jopa 3 vk

\end{solution}

\section{Potilastapaus}\label{potilastapaus-3}

Tk:n vuodeosastolle tulee 80-vuotias nainen, jolla on pitkälle edennyt dementia perussairautena ja tarvitsee apua kaikissa päivittäisissä toimissa. Potilas on otettu osastolle yleistilan laskun ja pyelonefriitin vuoksi. Liikkuminen on huonoa ja potilas on vuodepotilaana. Parin päivän osastohoidon jälkeen huomattiin sacrumin seudussa punoitusta sekä pinnallista ihorikkoa painehaavaan viitaten. Mikä on tärkein ja ensimmäinen hoito?

\begin{itemize}
\tightlist
\item
  \begin{enumerate}
  \def\labelenumi{\alph{enumi}.}
  \tightlist
  \item
    Hyvä ravitsemus ja proteiinilisät
  \end{enumerate}
\item
  \begin{enumerate}
  \def\labelenumi{\alph{enumi}.}
  \setcounter{enumi}{1}
  \tightlist
  \item
    Haavan paikallishoito
  \end{enumerate}
\item
  \begin{enumerate}
  \def\labelenumi{\alph{enumi}.}
  \setcounter{enumi}{2}
  \tightlist
  \item
    Plastiikkakirurgin konsultaatio
  \end{enumerate}
\item
  \begin{enumerate}
  \def\labelenumi{\alph{enumi}.}
  \setcounter{enumi}{3}
  \tightlist
  \item
    Asentohoito ja painehaavapatja
  \end{enumerate}
\end{itemize}

\begin{solution}
\leavevmode

Vastaus

\begin{verbatim}
  d
\end{verbatim}

Painehaava on paineen, venytyksen tai molempien yhdessä aiheuttama kudosvaurio, joka ilmenee painealueilla luisten ulokkeiden kohdalla, kuten istuinkyhmyn, ristiselän ja kantapäiden alueilla. Myötävaikuttavia tekijöitä painehaavan syntyyn ovat liikkumattomuus, ravitsemushäiriö, ruumiinlämmön lasku, vanhuus sekä paikalliset myötävaikuttavat kudosolot (verenkierron häiriöt, virheasennot). Hoidon tärkein osa-alue on paineen poistaminen alueelta; tässä ohjeina esim. asennonvaihdot 2-4h välein ja paineen vaihtavat patjat.

a: Hyvä ruokavalio ja varsinkin proteiini ovat tärkeitä mm. haavan paranemisessa, mutta ne eivät ole ensimmäinen hoito eivätkä estä painehaavan pahenemista ilman muita temppuja.

b: Haavan paikallishoito (suojaaminen ja tarvittaessa alueen puhdistaminen) on tärkeää, mutta jos asentohoitoa ei ole, niin millään ei ole väliä, koska painehaava vain pahenisi jatkuvan paineen takia.

c: Plastiikkakirurgia tietysti voi konsultoida, mutta ei ole tarpeen lievässä pinnallisessa painehaavassa.

d: Ehkäisyssä ja hoidossa on olennaisinta minimoida tai poistaa riskikohtiin kohdistuva paine (asentohoito, erikoispatjat, istuintyynyt). Painehaavapotilaalla käytetään AINA erittäin korkean riskin potilaille tarkoitettua patjaa. Kansainvälinen suositus jakaa makuualustat toiminnallisesti kahteen pääluokkaan: Reaktiiviset makuualustat reagoivat potilaan painoon, vartalon rakenteeseen ja liikkeeseen. Ne voivat olla tarvittaessa sähköllä toimivia tai esim. vaahtomuovisia makuualustoja. Aktiiviset makuualustat ovat sähköllä jatkuvasti toimivia, tyyppiesimerkkinä vaihtuvapaineiset makuualustat. Alla taulukko makuualustoista (ei tarvitse osata).

\pandocbounded{\includegraphics[keepaspectratio]{images/painehaavaluokittelu.png}}
\pandocbounded{\includegraphics[keepaspectratio]{images/makuualustat.png}}
\pandocbounded{\includegraphics[keepaspectratio]{images/painehaavakonshoito.png}}

\end{solution}

\section{Mikä seuraavista EI pidä paikkansa:}\label{mikuxe4-seuraavista-ei-piduxe4-paikkansa}

\begin{itemize}
\tightlist
\item
  \begin{enumerate}
  \def\labelenumi{\alph{enumi}.}
  \tightlist
  \item
    Appendisiittiperforaatio ja peritoniitti on henkeä uhkaava tila
  \end{enumerate}
\item
  \begin{enumerate}
  \def\labelenumi{\alph{enumi}.}
  \setcounter{enumi}{1}
  \tightlist
  \item
    Fekoliittiappendisiitin hoito on lähtökohtaisesti operatiivinen
  \end{enumerate}
\item
  \begin{enumerate}
  \def\labelenumi{\alph{enumi}.}
  \setcounter{enumi}{2}
  \tightlist
  \item
    Komplisoitumaton appendisiitti voi parantua konservatiivisestikin
  \end{enumerate}
\item
  \begin{enumerate}
  \def\labelenumi{\alph{enumi}.}
  \setcounter{enumi}{3}
  \tightlist
  \item
    Periappendikulaariabsessi vaatii aina operatiivisen hoidon akuutissa vaiheessa
  \end{enumerate}
\end{itemize}

\begin{solution}
\leavevmode

Vastaus

\begin{verbatim}
  d
\end{verbatim}

a: Totta, peritoniitti vaatii välitöntä leikkausta.

b: Jos appendixin suulla todetaan fekoliitti (kovettunut ulostekertymä), niin se viittaa korkeaan komplisoitumisriskiin (usein jopa määritellään suoraan komplisoituneeksi) ja tämän takia hoidetaan yleensä leikkauksella. Mikäli potilaalla on voimakkaat oireet, kuumetta, suuri C-reaktiivisen proteiinin pitoisuus (CRP) tai TT:ssa komplisoitumiseen liittyviä riskitekijöitä kuten fekoliitti tai tuumori, tulisi potilas hoitaa aina leikkauksella.

c: Totta, appendisiitti voidaan hoitaa konservatiivisesti antibioottihoidolla, jos diagnoosi on varmistettu ja komplikaatiot on poissuljettu TT:llä.

d:Arviolta noin 4-10 \%:lla potilaista on diagnoosivaiheessa kehittynyt umpilisäkkeen vieruspaise eli periappendikulaaripaise. Se syntyy kuolioituneen tai puhjenneen umpilisäkkeen seurauksena, kun elimistö pyrkii rajaamaan umpilisäkkeen ulkopuolelle leviävää bakteeri-infektiota. Periappendikulaarisen absessin jo kehityttyä hoito vaihtelee sairaaloittain ja on useimmiten aikaisessa vaiheessa konservatiivinen, vaikka onkin komplisoitunut tauti kyseessä. Mikäli periappendikulaariabsessi hoidetaan ensivaiheessa konservatiivisesti, yli 35-40-vuotiaalla potilaalla suositellaan umpilisäkkeen poistoa suunnitellusti myöhemmin, koska periappendikulaariabsessiin on todettu liittyvän selvästi suurempi umpisuolen/paksusuolen kasvainriski (5--20 \%) etenkin yli 35-40-vuotiailla potilailla.

\end{solution}

\section{Potilastapaus}\label{potilastapaus-4}

22-vuotias mies on tullut edellisenä päivänä alkaneen rintapiston vuoksi ensiapuun. Muuten kokee vointinsa hyväksi, mutta kertoi aamun lenkillä jotenkin vedon olleen poissa. SaO2 huoneilmalla 98\%, hengitys rauhallista ja vaivatonta. Auskultaatiossa vasemmalla on alentuneet hengitysäänet ja thx-kuvassa näkyy vasemmalla korkeintaan 1 cm ilmarinta apexissa. Mitä teet (valitse näistä mielekkäin vaihtoehto)?

\begin{itemize}
\tightlist
\item
  \begin{enumerate}
  \def\labelenumi{\alph{enumi}.}
  \tightlist
  \item
    kotiutan potilaan 4 tunnin seurannan jälkeen ja otan kontrollikuvan seuraavana päivänä
  \end{enumerate}
\item
  \begin{enumerate}
  \def\labelenumi{\alph{enumi}.}
  \setcounter{enumi}{1}
  \tightlist
  \item
    teen neula-aspiraation ja otan kontrollikuvan ja otan potilaan osastolle
  \end{enumerate}
\item
  \begin{enumerate}
  \def\labelenumi{\alph{enumi}.}
  \setcounter{enumi}{2}
  \tightlist
  \item
    laitan suurimman ea:sta löytyvän pleuradreenin potilaalle ja otan osastolle
  \end{enumerate}
\item
  \begin{enumerate}
  \def\labelenumi{\alph{enumi}.}
  \setcounter{enumi}{3}
  \tightlist
  \item
    otan potilaan osastolle seurantaan ja kontrolloin kuvan seuraavana päivänä
  \end{enumerate}
\end{itemize}

\begin{solution}
\leavevmode

Vastaus

\begin{verbatim}
  a
\end{verbatim}

Primaarinen spontaani ilmarinta (ilmarinta, joka ilmaantuu muuten tervekeuhkoiselle ilman traumaa) on tyypillisintä pitkillä ja hoikilla nuorilla miehillä. Ilmarinta on siis muistettava erityisesti nuoren aikuisen samoin kuin COPD-potilaan (sekundaarinen spontaani ilmarinta eli ilmaantuu muun keuhkosairauden päälle) äkillisen rintakivun ja hengenahdistuksen syynä. Keuhkokuva (mieluimmin PA-kuva seisten) on ensisijainen tutkimus diagnoosin varmistamiseksi, muita kuvantamisia tarvitaan vain harvoin.

Hoito voi olla konservatiivinen (tilanteen seuranta) ja on mahdollinen avohoidossa (useimmiten ESH) pienessä primaarisessa spontaanissa ilmarinnassa, jos kaikki seuraavat ehdot täyttyvät:

\begin{enumerate}
\def\labelenumi{\arabic{enumi}.}
\tightlist
\item
  potilas on muuten terve ja saturoi normaalisti (95\% tai yli)
\item
  potilas on oireeton tai vähäoireinen
\item
  ilmarinta on kooltaan ≤ 15 \% rintaontelon tilavuudesta, mikä vastaa ≤ 3 cm:n (joissain lähteissä alle 2,5cm) ilmasirppiä keuhkon yläosassa ilman lateraalista komponenttia (jos ilmarintaontelo ympäröi koko keuhkoa, se on jo suuri)
\item
  ilmarinta ei laajene seurannan aikana
\end{enumerate}

Seurannassa otetaan keuhkokuvia toistetusti. Ensimmäinen otetaan viimeistään parin päivän sisällä (joidenkin ohjeiden mukaan ensin 4-6 tunnin kuluttua ktrl-thx ja taas heti seuraavana päivänä) ja seuraava usein viikon kohdalla ilmarinnan ilmaantumisesta. Tämän jälkeen keuhkokuva uusitaan vähintään viikoittain, kunnes ilmarinta on resorboitunut ja keuhko on täysin laajentunut (joidenkin ohjeiden mukaan vain tarv 1 kk päästä, jos yhä pieni ilmasirppi). Yleensä seuranta on perusteltua järjestää erikoissairaanhoidossa. Aktiivisemmat hoitotoimenpiteet ovat perusteltuja, jos oireilu lisääntyy tai ilmarinta laajenee. Ilmaontelon tilavuudesta imeytyy tyypillisesti n.~1.25 \%-yksikön verran vuorokaudessa, eli 15 \%:n ilmarinta häviää n.~12 vrk:ssa.

Jos potilas vaatii kajoavaa hoitoa, niin se on pleuradreeni ja imu 3vrk tarvittaessa toistaen.
\pandocbounded{\includegraphics[keepaspectratio]{images/konservatiivinenilmarinta.png}}
\pandocbounded{\includegraphics[keepaspectratio]{images/diatvsterveysportti.png}}

\end{solution}

\section{Potilastapaus}\label{potilastapaus-5}

69-vuotiaalla potilaalla lääkityksenä dutasteridi seitsemän vuoden ajan. Nyt virtsasuihkun voima heikko ja jäännösvirtsaa 1650 ml. Hoidan hänet seuraavasti

\begin{itemize}
\tightlist
\item
  \begin{enumerate}
  \def\labelenumi{\alph{enumi}.}
  \tightlist
  \item
    kertakatetroit potilaan ja päästät hänet kotiin kehotuksella tulla tarvittaessa uudelleen.
  \end{enumerate}
\item
  \begin{enumerate}
  \def\labelenumi{\alph{enumi}.}
  \setcounter{enumi}{1}
  \tightlist
  \item
    laitat kestokatetrin, johon laitetaan venttiili. Potilas tyhjentää katetrin avaamalla venttiilin, kun tulee virtsahätä.
  \end{enumerate}
\item
  \begin{enumerate}
  \def\labelenumi{\alph{enumi}.}
  \setcounter{enumi}{2}
  \tightlist
  \item
    laitan kestokatetrin, joka yhdistetään pussiin ja potilas tulee 3 viikon kuluttua katetrin poistoon
  \end{enumerate}
\item
  \begin{enumerate}
  \def\labelenumi{\alph{enumi}.}
  \setcounter{enumi}{3}
  \tightlist
  \item
    lähetät hänet päivystyksenä erikoissairaanhoitoon.
  \end{enumerate}
\end{itemize}

\begin{solution}
\leavevmode

Vastaus

\begin{verbatim}
  c
\end{verbatim}

Akuutin ja kroonisen virtsaummen välitön hoito on virtsarakon katetrointi. Ensisijaisesti käytetään luonnollista reittiä eli viedään virtsaputken kautta katetri virtsarakkoon. Mikäli tämä jostain syystä ei onnistu, on asetettava paikallispuudutuksessa vatsanpeitteiden läpi suprapubinen katetri.

Mikäli kyseessä on selvä ulkoisen syyn takia aiheutunut ohimennyt virtsaumpi, voidaan rakko ainoastaan kertakatetroida tyhjäksi ja sitten seurata, että virtsaaminen onnistuu. Mikäli kyseessä on komplisoitunut tila tai huomattava ylivenyttyminen (\textgreater{} 1000 ml), on parempi jättää kestokatetri potilaalle esimerkiksi 1--2 viikon ajaksi. Rakon huomattava ylivenyttyminen tarvitsee pidemmän ajan palautuakseen, joten kertakatetrointi ei riitä. Katetri laitetaan pussiin, ei venttiiliin --- venttiiliä ei käytetä tällaisen retention yhteydessä, koska rakon tulee tyhjentyä vapaasti ilman painevaihtelua

Potilaan itse toteuttama toistokatetrointi voi myös tulla kyseeseen, mutta tämä vaatii ohjausta ja runsaasti kertakäyttökatetreja eikä täten yleensä ole helppo ratkaisu lyhyen päivystys- tai vastaanottohoidon yhteydessä.

\end{solution}

\section{Potilastapaus}\label{potilastapaus-6}

59-vuotias aiemmin terve mies tulee aluesairaalan ensiapuun äkillisesti alkaneen kovan repivän retrosternaalisen kivun vuoksi, kivun alkaessa menettänyt hetkeksi tajuntansa. Tutkittaessa takykardinen 110/min, EKG:ssa sinusrytmi, ei johtumishäiriöitä tai ST-tason muutoksia. RR 190/95 mmHg. Pulssit symmetriset. Thx-kuvassa ei ilmarintaa, ei nestettä pleurassa, ei inkompensaatiota. Auskultoiden diastolinen sivuääni gradus 3/6 aorttaläppäalueelta. Mikä EI pidä paikkaansa?

\begin{itemize}
\tightlist
\item
  \begin{enumerate}
  \def\labelenumi{\alph{enumi}.}
  \tightlist
  \item
    aortan -TT on syytä ottaa vahvan dissekaatioepäilyn vuoksi mikäli se on saatavissa nopeasti
  \end{enumerate}
\item
  \begin{enumerate}
  \def\labelenumi{\alph{enumi}.}
  \setcounter{enumi}{1}
  \tightlist
  \item
    B-tyypin dissekaatio voidaan hoitaa konservatiivisesti pyrkimällä alle normotension labetalolilla ja/tai nitroprussidilla
  \end{enumerate}
\item
  \begin{enumerate}
  \def\labelenumi{\alph{enumi}.}
  \setcounter{enumi}{2}
  \tightlist
  \item
    dissekaation ollessa kyseessä tulisi aloittaa labetaloli- ja/tai nitroprussidi-infuusio ja varmistaa, että verenpaineet saadaan normotensiivisiksi, vasta sitten voidaan potilasta lähteä siirtämään
  \end{enumerate}
\item
  \begin{enumerate}
  \def\labelenumi{\alph{enumi}.}
  \setcounter{enumi}{3}
  \tightlist
  \item
    A-tyypin dissekaatio tulee leikata viipymättä
  \end{enumerate}
\end{itemize}

\begin{solution}
\leavevmode

Vastaus

\begin{verbatim}
  c
\end{verbatim}

a: Aortan dissekoituman tärkein diagnostinen testi on aortan varjoainetehosteinen TT, jos se on nopeasti suoritettavissa. Myös ultralla voidaan tutkia akuutissa vaiheessa (lähinnä tamponaation ja aorttaläpän toiminnan arviona), mutta se ei ole poissulkututkimuksena täysin luotettava. Dissekoitumaan viittaavat löydökset tulisi varmentaa tietokonekerroskuvauksella (TT), mutta kriittisessä tilassa oleva potilas, jolla varmentuu sydäntamponaatio, voidaan viedä suoraan leikkaussaliinkin.

b: Stanford A-tyypin dissekoituman ensisijainen hoito on operatiivinen, kun taas B-tyypin dissekoituman ensisijainen hoito on konservatiivinen. Konservatiivisen hoidon tärkein osa on verenpaineen kontrolloiminen: tavoitetaso systolisessa paineessa on 100--120 mmHg ja sykkeessä n.~60/min. Nopea syke ja korkea verenpaine altistavat dissekoituman etenemiselle ja aortan ruptuuralle, liian alhainen verenpaine puolestaan pääte-elinten verenkiertovajeelle. Suonensisäinen natriumnitroprussidi, nitroglyseriini ja/tai labetaloli laskee tehokkaasti verenpainetta (labetaloli parempi jos syke \textgreater60, mutta verenpaine ei tavoitteessa ja nitroprussidi parempi, jos syke on jo valmiiksi matala, mutta verenpaine ei tavoitteessa). Konservatiivisesti hoidetun B-tyypin dissekoituman sairaalakuolleisuus on noin 10 \%.

B-tyypin dissekoituma voidaan hoitaa kajoavasti, jos tila komplisoituu (esim. akuutti tai uhkaava ruptuura). Ensisijaisesti kyseeseen tulee endograftihoito (TEVAR), jolla peitetään tyypillisesti laskevan torakaaliaortan yläosassa sijaitseva intimakerroksen repeämä (entry tear).

c: A-tyypin dissekoituman hoitona on välitön leikkaus sydänkirurgisessa yksikössä. Tärkein ensihoitotoimenpide akuuttia dissekoitumaa epäiltäessä on potilaan verenpaineen, syketason, kivun sekä pelon ja ahdistuksen hallinta suonensisäisellä lääkityksellä. Tavoitteena ei kuitenkaan ole sinänsä normotensio, vaan 100-120 mmHg systolista. Riittävä keskivaltimopaine on 60 mmHg ja systolinen painetaso 90 mmHg. \textbf{Jos TT:ssä todetaan epäilyn jälkeen A-tyypin dissekaatio tai B-tyypin dissekaatio komplisoituu, niin se vaatii välittömän leikkauksen ja potilas voidaan siirtää leikkaussaliin vaikka verenpaineet olisivatkin korkeat.}

d: Totta. A-tyypin dissekoituman leikkaushoidon tavoitteena on korvata dissekoitunut osa nousevaa aorttaa verisuoniproteesilla, erityisesti alue, jolla intimakerroksen repeämä (entry tear) sijaitsee. Tällöin estetään veren antegradinen kulkeutuminen väärään luumeniin. Lisäksi varmistetaan, että nousevan aortan alueella ei tapahtuisi uutta repeämää eikä aorttaläpän tai koronaariostiumien uutta katastrofia. Jos aortan tyvi sinus Valsalvae -tasolla sekä aorttaläppä ovat rakenteellisesti ehjät, riittää yleensä suora putkiproteesi, joka ommellaan proksimaalisesti aortan sinotubulaarijunktion tasoon. Tämä on yleisin (noin 70 \%) aortan dissekoituman leikkaustekniikka.

Aortan dissekoituman Stanford A -tyyppi = dissekoituma affisioi ainakin nousevaa aorttaa (voi olla myös laskevassa osassa)

Aortan dissekoituman Stanford B -tyyppi = dissekoituma affisioi vain laskevaa aorttaa

Aortan dissekoituman Stanford non-A-non-B -tyyppi = dissekoituma affisioi vähintään aortan kaarta, mutta ei nousevaa aorttaa

\pandocbounded{\includegraphics[keepaspectratio]{images/stanford.png}}

\end{solution}

\section{Mittaat vastaanotolla potilaan ABI-indeksin. Arvoksi saat vasemmasta jalasta 0.92, oikealla arvo on normaali 1.0. Mikä väittämistä on oikein?}\label{mittaat-vastaanotolla-potilaan-abi-indeksin.-arvoksi-saat-vasemmasta-jalasta-0.92-oikealla-arvo-on-normaali-1.0.-mikuxe4-vuxe4ittuxe4mistuxe4-on-oikein}

\begin{itemize}
\tightlist
\item
  \begin{enumerate}
  \def\labelenumi{\alph{enumi}.}
  \tightlist
  \item
    Kyseisen potilaan kardiovaskulaarinen ennuste on ABI indeksin perusteella huono
  \end{enumerate}
\item
  \begin{enumerate}
  \def\labelenumi{\alph{enumi}.}
  \setcounter{enumi}{1}
  \tightlist
  \item
    Potilaan verenpainelääkitys lopetetaan, jotta vasemman jalan painetta saadaan korjattua
  \end{enumerate}
\item
  \begin{enumerate}
  \def\labelenumi{\alph{enumi}.}
  \setcounter{enumi}{2}
  \tightlist
  \item
    Potilaan ABI arvo vasemmalla on normaaliarvojen rajoissa
  \end{enumerate}
\item
  \begin{enumerate}
  \def\labelenumi{\alph{enumi}.}
  \setcounter{enumi}{3}
  \tightlist
  \item
    Potilaalle pitäisi aloittaa ABI mittauksen perusteella heti kolesterolilääkitys
  \end{enumerate}
\end{itemize}

\begin{solution}
\leavevmode

Vastaus

\begin{verbatim}
  c
\end{verbatim}

Nilkka-olkavarsipainesuhteen (ABI) määrittäminen on pulssien palpoimisen ohella ASO-taudin tärkein tutkimus yleislääkärin näkökulmasta. ABI-mittaus tulisi suorittaa rutiinimaisesti tukkivaa valtimotautia epäiltäessä eikä potilasta tulisi lähettää erikoissairaanhoitoon ilman ABI-mittausta

Normaali ABI on yli 0.9 ja ABI-arvoa ≤ 0,9 pidetään yleisesti merkkinä alaraajan huonontuneesta valtimokierrosta (alle 0,90:n ABI-arvolla on 75 \%:n herkkyys ja 86 \%:n tarkkuus alaraajaiskemian diagnostiikassa, mutta herkkyys on huonompi diabeetikoilla ja vaikeaa munuaisten vajaatoimintaa sairastavilla valtimoiden seinämiä jäykistävän mediaskleroosin vuoksi). Normaali ABI-arvo ei sulje pois alaraajaiskemiaa, jos kliininen epäily on vahva. Tällöin tulee tehdä muita testejä diagnostiikan tueksi

Luotettavan viitealueen yläraja on 1,4. Tätä suuremmat arvot johtuvat usein mediaskleroosista (pseudohypertensio), mutta ne ovat myös merkki yleistyneestä valtimotaudista.

Alle 0,4 ABI-arvo tai alle 50 mmHg:n nilkkapaine viittaa krooniseen raajaa uhkaavaan iskemiaan, mutta mittaustulos tulee suhteuttaa oireisiin ja löydöksiin.

\end{solution}

\section{Kenelle seuraavista ohjelmoin kiireellisen gastroskopian?}\label{kenelle-seuraavista-ohjelmoin-kiireellisen-gastroskopian}

\begin{itemize}
\tightlist
\item
  \begin{enumerate}
  \def\labelenumi{\alph{enumi}.}
  \tightlist
  \item
    Uusi dyspepsia yli 50-vuotiaalla
  \end{enumerate}
\item
  \begin{enumerate}
  \def\labelenumi{\alph{enumi}.}
  \setcounter{enumi}{1}
  \tightlist
  \item
    Suolentoiminnan muutos
  \end{enumerate}
\item
  \begin{enumerate}
  \def\labelenumi{\alph{enumi}.}
  \setcounter{enumi}{2}
  \tightlist
  \item
    Närästys 30-vuotiaalla
  \end{enumerate}
\item
  \begin{enumerate}
  \def\labelenumi{\alph{enumi}.}
  \setcounter{enumi}{3}
  \tightlist
  \item
    Kirkas veri peräpäästä usean viikon ajan
  \end{enumerate}
\end{itemize}

\begin{solution}
\leavevmode

Vastaus

\begin{verbatim}
  a
\end{verbatim}

Gastroskopia on aiheellinen, jos potilaalla on yli 50--60-vuotiaana (useimmissa lähteissä 55-60, mutta usein yli 50v otetaan tähän myös mukaan) ilmaantunut ylävatsavaiva tai hänellä on hälyttäviä oireita tai löydöksiä.

b: Suolen toiminnan selvä ja muuttuminen on yleensä enemmän aihe kolonoskopialle. Esim. jos on suolen toiminnan selvä ja pitkäkestoinen (\textgreater{} 6 viikkoa) muuttuminen yli 50-vuotiaalla ja ulosteen hemoglobiinitesti (F-hHb) on raja-arvoinen (≥ 10 µg/g) tai positiivinen (\textgreater{} 25 µg/g), tehdään kolonoskopia. Samoin jos pitkittynyt ripuli (kesto \textgreater{} 4 viikkoa, \textgreater{} 3 löysää ulostetta/vrk) tai tulehduksellisen suolistosairauden epäily oireiden perusteella.

c: Uusi närästys 30-vuotiaalla ilman hälyttäviä oireita voidaan hoitaa PPI-hoitokokeilulla 4-8vk

d: Kirkas veri peräpäästä ilman selkeää proktologista syytä (esim. peräpukamat) on aihe ensisijaisesti kolonoskopialle, ei gastroskopialle

\pandocbounded{\includegraphics[keepaspectratio]{images/hälyttävätoireet.png}}

\end{solution}

\section{Potilastapaus}\label{potilastapaus-7}

Pienen terveyskeskuksen päiväpäivystykseen tulee 40-vuotias mies. Hänellä on alkanut edellisenä päivänä äkillinen, aaltomainen vasemman kyljen kipu, joka on voimistunut nyt niin, ettei hän pysty olemaan hetkeäkään paikalla. Epäilet virtsatiekiveä, koska potilaalla on niitä aikaiseminkin ollut. Potilas sanoo, että aikaisemmin yli 5mm:n kivet eivät ole tulleet itsenäisesti läpi. Mitä teet?

\begin{itemize}
\tightlist
\item
  \begin{enumerate}
  \def\labelenumi{\alph{enumi}.}
  \tightlist
  \item
    Otat terveyskeskuksessa pvk, crp, krea, plv ja lähetät potilaan erikoissairaanhoitoon TT-tutkimukseen, koska kiven koko pystytään määrittämään luotettavasti ainoastaan TT-tutkimuksessa
  \end{enumerate}
\item
  \begin{enumerate}
  \def\labelenumi{\alph{enumi}.}
  \setcounter{enumi}{1}
  \tightlist
  \item
    Otat terveyskeskuksessa pvk, crp, krea, plv ja kipulääkitset potilaan. Jos potilas tulee kivuttomaksi ja laboratoriotutkimukset ovat normaalit, laitat potilaan kotiin kipulääkitysreseptein ja teet kiireettömän lähetteen urologian poliklinikalle
  \end{enumerate}
\item
  \begin{enumerate}
  \def\labelenumi{\alph{enumi}.}
  \setcounter{enumi}{2}
  \tightlist
  \item
    Lähetät potilaan erikoissairaanhoidon päivystykseen virtsateiden TT-tutkimukseen, koska kiven koko pystytään määrittämään luotettavasti ainoastaan TT-tutkimuksessa
  \end{enumerate}
\item
  \begin{enumerate}
  \def\labelenumi{\alph{enumi}.}
  \setcounter{enumi}{3}
  \tightlist
  \item
    Otat terveyskeskuksessa virtsateiden UÄ-tutkimuksen ja hoidat potilaan konservatiivisesti, jos kivi on alle 5mm:n kokoinen. Jos kivi on kookkaampi, lähetät potilaan erikoissairaanhoidon päivystykseen
  \end{enumerate}
\end{itemize}

\begin{solution}
\leavevmode

Vastaus

\begin{verbatim}
  b
  
\end{verbatim}

Labroilla poissuljetaan infektio ja tarkistetaan munuaisten toiminta. Kuumeinen, yleisoireinen tai munuaisten toimintaan vaikuttava virtsatiekivi vaatii päivystyksellistä hoitoa. Kaikututkimus myös olisi hyvä tehdä avohoidossa (voi näyttää mm hydronefroosin ja jos se nähtäisiin, niin vaatisi päivystyksellistä hoitoa). Potilas on kivulias ja tarvitsee tämän takia kipulääkitystä. Jos kipu ohittuu ja labrat eivät osoita komplisoitumista, niin tilannetta voi jäädä seuraamaan ja tehdä kiireettömän lähetteen urologian poliklinikalle, jossa sitten voidaan miettiä esim. ESWL-hoitoa (extracorporeal shock wave lithotripsy) kiven ollessa suuri tai jos pienempi kivi ei poistu seurannassa.

a ja c: Virtsatiekivien ensisijainen diagnostinen tutkimus on kylläkin ilman varjoainetta kuvattava virtsateiden tietokonekerroskuvaus (vt-kivi-TT). Päivystykselliset toimet ovat kuitenkin liian aggressiivisia vastausvaihtoehtoja, jos potilas saadaan kivuttomaksi eikä punaisia lippuja (kuume, kohonnut krea, yksittäinen munuainen) ole todettavissa.

d: Ultraäänessä osa munuaiskivistä erottuu, mutta kiven koon arviointi on epäluotettavaa. Virtsanjohdinkivet erottuvat huonosti. Vaikka kivi erottuisi ja olisi selkeästi yli 5mm, niin se ei silti vaatisi päivystyksellistä hoitoa ilman komplisoitumisen merkkejä.

\end{solution}

\section{Potilastapaus}\label{potilastapaus-8}

Vastaanotollesi tulee uudestaan 70-vuotias nainen, jolla on toistuvasti tiheävirtsaisuutta ja virtsankirvellystä sekä toistetusti mikroskooppista hematuriaa. Virtsaviljelyssä ei edelleenkään ole bakteerikasvua. Mitä teet?

\begin{itemize}
\tightlist
\item
  \begin{enumerate}
  \def\labelenumi{\alph{enumi}.}
  \tightlist
  \item
    Mikroskooppinen hematuria ei ole uroteelisyövän oire ja kirjoitat potilaalle kokeeksi antibioottikuurin
  \end{enumerate}
\item
  \begin{enumerate}
  \def\labelenumi{\alph{enumi}.}
  \setcounter{enumi}{1}
  \tightlist
  \item
    Lähetät potilaan erikoissairaanhoitoon urologialle kystoskopiaan
  \end{enumerate}
\item
  \begin{enumerate}
  \def\labelenumi{\alph{enumi}.}
  \setcounter{enumi}{2}
  \tightlist
  \item
    Teetät potilaalle virtsaelinten UÄ-tutkimuksen. Jos siinä on jotain poikkeavaa, lähetät potilaan erikoissairaanhoitoon urologialle kystoskopiaan
  \end{enumerate}
\item
  \begin{enumerate}
  \def\labelenumi{\alph{enumi}.}
  \setcounter{enumi}{3}
  \tightlist
  \item
    Lähetät potilaan erikoissairaanhoitoon nefrologialle tarkempiin mikroskooppisen hematurian tutkimuksiin
  \end{enumerate}
\end{itemize}

\begin{solution}
\leavevmode

Vastaus

\begin{verbatim}
  b
  
\end{verbatim}

a: Yleisin uroteelisyövän oire on verivirtsaisuus, jota on ainakin 85 \%:lla rakkosyöpäpotilaista. Verivirtsaisuus voi olla makroskooppista eli silminnähtävää tai mikroskooppista eli virtsakokeen paljastamaa. Makroskooppisen hematurian mahdollinen väistyminen tai lieväasteisuus eivät poissulje pahanlaatuista uroteelikasvainta. Noin kolmanneksella rakkokasvainpotilaista on ärsytysoireita, kuten kivuliaisuutta virtsatessa, tiheä- ja yövirtsaisuutta sekä virtsapakon tunnetta.

Vaikka oireeton mikroskooppinen verivirtsaisuus voidaan usein jättää tutkimatta, kannattaa päätös tehdä yksilöllisesti riskitekijät huomioiden (katso alla olevista kuvista lähettämiskriteerit).

b: Polikliininen kystoskopia eli virtsarakon tähystys on virtsarakkosyövän perustutkimus, jossa nähdään mahdollisten tuumorien lukumäärä ja niiden koko. Ylävirtsateiden kuvantaminen kaikututkimuksella tai yleisemmin TT-urografialla tehdään myös kaikille potilaille toteamisen jälkeen. Koko vartalon TT tehdään levinneisyysselvittelynä epäiltäessä invasiivista syöpää.

c: Jos potilaalla todettaisiin UÄ:ssä epäilyttävää, niin se kylläkin olisi aihe lisätutkimuksille, mutta kystoskopia on ilman UÄ:täkin aiheellinen.

d: Mikroskooppisen hematurian selvittely on kylläkin usein aihe nefrologille lähettämiseen, mutta siihen tulee liittyä jotain munuaisten häiriöön liittyvää (esim. yleisoireet, proteinuria, heikentynyt GFR, kohonnut verenpaine varsinkin nuorella, anca-ab:t positiiviset tai epäily systeemitaudista)

\pandocbounded{\includegraphics[keepaspectratio]{images/lähettämiskriteeritrakkosyöpä.png}}
\pandocbounded{\includegraphics[keepaspectratio]{images/jatkotutkimuksetesh.png}}

\end{solution}

\section{Potilastapaus}\label{potilastapaus-9}

30--vuotias mies on ajanut humalassa lyhtytolppaa päin. Toimit ensiavun lääkärinä ja tutkit potilaan. RR 105/85, pulssi 110/min, SaO2 40\% lisähapella 85\%, hengitysfrekvenssi 25/min. Palpoiden aristaa vasenta puolta rintakehästä ja se myös rutisee palpaatiossa, auskultaatiossa hengitysäänet vaimeat kauttaaltaan mutta melussa vaikea saada hyvin kuunneltua, otsalla kookas kuhmu ja laseraatioita, tajunnan taso alentunut siten, että örisee kivulle. Mikä on oikea järjestys toiminnalle?

\begin{itemize}
\tightlist
\item
  \begin{enumerate}
  \def\labelenumi{\alph{enumi}.}
  \tightlist
  \item
    hätätorakotomia -\textgreater{} trauma-tt
  \end{enumerate}
\item
  \begin{enumerate}
  \def\labelenumi{\alph{enumi}.}
  \setcounter{enumi}{1}
  \tightlist
  \item
    pleuradreeni -\textgreater{} trauma-tt + pään TT -\textgreater{} tarvittaessa intubaatio
  \end{enumerate}
\item
  \begin{enumerate}
  \def\labelenumi{\alph{enumi}.}
  \setcounter{enumi}{2}
  \tightlist
  \item
    traumat-tt + pään TT -\textgreater{} intubaatio -\textgreater{} tarvittaessa pleuradreeni
  \end{enumerate}
\item
  \begin{enumerate}
  \def\labelenumi{\alph{enumi}.}
  \setcounter{enumi}{3}
  \tightlist
  \item
    natiivi-thx -\textgreater{} traumat-tt -\textgreater{} tarvittaessa pleuradreeni ja intubaatio
  \end{enumerate}
\end{itemize}

\begin{solution}
\leavevmode

Vastaus

\begin{verbatim}
  b
  
\end{verbatim}

Korkeaenerginen trauma, joka on todennäköisesti johtanut kylkiluumurtumaan ja sitä kautta ilmarintaan. Palpaatiossa myös kuuluu rutinaa, joka viittaa subkutaaniseen emfyseemaan. Saturaation heikentyminen, tajunnan alentuminen ja heikentynyt hemodynamiikka viittaavat mahdollisesti kehittyvään tensiopneumothoraxiin, joka vaatii välitöntä kajoavaa hoitoa. Jos paineilmarintaa epäillään traumapotilaalla tai elvytetyllä potilaalla (hengitys vaikeutuu, todetaan paineilmarintaan viittaavia löydöksiä), on syytä hoitaa tila, vaikkei diagnoosia varmentavaa keuhkokuvaa olisi mahdollista ottaa. Välitön hoito voi olla neulatorakosenteesi, mutta se harvoin riittää (tulisi asettaa useita, jos haluttaisiin riittävän), jonka takia voikin usein asettaa suoraan pleuradreenin paikoilleen.

Traumapotilas vaatii ilmarinnan hoidon lisäksi lisätutkimuksia, jotka korkeaenerginen tylppä vamma rintakehän alueella ja pään vammat huomioiden tulisi olla trauma-tt ja pään TT. Intubaatiota tulee myös vahvasti harkita, koska potilaan tajunta on alentumassa ja hengitysvaikeudet ovat selvät.

a: Hätätorakotomialle ei ole aihetta. Sen selkeä indikaatio olisi, jos todettaisiin sydämenpysähdys penetroivan vamman jälkeen.

c-d: Potilaan kuvantaminen ennen mahdollisen tensiopneumothoraxin hoitoa viivästyttää pleuradreenin asettamista ja altistaa potilaan tilan pahenemiselle. Siksi ensin dekompressio / pleuradreeni, jotta hänet voidaan turvallisesti kuvata ja tarvittaessa intuboida.

\end{solution}

\section{Potilastapaus}\label{potilastapaus-10}

65-vuotiaalla rouvalla on hankalat peräpukamat, jotka on suunniteltu leikattaviksi (ei pelkät Barronin ligatuurat). Hänellä on kuitenkin 2 vuotta sitten laitettu mekaaninen mitraaliläppäproteesi (INR hoitoalue 2,5-3,5). Mikä pitää paikkansa?

\begin{itemize}
\tightlist
\item
  \begin{enumerate}
  \def\labelenumi{\alph{enumi}.}
  \tightlist
  \item
    peräsuolen verenkierto on hyvä, eikä infektioriskiä ole yli vuoden vanhalla proteesilla mikäli se on mitraalipositiossa eli ei ab-profylaksin tarvetta, siltahoito pienimolekyläärisellä heparinilla kuitenkin tarvitaan
  \end{enumerate}
\item
  \begin{enumerate}
  \def\labelenumi{\alph{enumi}.}
  \setcounter{enumi}{1}
  \tightlist
  \item
    konsultoidaan kardiologia ja sydänkirurgia siitä voisiko mekaanisen proteesin vaihtaa biologiseksi
  \end{enumerate}
\item
  \begin{enumerate}
  \def\labelenumi{\alph{enumi}.}
  \setcounter{enumi}{2}
  \tightlist
  \item
    tauotetaan varfariinihoito ja tehdään toimenpide ab-profylaksiassa INR ollessa alle 2
  \end{enumerate}
\item
  \begin{enumerate}
  \def\labelenumi{\alph{enumi}.}
  \setcounter{enumi}{3}
  \tightlist
  \item
    tehdään toimenpide siltahoidossa ja ab-profylaksiassa
  \end{enumerate}
\end{itemize}

\begin{solution}
\leavevmode

Vastaus

\begin{verbatim}
  d
\end{verbatim}

Varfariini on vaikea lääke ja sen ongelmat vielä korostuvat perioperatiivisesti. Suuren vuotoriskin toimenpiteissä tulee harkita ns. siltahoitoa eli varfariinin tauotusta ja LMWH-hoidon aloittamista pitämään antikoagulaatiota yllä leikkaukseen asti. Siltahoitoa voinee harkita, jos lääkitystauko on pitkä toimenpiteeseen liittyvän vuotoriskin takia ja/tai potilaalla on erityisen suuri tukosriski. Siltahoidon kriteerit ovat viime vuosina tiukentuneet, ja sitä suositellaan nykyisin ainoastaan suuren tukosriskin potilaille.

Potilaan mekaaninen mitraalitekoläppä on keskimääräisen tukosriskin riskitekijä (korkea riski, jos olisi aivoinfarktin riskitekijöitä samalla). Potilas ei siis suoraan kuulu korkean tukosriskin kastiin potilastapauksen perusteella, joten periaatteessa siltahoitoa ei välttämättä tarvittaisi. Kysymyksen tekovuotena kuitenkin siltahoito on todennäköisesti ollut selkeämmin indikoitua eikä ole tarvinnut tuoda esille potilaan muita riskitekijöitä.

a: Keinomateriaalitekoläpät ovat erityisen alttiita infektioille ja tämän takia limakalvoa rikkovissa toimenpiteissä vaaditaan antibioottiprofylaksia.

b: Proteesia ei oltaisi todellakaan vaihtamassa peräpukamien leikkauksen takia.

c: Erityisesti jos potilaalla on mekaaninen tekoläppä tai muuten suuri tukosriski, varfariinihoitoa ei saa yleensä keskeyttää eikä keventääkään, ellei käytetä korvaavaa antikoagulanttia (LMWH:ta). Tämä ohje ei ole enää niin absoluuttisen pitävä, mutta kuitenkin hyvä perusohje pitää mielessä.

\pandocbounded{\includegraphics[keepaspectratio]{images/hoidonkeventäminen.png}}
\pandocbounded{\includegraphics[keepaspectratio]{images/siltahoidontarve.png}}
\pandocbounded{\includegraphics[keepaspectratio]{images/siltahoidontoteutus.png}}

\end{solution}

\section{Mikä väittämistä pitää paikkaansa?}\label{mikuxe4-vuxe4ittuxe4mistuxe4-pituxe4uxe4-paikkaansa}

\begin{itemize}
\tightlist
\item
  \begin{enumerate}
  \def\labelenumi{\alph{enumi}.}
  \tightlist
  \item
    Alaraajan laskimovajaatoiminnan oireisiin ei tukisukalla ole mitään apua
  \end{enumerate}
\item
  \begin{enumerate}
  \def\labelenumi{\alph{enumi}.}
  \setcounter{enumi}{1}
  \tightlist
  \item
    Alaraajan laskimovajaatoiminnassa kyse on siitä, että pintalaskimojärjestelmän paine on koholla silloin kun ihminen on pystyasennossa
  \end{enumerate}
\item
  \begin{enumerate}
  \def\labelenumi{\alph{enumi}.}
  \setcounter{enumi}{2}
  \tightlist
  \item
    Tupakointi on alaraajan laskimovajaatoiminnan tunnettu riskitekijä
  \end{enumerate}
\item
  \begin{enumerate}
  \def\labelenumi{\alph{enumi}.}
  \setcounter{enumi}{3}
  \tightlist
  \item
    Alaraajojen laskimovajaatoiminta on aina harmiton, kosmeettinen vaiva. Siitä ei voi aiheutua terveydelle haitallisia oireita
  \end{enumerate}
\end{itemize}

\begin{solution}
\leavevmode

Vastaus

\begin{verbatim}
  b
\end{verbatim}

Pintalaskimoiden vajaatoiminnan etiologia on tuntematon, mutta taustalla on inflammaatio, jonka pohjalta laskimoläpät eivät ole enää pitävät. Normaalisti pinnallisten laskimoiden normaali läppätoiminta estää laskimoveren takaisinvirtauksen (refluksi). Sulkeutuneet läpät katkaisevat laskimon sisällä olevan veripilarin muutaman sentin välein ja tämä vähentää hydrostaattista painetta. Läppien vajaatoiminnassa pintalaskimojärjestelmän paine on koholla ja tämä johtaa mm. laskimoiden turvotukseen (suonikohjut).

a: Käytännössä kaikille laskimovajaatoiminnasta (kannattaa huomioida kuitenkin dekompensoitunut sydämen vajaatoiminta ja kriittinen alaraajaiskemia vasta-aiheina) valittaville potilaille voi suositella kompressiohoitoa. \textbf{Kompressio ei estä taudin etenemistä, mutta helpottaa oireita.}

c: On jotain viitteitä siitä, että tupakointi voisi vaurioittaa laskimon seinämiä ja läppiä ja täten altistaa laskimovajaatoiminnalle, mutta tupakointi ei ole läheskään yhtä merkittävä riskitekijä laskimopuolen ongelmille kuin valtimopuolen ongelmille (arterioskleroosille).

d: Väärin; Laskimovajaatoiminta voi johtaa mm. haavaumiin ja siten infektioihin, eikä ole aina harmiton.

\pandocbounded{\includegraphics[keepaspectratio]{images/laskimopaine.png}}

\end{solution}

\section{Teet triagea kirurgian päivystyksessä. Kaikki seuraavat tilat vaativat nopeaa toimenpideinterventiota, mutta mikä niistä on kaikkein kiireellisin:}\label{teet-triagea-kirurgian-puxe4ivystyksessuxe4.-kaikki-seuraavat-tilat-vaativat-nopeaa-toimenpideinterventiota-mutta-mikuxe4-niistuxe4-on-kaikkein-kiireellisin}

\begin{itemize}
\tightlist
\item
  \begin{enumerate}
  \def\labelenumi{\alph{enumi}.}
  \tightlist
  \item
    strangulaatioepäily
  \end{enumerate}
\item
  \begin{enumerate}
  \def\labelenumi{\alph{enumi}.}
  \setcounter{enumi}{1}
  \tightlist
  \item
    akuutti scrotum
  \end{enumerate}
\item
  \begin{enumerate}
  \def\labelenumi{\alph{enumi}.}
  \setcounter{enumi}{2}
  \tightlist
  \item
    täyttöön reagoimaton aorta-aneurysmaruptuura
  \end{enumerate}
\item
  \begin{enumerate}
  \def\labelenumi{\alph{enumi}.}
  \setcounter{enumi}{3}
  \tightlist
  \item
    raju kolekystiitti
  \end{enumerate}
\end{itemize}

\begin{solution}
\leavevmode

Vastaus

\begin{verbatim}
  c
\end{verbatim}

Aneurysmaruptuura + ei vastetta nesteytykselle → suora hengenvaara. Kuolleisuus minuuteissa (ruptuurassa usein jopa kuolema välittömästi) riippuen ruptuuran tyypistä ja vakavuudesta, vaatii välitöntä leikkaussaliin vientiä. Tämä on selvästi kiireellisin kaikista.

a: Strangulaatioepäily -- erittäin kiireellinen, mutta ei lähes \emph{välittömästi} kuolettava kuten hoitamaton ruptuura

b: Akuutti scrotum (todennäköisesti testistorsio) -- kirurginen päivystysleikkaus. Leikkaus tulee tehdä viiden tunnin kuluessa oireiden alkamisesta.

d: Raju kolekystiitti -- myös päivystyksellinen leikkaus, mutta ei samalla tavalla \emph{välittömästi} hengenvaarallinen kuin aortan aneurysmaruptuura

\end{solution}

\section{Potilastapaus}\label{potilastapaus-11}

Tk-päivystykseen tulee 68-vuotias miespotilas, jolla on anamneesissa metabolinen oireyhtymä. Potilaalle on kivespussin alueelle tullut vuorokauden aikana nopeaan tahtiin punoitusta ja muutamia hemorragisia rakkuloita. Aristaa selvästi palpaatiota. Potilaalla on korkea kuume ad 39 astetta. Lab.kokeissa Leuk 23, CRP 75. Mitä epäilet diagnoosiksi?

\begin{itemize}
\tightlist
\item
  \begin{enumerate}
  \def\labelenumi{\alph{enumi}.}
  \tightlist
  \item
    Fournierin gangreena
  \end{enumerate}
\item
  \begin{enumerate}
  \def\labelenumi{\alph{enumi}.}
  \setcounter{enumi}{1}
  \tightlist
  \item
    Erysipelas
  \end{enumerate}
\item
  \begin{enumerate}
  \def\labelenumi{\alph{enumi}.}
  \setcounter{enumi}{2}
  \tightlist
  \item
    Epididymiitti ja orkiitti
  \end{enumerate}
\item
  \begin{enumerate}
  \def\labelenumi{\alph{enumi}.}
  \setcounter{enumi}{3}
  \tightlist
  \item
    Kivespussin selluliitti
  \end{enumerate}
\end{itemize}

\begin{solution}
\leavevmode

Vastaus

\begin{verbatim}
  a
\end{verbatim}

Fournier'n grangreena on käytännössä kivespussin ja/tai välilihan alueen nekrotisoiva faskiitti. Usein taustalla on jokin perineumin trauma tai ihorikkoon johtava asia ja muita vakaville infektioille altistavia tekijöitä, kuten alkoholismi, DM ja obesiteetti (metabolinen oireyhtymä kuten potilaalla) tai immunosuppressio. Tilalle tyypillisiä oireita ovat alkuun selluliitin kaltainen tilanne (potilaalla punoitusta nopeaan tahtiin), jonka jälkeen merkittävä kipu (potilasta aristaa palpaatiossa), yleistilan lasku (potilaalla korkea kuume) ja lopulta nekroottisten alueiden ilmentyminen (potilaalla hemorragisia rakkuloita, jotka voivat viitata tähän). Tila sopii Fournier'n gangreenaan ja vaatii välitöntä hoitoa.

Fournier'n gangreenin hoitolinjat ovat samanlaiset kuin kaasukuolion ja nekrotisoivan faskiitin: hoitona ovat radikaali kirurginen revisio ja laajakirjoinen antimikrobihoito sekä mahdollisesti ylipainehappihoito. Revisiossa kaikki kuolioitunut kudos poistetaan ja tehdään lisäksi tarvittavat laajat kudosavaukset tilan leviämisen selvittämiseksi ja hoitamiseksi. Usein on tarpeen tehdä paksusuoliavanne ulostekontaminaation poistamiseksi revidoidulta alueelta.

b: Pelkkä ruusu (erysipelas) ei oireilisi näin vakavana.

c: Epididymiitti (lisäkivestulehdus) ja orkiitti (kivestulehdus) aiheuttaisivat oireita keskittyen kivespussin sisälle eikä niinkään aiheuttaisi kivespussin ihomuutoksia tai näin vakavia yleisoireita.

d: Selluliitissa yleistila hyvä, mutta Fournier'n gangreenissa taas yleisoireita ja kipu suurempi. Aluksi Fournier'n gangreeni voi oireilla kuten kivespussin selluliitti, mutta edetessä yleistila laskee ja potilas menee septiseksi.

\pandocbounded{\includegraphics[keepaspectratio]{images/fournier.png}}

\end{solution}

\section{Akuutti haimatulehdus on patogeneettisesti}\label{akuutti-haimatulehdus-on-patogeneettisesti}

\begin{itemize}
\tightlist
\item
  \begin{enumerate}
  \def\labelenumi{\alph{enumi}.}
  \tightlist
  \item
    autoinflammatorinen tauti
  \end{enumerate}
\item
  \begin{enumerate}
  \def\labelenumi{\alph{enumi}.}
  \setcounter{enumi}{1}
  \tightlist
  \item
    autodigestiivinen tauti
  \end{enumerate}
\item
  \begin{enumerate}
  \def\labelenumi{\alph{enumi}.}
  \setcounter{enumi}{2}
  \tightlist
  \item
    autoimmuunitauti
  \end{enumerate}
\item
  \begin{enumerate}
  \def\labelenumi{\alph{enumi}.}
  \setcounter{enumi}{3}
  \tightlist
  \item
    infektio
  \end{enumerate}
\end{itemize}

\begin{solution}
\leavevmode

Vastaus

\begin{verbatim}
  b
\end{verbatim}

Akuutti pankreatiitti johtuu ennenaikaisesta haiman ruuansulatusentsyymien aktivoitumisesta ja sitä seuraavasta haiman autodigestiosta (eli parenkyymin itsetuhosta). Asinaarisoluissa tuotetut ruuansulatusentsyymit, kuten trypsinogeeni, vuotavat solujen ulkoiseen tilaan asinussoluvaurion seurauksena -\textgreater{} trypsinogeeni aktivoituu -\textgreater{} aktivoitunut trypsiini aktivoi muita vuotaneita entsyymejä, kuten elastaasin, lipaasin ja fosfolipaasin. Nämä entsyymit sitten aiheuttavat haiman parenkyymin nestemäistä hemorragista nekroosia ja peripankreaattisen rasvan rasvanekroosia. Kyseessä ei siis tyypillisesti ole infektion aiheuttama tila, vaan tulehdus on ns. steriiliä.

a-c: Haimatulehdus voi olla autoimmuunipohjainen, mutta se on harvinainen ja ilmenee tyypillisesti kroonisena pankreatiittina.

d: Pankreatiitti voi olla infektiivinen, mutta ne tapaukset ovat harvinaisia.

\end{solution}

\section{Potilastapaus}\label{potilastapaus-12}

Tk-lääkärin vastaanotollesi tulee 50-vuotias nainen, joka on tuntenut oikeassa rinnassaan uuden kyhmyn. Rintojen ja kainaloiden palpaatiossa toteat oikean rinnan ylälateraalineljänneksen alueella aristamattoman, mobiilin, noin 2 cm kokoisen kiinteän resistenssin. Potilaan sisko ja äiti ovat sairastaneet vastaavassa iässä rintasyövän. Mikä on tärkein hoitolinja?

\begin{itemize}
\tightlist
\item
  \begin{enumerate}
  \def\labelenumi{\alph{enumi}.}
  \tightlist
  \item
    Teet lähetteen kirurgille rintasyöpäepäilynä
  \end{enumerate}
\item
  \begin{enumerate}
  \def\labelenumi{\alph{enumi}.}
  \setcounter{enumi}{1}
  \tightlist
  \item
    Kehotat potilasta odottamaan aikaa seulontamammografiaan
  \end{enumerate}
\item
  \begin{enumerate}
  \def\labelenumi{\alph{enumi}.}
  \setcounter{enumi}{2}
  \tightlist
  \item
    Teet lähetteen geneetikolle perinnöllisyysselvitykseen
  \end{enumerate}
\item
  \begin{enumerate}
  \def\labelenumi{\alph{enumi}.}
  \setcounter{enumi}{3}
  \tightlist
  \item
    Ohjaat potilaan kiireelliseen rintojen mammografiaan ja ultraääniohjattuun paksuneulabiopsiaan
  \end{enumerate}
\end{itemize}

\begin{solution}
\leavevmode

Vastaus

\begin{verbatim}
  d
\end{verbatim}

Rintasyövän kolmoisdiagnostiikka: (1) Rinnan inspektio ja palpaatio, (2) Kuvantamistutkimukset (mammografia ja sitä täydentävät menetelmät) ja (3) Paksuneulanäyte. Kaikkien näiden osa-alueiden tulee yksiselitteisesti viitata hyvänlaatuisuuteen, jotta muutosta voidaan jäädä vain seuraamaan. Epävarmaksi jäänyt muutos poistetaan.

a: Ei vielä lähetettä kirurgialle, sitä ennen tulee tehdä muut kolmoisdiagnostiikkaan kuuluvat perusdiagnostiset tutkimukset.

b: Nyt ei enää jäädä odottamaan seulontamammografiaa, vaan tehdään lähetteet jatkotutkimuksiin. Potilas kylläkin on 50-vuotias, joten seulonta varmaan tulisi kohta (seulontaan kuuluu 50-69-vuotiaat naisen n.~kahden vuoden välein), mutta ei odotella sitä.
c: BRCA1/2-geeniselvitykset voivat olla myöhemmin aiheellisia potilaalla ja myös hänen sukulaisillaan, mutta eivät ole nyt heti tarpeellisia.

d: Kiireellinen mammografia ja UÄ-ohjattu paksuneulabiopsia kuuluvat kolmoisdiagnostiikkaan ja ovat tärkeimmät seuraavat toimenpiteet epäilyttävän palpaatiolöydöksen jälkeen.

\end{solution}

\section{Potilastapaus}\label{potilastapaus-13}

Potilaalle on tehty 3pv sitten laparotomiateitse anteriorinen resektio ylärektumin karsinooman takia. Ei avanteita, suora suolisauma. Potilas kuumeilee, vatsa aristaa kauttaaltaan. Mikä alla olevista löydöksistä viittaa suoraan lekaasiin eli suolisauman pettämiseen?

\begin{itemize}
\tightlist
\item
  \begin{enumerate}
  \def\labelenumi{\alph{enumi}.}
  \tightlist
  \item
    Vapaa ilma vatsaontelossa TT-kuvassa
  \end{enumerate}
\item
  \begin{enumerate}
  \def\labelenumi{\alph{enumi}.}
  \setcounter{enumi}{1}
  \tightlist
  \item
    CRP 200
  \end{enumerate}
\item
  \begin{enumerate}
  \def\labelenumi{\alph{enumi}.}
  \setcounter{enumi}{2}
  \tightlist
  \item
    Defance vatsanpeitteissä
  \end{enumerate}
\item
  \begin{enumerate}
  \def\labelenumi{\alph{enumi}.}
  \setcounter{enumi}{3}
  \tightlist
  \item
    Per rect annettu vesiliukoinen varjoaine karkaa sauman ulkopuolelle TT-kuvassa
  \end{enumerate}
\end{itemize}

\begin{solution}
\leavevmode

Vastaus

\begin{verbatim}
  d
\end{verbatim}

Kirurgista komplikaatiota on epäiltävä, mikäli potilas ei toivu odotetusti tai vointi huononee leikkauksen jälkeen! 1 kk sisään leikattu vatsapotilas -\textgreater{} muista aina epäillä kirurgista komplikaatiota -\textgreater{} kirurgin konsultaatio.

Saumalekaasi eli suolisauman pettäminen tapahtuu useimmiten 5. post-operatiivisena päivänä (POP) ja vaihteluväli yleensä n.~3-10 POP (on tietysti myös aikaisempia ja myöhäisempiäkin).

Tärkein lekaasitutkimus on vesiliukoinen varjoaine-TT. Proksimaalinen (ruokatorvi/ventrikkeli) ja distaalinen (rektum, sigman alaosa) sauma ovat helpoimmat kuvantamiskohteet (proksimaaliset p.o. juottovarjoaineella ja distaaliset per rectum varjoaineella), mutta ohutsuoli ja proksimaalinen paksusuoli ongelmallisempia. Jos kuvantamisessa nähdään, että varjoainetta karkaa suolesta, niin tiedetään, että sauma pettää karkaamiskohdasta.

b: CRP arvo postoperatiivisesti ennustaa lekaasia, mutta ei ole suora osoitus siitä. Laparotomisessa leikkauksessa huippu-CRP voi vielä normaalissakin tilanteessa osua 3. päivän kohdalle. Muuten tulehdusmittareihin liittyen niin potilaalla on kuumetta \textgreater48h leikkauksesta, mikä viittaa infektiokompliaatioon (1-48h leikkauksesta lämpöily johtuu yleensä toimenpiteestä).

c: Jokin muukin tekijä voisi aiheuttaa vatsanpeitteiden defancea. Kauttaaltaan aristavat vatsanpeitteet kylläkin viittaavat leikkauskomplikaatioon, mutta eivät ole suora osoitus suolisauman vuotamisesta.

d: Vatsan alueen leikkauksen jälkeen näkyy yleensä vielä pitkäänkin ilmaa intra-abdominaalisesti normaalitilanteessakin. Laparotomian jälkeen ilmaa usein ad 1vk operaatiosta, tämänkin jälkeen voi olla pieniä määriä nähtävissä jopa 3vk kohdalla. Laparoskopian jälkeen intra-abdominaalinen kaasu häviää tyypillisesti n.~3 vrk:ssa.

\end{solution}

\section{Potilastapaus}\label{potilastapaus-14}

60-vuotiaalle aiemmin terveelle miehelle on tehty Hartmanin toimenpide fekaaliperitoniittiin johtaneen sigman divertikkeliperforaation johdosta. Hän haluaisi kovasti avanteesta eroon. Kerrot hänelle, että:

\begin{itemize}
\tightlist
\item
  \begin{enumerate}
  \def\labelenumi{\alph{enumi}.}
  \tightlist
  \item
    Hänelle joudutaan tekemään J-pussileikkaus
  \end{enumerate}
\item
  \begin{enumerate}
  \def\labelenumi{\alph{enumi}.}
  \setcounter{enumi}{1}
  \tightlist
  \item
    Avanne on pysyvä
  \end{enumerate}
\item
  \begin{enumerate}
  \def\labelenumi{\alph{enumi}.}
  \setcounter{enumi}{2}
  \tightlist
  \item
    Lenkkiavanne voidaan sulkea paikallisesti pienestä viillosta puolen vuoden kuluttua
  \end{enumerate}
\item
  \begin{enumerate}
  \def\labelenumi{\alph{enumi}.}
  \setcounter{enumi}{3}
  \tightlist
  \item
    Avanne pyritään sulkemaan aikaisintaan puolen vuoden kuluttua laparoskopia-/laparotomiateitse
  \end{enumerate}
\end{itemize}

\begin{solution}
\leavevmode

Vastaus

\begin{verbatim}
  d
  
\end{verbatim}

Perforoituneen divertikuliitin, joka on aiheuttanut peritoniitin hoito on iv antibiootti ja päivystyksellinen leikkaus. Leikkauksessa poistetaan tulehtunut suolen osa, yleensä sigmasuoli. Lievimmissä tapauksissa ja hyväkuntoisilla potilailla suolen kontinuiteetti voidaan palauttaa, mutta useimmiten joudutaan turvautumaan paksusuolen pääteavanteeseen (Hartmannin leikkaus = poistetaan tulehtunut suolen osa, ommellaan peräsuolistumppi kiinni ja nostetaan suolen proksimaalinen osa pääteavanteeksi). Avanne suljetaan myöhemmässä vaiheessa elektiivisesti (joitakin kuukausia myöhemmin, yleensä aikaisintaan 6kk) tulehduksen rauhoituttua.

a: J-pussi rakennetaan tyypillisesti haavaisen koliitin leikkauksessa. Peräaukon säästävä proktokolektomia ja ileoanaalinen suoliliitos yhdistettynä suolisäiliön rakentamiseen (IPAA) on nykyisin käytetyin leikkausmenetelmä haavaista koliittia sairastavilla potilailla. Toimenpiteessä siis ohutsuolen loppuosasta muodostetaan yleensä J:n muotoinen säiliö, ns. J-pussi, joka yhdistetään peräaukkoon (kuvat leikkauksen periaatteista alla).

b: Avanne pyritään mahdollisuuksien ja potilaan toiveiden mukaan sulkemaan. Leikkaukseen liittyy kuitenkin noin 10-20 \%:n vakavan komplikaation riski.

c: Pääteavanteen sulkeminen suoritetaan ensisijaisesti laparoskooppisesti. Kyseessä ei siis Hartmannin toimenpiteen jälkeen ole lenkkiavanne (loop-avanne) kuten vastausvaihtoehdossa, vaan pääteavanne.

Pääteavanteella tarkoitetaan sellaista avannetta, jossa on ainoastaan yksi suolen ulkoaukko eli luumen. Lenkkiavanteessa on kaksi luumenia. Vain proksimaalinen luumen tuottaa ulostetta ja distaalinen luumen toimii limafistelinä (limafisteli = pääteavanne, josta ei ole yhteyttä käytössä olevaan osaan suolistoa). Lenkkiavanne voidaan myöhemmin sulkea paikallisesti avanteen kohdalta tekemättä suurempaa leikkaushaavaa. Pääteavanteen yhdistäminen esimerkiksi peräsuolistynkään ei onnistu ilman isoa leikkausta.

Pääte- ja lenkkiavanteen lisäksi on vielä ns. kaksipiippuinen avanne, jossa iholle nostetaan kaksi erillistä suolenpäätä. Vrt. lenkkiavanteeseen, jossa iholla olevat suolen päät ovat vielä yhteydessä toisiinsa pohjastaan, mutta kuitenkin ovat halkaistu auki iholle. Esimerkkikuvat avanteista alla.

\begin{figure}
\centering
\pandocbounded{\includegraphics[keepaspectratio]{images/hartmann.png}}
\caption{Hartmannin leikkaus}
\end{figure}

\pandocbounded{\includegraphics[keepaspectratio]{images/jpussi.png}}
\pandocbounded{\includegraphics[keepaspectratio]{images/avannetyypit.png}}

\end{solution}

\section{Potilastapaus}\label{potilastapaus-15}

Olet päivystävänä lääkärinä sairaalassa. Päivystykseen tulee kalpea, kipeän oloinen herra, jolla on kova selkään ja nivusiin säteilevä vatsakipu. Potilas pyörtyi ja heräämisen jälkeen hän koki voivansa huonosti. Myös kipu alkoi pyörtymisen yhteydessä. Miten EI pidä toimia?

\begin{itemize}
\tightlist
\item
  \begin{enumerate}
  \def\labelenumi{\alph{enumi}.}
  \tightlist
  \item
    Mietin, että kuulostaa virtsatiekiveltä ja potilaalla onkin virtsassa eryt +. Anna potilaalle kipulääkereseptin ja kotiutan potilaan. Lisäksi kerron, että vaiva kyllä korjaantuu todennäköisesti, kun virtsatiekivet ovat tulleet pois virtsajohtimesta
  \end{enumerate}
\item
  \begin{enumerate}
  \def\labelenumi{\alph{enumi}.}
  \setcounter{enumi}{1}
  \tightlist
  \item
    TT-kuvauksessa on diagnoosina rupturoitunut abdominaaliaortan aneurysma. Tämän vuoksi kutsun paikalle verisuonikirurgian päivystäjän ja anestesiapäivystäjän. Lisäksi välitän tiedon potilaasta päivystysleikkausyksikköön. Anestesiapäivystäjää odotellessa potilas on verenpaineseurannassa. Korjaan verenpainetta vain, jos potilaan tajunnan taso alkaa huonontua.
  \end{enumerate}
\item
  \begin{enumerate}
  \def\labelenumi{\alph{enumi}.}
  \setcounter{enumi}{2}
  \tightlist
  \item
    Tutkin potilaan kiireisenä potilaana ja pyydän ultraäänitutkimuksen tai jos mahdollista kiireisen vatsan TT-kuvauksen.
  \end{enumerate}
\item
  \begin{enumerate}
  \def\labelenumi{\alph{enumi}.}
  \setcounter{enumi}{3}
  \tightlist
  \item
    TT-kuvauksessa on diagnoosina rupturoitunut abdominaaliaortan aneurysma. Tämän vuoksi kutsun paikalle verisuonikirurgian päivystäjän ja anestesiapäivystäjän. Lisäksi välitän tiedon potilaasta päivystysleikkausyksikköön
  \end{enumerate}
\end{itemize}

\begin{solution}
\leavevmode

Vastaus

\begin{verbatim}
  a
  
\end{verbatim}

Vatsa-aortan aneurysman repeämälle (RAAA) tyypillistä on äkillisesti alkanut vatsakipu, joka säteilee selkään. Kipu voi joskus säteillä kylkeen, nivustaipeisiin, kiveksiin tai reiden yläosiin. Alkuvaiheessa voi oireena olla myös pyörtyminen (Jos repeäminen tapahtuu vatsaontelon takaiseen retroperitoneaalitilaan, vuoto ja kipu voi aiheuttaa ensin potilaan tajunnanmenetyksen, mutta vuoto voi tamponoitua ja potilas palaa sitten tajuihinsa tamponaation aiheuttaman vuodon vähenemisen seurauksena. Tila stabiloituu hetkeksi, kunnes vuoto taas jatkuu. Jos repeämä on aneurysman etuseinässä ja vuoto pääsee vapaaseen vatsaonteloon, potilas usein kuolee ennen kuin hän ehtii sairaalaan). Nämä kaikki sopivat potilaan tilanteeseen hyvin.

b: Leikkausta edeltävä nesteresuskitaatio pyritään optimoimaan siten, että potilas on kontrolloidussa hypotensiossa (hypotensiivinen hemostaasi, systolinen verenpaine 50--70 mmHg) ja leikkaussaliin siirrytään mahdollisimman nopeasti. Tämän takia verenapinetta ei tarvitse korjata, ellei tajunnan taso lähde rommaamaan. Ennen leikkausta systolista verenpainetta ei ole tarkoituksenmukaista nostaa yli 90 mmHg:n.~\textbf{Perussääntönä voidaan pitää, että jos potilas kommunikoi, verenpaine on hyväksyttävällä tasolla, vaikka se olisi kuinka matala tahansa.}

c: Vatsa-aortan anuerysmassa tehdään aina leikkaustekniikan päättämisen tueksi varjoainetehosteinen TT, ja tämä tehdään mahdollisuuksien mukaan myös rupturoituneessa AAAssa, koska vatsa-aortan repeämää epäiltäessä ultraäänitutkimus ei tyypillisesti ole riittävä. Hätätilanteessa kuitenkin ei aina ole aikaa/mahdollisuutta TT:lle ja ultra voi jäädä ainoaksi kuvantamistutkimukseksi.

d: Rupturoituneet vatsa-aortat tyypillisesti leikkaa verisuonikirurgi. Anestesiapäivystäjä myös tietysti tarvitaan paikalle potilaan elintoiminnoista huolehtimiseen ja potilaan saattamiseen toimenpidevalmiiksi. Kaikki RAAA-potilaat otetaan TYKSissä hybridisaliin 14 sulkupallovalmiudessa (REBOA, resuscitative endovascular balloon occlusion of the aorta). Repeytynyt vatsa-aortan aneurysma hoidetaan periaatteessa samalla toimenpidetekniikalla kuin repeytymätön aneurysma eli suonensisäinen hoito tehdään, jos se on saatavissa ja potilaan anatomia soveltuu stenttiproteesin asennukseen. Suonensisäinen hoito voidaan tehdä paikallispuudutuksessa yhdistettynä sedaatioon. Toimenpiteessä stenttiproteesi avataan munuaisvaltimoiden alapuoliseen aorttaan sulkupallon alapuolelle, ja sulkupallo siirretään munuaisvaltimoiden alapuolelle stenttiproteesin sisään heti, kun se on mahdollista.

\end{solution}

\section{Mihin allaolevista EI liity pahanlaatuistumisen riskiä nykytiedon mukaan?}\label{mihin-allaolevista-ei-liity-pahanlaatuistumisen-riskiuxe4-nykytiedon-mukaan}

\begin{itemize}
\tightlist
\item
  \begin{enumerate}
  \def\labelenumi{\alph{enumi}.}
  \tightlist
  \item
    Komplisoitumaton divertikuliitti
  \end{enumerate}
\item
  \begin{enumerate}
  \def\labelenumi{\alph{enumi}.}
  \setcounter{enumi}{1}
  \tightlist
  \item
    Barretin ruokatorvi
  \end{enumerate}
\item
  \begin{enumerate}
  \def\labelenumi{\alph{enumi}.}
  \setcounter{enumi}{2}
  \tightlist
  \item
    Tynkämaha
  \end{enumerate}
\item
  \begin{enumerate}
  \def\labelenumi{\alph{enumi}.}
  \setcounter{enumi}{3}
  \tightlist
  \item
    Colonin adenooma
  \end{enumerate}
\end{itemize}

\begin{solution}
\leavevmode

Vastaus

\begin{verbatim}
  a
  
\end{verbatim}

Jos potilaalla on komplisoitunut divertikuliitti (absessi, peritoniitti, fisteli, suolitukos), niin tulee suorittaa jatkossa kontrollitutkimuksena kolonoskopia rauhallisessa vaiheessa (n.~1kk jälkeen akuutista tulehduksesta). Tulee varmistaa, ettei kyseessä ole paksusuolisyöpä, jonka olisi ajateltu olevan divertikuliitti TT-kuvassa. N. 10\%:lla potilaista onkin oikeasti maligniteetti. Komplisoitumattomassa divertikuliittissa ei ole todettu olevan tällaista riskiä, jonka takia kontrollitutkimuksia ei tarvita.

b: Barretin esofagukselle on tyypillistä ruokatorven alaosien mukoosan intestinaalinen metaplasia (ei-keratinisoiva kerrostunut levyepiteeli muuttuu non-siliaariseksi lieriöepiteeliksi, jossa on pikarisoluja). Tämä tapahtuu tyypillisesti vasteena pitkäaikaiselle refluksitaudille, kun ruokatorven mukoosa muuttuu paremmin hapokasta refluksia kestäväksi.

Ruokatorven adenokarsinooma kehittyy lähes aina Barretin esofaguksen kautta (metaplasia-dysplasia-karsinoomasekvenssi). N. 3-5\%:lla potilaista Barretin ruokatorvi voi edetä ruokatorven adenokarsinoomaksi eliniän aikana.

c: Tynkämaha eli resekoitu mahalaukku (esim. lihavuusleikkaus) lisää riskiä mahasyövälle. Tyypillisesti ilmenee n.~15--20 vuoden kuluttua operaatiosta.

d: Paksusuolen adenoomat ovat polyyppityyppi (toiset ovat hyperplastisia polyyppejä), jotka voivat adenooma-karsinoomasekvenssin kautta malignisoitua. Erilaisilla polyypeillä on eriasteiset malignisoitumisriskit, mutta niitä ei voi luotettavasti erottaa kolonoskopian yhteydessä toisistaan, jonka takia ne tyypillisesti pyritään poistamaan. Tavallisesti poisto tapahtuu endoskooppisesti polypektomialla slingalla (EMR eli endoskooppinen mukoosaresektio) tai biopsiapihdillä (ESD eli endoskooppinen submukosaalinen dissektio).

\end{solution}

\section{Potilastapaus}\label{potilastapaus-16}

Potilasta on puukotettu noin 1 cm vasemman nännin alamediaalipuolelle pitkällä fileerausveitsellä. Tullessa ensiapuun RR 100/80, pulssi 110/min, SaO2 huoneilmalla 98\%. Natiivi-thx --kuva on otettu akuuttihoitohuoneessa istuen ja siinä näkyy vähäinen ilmarinta noin 1,5 cm apexissa ja nestettä vasemmassa pleurassa nousten kutakuinkin mamillatasolle. Hemoglobiini on astrupissa 85. Potilaalla on varfariinihoito mekaanisen keinoläpän vuoksi ja INR on 4,5. Puuttumatta nyt muihin hoitolinjauksiin, mitä teet AK-hoidon suhteen?

\begin{itemize}
\tightlist
\item
  \begin{enumerate}
  \def\labelenumi{\alph{enumi}.}
  \tightlist
  \item
    tauotan varfariinihoidon ja kontrolloin INR-arvon seuraavana päivänä
  \end{enumerate}
\item
  \begin{enumerate}
  \def\labelenumi{\alph{enumi}.}
  \setcounter{enumi}{1}
  \tightlist
  \item
    tauotan varfariinihoidon, annan hyytymistekijäkonsentraattia, kontrolloin INR-arvon annostelun jälkeen
  \end{enumerate}
\item
  \begin{enumerate}
  \def\labelenumi{\alph{enumi}.}
  \setcounter{enumi}{2}
  \tightlist
  \item
    tauotan varfariinihoidon, annan K-vitamiinia ja tarkistan INR --arvon annostelun jälkeen, aloitan LMWH --siltahoidon
  \end{enumerate}
\item
  \begin{enumerate}
  \def\labelenumi{\alph{enumi}.}
  \setcounter{enumi}{3}
  \tightlist
  \item
    potilaalla on indikaatio AK-hoidolle, en tauota
  \end{enumerate}
\end{itemize}

\begin{solution}
\leavevmode

Vastaus

\begin{verbatim}
  b
  
\end{verbatim}

Potilaalla on läpäisevä rintakehävamma, alkavaa hemodynamiikan epästabiliteettia ja selvästi hemopneumothorax sekä hemoglobiinin laskua. Kyseessä on siis henkeä uhkaava verenvuoto ja varfariinivaikutus on kumottava nopeasti.

Varfariinin nopea kumoaminen onnistuu hyytymistekijäkonsentraatilla (PCC, protrombiinikompleksikonsentraatti), joka kumoaa varfariinin n.~10-30 minuutissa.

a: Aivan liian hidasta, koska potilas vuotaa nyt eikä enää huomenna jos vuoto jatkuu.

c: Hitaammin varfariinin vaikutus voidaan kumota K-vitamiinilla (voidaan turvata oman hyytymistekijäsynteesin käynnistyminen 6--12 tunnin kuluessa; vaikutus voi myös alkaa aikaisemmin lähteestä riippuen n.~2h kohdalla). Tarvittaessa voidaan käyttää myös jääplasmaa. Tämä ei kuitenkaan näin vaarallisessa tilanteessa riitä vähentämään verenvuotoa.

d: Potilaalla kylläkin on indikaatio AK-hoidolle ja tärkeäkin sellainen. Akuutti verenvuoto kuitenkin ottaa prioriteetin potilaan terveyden suhteen, jos potilas vuotaa hengenvaarallisesti.\\

\end{solution}

\section{Millaisin tervekudosmarginaalein suositellaan PAD:lla varmistetun melanooma in situn poistoa?}\label{millaisin-tervekudosmarginaalein-suositellaan-padlla-varmistetun-melanooma-in-situn-poistoa}

\begin{itemize}
\tightlist
\item
  \begin{enumerate}
  \def\labelenumi{\alph{enumi}.}
  \tightlist
  \item
    1-2 cm Breslow'sta riippuen
  \end{enumerate}
\item
  \begin{enumerate}
  \def\labelenumi{\alph{enumi}.}
  \setcounter{enumi}{1}
  \tightlist
  \item
    1 cm
  \end{enumerate}
\item
  \begin{enumerate}
  \def\labelenumi{\alph{enumi}.}
  \setcounter{enumi}{2}
  \tightlist
  \item
    Diagnostinen marginaali 1-2 mm
  \end{enumerate}
\item
  \begin{enumerate}
  \def\labelenumi{\alph{enumi}.}
  \setcounter{enumi}{3}
  \tightlist
  \item
    5 mm
  \end{enumerate}
\end{itemize}

\begin{solution}
\leavevmode

Vastaus

\begin{verbatim}
  d
\end{verbatim}

Todetun melanooman tärkein hoito on melanooman poisto tai biopsia-arven re-ekskisio (0.5)1-2 cm marginaalilla Breslow-paksuuden mukaan. Syvyyssuunnassa poisto on aina faskiatasoon asti. Yleensä kyseessä on biopsia-arven re-ekskisio, koska melanooman diagnostinen näyte on yleensä koko ihomuutoksen poisto veneviillolla (vain liian laajat muutokset / hankalat anatomiset alueet biopsioidaan stanssibiopsialla).

\pandocbounded{\includegraphics[keepaspectratio]{images/melanoomamarginaali.png}}

\end{solution}

\section{Potilastapaus}\label{potilastapaus-17}

Terveyskeskusvastaanotollesi tulee 25-vuotiaalla mies, koska hänen vasen kiveksensä on suurentunut ja kipeä. Tunnet palpoiden siinä patin. Et ole avain varma patin etiologiasta. Potilas vakuuttaa, että uskoo patin tulleen kaksi vuotta sitten, kun tipahti pyörän tangolle. Mitä teet?

\begin{itemize}
\tightlist
\item
  \begin{enumerate}
  \def\labelenumi{\alph{enumi}.}
  \tightlist
  \item
    kontrolloit palpaatiolöydöksen 3-6kk:n päästä
  \end{enumerate}
\item
  \begin{enumerate}
  \def\labelenumi{\alph{enumi}.}
  \setcounter{enumi}{1}
  \tightlist
  \item
    laitat lähetteen laboratorioon hCG, LD, ja AFOS-tutkimuksia varten
  \end{enumerate}
\item
  \begin{enumerate}
  \def\labelenumi{\alph{enumi}.}
  \setcounter{enumi}{2}
  \tightlist
  \item
    tilaat kivespussin UÄ-tutkimuksen
  \end{enumerate}
\item
  \begin{enumerate}
  \def\labelenumi{\alph{enumi}.}
  \setcounter{enumi}{3}
  \tightlist
  \item
    saat ostopalveluna vatsan tt-tutkimuksen ja tilaat sen
  \end{enumerate}
\end{itemize}

\begin{solution}
\leavevmode

Vastaus

\begin{verbatim}
  c
\end{verbatim}

Kivessyövät ovat länsimaissa 15-40-vuotiaiden miesten yleisin maligniteetti (vuodesta riippuen Suomessa n.~110-160 tapausta). Suurentunut kives ja palpoituva resistenssi herättää epäilyn kivessyövästä. Kivessyöpä on useimmiten kivuton, mutta kivuliaisuus ei ole harvinaista.

Ensisijainen kivestuumorin selvittelyssä käytettävä tutkimus on UÄ; herkkyys on lähes 100\%. Kaikkia kiveksensisäisiä palpoitavia muutoksia on pidettävä kivessyöpänä, kunnes toisin on osoitettu -\textgreater{} tilaa UÄ jos epäilet kiinteää kasvainta tai jos kliininen arvio kivespussin resistenssistä jää epävarmaksi. Kivessyöpäepäilystä ei oteta UÄ-ohjattua neulabiopsiaa (altistaa syövän leviämiselle), vaan UÄ-löydöksen viitatessa syöpään edetään usein suoraan radikaaliin orkiektomiaan (koko kiveksen ja siemennuoran poisto kalvostoinaan).

a: Kivessyöpäepäilyä ei voi jäädä seuraamaan pelkällä palpaatiolla, se vaatii UÄ-kuvantamisen.

b: Kivessyövän diagnostiikassa on kylläkin apua syöpämerkkiaineista, mutta niitä ei oteta rutiinisti TK:ssa (normaalit merkkiainepitoisuudet eivät poissulje kivessyövän mahdollisuutta) eivätkä ne ole ensisijaisia \emph{diagnostisia} tutkimuksia (kuten UÄ). Tyypillisimmin mitattavat merkkiaineet verestä ovat AFP (alfafetoproteiini), HCG (istukkagonadotropiini) ja LDH (laktaattidehydrogenaasi). Nämä tutkitaan tyypillisesti erikoissairaanhoidossa ennen ja jälkeen orkiektomian ja myös seurannan yhteydessä.

d: Mikäli ultraäänitutkimus viittaa vahvasti kiveksen maligniteettiin, on syytä tehdä levinneisyystutkimuksena vartalon tietokonekerroskuvaus (TT). Pelkällä vatsan TT:llä ei oikein ole roolia kivessyövän diagnostiikassa eikä varsinkaan ennen UÄ-tutkimusta.

\end{solution}

\section{Potilastapaus}\label{potilastapaus-18}

67-vuotias mies tulee vaimonsa kanssa vastaanotolle, koska miehellä oli kolme viikkoa sitten yhden päivän kestänyt makroskooppinen verivirtsaisuus. Nyt virtsa on ollut täysin normaalin väristä. Miehellä ole ollut minkäänlaisia virtsaamisvaivoja verivirtsaisuuden aikana, eikä sen jälkeenkään. Mitä teet?

\begin{itemize}
\tightlist
\item
  \begin{enumerate}
  \def\labelenumi{\alph{enumi}.}
  \tightlist
  \item
    Tuseeraan potilaan ja teen hänelle virtsaelinten UÄ-tutkimuksen. Jos tutkimuksessa havaitaan poikkeavaa, teen lähetteet erikoissairaanhoitoon urologialle.
  \end{enumerate}
\item
  \begin{enumerate}
  \def\labelenumi{\alph{enumi}.}
  \setcounter{enumi}{1}
  \tightlist
  \item
    Tuseeraan potilaan ja tilaan hänelle virtsan sytologia-tutkimuksen, joka perustuu patologin arvioon virtsaan irtoavista soluista. Jos tutkimuksessa havaitaan poikkeavaa, teen lähetteen erikoissairaanhoitoon urologialle.
  \end{enumerate}
\item
  \begin{enumerate}
  \def\labelenumi{\alph{enumi}.}
  \setcounter{enumi}{2}
  \tightlist
  \item
    Teen lähetteen erikoissairaanhoitoon urologialle.
  \end{enumerate}
\item
  \begin{enumerate}
  \def\labelenumi{\alph{enumi}.}
  \setcounter{enumi}{3}
  \tightlist
  \item
    Tuseeraan potilaan ja kehotan häntä tulemaan heti uudestaan, jos makrohematuria uusiutuu.
  \end{enumerate}
\end{itemize}

\begin{solution}
\leavevmode

Vastaus

\begin{verbatim}
  c
  
\end{verbatim}

Yleisin uroteelisyövän oire on verivirtsaisuus, jota on ainakin 85 \%:lla rakkosyöpäpotilaista. Verivirtsaisuus voi olla makroskooppista eli silminnähtävää tai mikroskooppista eli virtsakokeen paljastamaa. \textbf{Makroskooppisen hematurian mahdollinen väistyminen tai lieväasteisuus eivät poissulje pahanlaatuista uroteelikasvainta.} Makroskooppinen hematuria vaatii lähes aina urologisia selvittelyitä ja ensisijainen tutkimus on pääsääntöisesti kystoskopia (etsitään rakkosyöpää).

a, b, d: Kaikissa vaihtoehdoissa urologiset jatkotutkimukset ja siten kystoskopia tapahtuvat vain tiettyjen ehtojen jälkeen. Tässä tapauksessa ei tarvitse enää löytää mitään muuta syytä kystoskopialle, koska sen indikaatio on jo täyttynyt (makroskooppinen hematuria, joka ei selity nuoren naisen virtsatieinfektiona).

Potilaan tosin kyllä saa tuseerata, sillä sen ainoat kontraindikaatiot ovat, että tutkijalla ei ole sormea tai potilaalla ei ole rectumia.

\pandocbounded{\includegraphics[keepaspectratio]{images/lähettämiskriteeritilmankeltaista.png}}
\pandocbounded{\includegraphics[keepaspectratio]{images/jatkotutkimuksetesh.png}}

\end{solution}

\section{Potilastapaus}\label{potilastapaus-19}

78-vuotiaalla potilaalla virtsasuihkun voima heikentynyt, nokturiaa 3 kertaa, lisäksi urge-oiretta ja tunnetta, että rakko ei tyhjene kunnolla. PSA 5,6 ug/l vapaan prosenttiosuus 12\%. Tuseeraten eturauhanen on suurentunut ja sileä. Aloitat potilaalle dutasteridin. Milloin kontrolloit, onko lääke auttanut?

\begin{itemize}
\tightlist
\item
  \begin{enumerate}
  \def\labelenumi{\alph{enumi}.}
  \tightlist
  \item
    Puolen vuoden kuluttua lääkkeen aloittamisesta
  \end{enumerate}
\item
  \begin{enumerate}
  \def\labelenumi{\alph{enumi}.}
  \setcounter{enumi}{1}
  \tightlist
  \item
    Dutasteridin aloittamisen jälkeen ei tarvita kontrollia
  \end{enumerate}
\item
  \begin{enumerate}
  \def\labelenumi{\alph{enumi}.}
  \setcounter{enumi}{2}
  \tightlist
  \item
    Vuoden kuluttua lääkkeen aloittamisesta
  \end{enumerate}
\item
  \begin{enumerate}
  \def\labelenumi{\alph{enumi}.}
  \setcounter{enumi}{3}
  \tightlist
  \item
    Kuukauden kuluttua lääkkeen aloittamisesta
  \end{enumerate}
\end{itemize}

\begin{solution}
\leavevmode

Vastaus

\begin{verbatim}
  a
\end{verbatim}

Potilaalla on hyvin todennäköisesti \textbf{eturauhasen hyvänlaatuinen hyperplasia (BPH).} Sille on tyypillistä ``kerääntymisoireet'' eli tihentynyt virtsaamistarve (pollakisuria), nokturia (yöllinen virtsaamistarve) ja pakkoinkontinenssi (urge-oire). Eturauhasen hyvänlaatuisen hyperplasian (BPH) kerääntymisoireiden taustalla on tavallisesti detrusorlihaksen hyperrefleksia johtuen suurentuneesta rakkopaineesta (prostata estää virtausta rakosta -\textgreater{} kohonnut paine rakossa) ja prostatahyperplasian aiheuttamien uretran muutosten aikaansaamasta sensorisesta ärsytyksestä. BPH:ssa ilmenee myös ``tyhjennysoireita'', joita ovat esim. virtsantulon viipyminen, heikentynyt virtsasuihku, ponnistelun tarve virtsatessa, virtsauksen keskeytyminen, rakon epätäydellisen tyhjenemisen tunne sekä vaikeimmassa tapauksessa jopa virtsaumpi. Palpaatiossa eturauhanen on kyhmytön, aristamaton ja suuri, jonka takia keskiuurretta voi olla vaikea tuntea.

Komplisoitumattomien BPH-potilaiden virtsaamisoireita hoidetaan ensisijaisesti lääkehoidolla. Ensisijaisia lääkkeitä ovat alfasalpaajat (alfa-adrenoreseptoriantagonistit) ja 5-alfareduktaasin estäjät (5-ARI). Alfasalpaajat (esim. tamsulosiini tai alfutsosiini) vaikuttavat rentouttamalla prostaattisen virtsaputken ja virtsarakon kaulan sileää lihaksistoa. Ne lievittävät nopeasti oireita (hoitovasteen arviointi tyypillisesti n.~1-3kk kohdalla), lisäävät virtsasuihkun huippuvirtaamaa ja vähentävät jäännösvirtsan tilavuutta.

5-alfareduktaasin estäjät (esim. dutasteridi tai finasteridi) vaikuttavat eturauhasen liikakasvuun estämällä testosteronin metaboloitumista dihydrotestosteroniksi (DHT), jolloin seerumin DHT:n pitoisuus pienenee. Tämä johtaa eturauhasen koon pienenemiseen ja oireiden lievittymiseen. Myös rakon ulosvirtauskanavan ahtauma pienenee eturauhasen koon pienentyessä. \textbf{5-ARI-lääkkeillä teho tulee hitaammin kuin alfasalpaajilla, jonka takia niiden vasteen arvioiminen suoritetaan myöhemmin (6kk kohdalla) kuin alfasalpaajilla.} 5-alfareduktaasin estäjiä (5-ARI) voidaan käyttää yhdessä alfasalpaajien kanssa, jolloin lääkkeiden teho paranee. \emph{Jos olet hoitanut BPH-potilasta lääkkeellisesti 6kk PTH:ssa, mutta oireet ovat potilaalla vieläkin vaikeahkot, niin tee lähete ESH.}

Potilaan PSA:sta huomio: Jos PSA on 2-15µg/l, niin lasketaan automaattisesti vapaa-PSA/kokonais-PSA-suhde, joka auttaa eturauhassyövän diagnostiikassa. Syöpäsolut tuottavat sitoutunutta PSA:a, jonka seurauksena suhde laskee ja tämä voi olla vihje eturauhassyövästä. Jos välillä 4-10 suhde \textless10\%, on riski syövälle yli 50\%. Jos suhde on \textgreater25\%, niin riski alle 10\%. Potilaan PSA:n ollessa 10-15\% on riski n.~28\%.

\pandocbounded{\includegraphics[keepaspectratio]{images/bphhoitoalgoritmi.png}}

\end{solution}

\section{Potilastapaus}\label{potilastapaus-20}

Vastaanotollesi tulee potilas, joka on Kannasta lukenut sairauskertomuksiaan. Hän kysyy sinulta, millainen toimenpide hänelle on sairaalassa tehty. Toimenpiteenä Kannan tietojen mukaan on ollut rec: a. poplitea-a. dorsalis pedis CVA converted l. dx. Mikä seuraavista väittämistä on VÄÄRIN?

\begin{itemize}
\tightlist
\item
  \begin{enumerate}
  \def\labelenumi{\alph{enumi}.}
  \tightlist
  \item
    Kerrot, että potilaan valtimo-ohitus oikealle on tehty omalla laskimosuonella
  \end{enumerate}
\item
  \begin{enumerate}
  \def\labelenumi{\alph{enumi}.}
  \setcounter{enumi}{1}
  \tightlist
  \item
    Kerrot, että ohitesuoni on ohituksen yhteydessä käännetty, jotta laskimoläpät eivät estä valtimovirtausta
  \end{enumerate}
\item
  \begin{enumerate}
  \def\labelenumi{\alph{enumi}.}
  \setcounter{enumi}{2}
  \tightlist
  \item
    Kerrot potilaalle, että nyt on tullut väärä käsitys. Oikeasti hänelle on tehty pallolaajennus kahteen oireisen oikeaan jalan suoneen
  \end{enumerate}
\item
  \begin{enumerate}
  \def\labelenumi{\alph{enumi}.}
  \setcounter{enumi}{3}
  \tightlist
  \item
    Kerrot, että ohitusleikkaus oikealle on tehty polvitaivevaltimosta jalkaterän pieneen valtimoon
  \end{enumerate}
\end{itemize}

\begin{solution}
\leavevmode

Vastaus

\begin{verbatim}
  c
\end{verbatim}

Kannan tekstissä lukee siis, että on tehty ohitus (rec, reconstructio), polvitaivevaltimosta (a. poplitea) jalkapöydän valtimoon (a. dorsalis pedis) potilaan omaa laskimograftia käyttäen (CVA, cum venam autogenam), laskimo on käännetty (converted; läppiä ei siis tuhottu, mutta kääntäminen mahdollistaa sen, että läpät eivät estä virtausta. Jos olisi converted sijaan in situ, niin läpät olisi tuhottu) ja toimenpide on tehty oikealla (l. dx eli laterum dextrum; l. sin eli laterum sinistrum taas olisi vasen).

a: Polven alapuolisiin valtimoihin tehtävissä ohituksissa ohitusmateriaalin valinta on ratkaiseva ja ensisijainen on potilaan oma laskimo (autologinen grafti), koska potilaan oma laskimo pysyy parhaiten auki ja infektioriski on pienempi. Vena saphena magna on aukipysyvyydeltään ylivoimainen (otetaan ensisijaisesti samasta alaraajasta kuin minne toimenpide tehdään); säären ja jalkaterän valtimo-ohituksissa tulisi aina käyttää potilaan omaa laskimoa

b: Pinnallisissa laskimoissa (kuten juuri todennäköisesti käytetyssä vena saphena magnassa) on läppiä, jotka estävät takaisinvirtausta (näiden läppien vajaatoiminta aiheuttaa laskimovajaatoimintaa ja johtaa mm. suonikohjuihin). Jos suonea ei käännettäisi tai läppiä tuhottaisi, niin läpät estäisivät virtausta toivottuun suuntaan eli valtimosuuntaan, kun laskimoa käytettäisiin ohitusgraftina.

c: Tästä ei ole viitteitä tekstissä. Pallolaajennusta tulee harkita ensimmäiseksi hoitovaihtoehdoksi, kun kysymyksessä on lyhyt valtimomuutos tai potilaan elinajanennuste on lyhyt ja leikkausriski on suuri. Leikkausta (endarterektomia tai ohitusleikkaus) taas tulee harkita ensimmäisenä hoitovaihtoehtona pitkissä, diffuuseissa valtimomuutoksissa, kun polvitaivevaltimon ulosvirtausvaltimot ovat tukossa ja leikkausriski on kohtuullinen, kunhan potilaalla on ohitusmateriaaliksi sopiva laskimo.

d: Vaskulaariset ohitusleikkaukset nimetään anatomisin perustein proksimaalisesta valtimosta distaaliseen valtimoon. Tekstissä lukee a. poplitea-a. dorsalis pedis eli ohitus on a. popliteasta ADP:hen.

\end{solution}

\section{Leikkaamattoman mahan okkluusion syynä on todennäköisimmin}\label{leikkaamattoman-mahan-okkluusion-syynuxe4-on-todennuxe4kuxf6isimmin}

\begin{itemize}
\tightlist
\item
  \begin{enumerate}
  \def\labelenumi{\alph{enumi}.}
  \tightlist
  \item
    kiinnikkeet
  \end{enumerate}
\item
  \begin{enumerate}
  \def\labelenumi{\alph{enumi}.}
  \setcounter{enumi}{1}
  \tightlist
  \item
    sappikivi
  \end{enumerate}
\item
  \begin{enumerate}
  \def\labelenumi{\alph{enumi}.}
  \setcounter{enumi}{2}
  \tightlist
  \item
    sigmavolvulus
  \end{enumerate}
\item
  \begin{enumerate}
  \def\labelenumi{\alph{enumi}.}
  \setcounter{enumi}{3}
  \tightlist
  \item
    paksusuolisyöpä
  \end{enumerate}
\end{itemize}

\begin{solution}
\leavevmode

Vastaus

\begin{verbatim}
  d
\end{verbatim}

Erityisesti vanhemman ihmisen leikkaamattoman mahan okkluusio on koolonkarsinooman aiheuttama, kunnes toisin osoitetaan.

a: Länsimaissa ohutsuolen tukoksen (SOB) selvästi yleisin syy on kiinnikkeet, mutta ne tulevat tyypillisesti aikaisemmasta leikkauksesta.

b: Sappikivi ei oikein aiheuta tyypillisesti suolen okkluusiota. On mahdollinen esimerkiksi sekundaarisena kolekystiitille, jonka yhteydessä tapahtuu sappirakon perforaatio duodenumiin ja muodostuu kolekystoduodenaalinen fisteli. Fisteliä pitkin suoleen pääsee suuri sappikivi, joka voi kulkea mm. iliokekaaliseen junktioon ja tukkia sen johtaen suoliobstruktioon (ns. sappikivi-ileus).

c: Volvulus eli suolen kiertyminen oman suolilipeensä ympärille voi johtaa affisioituneen suolen alueen (yleisimmin sigma vanhuksilla; keskisuolessa pikkulapsilla ja cecumissa nuorilla aikuisilla) obstruktioon ja jopa infarktiin. Voi ilmentyä leikkaamattomassa vatsassa, mutta ei kuitenkaan ole todellakaan yleisin syy mahan okkluusiolle.

\end{solution}

\section{Potilastapaus}\label{potilastapaus-21}

Ensiapuun tulee 75-vuotias mies, jolla todetaan hankala virtsatietulehdus. Prostata on jkv suurentunut, mutta aristamaton ja kyhmytön, pehmeä. Potilas kertoo, että viime aikoina tullut ilmaa virtsatessa virtsaputken kautta. Potilaalle ohjelmoidaan ensisijaisesti:

\begin{itemize}
\tightlist
\item
  \begin{enumerate}
  \def\labelenumi{\alph{enumi}.}
  \tightlist
  \item
    Virtsaelinten TT
  \end{enumerate}
\item
  \begin{enumerate}
  \def\labelenumi{\alph{enumi}.}
  \setcounter{enumi}{1}
  \tightlist
  \item
    Kolonoskopia/Kolon TT
  \end{enumerate}
\item
  \begin{enumerate}
  \def\labelenumi{\alph{enumi}.}
  \setcounter{enumi}{2}
  \tightlist
  \item
    Päivystysleikkaus
  \end{enumerate}
\item
  \begin{enumerate}
  \def\labelenumi{\alph{enumi}.}
  \setcounter{enumi}{3}
  \tightlist
  \item
    Eturauhasselvittelyt
  \end{enumerate}
\end{itemize}

\begin{solution}
\leavevmode

Vastaus

\begin{verbatim}
  b
\end{verbatim}

Pneumaturia eli ilma virtsassa viittaa vahvasti kolovesikaalifisteliin eli yhteyteen koolonista (tai muusta suolen osastakin mahdollisesti) rakkoon. Suolessa on ilmaa, joka pääsee fisteliä pitkin rakkoon ja tämä havaitaan ilmana virtsaputken kautta virtsatessa. Fisteli ilmenee myös toistuvina virtsatieinfektioina (bakteereita suolesta rakkoon) ja mahdollisesti jopa ulosteina virtsassa (tulee kysyä suoraan potilaalta).

Yleensä tämän aiheuttajana on reikä sigman huipusta virtsarakkoon divertikkeliperforaation takia. Diagnostinen tutkimus on kolonoskopia tai koolonin TT-kuvantaminen.

a: Ensisijainen syy on suolistoperäinen ja tämän takia tutkimuskin kohdistuu sinne.

c: Päivystysleikkaukselle ei ole aihetta, ellei potilas ole septinen tai peritoniitissa

d: Ei aiheellinen nyt ensisijaisesti, koska prostata on kliinisesti normaali ja pneumaturia ei johdu prostatan ongelmista.

\end{solution}

\section{Malignin melanooman diagnostisessa poistossa on tärkeintä}\label{malignin-melanooman-diagnostisessa-poistossa-on-tuxe4rkeintuxe4}

\begin{itemize}
\tightlist
\item
  \begin{enumerate}
  \def\labelenumi{\alph{enumi}.}
  \tightlist
  \item
    poistaa muutos ja saada selville melanooman paksuus.
  \end{enumerate}
\item
  \begin{enumerate}
  \def\labelenumi{\alph{enumi}.}
  \setcounter{enumi}{1}
  \tightlist
  \item
    ulottaa poisto lihasfaskian tasoon.
  \end{enumerate}
\item
  \begin{enumerate}
  \def\labelenumi{\alph{enumi}.}
  \setcounter{enumi}{2}
  \tightlist
  \item
    saada mahdollisimman laajat marginaalit.
  \end{enumerate}
\item
  \begin{enumerate}
  \def\labelenumi{\alph{enumi}.}
  \setcounter{enumi}{3}
  \tightlist
  \item
    Kaikki kysymyksen vastausvaihtoehdot ovat oikein.
  \end{enumerate}
\end{itemize}

\begin{solution}
\leavevmode

Vastaus

\begin{verbatim}
  a
  
\end{verbatim}

Melanooman diagnostisessa ekskisiossa tuumorimuutos poistetaan jo primaarivaiheessa kokonaan 1-2 mm:n kliinisin terveen kudoksen marginaalein veneviillosta ihonalaiseen rasvakerrokseen asti. Tämä lähetetään tutkimuksiin PAD-lausuntoa varten ja lausunnossa tärkein asia on Breslow'n mitta eli melanooman invaasion syvyys.

b: Melanooman hoidollisessa poistossa on tärkeää poistaa muutos lihasfaskiaan asti, mutta diagnostisessa ekskisiossa poisto on niin, että mukana on vain hieman ihonalaista rasvaa.

c: Marginaalit ovat n.~millin pari, tavoitteena on siis saada koko muutos poistettua. Ei tarvitse alkaa suoraan mittailemaan marginaaleja, mutta kunhan silmämääräisesti koko muutos on poistettu, niin poisto on hyvä.

\end{solution}

\section{Potilastapaus}\label{potilastapaus-22}

54-vuotiaalla potilaalla virtsasuihkun voima heikentynyt, nokturiaa 2 kertaa, ja tunnetta, että rakko ei tyhjene kunnolla. PSA 2,6 ug/l. Tuseeraten eturauhanen suurentunut ja sileä. Aloitat potilaalle tamsulosiinin. Milloin kontrolloit, onko lääke auttanut?

\begin{itemize}
\tightlist
\item
  \begin{enumerate}
  \def\labelenumi{\alph{enumi}.}
  \tightlist
  \item
    Puolen vuoden kuluttua lääkkeen aloittamisesta
  \end{enumerate}
\item
  \begin{enumerate}
  \def\labelenumi{\alph{enumi}.}
  \setcounter{enumi}{1}
  \tightlist
  \item
    Vuoden kuluttua lääkkeen aloittamisesta
  \end{enumerate}
\item
  \begin{enumerate}
  \def\labelenumi{\alph{enumi}.}
  \setcounter{enumi}{2}
  \tightlist
  \item
    Tamsulosiinin aloittamisen jälkeen ei tarvita kontrollia
  \end{enumerate}
\item
  \begin{enumerate}
  \def\labelenumi{\alph{enumi}.}
  \setcounter{enumi}{3}
  \tightlist
  \item
    Kuukauden kuluttua lääkkeen aloittamisesta
  \end{enumerate}
\end{itemize}

\begin{solution}
\leavevmode

Vastaus

\begin{verbatim}
  d
\end{verbatim}

Käyty jo aikaisemmassa hyvin samanlaisessa kysymyksessä asia läpi, joten katso sieltä, jos haluat verrata vastausta dutasteridin kontrollointiaikaan (n.~6kk päästä). Tamsulosiini on alfasalpaaja (selektiivinen alfa1-reseptorin antagonisti; samoin myös alfutsosiini) ja se vaikuttaa selvästi nopeammin kuin 5-alfareduktaasin estäjät (esim. dutasteridi tai finasteridi), mutta se ei pienennä eturauhasen kokoa. Alfasalpaajat vaikuttavat rentouttamalla prostaattisen virtsaputken ja virtsarakon kaulan sileää lihaksistoa ja tämän takia vaikutus ilmenee nopeasti. Vasteen kontrollointi voidaan täten tehdä n.~kuukauden kuluttua.

\pandocbounded{\includegraphics[keepaspectratio]{images/bphhoitoalgoritmi.png}}

\end{solution}

\section{Potilastapaus}\label{potilastapaus-23}

Vastaanotollasi on perusterve 35-vuotias mies, jolle on reiden alueelle kasvanut vuoden sisällä noin 7 cm halkaisijaltaan oleva kivuton pehmytkudospatti. Muutos on varsin tarkkarajainen ja kliinisesti sopisi lipoomaan. Miten toimit?

\begin{itemize}
\tightlist
\item
  \begin{enumerate}
  \def\labelenumi{\alph{enumi}.}
  \tightlist
  \item
    Teen lähetteen erikoissairaanhoitoon
  \end{enumerate}
\item
  \begin{enumerate}
  \def\labelenumi{\alph{enumi}.}
  \setcounter{enumi}{1}
  \tightlist
  \item
    Teen lähetteen ultraäänitutkimukseen ja pyydän paksuneulabiopsian
  \end{enumerate}
\item
  \begin{enumerate}
  \def\labelenumi{\alph{enumi}.}
  \setcounter{enumi}{2}
  \tightlist
  \item
    Sovin kontrollikäynnin puolen vuoden päähän
  \end{enumerate}
\item
  \begin{enumerate}
  \def\labelenumi{\alph{enumi}.}
  \setcounter{enumi}{3}
  \tightlist
  \item
    Poistan lipooman ja otan PAD-näytteeksi
  \end{enumerate}
\end{itemize}

\begin{solution}
\leavevmode

Vastaus

\begin{verbatim}
  a
\end{verbatim}

Mikäli pehmytkydoskasvain on mobiili, pehmeä, hitaasti kasvanut, ihonalainen ja alle 5 cm:n läpimittainen, se on hyvin todennäköisesti hyvänlaatuinen. Jos taas nämä eivät täyty ja muutos on kliinisesti pahanlaatuinen (kuten tässä, koska kasvanut nopeasti ja on \textgreater5cm), niin ei voi jäädä pelkälle seurantakannalle, vaan tulee tehdä lähete plastiikkakirurgille.

b: Jos muutos olisi kliinisesti poikkeava eli \textless5cm, mutta ei mobiili tai nopeakasvuinen, niin voitaisiin tehdä kaikukuvaus ja sen perusteella jatkaa. Jos UÄ:ssä muutos olisi ihonalainen ja hyvänlaatuinen, niin se voitaisiin poistaa TK:ssa, jos se vaikuttaisi helposti poistettavalta. Pahanlaatuisuuden merkkien ollessa läsnä UÄ:ssä tehtäisiin lähete plastiikkakirurgille.

c: Muutos ei ole kliinisesti hyvänlaatuinen, joten ei voi ehdottaa pelkkää seurantalinjaa.

d: Kliinisesti pahanlaatuiset pehmytkudoskasvaimet tulee lähettää ESH.

\pandocbounded{\includegraphics[keepaspectratio]{images/pehmytkudoskasvainalgoritmi.png}}

\end{solution}

\section{Potilastapaus}\label{potilastapaus-24}

62-vuotias mies tulee vastaanotollesi. Hänellä on verenpainetauti ja tyypin 2 diabetes. Kolme kuukautta sitten kirjoitit hänelle sildenafil-reseptin erektiohäiriöön. Kertoo, että auttoi erektioon kohtalaisen hyvin, mutta ei voi käyttää lääkettä, koska saa siitä kovan päänsäryn. Seksuaalisuus on hänelle erittäin tärkeää ja hän toivoo ratkaisua. Mitä teet?

\begin{itemize}
\tightlist
\item
  \begin{enumerate}
  \def\labelenumi{\alph{enumi}.}
  \tightlist
  \item
    Päänsärky ei ole tyypillinen haittavaikutus sildenafiilille ja kannustat potilasta jatkamaan sen käyttöä
  \end{enumerate}
\item
  \begin{enumerate}
  \def\labelenumi{\alph{enumi}.}
  \setcounter{enumi}{1}
  \tightlist
  \item
    Esittelet miten geelimuotoista PGE1-analogia käytetään ja kirjoitat siitä hänelle reseptin
  \end{enumerate}
\item
  \begin{enumerate}
  \def\labelenumi{\alph{enumi}.}
  \setcounter{enumi}{2}
  \tightlist
  \item
    Lähetät potilaan erikoissairaanhoitoon
  \end{enumerate}
\item
  \begin{enumerate}
  \def\labelenumi{\alph{enumi}.}
  \setcounter{enumi}{3}
  \tightlist
  \item
    Esittelet miten pistettävää PGE1-analogia käytetään ja kirjoitat siitä hänelle reseptin
  \end{enumerate}
\end{itemize}

\begin{solution}
\leavevmode

Vastaus

\begin{verbatim}
  b
\end{verbatim}

Useimmiten erektiohäiriöiden ensisijainen lääkkeellinen hoito on PDE5-estäjä, kuten sildenafiili (lyhytvaikutteisempi) tai tadalafiili (pitkävaikutteisempi). Näitä voi myös mahdollisesti yhdistellä vaikeassa erektiohäiriössä (esim. siltsua 100mg ennen yhdyntää ja pitkävaikutteisempi tadalafiili 5mg päivittäin pohjalla).

Jos nämä eivät auta tai ovat vasta-aiheisia (esim. potilas käyttää nitraatteja; lääkkeillä on synergistinen vaikutus ja voi aiheuttaa vaarallisen tasoista vasodilataatiota ja verenpaineen romahtamisen) tai ilmenee merkittäviä haittoja (yleisimpiä ovat juuri vasodilataation liittyvät päänsärky, huimaus, verenpaineen lasku ja naaman punoitus), niin voidaan kokeilla paikallisesti annosteltavia lääkkeitä. Näitä ovat esim. intrakavernoottinen injektiohoito tai siittimen kärkeen annosteltava voide (molemmat yleensä alprostadiilia eli synteettistä prostaglandiini E1:tä); näiden vaikutus ei riipu seksuaalisesta kiihottumisesta. \textbf{Geelin ruiskuttaminen kärkeen voi olla helpompaa opettaa ja miellyttävämpää käyttää, joten sitä voi miettiä ensisijaisena näistä;} se vain ei ole aivan yhtä tehokas kuin injektiohoito, joka on kaikkein tehokkain lääkehoito erektiohäiriöön. Jos injektiohoidon päätyy määräämään potilaalle, niin se tulisi kerran toteuttaa vastaanotolla ja antaa tarkat ohjeet injektioiden suorittamiseen.

a: Päänsärky on tyypillinen haittavaikutus ja jos se häiritsee potilasta, niin voidaan yrittää toisiakin hoitokeinoja, kuten juuri paikallishoitoa alprostadiililla.

c: Potilasta voi hyvin vielä yrittää hoitaa perusterveydenhuollossa, kunhan vain tietää mitä on määräämässä.

d: Intrakavernoottinen injektiohoito on äärimmäisen tehokas erektiohäiriön hoitokeino, mutta se vaatii opettamista ja geelihoitoa voi olla helpompaa kokeilla aluksi.

\pandocbounded{\includegraphics[keepaspectratio]{images/erektioalgoritmi.png}}

\end{solution}

\section{Virtsaretention vuoksi laitan miespotilaalle seuraavanlaisen virtsakatetrin:}\label{virtsaretention-vuoksi-laitan-miespotilaalle-seuraavanlaisen-virtsakatetrin}

\begin{itemize}
\tightlist
\item
  \begin{enumerate}
  \def\labelenumi{\alph{enumi}.}
  \tightlist
  \item
    kysymyksen kaikki vaihtoehdot ovat oikein
  \end{enumerate}
\item
  \begin{enumerate}
  \def\labelenumi{\alph{enumi}.}
  \setcounter{enumi}{1}
  \tightlist
  \item
    käyräkärkinen Ch16 Lofric-katetrin
  \end{enumerate}
\item
  \begin{enumerate}
  \def\labelenumi{\alph{enumi}.}
  \setcounter{enumi}{2}
  \tightlist
  \item
    pyöreäkärkinen Ch16 silikonikatetri, jonka balongiin 10 ml glyseroliliuosta
  \end{enumerate}
\item
  \begin{enumerate}
  \def\labelenumi{\alph{enumi}.}
  \setcounter{enumi}{3}
  \tightlist
  \item
    pyöreäkärkinen Ch22 silikonikatetri, jonka balongiin 10 ml glyseroliliuosta
  \end{enumerate}
\end{itemize}

\begin{solution}
\leavevmode

Vastaus

\begin{verbatim}
  c
\end{verbatim}

Virtsaretention vuoksi usein asetetaan kestokatetri eli balongillinen katetri (täytetty balongi estää katetrin poistumisen rakosta). Kestokatetrin pää on tyypillisesti pyöreä- ja suorakärkinen (Nelaton), mutta voi myös olla käyrän muotoinen eli ns. Tiemann-kärki. Aikuiselle tyypillisesti valitaan Ch 14-16-kokoinen katetri (12 matalimmillaan; huuhteluun tarvittavat kolmitiekatetrit väh. 18 ja usein Ch20--24) ja balongi täytetään 10\% glyserolilla (pallon koko vaihtelee katetrin mukaan 5-10 ml:sta 30-50 ml:aan, määrä on merkitty katetripakkaukseen; glyserolia on saatavissa valmiissa 10ml ruiskuissa).

b: Lofric-katetrit ovat kertakatetreja

d: Mitä isompi katetrin Ch-numero (Charriére), niin sitä paksumpi katetri on. Ch-luku kertoo katetrin ympärysmitan millimetreissä. Katetrin ulkohalkaisija saadaan jakamalla ympärysmitta kolmella, esimerkiksi Ch24-koon katetri on halkaisijaltaan noin 8 mm. French-yksiköt (Fr) tarkoittavat samaa. Tyypillisesti kaksitiekatetrit ovat kooltaan Ch12--16 ja huuhteluun tarvittavat kolmitiekatetrit Ch20--24.

\end{solution}

\section{Mikä vastaus on VÄÄRIN? Alaraajan laskimovajaatoiminnan hoidon selkeä indikaatio on:}\label{mikuxe4-vastaus-on-vuxe4uxe4rin-alaraajan-laskimovajaatoiminnan-hoidon-selkeuxe4-indikaatio-on}

\begin{itemize}
\tightlist
\item
  \begin{enumerate}
  \def\labelenumi{\alph{enumi}.}
  \tightlist
  \item
    Alaraajan laskimovajaatoiminnan hoitoon ei ole ehdottomia indikaatioita vaan pintalaskimoiden hoito on aina kosmeettista
  \end{enumerate}
\item
  \begin{enumerate}
  \def\labelenumi{\alph{enumi}.}
  \setcounter{enumi}{1}
  \tightlist
  \item
    Säären laskimoperäinen haava, joka on parantunut kompressiosidonnalla
  \end{enumerate}
\item
  \begin{enumerate}
  \def\labelenumi{\alph{enumi}.}
  \setcounter{enumi}{2}
  \tightlist
  \item
    Säären alaosan pigmenttimuutokset, jotka liittyvät pintalaskimovajaatoimintaan
  \end{enumerate}
\item
  \begin{enumerate}
  \def\labelenumi{\alph{enumi}.}
  \setcounter{enumi}{3}
  \tightlist
  \item
    Säären konservatiiviseen hoitoon reagoimaton laskimoperäinen haava
  \end{enumerate}
\end{itemize}

\begin{solution}
\leavevmode

Vastaus

\begin{verbatim}
  a
\end{verbatim}

Laskimovajaatoiminta voi aiheuttaa huomattavia komplikaatioita ja näiden hoitaminen ja estäminen on tärkeää. Myös komplisoitumatonta laskimovajaatoimintaakin voidaan siis hoitaa, jos potilaalla on merkittävää haittaa vajaatoiminnasta. Riippumatta siitä tarvitseeko potilas lopulta kajoavaa hoitoa, niin kompressiohoito tulee aloittaa välittömästi PTH:ssa ennen ESH:n arviota.

Kajoavan hoidon aiheita ja aiheita lähetteelle verisuonikirurgialle ovat merkittävä päivittäinen haitta, komplisoitunut tauti (C4-C6), vuotavat suonikohjut tai toistuvat pintalaskimotukokset/yksi laaja pintalaskimotukos.

\pandocbounded{\includegraphics[keepaspectratio]{images/pintalaskimovajaatoimintaalgoritmi.png}}

\end{solution}

\section{Potilastapaus}\label{potilastapaus-25}

58-vuotiaalla miehellä on kehittynyt ruma kliinisesti basalioomaksi sopiva max. 3 mm halkaisijaltaan muutos vasempaan hartiaan. Potilas on muutoin sangen terve, mutta jo vuosia sitten on asennettu mekaaninen aorttaläppäproteesi ja potilaalla on varfariinihoito. INR on pysynyt tasaisena jo vuosikymmenen tasolla 2-3 ja on tämänaamuisessa laboratoriokokeessa 2,1. Miten toimit?

\begin{itemize}
\tightlist
\item
  \begin{enumerate}
  \def\labelenumi{\alph{enumi}.}
  \tightlist
  \item
    otan ohutneulabiopsian muutoksesta
  \end{enumerate}
\item
  \begin{enumerate}
  \def\labelenumi{\alph{enumi}.}
  \setcounter{enumi}{1}
  \tightlist
  \item
    tauotan varfariinin, aloitan siltahoidon ja poistan muutoksen veneviillosta noin 5mm marginaalilla INR:n ollessa alle 1,2
  \end{enumerate}
\item
  \begin{enumerate}
  \def\labelenumi{\alph{enumi}.}
  \setcounter{enumi}{2}
  \tightlist
  \item
    poistan muutoksen veneviillosta PAD:ksi noin 5mm marginaalilla ja ompelen tiiviisti ihon kiinni
  \end{enumerate}
\item
  \begin{enumerate}
  \def\labelenumi{\alph{enumi}.}
  \setcounter{enumi}{3}
  \tightlist
  \item
    teen päivystyslähetteen plastiikkakirurgialle
  \end{enumerate}
\end{itemize}

\begin{solution}
\leavevmode

Vastaus

\begin{verbatim}
  c
  
\end{verbatim}

Pienen basaliooman voi poistaa terveyskeskuksessa ja hoitaa alusta loppuun itse. Hoito on ensisijaisesti 5mm makroskooppisilla marginaaleilla poisto veneviillosta. Tulee muistaa varmistaa PAD-vastaus ja puhtaat marginaalit. Jos sait koko muutoksen poistettua PAD:n mukaan ja leikkeessä on terveet marginaalit, niin hoito oli siinä. Seurantaakaan ei tarvitse, jos marginaalit olivat riittävät. Jos kyseessä on ollut suuren uusiutumisriskin (high grade) basaliooma tai ei-kirurgisesti hoidettu basaliooma, seuranta suunnitellaan harkinnan mukaan hoitavassa yksikössä. Jos basaliooma on leikkauksessa poistettu epätäydellisesti, pitää uusiutumisen ehkäisemiseksi suorittaa joko arven poistoleikkaus tai tilanteen mukaan ei-kirurginen hoito. Pelkkä seuranta ei ole näissä tapauksissa hyväksyttävää.

a: Basalioomasta ei oteta ohutneulabiopsiaa. Muutoksen poisto kokonaisuudessaan on paras. Jos poisto ei onnistu, niin koepalaksi riittää 3--5 mm:n stanssipala, mutta jos kyseessä on pieni, kliinisesti selvä basaliooma esimerkiksi vartalolla, kannattaa tuumori poistaa kerralla kokonaan riittävin marginaalein.

b: Potilas on varfariinilla, mutta INR 2,1 on terapeuttinen ja ihokirurgia voidaan turvallisesti tehdä varfariinia tauottamatta, kun kyseessä on pieni pinnallinen toimenpide.

d: Basaliooma ei ole päivystyksellistä hoitoa vaativa asia. Tyvisolusyöpä on käytännössä ystävällisin syöpä, minkä voi saada, koska se ei käytännössä koskaan metastasoi (alle 1\% riski). Se vaatii kuitenkin hoitoa, ei vain päivystyksellisesti ESH:ssa.

\end{solution}

\section{Potilastapaus}\label{potilastapaus-26}

Potilaalle on tehty oik. puolen hemikolektomia karsinooman takia laparoskooppisesti. Postoperatiivisesti 5 pv leikkauksesta nousee kuume ja vatsa kipeytyy. Leuk 14,6, CRP 150, thx kuvassa basaalisesti atelektaasia vasemmalla. Statuksessa Tax 39, Vatsa aristaa kauttaaltaan, jkv hengenahdistusta. Epäilet ensisijaisesti:

\begin{itemize}
\tightlist
\item
  \begin{enumerate}
  \def\labelenumi{\alph{enumi}.}
  \tightlist
  \item
    Abskessia
  \end{enumerate}
\item
  \begin{enumerate}
  \def\labelenumi{\alph{enumi}.}
  \setcounter{enumi}{1}
  \tightlist
  \item
    Virtsatulehdusta
  \end{enumerate}
\item
  \begin{enumerate}
  \def\labelenumi{\alph{enumi}.}
  \setcounter{enumi}{2}
  \tightlist
  \item
    Lekaasia
  \end{enumerate}
\item
  \begin{enumerate}
  \def\labelenumi{\alph{enumi}.}
  \setcounter{enumi}{3}
  \tightlist
  \item
    Pneumoniaa
  \end{enumerate}
\end{itemize}

\begin{solution}
\leavevmode

Vastaus

\begin{verbatim}
  c
\end{verbatim}

Viidennen päivän ''postoperatiivinen pneumonia'' vai sittenkin lekaasi? Saumalekaasi eli suolisauman pettäminen tapahtuu yleisimmin 5. POP (vaihteluväli yleensä 3.- 7. (-10.) POP; myös aikaisempia ja myöhäisempiä). 5. päivän pneumonia onkin siis saumalekaasi, kunnes on toisin todistettu.

Lekaasin oirekuva voi olla hämäävä ja skaalaltaan laaja, mutta potilaan korkea kuume, koholla olevat tulehdusarvot, diffuusi vatsakipu ja muutenkin yleistilan lasku viittaavat parhaiten lekaasiin ottaen huomioon 5. päivää sitten tapahtuneen hemikolektomian. Yleensä tämä tila johtaa uusintaleikkaukseen.

\end{solution}

\section{Potilastapaus}\label{potilastapaus-27}

62-vuotias mies tulee vastaanotollesi, koska haluaa, että hänen PSA arvonsa mitataan. Hänellä ei ole virtsaamisen oireita ja hänen PSA arvoaan ei ole koskaan mitattu? Mitä teet?

\begin{itemize}
\tightlist
\item
  \begin{enumerate}
  \def\labelenumi{\alph{enumi}.}
  \tightlist
  \item
    Kerrot, että PSA-seulontaa ei suositella ja eväät pelkän PSA-mittaamisen. Tuseeraat potilaan. Jos hänellä tuntuu kyhmy, mittaat PSA:n.~Muuten kieltäydyt mittaamisesta.
  \end{enumerate}
\item
  \begin{enumerate}
  \def\labelenumi{\alph{enumi}.}
  \setcounter{enumi}{1}
  \tightlist
  \item
    Selvität, onko miehellä lähisuvussa eturauhassyöpää. Kerrot, että PSA saattaa olla hyvälaatuisista syistä koholla ja hän saattaa joutua turhaan urologin vastaanotolle, jossa otetaan mahdollisesti koepalat eturauhasesta. Toisaalta tiedät kertoa, että eturauhassyöpä ei välttämättä aiheuta mitään oireita. Jos potilas keskustelun jälkeen haluaa edelleen PSA-mittaukseen, tuseeraat potilaan ja mittaat PSA:n.~
  \end{enumerate}
\item
  \begin{enumerate}
  \def\labelenumi{\alph{enumi}.}
  \setcounter{enumi}{2}
  \tightlist
  \item
    Kerrot, että PSA-seulontaa ei suositella ja eväät PSA:n mittaamisen.
  \end{enumerate}
\item
  \begin{enumerate}
  \def\labelenumi{\alph{enumi}.}
  \setcounter{enumi}{3}
  \tightlist
  \item
    Selvität, onko miehellä lähisuvussa eturauhassyöpää. Jos ei ole, kerrot, että PSA-mittaus aiheuttaa vain haittaa ja sitä ei tämän vuoksi suositella. Vetoat ylihoitoon ja et näin ollen suostu PSA-mittaukseen. Jos potilaalla on selvästi sukurasitus eturauhassyövälle, tuseeraat potilaan ja mittaat PSA:n.
  \end{enumerate}
\end{itemize}

\begin{solution}
\leavevmode

Vastaus

\begin{verbatim}
  b
\end{verbatim}

Suomessa ei ole kansallista eturauhassyövän seulontaohjelmaa ja PSA-arvoa hyödyntävä oireettomien potilaiden seulonta on edelleen kiistanalaista. Testin suurin ongelma ovat harmia aiheuttamattomien syöpien löytymisestä johtuva ylidiagnostiikka ja siihen liittyvät turhat hoidot ja komplikaatiot (esim. erektiohäiriöt ja inkontinenssi). Jos oireettoman miehen PSA-testausta harkitaan, hänelle pitää kertoa eturauhassyövän varhaiseen toteamiseen liittyvistä hyödyistä ja haitoista; ÄLÄ OTA OIREETTOMALTA POTILAALTA PSA-MITTAUSTA KERTOMATTA POTILAALLE AIKOMUKSESTASI JA SEN SEURAUKSISTA (älä sekoita PSA-testin käyttöön osana kliinistä tutkimusta potilaalla, jolla voidaan epäillä eturauhassyöpää). Jos potilas keskustelun jälkeen haluaa edelleen PSA-mittaukseen, tuseeraa potilas ja mittaa PSA. Älä suosittele PSA-testiä oireettomille yli 70-vuotiaille tai miehille, joilla on eliniänodotetta merkittävästi lyhentävä sairaus

a: PSA-seulontaa ei kannata rutiininomaisesti evätä sitä haluavilta, koska siitä voi myös olla merkittäviä hyötyjä potilaan kannalta. Parhaimmassa tapauksessa voidaan löytää kapselin ulkopuolelle levinnyt tai muuten suuren etenemisriskin omaava eturauhassyöpä, jonka radikaali kirurginen tai sädehoito saattaisi parantaa ja siten pelastaa potilaan hengen. Hyvin matala arvo voi myös toimia mieltä rauhoittavana löydöksenä, jos potilas stressaa eturauhassyöpämahdollisuudesta; syöpää ei kuitenkaan koskaan voida varmuudella poissulkea.

Vastauksessa on se osa oikein, että tuseeraat potilaan ja mittaat PSA:n jos kyhmy tuntuu, mutta PSA:ta ei tarvitse suoraan evätä, vaikka kyhmyä ei tuntuisikaan.

c: Ei voi yksiselitteisesti sanoa, että PSA-seulontaa ei suositella. Monet lääkärit ja urologit ovat alkaneet kääntyä PSA-seulonnan kannattajiksi viime vuosina.

d: Sama ongelma kuin A; ei siis tarvitse näin suoraan evätä PSA:n mittaamista.

\end{solution}

\section{Kilpirauhasen poiston eli totaalityreoidektomian indikaationa on}\label{kilpirauhasen-poiston-eli-totaalityreoidektomian-indikaationa-on}

\begin{itemize}
\tightlist
\item
  \begin{enumerate}
  \def\labelenumi{\alph{enumi}.}
  \tightlist
  \item
    Kaikki kysymyksen vastausvaihtoehdot ovat oikein.
  \end{enumerate}
\item
  \begin{enumerate}
  \def\labelenumi{\alph{enumi}.}
  \setcounter{enumi}{1}
  \tightlist
  \item
    kilpirauhasen karsinooma
  \end{enumerate}
\item
  \begin{enumerate}
  \def\labelenumi{\alph{enumi}.}
  \setcounter{enumi}{2}
  \tightlist
  \item
    molemminpuolinen struuma
  \end{enumerate}
\item
  \begin{enumerate}
  \def\labelenumi{\alph{enumi}.}
  \setcounter{enumi}{3}
  \tightlist
  \item
    hypertyreoosi
  \end{enumerate}
\end{itemize}

\begin{solution}
\leavevmode

Vastaus

\begin{verbatim}
  a
  
\end{verbatim}

Kilpirauhaskirurgian indikaatiot ovat mm. seuraavat:

Syöpä

Ohutneulabiopsiassa syöpäepäily (onb lk III)

Struuman aiheuttamat paineoireet

Retrosternaalistruuma

Uusiutunut hypertyreoosi

Hypertyreoosin radiojodihoito ei sovi. Esimerkiksi lääkehoidosta huolimatta uusiutunut Basedowin tauti voidaan siis hoitaa radiojodihoidolla (RAI) tai tyreoidektomialla (koko kilpirauhasen poisto). RAI ei sovi, jos potilaalla on silmäoireita, koska hoito voi pahentaa niitä. Kts. muut aiheet ja vasta-aiheet taulukosta alla

Kosmeettinen haitta

\pandocbounded{\includegraphics[keepaspectratio]{images/raiaiheet.png}}

\end{solution}

\section{Potilastapaus}\label{potilastapaus-28}

Olet terveyskeskuksen päiväpäivystäjä ja vastaanotollesi tulee 64-vuotias mies, jolla on tullut akuutti virtsaumpi. Toto 38.8 C. CRP 9, leuk 5.6, krea 78. U-Kemseul: eryt -, leuk -, nitr -, PSA 25. Katetroit potilaan ja katetripussiin tulee heti 800ml kirkasta virtsaa. Päästät hänet kotiin katetrin kanssa. Potilas pitää katetria kaksi viikkoa pussilla, jonka jälkeen hän tulee katetrin poistoon ja virtsakululliseen kontrolliin. Potilas on erittäin huolestunut PSA-arvosta. Mitä teet jatkossa?

\begin{itemize}
\tightlist
\item
  \begin{enumerate}
  \def\labelenumi{\alph{enumi}.}
  \tightlist
  \item
    Tuseeraat potilaan, teetät virtsaelinten UÄ-tutkimuksen, mittaat PSA uudelleen 2vkon päästä ja laitat lähetteen erikoissairaanhoitoon urologialle, jos ultraäänessä eturauhanen on kookas ja/tai PSA on edelleen koholla
  \end{enumerate}
\item
  \begin{enumerate}
  \def\labelenumi{\alph{enumi}.}
  \setcounter{enumi}{1}
  \tightlist
  \item
    Tuseeraat potilaan ja mittaat PSA uudelleen 1kk päästä ja laitat lähetteen erikoissairaanhoitoon urologialle, jos PSA on edelleen koholla
  \end{enumerate}
\item
  \begin{enumerate}
  \def\labelenumi{\alph{enumi}.}
  \setcounter{enumi}{2}
  \tightlist
  \item
    Tuseeraat potilaan ja laitat lähetteen erikoissairaanhoitoon urologialle.
  \end{enumerate}
\item
  \begin{enumerate}
  \def\labelenumi{\alph{enumi}.}
  \setcounter{enumi}{3}
  \tightlist
  \item
    Tuseeraat potilaan, mittaat PSA uudelleen 2vkon päästä ja laitat lähetteen erikoissairaanhoitoon urologialle, jos PSA on edelleen koholla
  \end{enumerate}
\end{itemize}

\begin{solution}
\leavevmode

Vastaus

\begin{verbatim}
  b
  
\end{verbatim}

PSA:ta voi nostaa monet hyvänlaatuiset asiat. Tärkeimpiä ovat VTI, eturauhastulehdus, virtsaumpi, suuri prostata ja katetrointi. Jos koholla olevan PSA-arvon taustalla epäillään sekundaarista syytä (esim. nyt tässä tapauksessa virtsaumpi ja vielä katetrointi ja katetrin poisto nyt vastaanotolla), kontrolloidaan PSA 4 vkon kuluttua.

\end{solution}

\section{Potilastapaus}\label{potilastapaus-29}

Vastaanotollesi tulee 69-vuotias potilas, jolla on distaalisessa sääressä n.~5 x 3 cm kokoinen pinnallinen, keltaisen fibriinikatteen peittämä, kostea ja kivuton säärihaava. Molemmissa alaraajoissa on pitting-ödeemaa säären yläkolmannekseen saakka. ADP ja ATP palpoituvat. Mikä on tärkein hoito?

\begin{itemize}
\tightlist
\item
  \begin{enumerate}
  \def\labelenumi{\alph{enumi}.}
  \tightlist
  \item
    Ihonsiirto
  \end{enumerate}
\item
  \begin{enumerate}
  \def\labelenumi{\alph{enumi}.}
  \setcounter{enumi}{1}
  \tightlist
  \item
    Kosteutta imevä paikallinen haavanhoitotuote
  \end{enumerate}
\item
  \begin{enumerate}
  \def\labelenumi{\alph{enumi}.}
  \setcounter{enumi}{2}
  \tightlist
  \item
    Turvotuksen hoito
  \end{enumerate}
\item
  \begin{enumerate}
  \def\labelenumi{\alph{enumi}.}
  \setcounter{enumi}{3}
  \tightlist
  \item
    Konsultaatio verisuonikirurgille
  \end{enumerate}
\end{itemize}

\begin{solution}
\leavevmode

Vastaus

\begin{verbatim}
  c
\end{verbatim}

Potilaalla on pinnallinen ja kivuton kostea haava distaalisessa sääressä. Haavan lokaatio ja kuvailtu ulkonäkö sekä laaja pitting-ödeema molemmissa alaraajoissa vievät ajatukset klassiseen laskimoperäiseen säärihaavaan. Alaraajapulssit palpoituvat poissulkien merkittävän alaraajojen tukkivan valtimotaudin.

Tärkein hoito laskimovajaatoiminnassa on aloittaa heti kompressiohoito, joka hoitaa turvotusta ja mahdollistaa haavan paranemisen. Kompressiohoidolle on hyvin vähän vasta-aiheita ja pääasiassa tulee huomioida, onko potilaalla dekompensoitu sydämen vajaatoiminta (ei haluta pahimmassa vaiheessa työntää kaikkia nesteitä ylös keuhkoihin pahentaen tilannetta) tai kriittistä alaraajaiskemiaa (ei tarvitse mitata ABI-arvoja rutiinisti ennen kompressiohoidon aloittamista, kuten joissain lähteissä sanotaan, vaan voi aloittaa kompressiohoidon ja seurata, että aiheuttaako se ongelmia. Tällä potilaalla vielä on merkittävä ASO-tauti poissuljettukin, niin ei tarvitse siitä huolehtia).

Kompressiohoidoksi voidaan aloittaa aluksi monikerrostukisidonta ja jos potilas on liikuntakykyinen niin aloitetaan vähäelastisella sidonnalla.

a: Ihonsiirto voi tulla kyseeseen, jos haava kroonistuu optimaalisesta kompressiohoidosta huolimatta.

b: Mikään paikallinen hoito ei hoida laskimovajaatoimintahaavaa ilman kompressiohoitoa.

d: Kompressiohoito voidaan aloittaa ilman verisuonikirurgin konsultaatiota.

\pandocbounded{\includegraphics[keepaspectratio]{images/säärihaavanparaneminen.png}}
\pandocbounded{\includegraphics[keepaspectratio]{images/kompressiohoitomalli.png}}
\pandocbounded{\includegraphics[keepaspectratio]{images/monikerrossidonta.png}}
\pandocbounded{\includegraphics[keepaspectratio]{images/sidontahaavaiselle.png}}

\end{solution}

\section{Potilastapaus}\label{potilastapaus-30}

Potilaalla todettu edellisellä viikolla haimasyöpä ja siihen liittyen maksametastaasi. Ei aiempia leikkauksia. Nyt potilas tulee päivystykseen keltaisuuden vuoksi. UÄ:ssä todetaan sappistaasi. Ensisijaisesti potilas tulee ohjata:

\begin{itemize}
\tightlist
\item
  \begin{enumerate}
  \def\labelenumi{\alph{enumi}.}
  \tightlist
  \item
    ERC-toimenpiteeseen
  \end{enumerate}
\item
  \begin{enumerate}
  \def\labelenumi{\alph{enumi}.}
  \setcounter{enumi}{1}
  \tightlist
  \item
    PTC-toimenpiteeseen
  \end{enumerate}
\item
  \begin{enumerate}
  \def\labelenumi{\alph{enumi}.}
  \setcounter{enumi}{2}
  \tightlist
  \item
    laparoskopiaan
  \end{enumerate}
\item
  \begin{enumerate}
  \def\labelenumi{\alph{enumi}.}
  \setcounter{enumi}{3}
  \tightlist
  \item
    laparotomiaan
  \end{enumerate}
\end{itemize}

\begin{solution}
\leavevmode

Vastaus

\begin{verbatim}
  a
\end{verbatim}

Haimasyöpä voi obstruktoida sappiteitä ja sitä kautta aiheuttaa sappistaasin ja obstruktiivisen ikteruksen. Haimakasvaimen aiheuttaman sappitietukoksen ensisijainen hoito, jos tuumorin leikkaushoito ei ole indikoitua (palliatiivinen hoito) tai potilasta ei voida nopeasti leikata, on tukoksen laukaiseminen endoskooppisessa retrogradisessa kolangiografiassa (ERC) asetettavalla stentillä.

b: Tukoksen toissijainen laukaisumenetelmä on ihon ja maksan läpi sappiteihin radiologisesti läpivalaisussa asetettava dreeni (PTD; perkutaaninen transhepaattinen dreeni, asetetaan PTC-teitse (perkutaaninen transhepaattinen kolangiografia))

c-d: Jos potilas leikattaisiin kasvaimen vuoksi nopeasti, turhaa stenttausta pyritään välttämään, koska se lisää leikkauksen jälkeisten komplikaatioiden riskiä. Tässä tapauksessa potilaalla on jo kuitenkin maksametastaaseja eikä hän todennäköisesti ole leikkaushoidon piirissä.

\end{solution}

\section{Kuinka monta prosenttia kolorektaalikarsinoomista löytyy tuseeramalla?}\label{kuinka-monta-prosenttia-kolorektaalikarsinoomista-luxf6ytyy-tuseeramalla}

\begin{itemize}
\tightlist
\item
  \begin{enumerate}
  \def\labelenumi{\alph{enumi}.}
  \tightlist
  \item
    \begin{enumerate}
    \def\labelenumii{\alph{enumii}.}
    \setcounter{enumii}{13}
    \tightlist
    \item
      30\%
    \end{enumerate}
  \end{enumerate}
\item
  \begin{enumerate}
  \def\labelenumi{\alph{enumi}.}
  \setcounter{enumi}{1}
  \tightlist
  \item
    \begin{enumerate}
    \def\labelenumii{\alph{enumii}.}
    \setcounter{enumii}{13}
    \tightlist
    \item
      5\%
    \end{enumerate}
  \end{enumerate}
\item
  \begin{enumerate}
  \def\labelenumi{\alph{enumi}.}
  \setcounter{enumi}{2}
  \tightlist
  \item
    \begin{enumerate}
    \def\labelenumii{\alph{enumii}.}
    \setcounter{enumii}{13}
    \tightlist
    \item
      10\%
    \end{enumerate}
  \end{enumerate}
\item
  \begin{enumerate}
  \def\labelenumi{\alph{enumi}.}
  \setcounter{enumi}{3}
  \tightlist
  \item
    \begin{enumerate}
    \def\labelenumii{\alph{enumii}.}
    \setcounter{enumii}{13}
    \tightlist
    \item
      20\%
    \end{enumerate}
  \end{enumerate}
\end{itemize}

\begin{solution}
\leavevmode

Vastaus

\begin{verbatim}
  d
\end{verbatim}

Koska noin 40\% paksu- ja peräsuolen syövistä sijaitsee nimenomaan peräsuolen alueella, on tuseeraus nopea ja helppo tapa varmistaa peräsuoli syöpäkasvaimen tai verenvuodon osalta. Tuseerauksella voidaan kuitenkin todeta vain hyvin selvät tapaukset ja suolen loppuosan syövät, joten diagnostisointimenetelmänä uuden suolistosyövän löytymisen suhteen, se on kovin rajallinen. Kaikista paksu- ja peräsuolen syövistä pelkästään tuseeraamalla pystytään löytämään 20 -- 25 \%. Peräsuolen kasvaimista arviolta n.~30\% on löydettävissä tuseeraamalla.

Hyvin näätä kysymys kylläkin, koska data vaihtelee lähteestä toiseen ja sillä ei ole käytännön eroa onko määrä nyt 20\% vai 30\%, koska tuseeraat potilaan suoliston sairautta epäillessäsi riippumatta tarkasta prosenttimäärästä.\\

\end{solution}

\section{Mikä seuraavista ei ole alaraajan pintalaskimovajaatoiminnan hoitomuoto?}\label{mikuxe4-seuraavista-ei-ole-alaraajan-pintalaskimovajaatoiminnan-hoitomuoto}

\begin{itemize}
\tightlist
\item
  \begin{enumerate}
  \def\labelenumi{\alph{enumi}.}
  \tightlist
  \item
    Sairaan päärungon termoablaatio (laser- tai Rf-tekniikka)
  \end{enumerate}
\item
  \begin{enumerate}
  \def\labelenumi{\alph{enumi}.}
  \setcounter{enumi}{1}
  \tightlist
  \item
    Sairaan laskimon ohitusleikkaus
  \end{enumerate}
\item
  \begin{enumerate}
  \def\labelenumi{\alph{enumi}.}
  \setcounter{enumi}{2}
  \tightlist
  \item
    Sairaiden laskimoiden skleroterapia
  \end{enumerate}
\item
  \begin{enumerate}
  \def\labelenumi{\alph{enumi}.}
  \setcounter{enumi}{3}
  \tightlist
  \item
    Sairaiden suonten poisto ja refluksireitin sulku kirurgisesti
  \end{enumerate}
\end{itemize}

\begin{solution}
\leavevmode

Vastaus

\begin{verbatim}
  b
\end{verbatim}

Kajoavan hoidon vaihtoehtoja ovat EVTA, UGFS tai avokirurgia.

a: EVTA ((endovascular thermal ablation) eli termoablaatio) on nykyään ensisijainen kajoava hoitomenetelmä pinnallisen laskimopäärungon vajaatoiminnassa ja on käytännössä korvannut avokirurgian. Se voidaan suorittaa laserablaationa (EVLA, endovenous laser ablation) tai radiotaajuusablaationa (FRA, radiofrequency segmental thermal ablation, rf-tekniikka eli radiofrekvenssitekniikka). Ensisijaisena näistä voidaan pitää laserablaatiota.

Termoablaatiossa hoidettava laskimo punktoidaan ja laserkuitu uitetaan sisään. Se viedään vajaatoimintamuutosten läpi proksimaaliselle puolelle safenofemoraalisen tai safenopopliteaalisen junktion distaalipuolelle pitäen turvamarginaalin syvään laskimoon. Hoidettavan laskimon ympärille injusoidaan puudutus. Peruutellessa ulos laseroidaan suonta, mikä johtaa laskimon endoteelin vaurioon ja johtaa pintalaskimon fibrotisoitumiseen kiinni, jolloin se ei enää voi aiheuttaa vajaatoiminta-oireita. Merkittävä osa termoablaatioista voidaan toteuttaa poliklinikkaoloissa. Erityistä jälkiseurantaa ei tarvita, ja potilas voi mobilisoitua heti. Komplikaatiot ovat harvinaisia ja tärkein tiedostaa on syvä laskimotukos, jonka takia hoito on vasta-aiheinen raskaana oleville, aktiivisen pintalaskimotukoksen aikana ja syvien laskimoiden obstruktiossa sekä niille, joilla on avoin foramen ovale (mahdollisuus paradoksiselle embolialle, kun laskimotrombi pääsee avoimen foramen ovalen kautta vasempaan eteiseen ja valtimoverenkiertoon).

b: Laskimoiden ohitusleikkaus ei ole indikoitua laskimovajaatoiminnan hoidossa.

c: Jos pintalaskimoiden termoablaatio ei ole teknisesti mahdollinen esimerkiksi hoidettavan laskimon mutkaisuuden takia, valitaan muu hoitomenetelmä, joka on yleensä vaahtoskleroterapia (UGFS, ultrasound guided foam sclerotherapy), jos kyseessä on suhteellisen pienet suonet. Hoitotulokset heikkenee mikäli suonen koko \textgreater{} 6 mm -\textgreater{} pääasiassa soveltuu pienten suonien hoitoon. Vaahtoskleroterapia toimii siten, että sklerosoiva vaahto syrjäyttää veren laskimosta, vaurioittaa endoteelin, aiheuttaa laskimon tromboosin ja lopulta fibrotisoitumisen.

d: Jos termoablaatio ja vaahtoskleroterapia on suljettu pois, valitaan avoleikkaus (High ligation and stripping).

\pandocbounded{\includegraphics[keepaspectratio]{images/evta.png}}
\pandocbounded{\includegraphics[keepaspectratio]{images/skleroterapia.png}}
\pandocbounded{\includegraphics[keepaspectratio]{images/highligation.png}}

\end{solution}

\section{Potilastapaus}\label{potilastapaus-31}

Vastaanotollesi tulee 69-vuotias potilas, jolla on distaalisessa sääressä n.~5 x 3 cm kokoinen pinnallinen, keltaisen fibriinikatteen peittämä, kostea ja kivuton säärihaava. Molemmissa alaraajoissa on pitting-ödeemaa säären yläkolmannekseen saakka. ADP ja ATP palpoituvat. Mikä on todennäköisin säärihaavan etiologia?

\begin{itemize}
\tightlist
\item
  \begin{enumerate}
  \def\labelenumi{\alph{enumi}.}
  \tightlist
  \item
    Martorellin ulkus
  \end{enumerate}
\item
  \begin{enumerate}
  \def\labelenumi{\alph{enumi}.}
  \setcounter{enumi}{1}
  \tightlist
  \item
    Laskimoperäinen haava
  \end{enumerate}
\item
  \begin{enumerate}
  \def\labelenumi{\alph{enumi}.}
  \setcounter{enumi}{2}
  \tightlist
  \item
    Neuropaattinen haava
  \end{enumerate}
\item
  \begin{enumerate}
  \def\labelenumi{\alph{enumi}.}
  \setcounter{enumi}{3}
  \tightlist
  \item
    Iskeeminen haava
  \end{enumerate}
\end{itemize}

\begin{solution}
\leavevmode

Vastaus

\begin{verbatim}
   b
   
\end{verbatim}

Kyseessä on sama potilas kuin aikaisemmassa kysymyksessä, jossa mietittiin, mitä hänelle kuuluisi tehdä. Potilaalla on pinnallinen ja kivuton kostea haava distaalisessa sääressä. Haavan lokaatio ja kuvailtu ulkonäkö sekä laaja pitting-ödeema molemmissa alaraajoissa vievät ajatukset klassiseen laskimoperäiseen säärihaavaan. Alaraajapulssit palpoituvat poissulkien merkittävän alaraajojen tukkivan valtimotaudin.

a: Hypertensiivinen säärihaava eli Martorellin haava on epätavallinen mutta ei harvinainen säärihaavan muoto. Martorellin haava on pitkään huonossa hoitotasapainossa olleen verenpainetaudin komplikaatio 40-80-vuotiailla potilailla, joista osa sairastaa myös diabetesta. Kliininen kuva on selkeä ja tunnusomainen: kyseessä on erittäin kivulias pinnallinen nekroottinen haava, jossa on purppuran punainen reunus. Paraneminen on yleensä hidasta.

Martorellin haavan taustalla on mikroangiopatia. Verenpainetautia sairastavilla on suuri ääreisvaltimoiden vastus huolimatta normaalista nilkka-olkavarsipainesuhteesta. Tämä johtuu pienten valtimoiden ahtautumisesta tai tukkeutumisesta. Seurauksena on kudosperfuusion vähentyminen, paikallinen iskemia sekä haavoja ja kipua. Pienet valtimot eivät reagoi normaalisti laajenemalla, mikä johtaa jopa ihon infarkteihin.

c: Neuropaattinen haava on tyypillinen diabeetikoilla johtuen distaalisesta polyneuropatiasta. Diabeetikon jalkahaava sijaitsee tyypillisesti nilkan distaalipuolella alueilla, joihin kohdistuu painetta. Polyneuropatian takia potilas ei kuitenkaan tunne paineen aiheuttamaa iskemiaa alueella, mikä johtaa lopulta haavaan. Noin puolessa diabeettisista haavoista on alentunut verenkierto myös osatekijänä. Tärkeimpiä tekijöitä diabeetikoiden haavoissa ovat siis verisuonien ahtautuminen ja siten kudosiskemia sekä diabeteksen neuropatia, joka peittää iskemian oireita.

d: Kriittinen iskemia voi aiheuttaa spontaanisti/vähäpätöisen vamman takia alaraajahaavan, jonka tyypillinen lokalisaatio on distaaliset kärkijäsenet ja/tai painealueet (esim. kantapää tai päkiät, malleolukset (erityisesti lateraalinen)). Valtimoperäiset haavat ovat myös yleensä tarkkarajaisempia ja pienempiä ja syvempiä kuin laskimoperäiset haavat.

\pandocbounded{\includegraphics[keepaspectratio]{images/alaraajahaavattyypit.png}}
\pandocbounded{\includegraphics[keepaspectratio]{images/martorelli.png}}

\end{solution}

\section{Potilastapaus}\label{potilastapaus-32}

Vastaanotollesi terveyskeskukseen tulee 65-vuotias mies. Hänelle on tehty eturauhassyövän vuoksi eturauhasen radikaali poistoleikkaus kolme vuotta sitten ja olet leikkaavan yksikön ohjeiden mukaisesti mitannut PSA-arvon vuosittain. Viimeisimmässä kontrollissa PSA 0.25. Mitä teet?

\begin{itemize}
\tightlist
\item
  \begin{enumerate}
  \def\labelenumi{\alph{enumi}.}
  \tightlist
  \item
    Teet virtsaelinten UÄ-tutkimuksen ja koska poistetun eturauhasen kohdalla ei näy poikkeavaa, mittaat PSA:n vuoden kuluttua ohjeiden mukaisesti
  \end{enumerate}
\item
  \begin{enumerate}
  \def\labelenumi{\alph{enumi}.}
  \setcounter{enumi}{1}
  \tightlist
  \item
    Mittaat PSA:n vuoden kuluttua ohjeiden mukaisesti
  \end{enumerate}
\item
  \begin{enumerate}
  \def\labelenumi{\alph{enumi}.}
  \setcounter{enumi}{2}
  \tightlist
  \item
    Lähetät potilaan erikoissairaanhoitoon
  \end{enumerate}
\item
  \begin{enumerate}
  \def\labelenumi{\alph{enumi}.}
  \setcounter{enumi}{3}
  \tightlist
  \item
    Tuseeraat potilaan ja koska kyhmyjä ei tunnu, mittaat PSA:n vuoden kuluttua ohjeiden mukaisesti
  \end{enumerate}
\end{itemize}

\begin{solution}
\leavevmode

Vastaus

\begin{verbatim}
   c
   
\end{verbatim}

Radikaalin prostatektomian jälkeen PSA-arvon odotetaan olevan mittaamattomissa (\textless{} 0,1 µg/l) 2 kuukauden kuluttua leikkauksesta. Tyypillisesti PSA-arvo määritetään 3 ja 6 kuukauden kuluttua leikkauksesta, sen jälkeen 6 kuukauden välein kolmanteen vuoteen asti ja sen jälkeen vuosittain. \textbf{Avoterveydenhuollon seurannassa yksittäinenkin PSA-arvon suureneminen mitattavaksi radikaaliprostatektomian jälkeen on indikaatio lähetteelle erikoissairaanhoitoon.}

a: Potilaan eturauhassyöpä on todennäköisesti uusiutunut, UÄ ei riitä nyt.

b ja d: PSA on noussut ja eturauhassyöpä on todennäköisesti uusiutunut, ei voi jäädä seuraamaan PSA-arvoja. Tilanne on erilainen, jos syöpä oltaisiin hoidettu sädehoidolla. Sädehoidon jälkeen biokemiallisena uusiutumisena pidetään PSA-arvon suurenemista 2 µg/l hoidon jälkeistä alinta arvoa (nadir) suuremmaksi. Määritelmä koskee myös potilaita, jotka ovat saaneet hormonaalista hoitoa. Kun potilas siirtyy perusterveydenhuollon seurantaan, tulee potilasasiakirjoihin kirjata erikoissairaanhoitoon lähettämisen raja-arvo.

\end{solution}

\section{Suturaatiolangan vahvuus}\label{suturaatiolangan-vahvuus}

Olet poistanut 45-vuotiaalta miespotilaaltasi selästä lapaluun kohdalta 1 cm kokoisen suspektin iholuomen diagnostisin marginaalein PAD-tutkimukseen. Mikä on sopiva sulamattoman langan vahvuus ihon sulkua varten?

\begin{itemize}
\tightlist
\item
  \begin{enumerate}
  \def\labelenumi{\alph{enumi}.}
  \tightlist
  \item
    3-0
  \end{enumerate}
\item
  \begin{enumerate}
  \def\labelenumi{\alph{enumi}.}
  \setcounter{enumi}{1}
  \tightlist
  \item
    8-0
  \end{enumerate}
\item
  \begin{enumerate}
  \def\labelenumi{\alph{enumi}.}
  \setcounter{enumi}{2}
  \tightlist
  \item
    5-0
  \end{enumerate}
\item
  \begin{enumerate}
  \def\labelenumi{\alph{enumi}.}
  \setcounter{enumi}{3}
  \tightlist
  \item
    2-0
  \end{enumerate}
\end{itemize}

\begin{solution}
\leavevmode

Vastaus

\begin{verbatim}
   a
\end{verbatim}

Ommellangan ominaisuuksia ovat vetolujuus, vahvuus ja läpimitta. Vetolujuudella tarkoitetaan langan lujuutta vetävää voimaa vastaan. Näitä ominaisuuksia määritetään eniten käytössä olevalla USP- eli United States Pharmacopeia -normilla. USP on numeerinen määrite, esimerkiksi 3-0, 2-0, 1-0, 0, 1. Tämän normin mukaisesti langan paksuus kerrotaan numerokoodilla eli mitä enemmän nollia koodissa on, sitä ohuempi ommellanka on kyseessä. Esimerkiksi 4-0-langasta käytetään puhekielessä nimitystä ''neljän nollan'' ja 5-0:sta ''viiden nollan'' lanka. Mitä suurempi langan koodi on, sitä pienempi vetolujuus sillä on (esim. 10-0 on todella pieni vetolujuus ja soveltuu vain mikrokirurgiaan).

Perusohjeita lankojen käytölle on (ainakin dioissa näin, eri lähteissä hieman erilain):

10-0 \ldots8-0 mikrokirurgia

6-0 kasvot

5-0 kasvot, kaula, käsi

4-0 peruslanka -\textgreater{} vartalo, raajat

3-0 päänahka, jalkapohja, selkä

2-0 lihaksiin tai faskiaan

1-0, 0, 1, 2, \ldots{}

\end{solution}

\section{On tehty totaali gastrektomia diffuusin ventrikkelikarsinooman takia. Leikkaus ollut makroskooppisesti radikaali. Lopullisessa PAD-vastauksessa mainitaan, että poisto on R1-tasoinen. Se tarkoittaa, että}\label{on-tehty-totaali-gastrektomia-diffuusin-ventrikkelikarsinooman-takia.-leikkaus-ollut-makroskooppisesti-radikaali.-lopullisessa-pad-vastauksessa-mainitaan-ettuxe4-poisto-on-r1-tasoinen.-se-tarkoittaa-ettuxe4}

\begin{itemize}
\tightlist
\item
  \begin{enumerate}
  \def\labelenumi{\alph{enumi}.}
  \tightlist
  \item
    Imusolmukepoisto on riittämätön
  \end{enumerate}
\item
  \begin{enumerate}
  \def\labelenumi{\alph{enumi}.}
  \setcounter{enumi}{1}
  \tightlist
  \item
    Kasvain on täysin pois
  \end{enumerate}
\item
  \begin{enumerate}
  \def\labelenumi{\alph{enumi}.}
  \setcounter{enumi}{2}
  \tightlist
  \item
    Imusolmukepoisto on riittävä
  \end{enumerate}
\item
  \begin{enumerate}
  \def\labelenumi{\alph{enumi}.}
  \setcounter{enumi}{3}
  \tightlist
  \item
    Kasvain kasvaa mikroskooppisesti preparaatin katkaisulinjaan
  \end{enumerate}
\end{itemize}

\begin{solution}
\leavevmode

Vastaus

\begin{verbatim}
   d
\end{verbatim}

Syöpien leikkaushoidossa poiston jälkeen voidaan arvioida resektion (R) tasoa. Diffuusin ventrikkelikarsinooman hoidossa tavoitellaan täydellistä poistoa eli R0.

R0 = kaikki kasvainkudos poistettu

R1 = mikroskooppista tuumorikasvua jäljellä resektiolinjassa

R2 = makroskooppista tuumorikasvua jäljellä

\end{solution}

\section{Potilastapaus}\label{potilastapaus-33}

Perusterve 40-vuotias mies. Vasen alavatsakipu 3 pv. Yleiskunto hyvä, lievää lämpöilyä, ei selvää defancea. Leuk 13, CRP 50. TT:ssä komplisoitumaton akuutti divertikuliitti. Aloitatko antibiootit? Entä tarvitaanko diagnoosin varmistamiseksi rauhallisen vaiheen kolonoskopiaa?

\begin{itemize}
\tightlist
\item
  \begin{enumerate}
  \def\labelenumi{\alph{enumi}.}
  \tightlist
  \item
    Antibiootit kyllä, ei kolonoskopiaa
  \end{enumerate}
\item
  \begin{enumerate}
  \def\labelenumi{\alph{enumi}.}
  \setcounter{enumi}{1}
  \tightlist
  \item
    Ei antibiootteja, ei kolonoskopiaa
  \end{enumerate}
\item
  \begin{enumerate}
  \def\labelenumi{\alph{enumi}.}
  \setcounter{enumi}{2}
  \tightlist
  \item
    Molemmat tarvitaan
  \end{enumerate}
\item
  \begin{enumerate}
  \def\labelenumi{\alph{enumi}.}
  \setcounter{enumi}{3}
  \tightlist
  \item
    Ei antibiootteja, kolonoskopia kyllä
  \end{enumerate}
\end{itemize}

\begin{solution}
\leavevmode

Vastaus

\begin{verbatim}
 b  
\end{verbatim}

Jos potilaalla on komplisoitumaton divertikuliitti, niin se voidaan hoitaa oireenmukaisesti (NSAID + parasetamoli) ilman antibioottia. Jos kyseessä on potilaan ensimmäinen divertikuliitti (kuten nyt tässä), niin diagnoosi tulee varmistaa TT:llä. Jos taas uusiutunut lievällä tyypillisellä taudinkuvalla, niin ei tarvitse TT:tä ja voidaan hoitaa oireenmukaisesti ilman lisädiagnostiikkaa. Paastoa tai ruokarajoituksia ei tarvita, vaan potilas voi syödä vapaasti. Mikäli oireet eivät helpota parissa päivässä tai vointi heikkenee, tulee diagnoosia tarkentaa TT:lla ja tarvittaessa aloittaa antibiootti.

a: Antibiootti kannattaa varovaisuusperiaatteen mukaan aloittaa raskaana olevilla tai jos potilaan immuunipuolustus on heikentynyt esimerkiksi solunsalpaajahoidon, immuunipuolustusta muokkaavan lääkityksen, maksakirroosin tai diabeteksen takia. Antibiootti, joka kattaa tavallisimmat suolistopatogeenit, voidaan toteuttaa Suomessa useimmiten kefalosporiinin ja metronidatsolin yhdistelmällä. Hoito määräytyy potilaan yleistilan mukaan. Perusterveydenhuollossa hoitona on suun kautta otettavat kefaleksiini 500 mg x 3 ja metronidatsoli 400 mg x 3. Erikoissairaanhoidossa aloitetaan suonensisäisesti kefuroksiimi 1,5 g x 3 ja metronidatsoli 500 mg x 3. Yleensä viikon hoito riittää

c: Molemmat tarvitaan, jos kyseessä on komplisoitunut divertikuliitti. Komplisoitunut divertikuliitti vaatii hoidoksi aina vähintään i.v. antibiootin (yleensä kefuroksiimi ja metronidatsoli) ja joskus myös jopa leikkauksen tai joidenkin paiseiden tapauksessa dreneerauksen. Komplisoituneen divertikuliitin jälkeen kontrollitutkimuksena tulee suorittaa kolonoskopia rauhallisessa vaiheessa (n.~1kk jälkeen akuutista tulehduksesta), jotta voidaan varmistaa, ettei kyseessä ole paksusuolisyöpä. Komplisoitumattomassa divertikuliitissa mitään kontrolleja ei tarvita.

\pandocbounded{\includegraphics[keepaspectratio]{images/divertikuliittiluokatjahoito.png}}

\end{solution}

\section{Aortan B-tyypin dissekaation hoidossa seuraavista EI pidä paikkaansa:}\label{aortan-b-tyypin-dissekaation-hoidossa-seuraavista-ei-piduxe4-paikkaansa}

\begin{itemize}
\tightlist
\item
  \begin{enumerate}
  \def\labelenumi{\alph{enumi}.}
  \tightlist
  \item
    B-tyypin dissekaatio voi dilatoitua ja rupturoitua vuosia ensi-ilmaantumisen jälkeen
  \end{enumerate}
\item
  \begin{enumerate}
  \def\labelenumi{\alph{enumi}.}
  \setcounter{enumi}{1}
  \tightlist
  \item
    B-tyypin dissekaatio on vähemmän yleinen kuin A-tyypin dissekaatio
  \end{enumerate}
\item
  \begin{enumerate}
  \def\labelenumi{\alph{enumi}.}
  \setcounter{enumi}{2}
  \tightlist
  \item
    valtaosa B-tyypin dissekaatioista joudutaan leikkaamaan hätäleikkauksena tai vähintään lähipäivinä
  \end{enumerate}
\item
  \begin{enumerate}
  \def\labelenumi{\alph{enumi}.}
  \setcounter{enumi}{3}
  \tightlist
  \item
    laskeva aortta voidaan hoitaa stentillä
  \end{enumerate}
\end{itemize}

\begin{solution}
\leavevmode

Vastaus

\begin{verbatim}
 c
\end{verbatim}

Komplisoitumattoman B-tyypin dissekoituman hoito on ensisijaisesti lääkkeellinen verenpaineen hallinta. B-tyypin dissekoituma komplisoituu akuutissa vaiheessa noin \textbf{16 \%:ssa} tapauksista (ei siis valtaosassa), ja tällöin on kajoava hoito perusteltu.

a: Totta. Tämän takia dissekoitumapotilaita seurataan sairaalahoidon jälkeenkin. Tehdään aortan TT tai MK (MK nuoremmille) 3 kk, 6 kk ja 12 kk dissekoitumasta, jonka jälkeen vuosittain ainakin 5 v ajan. Elinikäinen kuvantamisseuranta on yleensä perusteltu, mutta väliä voi harventaa 2--3 vuoteen, jos tilanne on ollut vakaa viiden vuoden ajan. Dissekoituman jälkeen osalle potilaista kehittyy toimenpidehoitoa vaativa aortan dilataatio 5 v:n kuluessa, joten seuranta on tärkeää. Seuranta toteutetaan leikanneessa yksikössä tai oman alueen sairaalassa leikanneen yksikön ohjeiden mukaan.

b: Stanfordin A-tyypin dissekoituma on yleisin. Se on myös vaarallisempi.

d: Ensisijaisesti komplisoituneen B-tyypin dissekoituman hoidossa kyseeseen tulee endograftihoito eli stentti (TEVAR, Thoracic Endovascular Aortic Repair), jolla peitetään tyypillisesti laskevan torakaaliaortan yläosassa sijaitseva intimakerroksen repeämä (entry tear).

A-tyypin dissekoituman hoito on taas ensisijaisesti operatiivinen ja tehdään avoleikkauksessa (tavoitteena on korvata dissekoitunut osa nousevaa aorttaa verisuoniproteesilla, erityisesti alue, jolla intimakerroksen repeämä (entry tear) sijaitsee). Jos aortan tyvi sinus Valsalvae -tasolla sekä aorttaläppä ovat rakenteellisesti ehjät, riittää yleensä suora putkiproteesi, joka ommellaan proksimaalisesti aortan sinotubulaarijunktion tasoon. Tämä on yleisin (noin 70 \%) aortan dissekoituman leikkaustekniikka.

\pandocbounded{\includegraphics[keepaspectratio]{images/tevar.png}}
\pandocbounded{\includegraphics[keepaspectratio]{images/stanford.png}}

\end{solution}

\section{Rintasyövän kirurgisen hoidon yleisin leikkaustekniikka on}\label{rintasyuxf6vuxe4n-kirurgisen-hoidon-yleisin-leikkaustekniikka-on}

\begin{itemize}
\tightlist
\item
  \begin{enumerate}
  \def\labelenumi{\alph{enumi}.}
  \tightlist
  \item
    Halstedin radikaali mastektomia
  \end{enumerate}
\item
  \begin{enumerate}
  \def\labelenumi{\alph{enumi}.}
  \setcounter{enumi}{1}
  \tightlist
  \item
    ihoa säästävä mastektomia ja välitön rintarekonstruktio
  \end{enumerate}
\item
  \begin{enumerate}
  \def\labelenumi{\alph{enumi}.}
  \setcounter{enumi}{2}
  \tightlist
  \item
    resektio ja vartijaimusolmukebiopsia
  \end{enumerate}
\item
  \begin{enumerate}
  \def\labelenumi{\alph{enumi}.}
  \setcounter{enumi}{3}
  \tightlist
  \item
    ablaatio ja vartijaimusolmukebiopsia
  \end{enumerate}
\end{itemize}

\begin{solution}
\leavevmode

Vastaus

\begin{verbatim}
 c
 
\end{verbatim}

Rintasyövän tavallisin kirurginen hoitomuoto on siis rinnan säästävä leikkaus eli resektio (breast conserving surgery). Tehdään siis vartijaimusolmuketutkimus, lumpektomia (partiaalinen mastektomia) ja lopuksi sädehoito. Sädehoito vähentää paikallista uusiutumisriskiä jäljelle jääneessä rauhaskudoksessa ja on siten olennainen osa rinnan säästävää hoitoa.

a: William Halsted (1852-1922) oli kirurgi, joka alkoi hoitaa rintasyöpäpotilaitaan poistamalla rinnan ja kainalon imusolmukkeiden kokonaan. Näiden lisäksi poistettiin myös pectoralislihakset. Halstedin potilaista oli elossa 51 \% kolme vuotta leikkauksen jälkeen, mikä oli massiivinen kehitys aikaisempiin hoitoihin verrattuna. Halstedin doktriini siitä, että laaja leikkaus olisi rintasyöpäpotilaille aina välttämätön oli pitkään syvälle juurtunut käsitys, mutta 1970-luvulta lähtien ensin luovuttiin ''radikaalista'' rinnan poistosta (Maddenin mastektomia, pectoralikset säästettiin) ja sitten myöhemmin tutkimuksilla osoitettiin rintaa säästävän leikkauksen turvallisuus.

b: Tavallisin indikaatio ihoa säästävälle mastektomialle on laaja DCIS ja halu tehdä välitön rintarekonstruktio. Välittömässä rintarekonstruktiossa syöpäleikkaus eroaa tavallisesta mastektomiasta siten, että rinnan oma iho säästetään kokonaan tai pääosin. Soveltuu esim. rintasyövälle altistavan geenimutaation kantajalle riskiä vähentävänä toimenpiteenä tai hyväennusteiselle rintasyövälle, jossa kuitenkin halutaan poistaa koko rintakudos.

Välitön rintarekonstruktio = rakennetaan uusi rinta samassa leikkauksessa kuin rinta poistetaan

Myöhäisrekonstruktio = potilaalle on tehty rintasyövän vuoksi rinnan ablaatio ja sen jälkeen annettu liitännäishoidot syöpätaudeilla ja vasta näiden hoitojen jälkeen tehdään rekonstruktio

d: Rinnan poisto eli mastektomia eli ablaatio on aiheellinen, jos invasiivinen rintasyöpä tai duktaalinen karsinooma in situ (DCIS) on niin laaja, että säästävällä leikkauksella ei saavuteta joko riittäviä tervekudosmarginaaleja tai hyväksyttävää esteettistä tulosta. Mastektomia tulee tehdä myös, jos potilas toivoo sitä vielä sen jälkeen, kun hän on saanut perusteellisen informaation leikkausmenetelmistä ja niiden eduista sekä haittapuolista. Mastektomia tulee myös tehdä, jos kyseessä on inflammatorinen rintakarsinooma (ei hoideta osapoistolla).

Lisäksi mastektomiaa on harkittava, mikäli sädehoitoa ei voida toteuttaa (säästävässä leikkauksessa leikkauksenjälkeinen sädehoito on tärkeä osa hoitoa). Tällaisia syitä ovat aiemmin annettu sädehoito (aiempi rintasyöpä tai Hodgkinin tauti), varhaisraskaus, tietyt autoimmuunisairaudet tai potilaan haluttomuus käydä sädehoidossa. Sädehoito pienentää paikallisen uusiutumisen riskiä myös iäkkäillä, mutta jos iäkäs potilas ei jaksa käydä sädehoidossa, se voidaan myös jättää pois hoitosuunnitelmasta.

\end{solution}

\section{Potilastapaus}\label{potilastapaus-34}

Vastaanotollesi tulee potilas, joka on käynyt sappikivitautioireiden vuoksi tilaamassasi ultraäänitutkimuksessa. Sattumalöydöksenä lausunnossa kerrotaan, että potilaalla on abdominaaliaortan aneurysma, jonka maksimihalkaisija on 3.5 cm. Mikä potilaalle kerrotuista tiedoista ei pidä paikkaansa?

\begin{itemize}
\tightlist
\item
  \begin{enumerate}
  \def\labelenumi{\alph{enumi}.}
  \tightlist
  \item
    Vaikka pullistuma jatkuu yli 10 cm matkalla, ei pullistuneen osan pituudella ole merkitystä pullistuman ennusteeseen. Pullistuneen valtimon halkaisija määrittää suonen puhkeamisen riskin
  \end{enumerate}
\item
  \begin{enumerate}
  \def\labelenumi{\alph{enumi}.}
  \setcounter{enumi}{1}
  \tightlist
  \item
    Teillä on pullistunut valtimo, jonka vuoksi lähetän teidät ambulanssilla heti päivystykseen
  \end{enumerate}
\item
  \begin{enumerate}
  \def\labelenumi{\alph{enumi}.}
  \setcounter{enumi}{2}
  \tightlist
  \item
    Tupakoinnin lopettamisella on suotuisa vaikutus myös pullistuneen ison valtimon ennusteeseen
  \end{enumerate}
\item
  \begin{enumerate}
  \def\labelenumi{\alph{enumi}.}
  \setcounter{enumi}{3}
  \tightlist
  \item
    Näin pieni pullistuma ei vielä vaadi kirurgista toimenpidettä. Seuraamme pullistumaa nyt säännöllisesti ja jos pullistumassa tapahtuu merkittävää kasvua, niin sitten harkitaan pullistuman hoitoa kirurgisesti
  \end{enumerate}
\end{itemize}

\begin{solution}
\leavevmode

Vastaus

\begin{verbatim}
 b
 
\end{verbatim}

Näin pieni vatsa-aortan aneurysma ei ole ruptuurariskissä eikä potilaalla ole sen suhteen oireitakaan. Ei siis tarvitse lähettää potilasta päivystykseen, vaan voidaan jatkaa seurantalinjalla muutoksen suhteen (tässä tapauksessa läpimitan perusteella n.~2-3 vuoden välein).

a: Vatsa-aortan aneurysman (AAA) ruptuurariskin määrittää enimmäisläpimitta ja sen perusteella hoitopäätökset tehdään. AAAn määritelmä on \textgreater30mm läpimitta (eli 1,5x normaalin läpimitan). Käytännön toimenpiderajaksi on asetettu 55 mm miehillä (50 mm naisilla), sillä tätä pienemmät aneurysmat repeävät harvoin.

c: Vatsa-aortan aneurysma (AAA) -potilaille on aina aiheellista seuraavat noninvasiiviset hoidot: tupakoinnin lopettaminen, elintapaohjeet (ml. liikunta- ja ruokavalio-ohjeet), verenpaineen hoito, statiinilääkitys, verihiutaleiden estäjälääkitys (ASA). Tärkein riskitekijä on tupakointi ja lähes 90 \% aneurysmapotilaista tupakoi tai on tupakoinut jossain vaiheessa elämää. Tupakointi suurentaa paitsi vatsa-aortan aneurysman ilmaantuvuutta myös sen kasvunopeutta ja repeämisriskiä.

d: Käytännön toimenpiderajaksi on asetettu 55 mm miehillä (50 mm naisilla), sillä tätä pienemmät aneurysmat repeävät harvoin. Tämän ulkopuolella muutosta kyllä seurataan ja jos vatsa-aortan aneurysman (AAA) kasvu on yli 10 mm vuodessa, tulee potilas lähettää verisuonikirurgiseen yksikköön (normaalisti aneurysma kasvaa n.~2-4mm vuodessa). Seuranta suoritetaan TK:ssa vatsan UÄ-tutkimuksella.

\pandocbounded{\includegraphics[keepaspectratio]{images/enimmäisläpimitta.png}}
\pandocbounded{\includegraphics[keepaspectratio]{images/seurantaaaa.png}}

\end{solution}

\section{Potilastapaus}\label{potilastapaus-35}

Vastaanotollesi tulee potilas, joka on Kannasta lukenut sairauskertomuksiaan. Toimenpiteenä Kannan tietojen mukaan on ollut rec: a. poplitea -- a. dorsalis pedis CVA converted. Potilas on hiukan epävarma, toimenpiteen jälkeisistä elämäntapaohjeista. Mikä seuraavista ohjeista ei paranna potilaan ennustetta?

\begin{itemize}
\tightlist
\item
  \begin{enumerate}
  \def\labelenumi{\alph{enumi}.}
  \tightlist
  \item
    Sairaalasta määrätty kolesterolilääke parantaa valtimotaudin ennustetta
  \end{enumerate}
\item
  \begin{enumerate}
  \def\labelenumi{\alph{enumi}.}
  \setcounter{enumi}{1}
  \tightlist
  \item
    Nyt kun teidät on ohitettu ja haavatkin on parantuneet, voitte jatkaa tupakointia, mikäli tupakoimatta olo tuntuu vaikealta
  \end{enumerate}
\item
  \begin{enumerate}
  \def\labelenumi{\alph{enumi}.}
  \setcounter{enumi}{2}
  \tightlist
  \item
    Toimenpiteen jälkeen liikkuminen ja erityisesti kävely on hoitotuloksen kannalta hyväksi
  \end{enumerate}
\item
  \begin{enumerate}
  \def\labelenumi{\alph{enumi}.}
  \setcounter{enumi}{3}
  \tightlist
  \item
    Sairaalassa aloitettu Primaspan-lääkitys on pysyvä lääkitys ja ehkäisee sydän- ja verisuonisairauksien ilmaantumista
  \end{enumerate}
\end{itemize}

\begin{solution}
\leavevmode

Vastaus

\begin{verbatim}
 b
 
\end{verbatim}

Tupakointi ei tietystikään paranna potilaan ennustetta.

Potilas on siis sama, jonka kantatekstiä jo tulkittiin aikaisemmassa kysymyksessä. Hänelle on tehty ASO-taudin takia ohitusleikkaus käyttäen laskimograftia.

a ja d: Oireellisesta ASO-taudista kärsiville tulisi vähintään aloittaa aina statiini ja ASA. Suurentuneen vuotoriskin vuoksi ei preventiomielessä aloiteta ASA, jos potilaalla on jo käytössä AK tai muu antitrombootti muista syistä.

c: ASO-taudin hoito-ohje on aina \textbf{``stop smoking, start walking''.} Tämä pätee myös kajoavasti hoidettujen kanssa.

\end{solution}

\section{Potilastapaus}\label{potilastapaus-36}

56-vuotiaalla potilaalla on kivespussi suurentunut. Toteat transilluminaatiossa oikealla puolella kivespussissa selkeän vaaleanpunaisen/oranssisen löydöksen. Potilas ei bakteerikammoisena halua erikoissairaanhoitoon vaan toivoo hoitoa terveyskeskuksessa. Hoidan hänet seuraavasti?

\begin{itemize}
\tightlist
\item
  \begin{enumerate}
  \def\labelenumi{\alph{enumi}.}
  \tightlist
  \item
    punktoin mahdollisen nesteen kivespussista
  \end{enumerate}
\item
  \begin{enumerate}
  \def\labelenumi{\alph{enumi}.}
  \setcounter{enumi}{1}
  \tightlist
  \item
    punktoin mahdollisen nesteen kivespussista ja laitan tilalle puudutusainetta
  \end{enumerate}
\item
  \begin{enumerate}
  \def\labelenumi{\alph{enumi}.}
  \setcounter{enumi}{2}
  \tightlist
  \item
    lähetän potilaan kivespussin ultraäänitutkimukseen kivessyövän poissulkemiseksi
  \end{enumerate}
\item
  \begin{enumerate}
  \def\labelenumi{\alph{enumi}.}
  \setcounter{enumi}{3}
  \tightlist
  \item
    punktoin mahdollisen nesteen kivespussista ja laitan tilalle puudutusaineen ja etoksisklerolin sekoitusta
  \end{enumerate}
\end{itemize}

\begin{solution}
\leavevmode

Vastaus

\begin{verbatim}
 d
\end{verbatim}

Potilaalla on todennäköisesti hydroseele eli vesikives, jossa tunica vaginaliksen eli kiveksen tuppikalvon viskeraali- ja parietaalikalvojen väliin on kehittynyt nestekertymä. Kyseessä ei ole pahanlaatuinen muutos, mutta se tulee tunnistaa ja osaa hoitaa tarvittaessa. Sen diagnoosiin ei tarvitse ultraääntä (ellei ole epäselvyyksiä) ja voidaan diagnosoida palpaatiolla ja transilluminaatiolla (valaisemalla kivespussi valolla ja tarkistelemalla, miten valo kulkee sen läpi).

Oireetonta hydroseeleä ei välttämättä tarvitse hoitaa, mutta mikäli vesikiveksestä aiheutuu oireita, hoito on perusteltua.

Ensisijainen hoitokeino pienessä (\textless2.5dl) nestekertymässä on punktio ja skleroterapia. Pelkkä nesteen poisto punktoimalla ei ole tehokasta, sillä neste palautuu nopeasti. Tämän vuoksi punktioon tulee yhdistää skleroterapia. Skleroterapiassa tyhjennyksen jälkeen tunica vaginaliksen sisään ruiskutetaan lääkeainetta (esim. polidokanoli+lidokaiini), joka aiheuttaa tunica vaginaliksen kiinnittymisen kivekseen ja estää näin vesikiveksen uusiutumisen. Jotta liima voisi pitää, kivespussia ulkoisesti tukevien housujen käyttö kuuluu keskeisesti välittömään jälkihoitoon.

Kookkaan tai toistetun skleroterapian jälkeen uusiutuvan vesikiveksen hoito on kirurginen. Tällöin tunica vaginalis avataan ja nesteontelo tyhjennetään. Tämän jälkeen tunica vaginalis voidaan poistaa lisäkiveksen reunojen myötäisesti, kääntää lisäkiveksen taakse (Winkelmanin leikkaus) tai rypyttää ompeleilla (Lordin leikkaus).

\pandocbounded{\includegraphics[keepaspectratio]{images/vesikives.png}}
\pandocbounded{\includegraphics[keepaspectratio]{images/kivespussinpatithoito.png}}

\end{solution}

\section{Olet keskussairaalan päivystyksessä kirurgian eval-sijaisena, ja ensihoitohuoneeseen tuodaan huonokuntoinen monivammapotilas. Minkä seuraavista tutkit ensimmäiseksi:}\label{olet-keskussairaalan-puxe4ivystyksessuxe4-kirurgian-eval-sijaisena-ja-ensihoitohuoneeseen-tuodaan-huonokuntoinen-monivammapotilas.-minkuxe4-seuraavista-tutkit-ensimmuxe4iseksi}

\begin{itemize}
\tightlist
\item
  \begin{enumerate}
  \def\labelenumi{\alph{enumi}.}
  \tightlist
  \item
    thoraxin
  \end{enumerate}
\item
  \begin{enumerate}
  \def\labelenumi{\alph{enumi}.}
  \setcounter{enumi}{1}
  \tightlist
  \item
    abdomenin ja lantion
  \end{enumerate}
\item
  \begin{enumerate}
  \def\labelenumi{\alph{enumi}.}
  \setcounter{enumi}{2}
  \tightlist
  \item
    kaularangan
  \end{enumerate}
\item
  \begin{enumerate}
  \def\labelenumi{\alph{enumi}.}
  \setcounter{enumi}{3}
  \tightlist
  \item
    neurologisen statuksen
  \end{enumerate}
\end{itemize}

\begin{solution}
\leavevmode

Vastaus

\begin{verbatim}
 a
\end{verbatim}

ABCDE-periaatetta miettien: Airway, Breathing \ldots{} -\textgreater{} näistä vaihtoehdoista thorax ensimmäisenä

Kaularanka myös tietysti tärkeä ja siihen usein on jo valmiiksi asetettu kauluri traumapotilaille.

Abdomenin ja lantion alueella voi olla verenvuotoja. Abdomenin vuodot FASTilla, lantion vuodot voi näkyä hematoomina ja kannattaa epäillä lantiomurtumissa.

Neurologinen status kuuluu myös asiaan, mutta ei ole ensimmäinen asia, joka huolestuttaa.

\end{solution}

\chapter{2022 (Cupidus)}\label{cupidus}

Suurin osa uusia kysymyksiä, mutta muutama jo aikaisemmissa tärpeissä ollut jätetty nyt tässä mainitsematta. Tästä vuodesta eteenpäin ei ole wikissä tarjolla suurimmassa osassa kysymyksistä tarkkaa kysymyksenasettelua eikä varsinkaan vaihtoehtoja, jonka takia olen joutunut useassa paikassa vain kirjoittamaan aiheesta yleisesti.

\section{Lymfödeema -- tärkein hoito?}\label{lymfuxf6deema-tuxe4rkein-hoito}

Ei vaihtoehtoja wikissä, mutta koita vastata ilman vinkkejä: Mikä on siis lymfödeeman hoidon kulmakivi?

\begin{solution}
\leavevmode

Vastaus

\begin{verbatim}
 Kompressiohoito
\end{verbatim}

Lievässä turvotusoireistossa hoito voidaan aloittaa suoraan kompressiohihalla tai -sukalla.

Mikäli raajassa on huomattava nesteylimäärä, voidaan turvotusta vähentää manuaalisella lymfakäsittelyllä ennen yksilöllisen kompressiohihan tai -sukan mittaamista.

Kompressiotekstiilit tulee uusia riittävän usein, alaraajassa noin 3 kuukauden ja yläraajassa 6 kuukauden välein riittävän kompression takaamiseksi.

Hoidon tavoitteena on estää nesteen kertyminen kudokseen ja pysyvien kudosmuutosten syntyminen. Ihon kunnosta huolehtiminen sekä liikunta ja ylipainoisilla potilailla lisäksi painonhallinta ovat tärkeitä potilaan itsehoidossa.

\textbf{Tietoa lymfödeemasta}

Lymfödeemalla eli imunesteturvotuksella tarkoitetaan imunesteen epänormaalia kertymistä ihonalaiskudokseen imunestekierron häiriön seurauksena. Normaali imunestekierto voi häiriintyä leikkauksen, sädehoidon, infektion tai vamman takia (sekundaarinen imunesteturvotus). Mikäli imutiet tai -solmukkeet ovat vaurioituneet, proteiinipitoinen imuneste kertyy kudoksiin. Imunesteturvotus voi johtua myös synnynnäisestä imusuoniston poikkeavuudesta (primaarinen imunesteturvotus).

Länsimaissa imunesteturvotuksen tavallisin syy on syövän takia tehty imusolmukkeiden poisto ja mahdollinen sädehoito (maailmanlaajuisesti yleisin syy on filariaasi eli rihmamadon (tavallisimmin Wuchereria bancrofti) aiheuttama parasiitti-infektio imuteissä). Esimerkiksi rintasyöpäpotilaista, joille on tehty kainalon imusolmukkeiden poisto, 20--40 prosentille kehittyy yläraajan imunesteturvotus. Rintasyöpään liittyvä imunesteturvotus kehittyy tavallisimmin parin ensimmäisen vuoden aikana, mutta se voi ilmaantua myös vuosien viiveellä. Alaraajan imunesteturvotukselle altistavat erityisesti gynekologisten ja urologisten syöpien sekä melanooman vuoksi tehty lantion ja/tai nivusalueen imusolmukkeiden poisto ja sädehoito.

Taudin alkuvaiheessa imunesteturvotukselle on tyypillistä ihon painamistestillä todennettava kuoppaturvotus. Taudin edetessä turvotus muuttuu fibrotisoitumisen ja rasvakertymän myötä kiinteämmäksi eikä kuoppaturvotusta enää saada esiin. Imunesteturvotuksen diagnoosi varmistetaan imuteiden gammakuvauksella (lymfoskintigrafia).

\ldots..

Imunesteturvotuksen kirurgista hoitoa voidaan harkita, mikäli konservatiivisella hoidolla ei saavuteta oireiden riittävää hallintaa.

Kirurgiset hoitovaihtoehdot voidaan jakaa kahteen kategoriaan: massanpoistoleikkaukset ja imuteiden toimintaa korjaavat leikkaukset, joita voidaan myös tapauskohtaisesti yhdistellä. Imuteiden toimintaa korjaavista leikkauksista on viime vuosina kertynyt näyttöä jo isoista potilasaineistoista myös pitkäaikaistulosten osalta.

Massanpoistoleikkaukset: Imunesteturvotuksen myöhäisvaiheessa kudoksiin kertyy ylimääräistä rasvaa ja sidekudosta, jota voidaan poistaa rasvaimulla. Rasvaimulla voidaan saavuttaa huomattava raajan keveneminen ja toimintakyvyn paraneminen. Pysyvä hoitotulos edellyttää potilaalta sitoutumista kompressiohoitoon myös leikkauksen jälkeen. Erittäin vaikeissa ja pitkälle edenneissä tilanteissa voidaan tehdä vaurioituneen ihoalueen radikaali poisto joko pitkittäissuuntaisesti tai tarvittaessa korvata ihoalue ihosiirteellä.

Uusia korjaavia hoitomuohoa ovat siis mm. mikrovaskulaarinen imusolmukesiirto ja lymfa-venööttinen anastomoosi (LVA) eli imuteiden yhdistäminen laskimoon; voidaan harkita potilaille, joilla imunesteturvotus ei ole vielä edennyt fibroottiseksi.

\pandocbounded{\includegraphics[keepaspectratio]{images/lymfödeema.png}}
\pandocbounded{\includegraphics[keepaspectratio]{images/lymfedeemakaavio.png}}

\end{solution}

\section{Ruokatorvisyöpä, mikä on totta insidenssin suhteen}\label{ruokatorvisyuxf6puxe4-mikuxe4-on-totta-insidenssin-suhteen}

\begin{itemize}
\tightlist
\item
  \begin{enumerate}
  \def\labelenumi{\alph{enumi}.}
  \tightlist
  \item
    Adenokarsinooma on yleistynyt ja levyepiteelikarsinooma yleistynyt
  \end{enumerate}
\item
  \begin{enumerate}
  \def\labelenumi{\alph{enumi}.}
  \setcounter{enumi}{1}
  \tightlist
  \item
    Adenokarsinooma on vähentynyt ja levyepiteelikarsinooma vähentynyt
  \end{enumerate}
\item
  \begin{enumerate}
  \def\labelenumi{\alph{enumi}.}
  \setcounter{enumi}{2}
  \tightlist
  \item
    Levyepiteelikarsinooma on yleistynyt ja adenokarsinooma vähentynyt
  \end{enumerate}
\item
  \begin{enumerate}
  \def\labelenumi{\alph{enumi}.}
  \setcounter{enumi}{3}
  \tightlist
  \item
    Levyepiteelikarsinooma on vähentynyt ja adenokarsinooma yleistynyt
  \end{enumerate}
\end{itemize}

\begin{solution}
\leavevmode

Vastaus

\begin{verbatim}
 d
\end{verbatim}

Vuonna 2020 Suomessa todettiin 361 uutta ruokatorven syöpää. Taudin kaksi yleisintä histologista tyyppiä ovat levyepiteelikarsinooma ja adenokarsinooma.

Levyepiteelikarsinooma on maailmanlaajuisesti yleisin tyyppi. Länsimaissa levyepiteelikarsinooman tärkeimmät riskitekijät ovat tupakointi, alkoholin suurkulutus ja etenkin näiden runsas yhteiskäyttö. Tupakoinnin vähentyessä levyepiteelikarsinooman insidenssi on vähentynyt ja ei enää ole länsimaissa yleisin esofagussyövän tyyppi.

Suomessa nykyään yleisin ruokatorven syövän tyyppi on adenokarsinooma. Adenokarsinooman tärkein riskitekijä on pitkään jatkunut refluksitauti.

\pandocbounded{\includegraphics[keepaspectratio]{images/ruokatorvisyöpäinsidenssit.png}}

\end{solution}

\section{Abdominal plastia: lihavuus pannikulektomia}\label{abdominal-plastia-lihavuus-pannikulektomia}

Kysymyksenasettelu jättää tärpin hyvin epäselkeäksi, joten tässä abdominaaliplastiasta ja pannikulektomiasta tärkeimmät:

Merkittävä painonnousu venyttää ihoa ja ihonalaiskudoksia. Massiivisen painonpudotuksen myötä venyttyneet kudokset eivät palaudu entiselleen, vaan jäävät roikkumaan tyhjinä ihopoimuina. Tyhjät ihopoimut saattavat aiheuttaa terveydellisiä, toiminnallisia sekä psykososiaalisia oireita. Poimuissa voi esiintyä hautumia, haavaumia sekä ihon infektioita. Painavat poimut hankaloittavat usein liikkumista ja vaikeuttavat päivittäisen hygienian ylläpitoa.

\begin{itemize}
\tightlist
\item
  Lisätyllä liikunnalla tai painonpudotuksella ei ole merkittävää vaikutusta venyttyneen ihon häviämiseen. Massiivisen laihtumisen jälkeinen vartalonmuovauskirurgia (post-bariatrinen kirurgia) on ainoa tehokas hoitokeino tyhjien ihopoimujen poistamiseksi. Suomen julkisessa terveydenhuollossa tehdään pääsääntöisesti vatsan, kylkien ja alaselän muovaustoimenpiteitä (abdominoplastia/bodylift). Muut vartalonmuovaustoimenpiteet tehdään yksityisellä sektorilla, poikkeuksena merkittävää terveydellistä haittaa tai esimerkiksi ammatinharjoittamista rajoittavien ihopoimujen korjaaminen. Pelkkä esteettinen haitta ei oikeuta pääsyä toimenpiteeseen julkisessa terveydenhuollossa.

  \begin{itemize}
  \tightlist
  \item
    Vartalonmuovaustoimenpiteisiin on muodostettu yhtenäisiä leikkauskriteerejä, joiden tarkoituksena on taata tasa-arvoinen pääsy vartalonmuovaustoimenpiteisiin, pienentää komplikaatioriskejä sekä varmistaa hyvä esteettinen lopputulos. Kts. kriteerit alla olevasta taulukosta.
  \end{itemize}
\end{itemize}

Abdominoplastiassa eli riippuvatsan korjausleikkauksessa (``tummy tuck'') poistetaan alavatsan roikkuva ihopoimu. Tavoitteena on palauttaa vatsanpeitteiden ihon venyttymistä edeltänyt anatomia.

\begin{itemize}
\tightlist
\item
  Leikkaustekniikoita on useita, yleisimmin käytetty on kyljestä kylkeen viilto pubiksen yläpuolella ja ylimääräinen iho-ja pehmytkudos poistetaan. Napa jätetään kiinni faskiaan ja nostetaan iholle erillisestä viillosta (napaplastia). Suorien vatsalihasten erkauma (rektusdiastaasi) voidaan korjata abdominoplastian yhteydessä lihaskalvon ompelemisella (rektusplikaatio) tai käyttämällä verkkoa.

  \begin{itemize}
  \tightlist
  \item
    Abdominoplastia voidaan tehdä myös fleur de lis -tyyppisesti, jolloin perinteisen abdominoplastian lisäksi poistetaan vatsan keskilinjan ihoylimäärä erillisen pitkittäisen leikkausviillon avulla. Lopputuloksena on tällöin alavatsalla kulkevan horisontaalisen arven lisäksi keskiviiltoarpi.
  \item
    Alavartalon kohotusleikkauksessa (bodylift/ belt lipektomia, circumferential abdominoplasty) poistetaan alavatsan ihopoimun lisäksi kylkien ja alaselän alueilla olevat roikkuvat poimut (voivat pahimmillaan haitata istumista tai ulostamista).
  \end{itemize}
\end{itemize}

Jos potilaalla on kookas ja painava alavatsapoimu, joka aiheuttaa toistuvia selluliitteja, kroonista haavautumista ja imunestekiertohäiriötä tai vaikeuttaa liikkumista, voidaan potilaalle harkita \textbf{pannikulektomiaa.} \emph{Toimenpide soveltuu potilaille, jotka ovat monisairaita tai joilla on merkittävä obesiteetti, jolloin abdominoplastian leikkauskriteerit eivät painoindeksin suhteen täyty.}

\begin{itemize}
\tightlist
\item
  Toimenpiteessä riippuvatsa poistetaan amputaation tapaan, jolloin napa dislokoituu kaudaalisesti tai se poistetaan kokonaan. Tällaisessa toimenpiteessä ei huomioida vartalon muodon esteettisiä tarpeita eli kosmeettinen lopputulos on huonompi.
\item
  Harvinainen toimenpide ja tapauskohtainen arvio
\end{itemize}

\pandocbounded{\includegraphics[keepaspectratio]{images/vartalonmuovauskriteerit.png}}
\pandocbounded{\includegraphics[keepaspectratio]{images/abdominoplastiatyypit.png}}
\pandocbounded{\includegraphics[keepaspectratio]{images/pannikulektomia.png}}

\section{Mikä on penissyövän epätyypillisin kasvupaikka?}\label{mikuxe4-on-penissyuxf6vuxe4n-epuxe4tyypillisin-kasvupaikka}

Ei vaihtoehtoja, mutta koita miettiä, mikä on harvinaisin sijainti peniksessä syövälle?

\begin{solution}
\leavevmode

Vastaus

\begin{verbatim}
 Varsi
\end{verbatim}

Penissyövän tyypillisin esiintymispaikka on terska tai terskan tyven sulcus -alue. Joskus penissyöpä esiintyy ainoastaan esinahan sisälehdessä. Pelkästään peniksen varressa sijaitseva kasvain on harvinainen.

Arviolta noin 50 \% karsinoomista esiintyy terskan alueella, kolmannes esinahan alueella ja loput peniksen varressa tai multifokaalisesti.

\end{solution}

\section{Uroteelisyöpä -- mikä parhain kuvantamismuoto?}\label{uroteelisyuxf6puxe4-mikuxe4-parhain-kuvantamismuoto}

Ei vaihtoehtoja, mutta koita vastata ilman vinkkejä.

\begin{solution}
\leavevmode

Vastaus

\begin{verbatim}
 Kystoskopia
\end{verbatim}

Uroteelisyöpä esiintyy valtaosin (n.~95\%) virtsarakossa. Uroteelin syövistä ainoastaan 5--10 \% ilmaantuu ylävirtsateihin. Noin joka kuudennella ylävirtsatietuumoripotilaalla todetaan samanaikainen virtsarakkotuumori.

Yleisin uroteelisyövän oire on verivirtsaisuus, jota on ainakin 85 \%:lla rakkosyöpäpotilaista. Verivirtsaisuus voi olla makroskooppista eli silminnähtävää tai mikroskooppista eli virtsakokeen paljastamaa. Makroskooppisen hematurian mahdollinen väistyminen tai lieväasteisuus eivät poissulje pahanlaatuista uroteelikasvainta. Kun todetaan makroskooppinen verivirtsaisuus, niin yleensä jatkotutkimuksena tehdään sedimenttitutkimus (U-solut) ja virtsan irtosolututkimus (U-sytologia), joka on hyvin spesifinen (95-100\%), mutta ei negatiivisena poissulje kasvaimen mahdollisuutta. Polikliininen kystoskopia eli virtsarakon tähystys on virtsarakkosyövän perustutkimus, jossa nähdään mahdollisten tuumorien lukumäärä ja niiden koko.

Rakkotuumorin ensivaiheen hoito on virtsarakon höyläysleikkaus eli TUR-BT-toimenpide (transurethral resection of bladder tumor). Ennen höyläystä tehdään TT-urografia eli virtsateiden varjoainetehosteinen tietokonekerroskuvaus (TT), jolla voidaan tarkistaa, onko ylävirtsateiden alueella kasvaimia tai virtsankulun estettä. Jos epäillään lihakseen tunkeutuvaa tai etäpesäkkeistä syöpää, tarkistetaan vielä keuhkojen tilanne etäpesäkkeiden varalta keuhkojen TT-kuvalla.

\pandocbounded{\includegraphics[keepaspectratio]{images/kystoskopiassa.png}}

\end{solution}

\section{Kiveskipu}\label{kiveskipu}

Ei tätä tarkemmin wikissä, ei myöskään vastausta annettu, mikä voisi antaa vinkkiä siitä, mitä on tarkalleen kysytty. Tässä muutamia periaatteita akuutista kiveskivusta:

Akuutti kiveskipu johtuu yleensä kiveksen kiertymästä, kiveslisäkkeen (appendix testis) kiertymästä tai epididymiitistä

\begin{itemize}
\tightlist
\item
  Näistä tärkein tunnistaa ja hoitaa kiireellisesti on kiveskiertymä eli testistorsio. Äkillisen kiveskivun syy on siis kiertymä, kunnes toisin on osoitettu.

  \begin{itemize}
  \tightlist
  \item
    Tyypillinen potilas on lapsi tai nuori, mutta kives voi kiertyä myös aikuisilla
  \item
    Alkaa äkillisenä kipuna (voi aluksi tuntua vain alavatsalla ja siirtyä kivespussiin) sekä unilateraalisena kivespussin turvotuksena. Kives voi nousta ylöspäin ja olla horisontaalisessa asennossa, mutta alkuvaiheen tärkein löydös on usein ainoastaan \emph{todella} kipeä ja aristava kives. Cremaster-refleksi puuttuu tyypillisesti.
  \item
    Kaikututkimus epäselvissä tapauksissa, mutta jos sen saamisessa kestää, niin sitä ei tarvita (normaali doppler-uä-löydös ei myöskään poissulje torsiota), vaan potilaalle tulee tehdä välitön päivystysleikkaus, jossa kiertymä oikaistaan (detorsio) ja \emph{molemmat} kivekset kiinnitetään (taustalla oleva altistava poikkeavuus saattaa olla molemminpuolinen, joten toinenkin puoli on tarkistettava ja kiinnitettävä). Joissakin tapauksissa kiveksen kiertymä menee ohi spontaanisti tai kivespussin tutkimuksen yhteydessä. Näissäkin tapauksissa kiinnitysleikkaus on myöhemmin aiheellinen.
  \end{itemize}
\item
  Kivelisäkkeen kiertymä (appendix testis torsio) antaa samanlaisia mutta lievempiä oireita kuin itse kiveksen kiertymä.

  \begin{itemize}
  \tightlist
  \item
    Arviolta 90 \%:lla miehistä on kiveksen yläosissa tai lisäkiveksessä sikiökaudelta jäänyt muutaman millimetrin mittainen varrellinen lisäke, joka saattaa kiertyä oman vartensa ympäri ja mennä nekroosiin.
  \item
    Ennen kuin kivespussin turvotus on ehtinyt ilmaantua voi kivespussin läpi nähdä kuolioon menneen mustan kudoskappaleen. Tämä ``blue dot sign'' on lisäkkeen kiertymälle diagnostinen. Tumma alue on palpoiden arka toisin kuin muu osa kivestä.
  \item
    Lisäkkeen kiertymää ei tarvitse leikata. Yleensä erotusdiagnostiikka testistorsion suhteen on kuitenkin niin vaikea, että eksploratiivinen leikkaus tehdään varmuuden vuoksi varsinkin jos oireiden alusta on alle vuorokausi. Kuoliossa oleva lisäke tyypillisesti poistetaan, mikä helpottaa potilaan oireita ja nopeuttaa paranemista, mutta kivestä tarvitse kiinnittää kuten testistorsiossa.
  \end{itemize}
\item
  Epididymiitissä (lisäkiveksen tulehduksessa) turvotus ja aristus paikantuvat lisäkivekseen, mutta myös kives voi aristaa (epididymo-orkiitti; kiveksenkin tulehdus). Kivespussi on usein turpea, punoittava ja kuumoittava.

  \begin{itemize}
  \tightlist
  \item
    Usein yhteydessä on virtsaamisoireita: kipua, kirvelyä ja tihentynyttä tarvetta; pääasiallisesti johtuu siitä, että aiheuttajia ovat virtsatietulehdusbakteerit, seksuaalisesti aktiivisilla klamydia ja joskus gonokokki. Epididymiittiä esiintyy myös ennen seksuaalisesti aktiivista ikää (tulee siis muistaa nuorilla pojillakin) kiveskivun syynä! (Lapsilla epididymiitti aiheutuu ilmeisesti steriilin tai infektoituneen virtsan kulkeutumisesta siemenjohtimeen. Taudin uusiessa kannattaa tehdä ainakin virtsateiden kaikututkimus mm. ektooppisen ureterin varalta. Lisäksi kannattaa kiinnittää huomiota mahdolliseen kasteluun tai virtsaamisvaikeuksiin.)
  \item
    Hoitona on antibioottihoito 2vk. Kivespussia tukevat (riittävän kireät) alushousut, viileät kääreet ja tulehduskipulääkkeet lievittävät kipua.
  \end{itemize}
\end{itemize}

\pandocbounded{\includegraphics[keepaspectratio]{images/akuuttikives.png}}

\section{Ihosiirteet}\label{ihosiirteet}

Ei vaihtoehtoja tai tarkempaa kysymyksenasettelua. Tässä hieman ihosiirteistä:

Ihosiirteet luokitellaan kokoihosiirteiksi tai osaihosiirteiksi, riippuen siitä, miten suurta osaa ihon paksuudesta käytetään siirteenä.

\begin{itemize}
\tightlist
\item
  Ihon osista dermis vastaa ihon viskoelastisista ominaisuuksista ja on sen vuoksi ratkaiseva parantuneen ihosiirteen stabiliteetissa.
\item
  Kokoihosiirteessä on mukana koko epidermaalinen ja dermaalinen osuus mukaan lukien ihon apuelimet ja karvatupet.

  \begin{itemize}
  \tightlist
  \item
    Kokoihosiirteiden saatavuus on rajallinen, ja niitä käytetään yleensä rekonstruoitaessa esteettisesti (kasvot) tai funktionaalisesti (esim. kädet) tärkeitä kehonosia.
  \end{itemize}
\item
  Osaihosiirre koostuu epidermiksestä ja osapaksusta dermiksestä. Siirre voidaan ottaa joko paksuna (sisältää enemmän karvatuppia ja hikirauhasia; liittyy vähemmän arpeutumista) tai ohuena (enemmän arpeutumista, vaatii jatkossa jatkuvaa ihonhoitoa ja rasvausta perusvoiteella hiki- ja talirauhasten puuttumisen vuoksi; vähentää kyllä ottokohdan morbiditeettia, ja ottokohta paranee hyvällä paikallishoidolla 10--14 päivän kuluessa. Tarvittaessa samaa ottokohtaa voidaan käyttää uudestaan kahden viikon kuluttua siirteen ottamisesta, mikä voi olla tarpeen hoidettaessa laajoja palovammoja, joissa ihon ottokohdista on puutetta.)

  \begin{itemize}
  \tightlist
  \item
    Osaihosiirteitä käytetään laajoissa ihopuutoksissa, joissa haavan suora sulkeminen ei ole mahdollista. Tällaisia ovat palovammat, traumaattiset kudospuutokset, faskiotomiahaavat tai laajat pehmytkudosinfektioiden revisiot. Osaihosiirteitä käytetään myös mikrovaskulaaristen tai pedikulaaristen lihaskielekkeiden päälle.
  \end{itemize}
\end{itemize}

\section{Latissimus dorsi -implantaatio komplikaatio}\label{latissimus-dorsi--implantaatio-komplikaatio}

Ei tarkempaa kysymyksenasettelua tai vaihtoehtoja wikissä. Tässä rintarekonstruktiosta ja yleisimmistä komplikaatioista:

Rintarekonstruktiot esim. rintasyövän takia tehdyn mastektomian jälkeen (joko välittömästi saman leikkauksen yhteydessä tai vasta myöhäisrekonstruktiona) voidaan toteuttaa omakudosrekonstruktioina tai implanteilla tai molemmilla. Suomessa rintasyövän rintarekonstruktiot tehdään useimmiten omakudoksesta (maailmalla enemmän implanteilla).

\begin{itemize}
\tightlist
\item
  Rekonstruktiotekniikka riippuu potilaasta ja hänen ``varastoistaan''. Yleisimmät rintarekonstruktiossa käytettävät omakudosrekonstruktiot ovat:

  \begin{itemize}
  \tightlist
  \item
    mikrovaskulaariset kielekkeet (vatsa/reisi)
  \item
    varrelliset kielekkeet (selkäkieleke, ensisijaisesti latissimus dorsi -kieleke eli LD-kieleke)
  \item
    rasvansiirto
  \end{itemize}
\item
  Mikrovaskulaarisista tekniikoista yleisin nykyään on DIEP-perforanttikieleke (deep inferior epigastric perforator flap; kielekkeen suonitus perustuu syviin inferiorisiin epigastrisiin suoniin), jossa nostetaan alavatsakieleke rinnaksi. Kielekkeen nostolla ja ottokohdan sulkemisella voidaan samalla toteuttaa vartaloa muovaavaa kirurgiaa (saadaan vatsalta rasvaa pois), ottokohdan ongelmia on vähän, ja omakudosrinta on rekonstruktiovaihtoehtona potilaalle varsin luonnollinen.
\item
  Latissimus dorsista saatava volyymi (muodostuu lapaluun alapuolella olevasta iho-rasvapoimusta ja LD-lihaksesta, joka saa verisuonituksensa torakodorsaali-suonista) ei ole kovinkaan suuri, jonka takia LD-kieleke soveltuu hoikille, pienirintaisille naisille (usein paksummilla ehkä alavatsakieleke parempi, kun siellä varastoa) ja toisaalta keskivartaloltaan tukevarakenteisille, joilla toinen rinta ei ole kovin suuri. Lisävolyymiksi voidaan kielekkeen alle asettaa samassa leikkauksessa silikoniproteesi, mutta yhä useammin volyymia täydennetään vapaalla rasvansiirrolla, jolloin uusi rinta on kokonaan omakudossiirre.
\item
  Rasvansiirto harvoin riittää pelkäksi rekonstruktiokeinoksi, mutta sitä voidaan käyttää muiden tekniikoiden täydentämisessä.
\end{itemize}

\textbf{Komplikaatioista:}

\begin{itemize}
\tightlist
\item
  Laajan rintaleikkauksen yleisin postoperatiivinen komplikaatio on serooma eli kudosnestekertymä leikkaushaavaaonteloon.

  \begin{itemize}
  \tightlist
  \item
    Seroomaeritys on voimakkainta heti leikkauksen jälkeisinä päivinä, ja siksi leikkausalueelle laitetaan tavallisesti laskuputki eli dreeni noin viikon ajaksi
  \item
    Mikäli serooman erittyminen on runsasta vielä laskuputken poistamisen jälkeen, sitä on poistettava leikkausalueelta punktoimalla. Mikäli serooma on väriltään kirkkaan kellertävää, ei punktion yhteydessä ole tarvetta aloittaa antibioottia tai tarkistaa tulehdusarvoja.
  \end{itemize}
\item
  Leikkauksenjälkeistä hematoomaa voi tarvittaessa punktoida, mutta paras ensihoito on kylmä + kompressio. Jos pingottava, niin evakuaatio → kirurgille.
\item
  Rintasyöpäleikkauksissa ja \textbf{varsinkin LD-kielekkeiden teon aikana voidaan vaurioittaa n.~thoracicus longusta, joka aiheuttaa lapaluun siirotusta (enkelinsiipideformaatio)}
\item
  Haavan paranemisen ongelmat
\item
  Haavareunan nekroosi -\textgreater{} konsultoi kirurgia (Hoitona revisioleikkaus, mahd. VAC (vacuum assisted closure). Myöhemmin kielekkeen uudelleen muotoilu ja/tai ihonsiirto tai rasvansiirto)
\item
  Jos kosmeettisesti ruma lopputulos ja potilasta häiritsee tämä, niin voi tehdä lähetteen plastiikkakirurgille (rasvansiirto ja arven vapauttelu yms mahdollista).
\end{itemize}

\pandocbounded{\includegraphics[keepaspectratio]{images/diep.png}}
\pandocbounded{\includegraphics[keepaspectratio]{images/ld.png}}
\pandocbounded{\includegraphics[keepaspectratio]{images/siirotus.png}}

\section{10 cm kova suoni}\label{cm-kova-suoni}

Ei tarkempaa kysymyksenasettelua tai vaihtoehtoja. Todennäköisesti kyseessä todettu 10cm pitkä pintalaskimotukos ja kysytty, mitä tehdään.

\begin{itemize}
\tightlist
\item
  Pintalaskimotukos eli tromboflebiitti syntyy tavallisimmin kohjuiseen laskimoon, joka on altistunut mekaaniselle ärsytykselle. Suoneen kehittyy nopeasti akuutti ei-bakteeriperäinen tulehdus, joka saattaa levitä pinnallista päärunkoa myöten jopa safenofemoraaliseen tai safenopopliteaaliseen junktioon ja syvään systeemiin saakka.
\item
  Syvä laskimotukos komplisoi tilannetta lähes 20 \%:lla ja keuhkoembolia puolestaan noin 5 \%:lla potilaista, taustalla oleva hyytymishäiriö lisää tätä alttiutta. Terveeseen laskimoon spontaanisti kehittyvä migroiva flebiitti on usein paraneoplastinen ilmiö, jonka taustalta voi löytyä maligniteetti tai hematologinen sairaus.
\item
  Tromboflebiitti diagnosoidaan kliinisesti. Havaitaan aristava, punoittava ja kuumottava resistenssi. Mikäli tukos on edennyt reiden yläosaan tai lähelle polvitaivetta, se on selvästi pitkittynyt tai potilaalla on tiedossa oleva hyytymishäiriö, syvä laskimotukos kannattaa sulkea pois ultraäänitutkimuksella.

  \begin{itemize}
  \tightlist
  \item
    Erotusdiagnostiikassa keskeistä on sulkea pois bakteeritulehduksen (ruusu eli erysipelas) mahdollisuus.
  \end{itemize}
\item
  Syvä laskimotukos ja keuhkoembolia ovat pinnallisen laskimotukoksen mahdollisia komplikaatioita. Laaja-alainen tai safeenalaskimoiden tyven seutuun ulottuva tromboflebiitti saattaa edetä syvään laskimoverkostoon tai keuhkoemboliaksi. Epäilyttävissä tapauksissa potilas lähetetään päivystyspoliklinikkaan arvioon.
\end{itemize}

\textbf{Akuutin pinnallisen laskimotukoksen hoidossa käytetään oraalisia antikoagulantteja (ensisijainen) tai pienimolekyylistä hepariinia (LMWH).}

\begin{itemize}
\tightlist
\item
  \textbf{Verenohennuslääkitys rajataan tapauksiin, joissa tukos on laaja-alainen tai se on edennyt päärunkoihin, isoon tai lyhyeen kehräslaskimoon yli 5 cm:n matkalle.}

  \begin{itemize}
  \tightlist
  \item
    Potilaan 10cm tukoksessa hoito on siis AK-hoito, koska muutos on yli 5cm
  \item
    AK-hoidon kesto on 6vk ja annoksena tromboosiprofylaksia-annos (esim Xarelto 10mg 1x1 (ensisijainen) tai jos vuotoriski korostunut niin Eliquis 2,5mg 1x2 tai jos LMWH-hoito on indikoitua (esim. hyytymishäiriö tai aktiivinen syöpäsairaus tai jos on raskaana) niin enoksapariini 40mg 1x1 riippuen potilaan painosta ja munuaistoiminnasta)
  \end{itemize}
\item
  Taudin oireita ja leviämistä kannattaa yrittää lievittää kevyellä lääkinnällisellä hoitosukalla ja anti-inflammatorisilla analgeeteilla. Mikrobilääkettä ei tarvita.
\item
  Laaja-alaisesta pintalaskimotukoksesta tai toistuvista pienemmistäkin tukoksista kärsivä henkilö ohjataan akuuttivaiheen jälkeen erikoissairaanhoitoon, jotta taustalla mahdollisesti oleva laskimovajaatoiminta tulee hoidetuksi. Kajoavaa hoitoa ei suositella akuutissa vaiheessa.
\end{itemize}

\pandocbounded{\includegraphics[keepaspectratio]{images/tromboflebiittialgoritmi.png}}

\section{Kilpirauhasen kysta yli 3 cm, painaa -- mikä hoitomuoto?}\label{kilpirauhasen-kysta-yli-3-cm-painaa-mikuxe4-hoitomuoto}

Ei vaihtoehtoja, mutta koita vastata ilman vinkkejä

\begin{solution}
\leavevmode

Vastaus

\begin{verbatim}
 Aspiraatio
\end{verbatim}

Paineoireita aiheuttava kysta voidaan tyhjentää aspiraatiolla. Toimenpideradiologi tyhjentää kystan ja täyttää ontelon joko etanolilla/polidokanolilla (pelkän tyhjennyksen jälkeen tilanne uusii todella usein ja nopeasti). Skleroterapia ei tosin poista uusiutumisriskiä. Tarkoitus on hoitaa oireettomaksi, ei hävittää kystaa kokonaan.

Leikkaus voi joskus olla tarpeen, jos kysta oireilee (toistetusta) skleroterapiasta riippumatta jatkuvasti täyttymällä uudestaan. Voidaan koittaa aluksi toimenpiteenä radiofrekvenssiohoitoa. Teho on paras yksilokeroisissa puhtaasti kystisissä muutoksissa: suuret kystat voivat tarvita useampia käsittelyitä.

Kystat ovat hyvin yleisiä. Erilaisia kilpirauhaskyhmyjä on uä:llä tutkittuna ad 75\% ihmisistä, naisilla enemmän. Näistä kystisiä on 15-25\% ja benignejä 95\%.

\end{solution}

\section{Mikä on yleisin kilpirauhaskarsinooman tyyppi?}\label{mikuxe4-on-yleisin-kilpirauhaskarsinooman-tyyppi}

Ei vaihtoehtoja, mutta koita vastata ilman vinkkejä

\begin{solution}
\leavevmode

Vastaus

\begin{verbatim}
 Papillaarinen (80-90%)
\end{verbatim}

Syövät jaetaan erilaistuneeseen kilpirauhassyöpään (DTC, differentiated thyroid cancer, papillaarinen ja follikulaarinen karsinooma), medullaariseen ja huonosti erilaistuneeseen anaplastiseen syöpään. Erilaistuneen kilpirauhassyövän tyypillisin alaluokka on papillaarinen karsinooma (PTC). Harvinaisena löydöksenä kilpirauhasessa voidaan todeta lymfooma tai etäpesäke.

PTC (papillary thyroid carcinoma) on tyypillinen nuorilla naisilla (30-40v). Tärkein riskitekijä on ionisoiva säteily. Kasvaa hitaasti, käyttäytyy yleensä varsin ''hyvänlaatuisesti'' ja 5v-ennuste \textgreater99.5\% (joissain lähteissä 97\%). Voi levitä kaulan alueen imusolmukkeisiin, mutta etäpesäke ei yleensä heikennä ennustetta kovinkaan paljoa

Joissakin tilanteissa hoidoksi voi riittää pelkästään toisen lohkon poisto (riippuu mm. kasvaimen koosta -- ei tarvitse tietää tarkemmin).

\pandocbounded{\includegraphics[keepaspectratio]{images/kilpirauhassyövät.png}}

\end{solution}

\section{Post barronisaatio}\label{post-barronisaatio}

Ei vaihtoehtoja tai tarkempaa kysymyksenasettelua wikissä. Tässä hieman barronisaatiosta:

Barronisaatio viittaa sisäisten peräpukamien hoitokeinoon, jossa peräpukamat hoidetaan ligeeramalla ne kumilenkeillä. Se on nykyään eniten käytetty toimenpide I-III asteen sisäisten pukamien hoitoon. Oireettomia peräpukamia ei tarvitse hoitaa.

\begin{itemize}
\tightlist
\item
  Peräpukamat siis jaetaan sisäisiin ja ulkoisiin sen mukaan ovatko ne lähtöisin linea dentatan yläpuolelta (sisäiset) vai alapuolelta (ulkoiset).

  \begin{itemize}
  \tightlist
  \item
    Linea dentata on tärkeä kohta anaalikanavassa:

    \begin{itemize}
    \tightlist
    \item
      sen yläpuolella laskimoverenkierto kulkeutuu porttilaskimojärjestelmään ja alapuolella systeemiverenkierron puolelle
    \item
      alapuolella somaattinen hermotus ja yläpuolella viskeraalinen hermotus
    \end{itemize}
  \item
    Koska vasta alapuolella on somaattinen hermotus, niin ulkoiset peräpukamat (linean alapuolella) ovat tyypillisesti kivuliaampia ja sisäiset peräpukamat (linean yläpuolella) ovat tyypillisesti vähemmän kivuliaita
  \end{itemize}
\item
  Sisäiset pukamat luokitellaan niiden ulosluiskahtamistaipumuksen mukaan neljään asteeseen:

  \begin{itemize}
  \tightlist
  \item
    Aste 1 = Eivät luiskahda esiin ponnistaessa, mutta voivat vuotaa verta
  \item
    Aste 2 = Luiskahtavat esiin ponnistaessa, mutta vetäytyvät spontaanisti peräaukkokanavaan
  \item
    Aste 3 = Luiskahtavat esiin ponnistettaessa, mutta ovat työnnettävissä takaisin peräaukkokanavaan
  \item
    Aste 4 = Sijaitsevat peräaukon ulkopuolella, eivätkä ole työnnettävissä takaisin peräaukkokanavaan (pysyvästi prolaboitunut)
  \item
    Pukamien ulos luiskahtamista voidaan tarkastella pyytämällä potilasta ponnistamaan proktoskooppia ulos vedettäessä
  \end{itemize}
\end{itemize}

Barronisaatiota ei oikein käytetä ulkoisiin peräpukamiin, koska ulkoisia pukamia hermottaa somaattiset hermot -\textgreater{} ligaatio ja sen aiheuttama kuolio aiheuttaisi massiivista kipua. Ulkoisten peräpukamien ensisijainen toimenpiteellinen hoito on kirurginen poisto (hemorroidektomia), jos kajoavaa hoitoa tarvitaan.

\begin{itemize}
\tightlist
\item
  Jos potilas on kivulias heti peräpukamien ligeeraushoidon jälkeen, on silikonilenkki todennäköisesti liian lähellä iho-limakalvorajaa (linea dentata) ja se kannattaa poistaa katkaisemalla kapeakärkisillä saksilla.
\item
  IV asteen eli pysyvästi ulos luiskahtaneen sisäisen pukaman hoitoon ligeeraus ei myöskään sovellu (leikkaushoitoa suositellaan oireileviin IV asteen ja osaan III asteen peräpukamista), koska ligatuurat eivät riitä vetämään pysyvästi peräaukon ulkopuolelle luiskahtanutta pukamaa takaisin sisään.
\end{itemize}

Samalla kertaa voidaan ligeerata 3 pukamaa. Hoito voidaan uusia 3--4 kertaa n.~kuukauden välein, jos oireisia pukamia on vielä jäljellä (n.~50 \%:lle riittää yksi hoitokerta). Jos vaivoja on vielä tämän jälkeen, on harkittava leikkaushoitoa.

\begin{itemize}
\tightlist
\item
  Antikoagulaatiohoitoa ei tarvitse tauottaa peräpukamien barronisaatiota edeltävästi.
\item
  On pieni riski verenvuodolle erityisesti jos käytössä on antikoagulantti, mutta AK-hoitoa ei pääsääntöisesti tarvitse tauottaa. Lenkin irrotessa voi myös tulla vähän verenvuotoa, mutta runsas vuoto on harvinaista.
\end{itemize}

Mahdollisia komplikaatioita ovat siis mm. lievä verenvuoto ja infektiot sekä kipu. Ei tyypillisesti aiheuta inkontinenssia toisin kuin monet muut operatiiviset hoidot anaalikanavan alueella.

\begin{itemize}
\tightlist
\item
  Ligatuurat eivät yleensä aiheuta sairausloman tarvetta.
\end{itemize}

\pandocbounded{\includegraphics[keepaspectratio]{images/barronisaatio.png}}
\pandocbounded{\includegraphics[keepaspectratio]{images/hemorrhoids.png}}

\section{Bariatric surgery}\label{bariatric-surgery}

Ei vaihtoehtoja tai tarkempaa kysymyksenasettelua. Tässä jotain luettavaa:

Lihavuuskirurgiaan eli bariatriseen kirurgiaan sisältyy useita vatsaelinkirurgisia leikkauksia, joiden tavoitteena on vaikeasti lihavan potilaan hyvä ja pysyvä painonlasku pitkäaikaisseurannassa

\begin{itemize}
\tightlist
\item
  Vaikean lihavuuden hoidossa lihavuuskirurgia on ainoa hoitomuoto, jolla osoitetusti voidaan aikaansaada hyvä ja pysyvä painonlasku pitkäaikaisseurannassa
\end{itemize}

Kirurgista hoitoa harkitaan aina yksilöllisesti, mutta lihavuuskirurgialle on asetettu aiheet ja vasta-aiheet, jotka ovat Käypä Hoitoon kirjoitettuja.

\begin{itemize}
\tightlist
\item
  Leikkaushoidon edellytyksenä on asianmukainen aikaisempi konservatiivinen (ei-kirurginen) hoito

  \begin{itemize}
  \tightlist
  \item
    Käypä hoito -työryhmän harkitsema sovelias konservatiivinen hoito on esimerkiksi seuraava: terveydenhuollon toimintayksikön toteuttama noin 6 kuukauden yksilö-, ryhmä- tai internetpohjainen hoito, joka on johtanut elämäntapamuutoksiin ja \textbf{noin 5 \%:n suuruiseen laihtumiseen mutta jonka tulos ei ole ollut riittävä terveyden kannalta tai jonka jälkeen paino on noussut uudestaan.} Kriteerin tarkoituksena on varmistaa potilaan sitoutuminen hoitoon sekä leikkauksen edellyttämiin ruoka- ja liikuntatottumusten muutoksiin. Hoidosta ei saisi olla kulunut yli 5 vuotta.

    \begin{itemize}
    \tightlist
    \item
      5\% muutosta tärkeämpää on kuitenkin se, että potilas on osoittanut pystyvänsä tekemään elintapamuutoksia eikä laihtumistulos ole pelkästään esimerkiksi ENE-dieettiin liittyvä painon väheneminen
    \item
      \textbf{Konservatiivinen hoito toteutetaan ensisijaisesti perusterveydenhuollossa ennen erikoissairaanhoitoon lähettämistä}
    \end{itemize}
  \end{itemize}
\end{itemize}

Leikkaushoidon painoindeksirajat:

\begin{itemize}
\tightlist
\item
  yli 40 kg/m2 tai
\item
  yli 35 kg/m2 ja potilaalla on lihavuuden liitännäissairaus tai sen vaaratekijöitä, kuten tyypin 2 diabetes, hypertensio, uniapnea, kantavien nivelten nivelrikko, munasarjojen monirakkulatauti (PCOS), tai muu sairaus, jonka voidaan olettaa lievittyvän lihavuusleikkauksella (esim. refluksitauti, kun potilaalle suunnitellaan mahalaukun ohitusleikkausta)

  \begin{itemize}
  \tightlist
  \item
    Maailmalla monessa paikkaa BMI yli 35 on jo leikkausaihe yksinään ilman liitännäissairauksia
  \item
    Tyypin 2 diabetesta sairastavalla voidaan lisäksi harkita leikkausta jo painoindeksillä 30--35 kg/m2, jos lihavuuden ja diabeteksen konservatiivinen hoito ei ole tuottanut riittävää tulosta, sillä satunnaistetuissa tutkimuksissa näillä potilailla tulokset ovat olleet vastaavia kuin potilailla, joiden painoindeksi on yli 35 kg/m2
  \item
    Painoindeksinä käytetään leikkausharkintaa edeltävää painoindeksiä. Jos potilas laihduttaa tämän jälkeen niin, että painoindeksi on ennen leikkausta pienempi kuin edellä mainitut rajat, leikkaus voidaan tehdä potilaan niin toivoessa
  \end{itemize}
\end{itemize}

\emph{Leikkauksen edellytyksenä on myös, ettei potilaalla ole päihdeongelmaa!}

\textbf{Leikkausmenetelmistä:}

Yleisimmät kaksi leikkausmenetelmää ovat mahalaukun ohitusleikkaus (gastric bypass, Roux-en-Y, RYGB) ja mahalaukun kavennusleikkaus (sleeve gastrectomy). Leikkaukset tehdään ensisijaisesti laparoskooppisesti.

\begin{itemize}
\tightlist
\item
  Nämä ovat usein hyvinkin vertailukelpoisia, mutta molemmilla on hyvät ja huonot puolet (taulukossa alimmassa kuvassa)
\end{itemize}

\pandocbounded{\includegraphics[keepaspectratio]{images/lihavuuskirurgiakriteerit.png}}
\pandocbounded{\includegraphics[keepaspectratio]{images/lihavuusleikkaustekniikat.png}}
\pandocbounded{\includegraphics[keepaspectratio]{images/sleevepass.png}}

\section{Crohnin tauti ja leikkaus eli mikä on tehty}\label{crohnin-tauti-ja-leikkaus-eli-mikuxe4-on-tehty}

Ei vaihtoehtoja wikissä. Tässä pari periaatetta tulehduksellisten suolistosairauksien (IBD) leikkaushoidosta:

Tärkein asia muistaa tulehduksellisten suolistosairauksien kahdesta päätyypistä (Crohnin tauti ja haavainen paksusuolitulehdus eli colitis ulcerosa) on se, että mitä osaa suolistosta ne affisioivat

\begin{itemize}
\tightlist
\item
  Colitis ulcerosa affisioi vain paksusuolta ja vaikuttaa siihen jatkuvaan tyyliin siten, että tauti lähtee rektumista ja voi levitä proksimaalisesti cecumiin asti.
\item
  Crohnin tauti taas voi affisioida koko GI-kanavaa suusta anukseen ja leesio ei ole yhtenäinen, vaan Crohnin taudin ns. ``skip-leesioita'' voi ilmentyä yhdessä osassa suolta ja sitten pitkän terven suolen osan jälkeen toisessa osassa suolta.
\end{itemize}

Tämän tärkeän eron takia tautien leikkaushoidon periaatteet eroavat suuresti:

\begin{itemize}
\tightlist
\item
  Koska haavainen koliitti voi affisioida vain paksusuolta, niin taudin kannalta parantava lopputulos saavutetaan, jos koko paksusuoli poistetaan eli potilaalle tehdään proktokolektomia -\textgreater{} ei enää ole paksusuolta, jota tauti voisi affisioida

  \begin{itemize}
  \tightlist
  \item
    Haavaisen koliitin leikkausaiheet jaetaan kolmeen pääryhmään: fulminantti koliitti, krooninen lääkehoitoon huonosti reagoiva koliitti ja koliittiin liittyvä syöpä tai sen esiaste. Krooninen lääkehoitoon reagoimaton koliitti on tavallisin leikkausaihe (65 \%).
  \item
    Nykyisin selvästi yleisin leikkausmenetelmä on proktokolektomia, johon yhdistetään ileumsäiliö ja ileoanaalinen liitos (IPAA, ileal pouch anal anastomosis). Tavallisimmin käytetty säiliö on J:n muotoinen eli ns. J-pussi. Tarvittaessa leikkauksessa tehdään väliaikainen ohutsuolen lenkkiavanne, joka suojaa tehtyä ileoanaalista liitosta. Avanne suljetaan myöhemmässä leikkauksessa, kun on ensin varmistettu tähystyksellä, että ileoanaaliliitos on parantunut.
  \end{itemize}
\item
  Crohnin taudin leikkaushoidossa minkään suolen osan poisto ei \emph{paranna} tautia, koska mikä tahansa suoliston osa voi affisioitua -\textgreater{} totaalikolektomiaan ei yleensä edetä (suurentaa riskiä taudin leviämisestä ohutsuolen puolelle) ja pyritään ''suolta säästävään'' kirurgiaan (vain affisioitunut suolen osan poisto ja plastiat ahtaumiin)

  \begin{itemize}
  \tightlist
  \item
    Leikkaushoidon indikaatioita ovat pääasiassa taudin komplikaatiot (esim. ahtauma, fisteli, verenvuoto ja anemisoituminen ja tarve jatkuville punasolusiirroille, Crohnin tautiin liittyvä syöpä tai esiaste) ja lääkitykseen reagoimaton tauti. Mahdollisesti kehittyvät perianaalipaiseet avataan välittömästi, mutta muiden Crohnin taudin perianaalimuutosten leikkaushoitoa harkitaan vain, jos ne aiheuttavat potilaalle invalidisoivia oireita. Suolen puhkeama ja siihen liittyvä vatsakalvotulehdus, kuten täydellinen suolitukos, vaativat välitöntä leikkaushoitoa. Leikkauksessa poistetaan sairas suolenosa ja tehdään suoliliitos suolen päiden välille.

    \begin{itemize}
    \tightlist
    \item
      Tavallisin Crohnin taudin ilmentymä on aivan distaalisen sykkyräsuolen (ileum) usein lyhyehkö ahtauma. Niitä esiintyy 75 \%:lla sairastuneista. Jos ahtauma aiheuttaa oireita, kuten oksentelua, kouristavia vatsakipuja ja laihtumista, tehdään ileosekaalinen suolentypistys. Suurimmalle osalle potilaista toimenpiteen voi tehdä tähystysleikkauksena.
    \item
      Noin kolmasosalla potilaista on tulehdus paksusuolen alueella. Tyypillisimmillään tulehdus on paksusuolen oikealla puolella, mutta se voi olla missä tahansa paksusuolen osassa. Jos Crohnin tauti rajoittuu paksusuolen oikeaan puoleen, voidaan tehdä oikeanpuoleinen paksusuolen typistys. Mikäli tulehtunut alue on laajempi eikä tautia ole peräsuolen alueella, tehdään subtotaali kolektomia ja ileosigmoidaalinen tai ileorektaalinen liitos. Jos myös peräsuolen alueella on vaikea tulehdus, anaalikanava on ahtautunut tai potilaalla on anaalifisteleitä, poistetaan peräsuoli ja peräaukko sekä tehdään pysyvä pääteavanne.
    \item
      Diagnoosiltaan selkeässä Crohnin taudissa ei suositella proktokolektomiaa yhdessä ileoanaaliliitoksen kanssa, koska liitosalueella taudin uusiutumisriski on suuri ja noin puolet tehdyistä liitoksista joudutaan myöhemmin purkamaan ja siirtymään pysyvään ileostomiaan.
    \end{itemize}
  \end{itemize}
\end{itemize}

\textbf{Tärkeimmät tiedot siis muistaa IBD-tautien leikkaushoidoista:}

\begin{itemize}
\tightlist
\item
  Colitis ulcerosassa leikkaushoito on kuratiivinen (CU affisioi vain paksusuolta), Crohnin taudissa ei (Crohn affisio koko suoliston aluetta)
\item
  Colitis ulcerosan tärkein leikkaustekniikka on IPAA (ileal pouch anal anastomosis) eli proktokolektomia, johon yhdistetään ileumsäiliö (tavallisimmin J-pussi) ja ileoanaalinen liitos
\item
  Crohnin taudissa leikkaushoidon periaate on se, että tarvittaessa leikataan pois vain affisioitunut osa (yleisimmin terminaalinen ileum -\textgreater{} ileosekaalinen typistys) ja ahtaumat laajennetaan (suolta säästävä kirurgia)
\item
  Leikkaushoidon tärkein indikaatio on lääkehoitoon reagoimaton tauti ja komplikaatiot (niin akuutit (esim. suolen puhkeama Crohnissa tai fulminantti koliitti colitis ulcerosassa) kuin krooniset (IBD-liittyvä syöpä tai esiaste, Crohnin taudissa ahtaumat tai fistelit\ldots))
\end{itemize}

\section{Anteriorinen resektio tehty, epäilet saumalekaasia. Mikä kuvantamistutkimus paras?}\label{anteriorinen-resektio-tehty-epuxe4ilet-saumalekaasia.-mikuxe4-kuvantamistutkimus-paras}

Ei vaihtoehtoja wikissä, mutta samankaltainen kysymys oli jo aikasemmassa tentissä. Koita vastata ilman vinkkejä.

\begin{solution}
\leavevmode

Vastaus

\begin{verbatim}
 TT ja varjoaine p.r.
\end{verbatim}

Rektumin tai sigman alaosan lekaasiin eli suolisauman pettämiseen viittaa parhaiten siis se, että per rect annettu vesiliukoinen varjoaine karkaa sauman ulkopuolelle TT-kuvassa. Jos taas lekaasiepäily olisi ylempänä GI-kanavassa (varsinkin ruokatorven/ventrikkelin alueella), niin tutkimus tehtäisiin p.o. varjoaineella. Ohutsuoli ja proksimaalinen paksusuoli ovat hankalampia kuvantaa.

Anteriorinen resektio on tyyppileikkaus peräsuolisyövässä. Toinen tyypillinen leikkaus olisi abdominoperineaalinen resektio (rectumamputaatio). Rectumamputaatiossa tehdään pysyvä avanne, koska peräsuoli ja peräaukko poistetaan täysin.

Peräsuolisyövän leikkaustavan määrittävät kasvaimen sijainti, sen paikallinen levinneisyys ja potilaan kunto. Jos kasvain ei kasva liian lähelle sulkijalihaksia, potilaalle tehdään peräsuolen anteriorinen resektio (poistetaan peräsuoli suoliliepeineen ja tehdään suoliliitos katkaistun paksusuolen pään ja peräsuolityngän välille). Anteriorinen resektio voidaan tehdä avoleikkauksena, laparoskooppisesti tai robottiavusteisesti. Merkittäviä eroja komplikaatioluvuissa ja onkologisissa tuloksissa ei tekniikkojen välillä ole. Jos kasvain kasvaa peräaukon sulkijalihaksiin tai hyvin lähelle niitä, leikkauksessa on poistettava peräsuolen ja suoliliepeen lisäksi peräaukkokanava ja sulkijalihakset ja potilaalle tehdään pysyvä paksusuoliavanne.

Ns. LARS-oireet (Low Anterior Resection Syndrome) ovat yleisiä peräsuolen anteriorisen resektion jälkeen ja näistä yleisimpiä ovat mm. ulostamisfrekvenssin nousu, ulostamispakko (urge), pidätyskyvyn ongelmat (ulosteinkontinenssi), virtsaamisen ongelmat ja erektio-ongelmat.

\pandocbounded{\includegraphics[keepaspectratio]{images/anteriorinenresektio.png}}

\end{solution}

\section{Laskimovajaatoiminnan riskitekijät}\label{laskimovajaatoiminnan-riskitekijuxe4t}

Ei vaihtoehtoja wikissä, tässä tärkeimpiä riskitekijöitä:

\begin{itemize}
\tightlist
\item
  Naissukupuoli
\item
  Raskaudet
\item
  Ylipaino
\item
  Sukurasitus
\item
  Sairastettu syvä laskimotukos
\end{itemize}

Hyvä huomata, että tupakointia ei tyypillisesti mainita tässä. Tupakointi ei siis tyypillisesti affisioi laskimopuolta niin paljoa kuin valtimopuolta ja se ei ole kovinkaan tärkeä riskitekijä laskimovajaatoiminnan suhteen (joidenkin lähteiden mukaan ei lisäisi riskiä ollenkaan).

\section{Hb 82, melena, ei verioksennusta. Missä vuoto todennäköisimmin?}\label{hb-82-melena-ei-verioksennusta.-missuxe4-vuoto-todennuxe4kuxf6isimmin}

Ei vaihtoehtoja wikissä, mutta koita vastata ilman vinkkejä.

\begin{solution}
\leavevmode

Vastaus

\begin{verbatim}
 Ylä-GI, tavallisimmin mahalaukun/duodenumin ulkus
\end{verbatim}

Akuutin suolistoverenvuodon oireita ovat verioksennus (hematemeesi), meleena eli tumma veriuloste, kirkas verenvuoto peräsuolesta (hemorrhagia ex ano), kollapsi tai anemisoituminen. Meleena on useimmiten lähtöisin ylä-GI-kanavasta ja hemorrhagia ex ano ala-GI-kanavasta.

Noin 80--90 \% ruoansulatuskanavan verenvuodoista on peräisin ruoansulatuskanavan yläosasta (Treitzin ligamentin yläpuolelta), 1--5 \% ohutsuolesta ja 15 \% paksu- ja peräsuolesta.

\pandocbounded{\includegraphics[keepaspectratio]{images/givuodottaulukko.png}}

\end{solution}

\section{Potilastapaus}\label{potilastapaus-37}

\emph{Kysymys todennäköisesti ollut jo 2022, mutta wikissä kirjoitettu vain aihe ylös. Tämä potilastapausteksti on vuoden 2024 tentistä, jossa todennäköisesti on taas kysytty sama kysymys kuin aiemmin.}

60-vuotias alkotaustainen mies päivystykseen, 40 astetta kuumetta, oikean kyljen kipu, yskä, saturoituu huoneilmalla 92-94, CRP 240, thx-rtg:ssä oikean puolen diffuusi pneumonia ja pleurassa runsaasti nestettä, mikä ensilinjan diagnoosi ja hoito?

\begin{itemize}
\tightlist
\item
  \begin{enumerate}
  \def\labelenumi{\alph{enumi}.}
  \tightlist
  \item
    pneumonia ja laajakirjoinen ab, CPAP
  \end{enumerate}
\item
  \begin{enumerate}
  \def\labelenumi{\alph{enumi}.}
  \setcounter{enumi}{1}
  \tightlist
  \item
    empyeema, pleuradreeni, laajakirjoinen ab
  \end{enumerate}
\item
  \begin{enumerate}
  \def\labelenumi{\alph{enumi}.}
  \setcounter{enumi}{2}
  \tightlist
  \item
    joku (ei wikissä)
  \end{enumerate}
\end{itemize}

\begin{solution}
\leavevmode

Vastaus

\begin{verbatim}
 b
\end{verbatim}

Empyeema (keuhkopussin märkäkertymä) on usein pneumonian komplikaatio (muita aiheuttajia esim. maligniteetti, keuhkoabskessin rupturoituminen, esofagusruptuura, pneumothorax, hemothorax, leikkauskomplikaatio tai subfreeninen abskessi) ja sen esiintyvyys on kasvussa erityisesti päihdeongelmaisilla.

Empyeeman tyypillisiä oireita ovat korkea kuume, hengitysvaikeudet, kipu kyljessä. Yleisesti siis potilas on todella sairas ja kun empyeema on klassisesti pneumonian komplikaatio, niin tyypillistä tilalle on, että pneumoniapotilaan vointi heikkenee antibioottihoidosta huolimatta.

Empyeema diagnosoidaan thoraxkuvalla, mutta TT antaa tarkemman kuvan taudin syystä, tulehdusontelon laajuudesta sekä lokeroisuudesta. Ultraäänitutkimuksesta on eniten hyötyä diagnostisen punktion kohdentamisessa. Diagnostiikassa voidaan kuitenkin hyödyntää nesteen valkosolumääriä, happamuutta sekä glukoosi- että laktaattidehydrogenaasipitoisuutta. Viljelyvastaus ohjaa mikrobilääkehoitoa.

Empyeeman hoidon perusideologiana on tehokas mikrobilääkehoito (keskimäärin kestää n.~3 viikkoa) ja märkäkertymien dreneeraus. Dreneerausmenetelmän valintaan vaikuttavat taudin etiologia, vaikeusaste (kertymien laajuus ja lokeroisuus), potilaan yleistila ja pitkäaikaissairaudet sekä keuhkon infektiotilanne ja mahdollinen fisteli keuhkosta keuhkopussiin. Empyeeman eksudativisessa vaiheessa pleuraan (keuhkopussi) kertyvä neste on dreneerattavissa päivittäisillä kaikuohjatuilla punktioilla. Usein päädytään kuitenkin jo varhaisvaiheessa pleuradreenin asettamiseen.

Jos pleura ei tyhjene dreenillä -\textgreater{} heti leikkaus. Samoin jos mikrobilääkityksellä ja dreneerauksella tauti ei osoita paranemisen merkkejä, niin leikkaukseen tulisi edetä viimeistään viikossa. Ensisijaisesti kirurginen hoito tulisi toteuttaa tähystysleikkauksena (pidemmälle edennyt avoleikkauksena). Leikkaushoidon tavoitteena on tyhjentää märkä ja kuoria paksu fibriini keuhkon päältä (eli dekortikaatio)-- \textgreater{} jos keuhko ei laajene, tyhjä tila täyttyy herkästi mädällä. Empyeemapotilaista noin puolet tarvitsee kirurgista hoitoa.

\end{solution}

\section{Munuaissyövät}\label{munuaissyuxf6vuxe4t}

Ei tarkempaa kysymyksenasettelua tai vaihtoehtoja, vain aihe kirjoitettu. Tässä hieman munuaissyövistä:

Munuaissyövät ovat yleensä oireettomia ja suurin osa löytyy sattumalta vatsan kuvantamistutkimuksissa

\begin{itemize}
\tightlist
\item
  Mahdollisia oireita ovat kuitenkin selkä- ja kylkikipu, hematuria, metastaasien oireet, yleisoireet; hypersedimentaatio, anemia ja mikroskooppinen verivirtsaisuus ovat tavallisia löydöksiä

  \begin{itemize}
  \tightlist
  \item
    Kaikukuvaus on suositeltavin seulontatutkimus epäiltäessä munuaissyöpää. Laboratoriotutkimukset: La, PVKT, Krea, AFOS ja U-KemSeul. Kasvainlöydös varmistetaan tavallisesti vartalon varjoainetehosteisella tietokonetomografialla (TT). Histologinen varmistus biopsialla tulee ottaa aina ennen ablatiivisen (radiofrekvenssi- tai kryoablaatio) tai onkologisen hoidon aloitusta. Kudosnäytteiden ottaminen munuaiskasvaimista on yleistynyt, koska ennustetta sekä mahdollisen leikkauksen tai muun hoidon hyötyjä ja haittoja joudutaan punnitsemaan yhä tarkemmin.

    \begin{itemize}
    \tightlist
    \item
      Munuaissyöpä syntyy tubulusepiteelin solujen muuttuessa kumuloituvien geenimutaatioiden kautta pahanlaatuisiksi. Munuaissyövän tärkeimmät histologiset alatyypit ovat \textbf{kirkassoluinen karsinooma (75 \%), papillaarinen karsinooma (10 \%) ja kromofobinen karsinooma (5 \%)}
    \end{itemize}
  \end{itemize}
\end{itemize}

Tärkein riskitekijä on tupakointi. Muita riskitekijöitä ovat lihavuus ja korkea verenpaine. Myös perinnöllisillä tekijöillä on vaikutusta.

Ainoa munuaissyövän kuratiivinen hoito on leikkaus, jossa poistetaan joko koko munuainen (nefrektomia) tai osa munuaista (resektio). Resektio tehdään jos on mahdollista, ja nykyisin valtaosa tehdään robottiavusteisesti laparoskooppisesti.

\begin{itemize}
\tightlist
\item
  Munuaissyövän työntyminen munuaislaskimoon ja alaonttolaskimoon ei ole leikkausten vasta-aihe vaan tuumoritapin poistaminen laskimosta kannattaa.
\end{itemize}

Kaikkia pieniä (\textless{} 4 cm) munuaiskasvaimia ei tarvitse hoitaa erityisesti, jos potilas on iäkäs ja muita sairauksia on jo kertynyt merkittävästi.

\begin{itemize}
\tightlist
\item
  Pienen, paikallisen (T1a) munuaiskasvaimen ennuste on vanhuksilla hyvä, kasvunopeus on yleensä hidas ja taudin leviämisen riski pieni.

  \begin{itemize}
  \tightlist
  \item
    Jos munuaissyöpä on ollut kooltaan pieni (alle 4 cm) ja paikallinen (T1), viiden vuoden kuluttua elossa on yli 90 \% radikaalisti leikatuista potilasta.
  \end{itemize}
\item
  Lisäksi kajoaviin toimenpiteisiin liittyy haittoja, kuten munuaisten vajaatoiminnan kehittyminen ja siitä seuraava sydän- ja verisuonitautiriskin suureneminen sekä leikkauskomplikaatioiden mahdollisuus.
\item
  Aktiivinen seuranta on hyväksytty vaihtoehto valikoiduille potilaille. Jos leikkausta ei voi tehdä, mutta päädytään kuitenkin aktiiviseen hoitoon, sopivassa paikassa sijaitsevia pieniä munuaiskasvaimia voi tuhota ihon kautta radiologisessa ohjauksessa tapahtuvalla ablaatiolla (radiotaajuus-, mikroaalto-, laser- ja kryoablaatio).

  \begin{itemize}
  \tightlist
  \item
    Hyvin pienten, noin senttimetrin kokoisten munuaiskasvainten kohdalla voidaan aluksi suositella seurantaa myös nuorten ja hyväkuntoisten potilaiden kohdalla, jolloin vältetään epäselvien ja merkityksettömien muutosten turha leikkaushoito.
  \end{itemize}
\item
  Toteamisvaiheessa metastasoineen syövän hoito on palliatiivinen: munuaisen poisto ja/tai lääkehoidot. Pienen ja keskisuuren riskin metastasoineessa syövässä käytetään munuaisen poiston ja lääkehoidon yhdistelmää, mutta suuren riskin metastasoineessa syövässä munuaisen poistosta ei ole hyötyä.
\end{itemize}

\pandocbounded{\includegraphics[keepaspectratio]{images/munuaistuumorialgoritmi.png}}

\section{Vatsa-aortan aneurysmat -- miten oireilee? +hoito}\label{vatsa-aortan-aneurysmat-miten-oireilee-hoito}

Ei vaihtoehtoja wikissä, mutta tässä tärkeimmät:

Vatsa-aortan aneurysma (AAA) on lähes aina oireeton, kunnes se rupturoituu (jonka oireita on taas mm. äkillisesti alkanut voimakas vatsakipu, joka säteilee selkään ja joskus kylkeen/nivustaipeisiin\ldots, hypotensio, synkopee yms). Joskus AAA voi kuitenkin aiheuttaa kompressio-oireita ja kipua.

\begin{itemize}
\tightlist
\item
  Laskimokompressiosta voi aiheutua turvotus ja/tai syvän laskimon tukos alaraajoissa sekä hydronefroosi (virtsajohtimen kompressio)
\item
  Jos aneurysma kipuilee, sen katsotaan olevan merkki repeämisvaarasta
\item
  Iso aneurysma on usein helposti palpoitavissa sykkivänä resistenssinä vatsaa tunnusteltaessa
\end{itemize}

Vatsa-aortan aneurysman hoito on ensisijaisesti tilanteen seuranta ja riskitekijöiden hoito, joka sisältää tupakoinnin lopettamisen, asetyylisalisyylihappo- tai klopidogreelilääkityksen, verenpaineen hoidon ja statiinilääkityksen. Seuranta toteutetaan ensisijaisesti terveyskeskuksessa UÄ-tutkimuksella, joka suoritetaan

\begin{itemize}
\tightlist
\item
  3 vuoden välein jos aneurysman koko on 30-35 mm
\item
  2 vuoden välein jos aneurysman koko on 36-40 mm
\item
  1 vuoden (3-12kk) välein jos aneurysman koko on \textgreater40 mm
\end{itemize}

Jos koko \textgreater{} 50 mm, niin lähete vkir (naisilla 45 mm) ja aneurysma mahdollisesti hoidetaan operatiivisesti. Käytännön toimenpiderajaksi on asetettu 55 mm miehillä (50 mm naisilla), sillä tätä pienemmät aneurysmat repeävät harvoin.

\begin{itemize}
\tightlist
\item
  Hoitovaihtoehdot:

  \begin{itemize}
  \tightlist
  \item
    Suonensisäisen stenttiproteesin (EVAR, endovascular aneurysm repair) asentaminen on nykyään ensisijainen AAAn hoito, sillä sen aiheuttama välitön riski potilaalle on pienempi, erityisesti revenneen AAAn osalta
  \item
    Osa aneurysmista avoleikataan
  \item
    Hoitomuodon valintaan vaikuttavat potilaan elinajan ennuste (nuoremmat voidaan avoleikata useammin), yleistila sekä aneurysman muoto ja sijainti suhteessa aortan valtimohaaroihin

    \begin{itemize}
    \tightlist
    \item
      Avoleikkaus (Y-proteesi) ei vaadi jatkoseurantaa, koska myöhäiskomplikaatiot avoleikkauksen jälkeen ovat harvinaisia. Akuutimmat komplikaatiot taas yleisempiä ja toimenpiteeseen liittyvä kuolleisuus on n.~3-4 \% kirjallisuuden mukaan. Kuitenkin on hyvä pitkäaikaisennuste jos akuutista vaiheesta selviää -\textgreater{} nuorempia potilaita leikataan enemmän Y-proteesilla kuin vanhoja potilaita.

      \begin{itemize}
      \tightlist
      \item
        Y-protetisaatiossa voi ilmentyä komplikaationa esim. proteesi-infektio tai aortoenteerinen disteli, jossa on fisteli Y-proteesisauman ja suolen välillä. Ovat vakavia mutta harvinaisia komplikaatiota, joiden hoitoon liittyy merkittävä sairastuvuus ja kuolleisuus.
      \end{itemize}
    \item
      EVAR:n kanssa taas on enemmän myöhäiskomplikaatioita, jonka takia EVAR vaatii loppuiän seurantaa. Komplikaatiot liittyvät yleisimmin niin sanottuun endoleakiin (EL) eli siihen, että aneurysmasäkki paineistuu ja sinne pääsee verenkiertoa. Mikäli endoleak aiheuttaa merkittävää aneurysmasäkin kasvua, se pitää hoitaa. Yleensä hoito on suonensisäinen lisätoimenpide, mutta joskus joudutaan tekemään konversio eli stenttiproteesin poisto ja aortan korjaus avoimesti proteesilla.
    \end{itemize}
  \end{itemize}
\end{itemize}

\section{Ihosyövät}\label{ihosyuxf6vuxe4t}

Ei tarkempaa kysymyksenasettelua tai vaihtoehtoja wikissä, ja kyseessä on aika laaja aihealue. Tässä nyt joitain pieniä hippuja ihosyöpien jaottelusta yms:

Ihosyöpä on yleisin syöpätyyppi, mutta vain 1\% syöpäkuolemista johtuu ihosyövistä

\begin{itemize}
\tightlist
\item
  Tärkein ulkoinen riskitekijä on runsas elinaikainen altistuminen auringon UV-säteilylle sekä useat ihon palamiskerrat. Ihosyöpää voi ilmentyä lapsilla ja tummaihoisillakin.
\item
  Riskiä lisää myös aikaisemmin sairastettu ihosyöpä itsellä tai lähisukulaisella.
\end{itemize}

Tärkeimmät tyypit ovat melanooma ja ei-melanoottiset syövät kuten basaliooma (tyvisolusyöpä) ja okasolusyöpä (levyepiteelisyöpä). \textbf{Basaliooma on yleisin ihosyöpä, sen jälkeen okasolusyöpä ja melanooma}

\begin{itemize}
\tightlist
\item
  Tosin Syöpärekisterin mukaan toiseksi yleisin vallitsevuuden perusteella on melanooma ja sen jälkeen okasolusyöpä, mutta ilmaantuvuudessa ne ovat tasoissa (v. 2018 melanoomaa 1658 kpl ja okasolusyöpää 1700 kpl)
\end{itemize}

\section{Gastroskopiat}\label{gastroskopiat}

Ei tarkempaa kysymyksenasettelua tai vaihtoehtoja wikissä. Tärkeimmät asiat tietää gastroskopioista on tutkimuksen indikaatiot. Vasta-aiheita ovat kompensoimaton sydämen vajaatoiminta ja vaikea keuhkosairaus. Tuore sydäninfarkti on suhteellinen vasta-aihe, vaikka endoskopia aiheuttaa harvoin iskemiaa hemodynaamisesti vakaassa tilanteessa. Raskaus ei ole gastroskopian vasta-aihe.

\pandocbounded{\includegraphics[keepaspectratio]{images/gskopiaindikaatiot.png}}

\section{Mahahaava ja maligniteetit}\label{mahahaava-ja-maligniteetit}

Peptisen ulkustaudin hoidon onnistuminen tarvitsee varmentaa gastroskopialla, jos kyseessä on ollut ventrikkeliulkus

\begin{itemize}
\tightlist
\item
  Ventrikkeliulkuspotilaalla PPI-lääkitystä jatketaan, kunnes haavan parantuminen on varmistettu ja koepaloin todettu, ettei kyseessä ole maligniteetti
\item
  Yleensä gastroskopialla kontrolli 2-3kk kohdalla, varmistetaan ettei haavan taustalla ole karsinoomaa tai esiasteita
\item
  Jos tähystyksessä Forrest I haava (suihkuvuoto tai valuva vuoto; indikaatio endoskooppiselle hoidolle), niin yleensä kontrollitähystys saman osastojakson aikana
\end{itemize}

Duodenaaliulkuksien taustalla ei käytännössä koskaan ole malignisoitumisriskiä -\textgreater{} ei tarvetta kontrollitähystyksille

\begin{itemize}
\tightlist
\item
  Mahdollisen H. pylorin häätöhoidon onnistuminen kuitenkin varmistetaan avohoidossa ulostetestillä (huom! vasta-ainemääritys ei sovellu hoidon onnistumisen seurantaan); Ennen kontrollinäytettä on muistettava vähintään 2 viikon PPI-tauko
\end{itemize}

\section{Milloin sarkoomaa kannattaa epäillä}\label{milloin-sarkoomaa-kannattaa-epuxe4illuxe4}

Ei vaihtoehtoja wikissä, mutta tässä tärkeimmät:

Sarkoomat jaetaan pehmytkudossarkoomiin ja luusarkoomiin, joiden käyttäytyminen ja hoitoperiaatteet eroavat toisistaan

\begin{itemize}
\tightlist
\item
  Todennäköisesti tässä on tarkoitettu nyt pehmytkudossarkoomia, kuten liposarkoomat, erilaistumaton pleomorfinen sarkooma, leiomyosarkooma\ldots{}

  \begin{itemize}
  \tightlist
  \item
    Ovat harvinaisia ja todetaan Suomessa n.~220 vuodessa
  \item
    Todetaan kaikenikäisillä
  \end{itemize}
\end{itemize}

Sarkooma voi sijaita pinnallisesti raajoissa tai vartalolla tai syvemmällä retroperitoneaali- tai peritoneaalitilassa

\textbf{Pehmytkudossarkooma on yleensä oireeton kyhmy, joka kasvaa. Yleisoireita on harvoin. Sarkoomaa on epäiltävä erityisesti, jos kasvava kyhmy on atraumaattinen, lihaskalvon alla, alustaansa kiinnittynyt, kova ja/tai yli 5 cm:n läpimittainen}

\begin{itemize}
\item
  UÄ voi näyttää hematoomalta ja potilas tulee lähettää jatkotutkimuksiin, jos se ei vastaa kliinistä löydöstä.
\item
  Pehmytkudossarkooman epäilyn herättyä potilas tulee ohjata yliopistosairaalan sarkoomatyöryhmän konsultaatioon
\end{itemize}

\section{Sarkoomaepäily, mitä tehdään tk:ssa?}\label{sarkoomaepuxe4ily-mituxe4-tehduxe4uxe4n-tkssa}

\begin{itemize}
\tightlist
\item
  \begin{enumerate}
  \def\labelenumi{\alph{enumi}.}
  \tightlist
  \item
    stanssibiopsia
  \end{enumerate}
\item
  \begin{enumerate}
  \def\labelenumi{\alph{enumi}.}
  \setcounter{enumi}{1}
  \tightlist
  \item
    poistan koko muutoksen PAD-näytteeksi
  \end{enumerate}
\item
  \begin{enumerate}
  \def\labelenumi{\alph{enumi}.}
  \setcounter{enumi}{2}
  \tightlist
  \item
    sarkoomaepäilyssä ei koskaan saa ottaa näytettä tk:ssa
  \end{enumerate}
\item
  \begin{enumerate}
  \def\labelenumi{\alph{enumi}.}
  \setcounter{enumi}{3}
  \tightlist
  \item
    joku (ei wikissä)
  \end{enumerate}
\end{itemize}

\begin{solution}
\leavevmode

Vastaus

\begin{verbatim}
 c
 
\end{verbatim}

Älä koskaan ota neulanäytettä/stanssibiopsiaa/poistoa patista, joka uä:n tai kliinisen tutkimuksen mukaan voisi ehkä mahdollisesti olla sarkooma. Sarkoomaepäilyt lähetetään ESH, jossa MRI ja sen perusteella kohdennetaan paksuneulabiopsia.

\pandocbounded{\includegraphics[keepaspectratio]{images/pehmytkudoskasvainalgoritmi.png}}

\end{solution}

\section{Sarkooma on}\label{sarkooma-on}

\begin{itemize}
\tightlist
\item
  \begin{enumerate}
  \def\labelenumi{\alph{enumi}.}
  \tightlist
  \item
    aina lihaksesta peräisin
  \end{enumerate}
\item
  \begin{enumerate}
  \def\labelenumi{\alph{enumi}.}
  \setcounter{enumi}{1}
  \tightlist
  \item
    yleensä perinnöllinen
  \end{enumerate}
\item
  \begin{enumerate}
  \def\labelenumi{\alph{enumi}.}
  \setcounter{enumi}{2}
  \tightlist
  \item
    yleensä traumaperäinen
  \end{enumerate}
\item
  \begin{enumerate}
  \def\labelenumi{\alph{enumi}.}
  \setcounter{enumi}{3}
  \tightlist
  \item
    yleensä syvällä
  \end{enumerate}
\end{itemize}

\begin{solution}
\leavevmode

Vastaus

\begin{verbatim}
 d
 
\end{verbatim}

a: Sarkoomat eivät ole aina lihaksesta peräisin (myo- tai leiomyosarkoomia). Yleisin yksittäinen pehmytkudossarkooma on liposarkooma ja leiomyosarkooma eli sileän lihaksen sarkooma on toiseksi yleisin.

b: Vain pieni osa (noin 5-10 \%) liittyy perinnöllisiin syndroomiin (esim. Li-Fraumenin oireyhtymä tai tyypin 1 neurofibromatoosi) ja valtaosa on sporadisia.

c: Trauma ei aiheuta sarkoomaa, vaikka potilas joskus tulee lääkäriin trauman jälkeen huomatun patin vuoksi, joka voi olla pehmytkudosvauriosta johtuvaa tai trauma on voinut tuoda huomiota aikaisemmin jo paikalla olleeseen sarkoomaan.

d: Totta. Sarkoomat tyypillisimmin kasvavat syvän faskian alapuolella eivätkä suoraan ihon alla.

\pandocbounded{\includegraphics[keepaspectratio]{images/pehmytkudoskasvainalgoritmi.png}}

\end{solution}

\section{Lymfakierron ongelmien etiologia}\label{lymfakierron-ongelmien-etiologia}

Ei vaihtoehtoja, mutta tässä tärkeimmät:

Imunestekierto kuljettaa nesteitä ja isoja molekyylejä kudoksista verenkiertoon ja imusolmukkeisiin. Lymfakierron estyminen syystä tai toisesta johtaa lymfaturvotukseen eli lymfedeemaan kudoksessa, josta nestettä ei saada kierrätettyä ulos. Se johtaa ensin imunesteen ja myöhemmässä vaiheessa side- ja rasvakudoksen kertymiseen oireiseen raajaan.

\begin{itemize}
\tightlist
\item
  Syy voi olla primaarinen (geneettinen, esim. Milroyn tauti tai Meigen tauti) tai sekundaarinen (paljon yleisempi)
\item
  \textbf{Yleisin sekundaarinen syy (ainakin länsimaissa) on imusolmukkeiden poisto ja/tai sädehoito syövän hoidon yhteydessä.} Yleisin potilasryhmä ovat kainaloon levinneen rintasyövän vuoksi hoidetut potilaat.

  \begin{itemize}
  \tightlist
  \item
    Rintasyöpäpotilaista, joille on tehty kainalon imusolmukkeiden poisto, 20--40 prosentille kehittyy yläraajan imunesteturvotus. Vartijaimusolmukkeen poiston jälkeisen imunesteturvotuksen riski on 4--10 \%. Leikkauksen jälkeinen sädehoito suurentaa imunesteturvotuksen riskiä. Rintasyöpään liittyvä imunesteturvotus kehittyy tavallisimmin parin ensimmäisen vuoden aikana, mutta se voi ilmaantua myös vuosien viiveellä.
  \item
    Alaraajan imunesteturvotukselle altistavat erityisesti gynekologisten ja urologisten syöpien sekä melanooman vuoksi tehty lantion ja/tai nivusalueen imusolmukkeiden poisto ja sädehoito. Myös muiden syöpien kuten sarkooman hoidon seurauksena voi kehittyä imunesteturvotus.
  \end{itemize}
\item
  \textbf{Kehitysmaissa (ja siten maailmanlaajuisesti) yleisin syy on filariaasi eli rihmamadon (tavallisimmin Wuchereria bancrofti) aiheuttama parasiitti-infektio imuteissä.} Maailman terveysjärjestön WHO:n tehokkaan lääkehoitokampanjan myötä infektioiden määrä on saatu viime vuosina selvään laskuun.
\end{itemize}

Myös mm. trauma ja ylipaino voivat olla sekundaarisen lymfedeeman taustalla.

\section{ERCP komplikaatiot}\label{ercp-komplikaatiot}

Ei vaihtoehtoja, mutta tässä tärkeimmät:

ERCP:hen liittyviä komplikaatioita ovat

\begin{itemize}
\tightlist
\item
  haimatulehdus (2-5\%, joissain lähteissä 3,5--9,7 \%)
\item
  kolangiitti (1 \%)
\item
  sfinkterotomian jälkeinen verenvuoto (1 \%)
\item
  tiehyen tai suolen puhkeama (alle 1 \%)
\end{itemize}

ERCP eli endoskooppinen retrogradinen kolangiopankreatografia on toimenpide, jossa sivulle katsovalla duodenoskoopilla voidaan tutkia ja hoitaa sappi- ja haimatiehyen sairauksia läpivalaisulaitteen avulla. \textbf{ERCP tulee kohdentaa potilaille, joille on tarkoitus tehdä samalla jokin hoidollinen toimenpide.} Diagnostisena tutkimuksena ERC:ta käytetään sklerosoivan kolangiitin etenemisen seurannassa kuvauslöydöksen ja harjairtosolunäytteiden avulla.

\pandocbounded{\includegraphics[keepaspectratio]{images/ercp.png}}
\pandocbounded{\includegraphics[keepaspectratio]{images/ercpkuvaus.png}}

\section{Laskimoiden termoablaatio}\label{laskimoiden-termoablaatio}

Ei vaihtoehtoja, mutta tässä tärkeimmät termoablaatiosta:

Laskimovajaatoiminnan kajoavan hoidon vaihtoehtoja ovat EVTA (termoablaatio), UGFS (sleroterapia; sekundaarinen, jos termoablaatio ei ole teknisesti mahdollinen esimerkiksi hoidettavan laskimon mutkaisuuden takia) tai avokirurgia (viimeinen vaihtoehto jos edellä mainitut eivät sovi)

\begin{itemize}
\tightlist
\item
  EVTA (endovascular thermal ablation) on nykyään ensisijainen kajoava hoitomenetelmä pinnallisen laskimopäärungon vajaatoiminnassa ja on käytännössä korvannut avokirurgian. Se voidaan suorittaa laserablaationa (EVLA, endovenous laser ablation) tai radiotaajuusablaationa (FRA, radiofrequency segmental thermal ablation, rf-tekniikka eli radiofrekvenssitekniikka). \textbf{Ensisijaisena näistä voidaan pitää laserablaatiota.}
\end{itemize}

\textbf{Termoablaatiossa hoidettava laskimo (useimmiten vena saphena magna tai parva) punktoidaan ja laserkuitu uitetaan sisään.} Se viedään vajaatoimintamuutosten läpi proksimaaliselle puolelle safenofemoraalisen tai safenopopliteaalisen junktion distaalipuolelle pitäen turvamarginaalin syvään laskimoon. Hoidettavan laskimon ympärille injisoidaan puudutus. Peruutellessa ulos laseroidaan suonta, mikä johtaa laskimon endoteelin vaurioon ja johtaa pintalaskimon fibrotisoitumiseen kiinni, jolloin se ei enää voi aiheuttaa vajaatoiminta-oireita.

\begin{itemize}
\tightlist
\item
  Merkittävä osa termoablaatioista voidaan toteuttaa poliklinikkaoloissa. Erityistä jälkiseurantaa ei tarvita, ja potilas voi mobilisoitua heti.
\item
  \textbf{Komplikaatiot ovat harvinaisia ja tärkein tiedostaa on syvä laskimotukos,} jonka takia hoito on vasta-aiheinen raskaana oleville, aktiivisen pintalaskimotukoksen aikana ja syvien laskimoiden obstruktiossa sekä niille, joilla on avoin foramen ovale (mahdollisuus paradoksiselle embolialle, kun laskimotrombi pääsee avoimen foramen ovalen kautta vasempaan eteiseen ja valtimoverenkiertoon).
\end{itemize}

\textbf{Toimenpiteen jälkeinen pintalaskimotukos hoidetulla alueella on normaalilöydös eikä komplikaatio (suonihan on tarkoituksellisesti tukittu ja post-ablatiivisesti pinnallinen tromboflebiitti on odotettu ja tavallinen löydös 1-2 vk kuluttua laserablaatiosta tai vaahdotuksesta)} -- hoitamattomassa suonessa kohdellaan kuten pintalaskimotukos yleensä.

\begin{itemize}
\tightlist
\item
  Hoidetun suonen alueella voidaan myös usein todeta \textbf{suonen kulkua myötäilevä pigmentaatiomuutos} koska lämpö voi vaikuttaa ihon alla oleviin verisuoniin ja pigmenttisoluihin. Nämä värimuutokset ovat yleensä vaarattomia, mutta voivat olla pysyviä, joka tulee pitää mielessä, jos suonikohjuja lähdetään korjaamaan vain esteettisistä syistä (esteettinen ongelma voi siis vain muuttua toisenlaiseksi luonteeltaan).
\end{itemize}

\pandocbounded{\includegraphics[keepaspectratio]{images/evta.png}}

\section{Laskimovajaatoiminnan hoitoindikaatiot}\label{laskimovajaatoiminnan-hoitoindikaatiot}

Käyty jo läpi aikaisemmissa tärpeissä, mutta koita nyt taas tuoda indikaatiot (sekä konservatiiviselle että operatiiviselle) mieleesi ilman vinkkejä

\begin{solution}
\leavevmode

Vastaus

\begin{verbatim}
 Alla 
 
\end{verbatim}

Oireetonta laskimovajaatoimintaa ei sinänsä tarvitse hoitaa konservatiivisestikaan. Jos potilas kokee oireita, mutta ei vielä täytä kajoavan hoidon kriteerejä, niin hoito on pääasiassa vain kompressiohoitoa (paljon mitään muuta ei voida tehdä). Kompressiohoito ei kuitenkaan näytä estävän suonikohjuja pahenemasta, mutta se vähentää laskimovajaatoiminnan oireita ja hyödyttää monia potilaita. Kompressiohoidolla on kuitenkin heikko hoitomyöntyvyys. Komplisoituneessa taudissa kompressiohoito on kuitenkin äärimmäisen tärkeä ja siitä on hyötyä laskimohaavojen hoidossa ja ehkäisyssä.

\textbf{Kajoavan hoidon indikaatiot ovat seuraavat: komplisoitunut tauti (C4-C6), toistuvat pintalaskimotukokset/laaja pintalaskimotukos, vuotavat suonikohjut tai merkittävää haittaa aiheuttava komplisoitumaton tauti (kunhan BMI \textless35).} Merkittävä haitta tarkoittaa haittaluokkaa 2-3 (potilas ei pysty työskentelemään ilman kompressiohoitoa tai ei pysty työskentelemään siitä huolimatta).

\pandocbounded{\includegraphics[keepaspectratio]{images/pintalaskimovajaatoimintaalgoritmi.png}}
\pandocbounded{\includegraphics[keepaspectratio]{images/laskimovajaatoimintahaittaluokka.png}}

\end{solution}

\section{Ulcusperforaatio}\label{ulcusperforaatio}

Ei vaihtoehtoja tai kysymyksenasettelua wikissä. Tässä tärkeimmät:

Toiseksi yleisin peptisen ulkustaudin komplikaatio on perforaatio (yleisin komplikaatio on ulkusvuoto n.~20-25\% potilaista). On arvioitu, että 5--10 \%:lle ulkuspotilaista kehittyy jossain vaiheessa ulkusperforaatio.

\begin{itemize}
\tightlist
\item
  Ulkuspotilaalle (tai potilaalle, jolla epäilet olevan taustalla ulkus) ilmestyvä äkillinen kova vatsakipu viittaa ulkuksen perforaatioon. Suurimmalla osalla ulkusperforaatiopotilaista on ollut aikaisemmin ulkus tai ainakin ulkustyyppisiä vaivoja, mutta noin 10 \%:lla potilaista ulkusperforaatio tapahtuu ulkustaudin ensimmäisenä kliinisenä manifestaationa.
\item
  Kipu on luonteeltaan jatkuvaa, erittäin voimakasta. Se tuntuu aluksi epigastriumissa tai oikean kylkikaaren alla, mutta leviää nopeasti koko vatsan alueelle. Myös hartiakipua saattaa esiintyä merkkinä palleaärsytyksestä (Kehrin oire)
\end{itemize}

Ulkusperforaatiossa todetaan vapaata ilmaa vatsan TT:ssä (kuten muutenkin suoliperforaatiossa). Ulkusperforaatio on tavallisin pneumoperitoneumin aiheuttaja. Perforaatio sijaitsee tavallisimmin bulbus duodenin alueella (35--65 \%) ja toiseksi yleisimmin pyloruksessa (25--45 \%). Perforaatioista 5--25 \% sijaitsee muualla ventrikkelissä.

Hoidossa on olennaisinta varhainen diagnoosi, jonka jälkeen aloitetaan kiireellisesti laajakirjoinen mikrobilääkehoito sekä nesteresuskitaatio. Itse hoito on päivystyksellinen leikkaus

\begin{itemize}
\tightlist
\item
  Leikkaukseksi riittää lähes aina vatsaontelon huuhtelu ja yksinkertainen perforaation sulku knoppiompeleiden ja omenttipedikkelin kanssa tai ilman sitä. Alle 10 \% potilaista tarvitsee ventrikkeliresektiota. Leikkauksen yhteydessä ventrikkeliulkuksesta otetaan histologinen näyte, sillä mahasyöpä voi joskus tulla ilmi ulkusperforaation muodossa. Ventrikkeliulkuksen tapauksessa tulee myös tehdä kontrolligastroskopia maligniteettiriskin takia.

  \begin{itemize}
  \tightlist
  \item
    Laparoskooppinen kirurgia on yleistynyt, ja sen etuna on leikkauksen jälkeisen kipulääkityksen vähäisempi tarve, vähäisempi kuolleisuus ja komplikaatioriski sekä lyhyempi sairaalassaoloaika. Toisaalta laparoskooppisella tekniikalla tehtyjen leikkausten kesto on pitempi verrattuna avoleikkauksiin.
  \end{itemize}
\end{itemize}

\section{Haimatulehduksen vaikeus, mikä korreloi?}\label{haimatulehduksen-vaikeus-mikuxe4-korreloi}

\begin{itemize}
\tightlist
\item
  \begin{enumerate}
  \def\labelenumi{\alph{enumi}.}
  \tightlist
  \item
    Amylaasi
  \end{enumerate}
\item
  \begin{enumerate}
  \def\labelenumi{\alph{enumi}.}
  \setcounter{enumi}{1}
  \tightlist
  \item
    Haimaperäinen amylaasi
  \end{enumerate}
\item
  \begin{enumerate}
  \def\labelenumi{\alph{enumi}.}
  \setcounter{enumi}{2}
  \tightlist
  \item
    Lipaasi
  \end{enumerate}
\item
  \begin{enumerate}
  \def\labelenumi{\alph{enumi}.}
  \setcounter{enumi}{3}
  \tightlist
  \item
    CRP
  \end{enumerate}
\end{itemize}

\begin{solution}
\leavevmode

Vastaus

\begin{verbatim}
 d 
 
\end{verbatim}

Yksittäisistä laboratoriokokeista paras ennustearvo on C-reaktiivisella proteiinilla (CRP). Sairaalaan tulovaiheessa CRP \textgreater150 mg/l ennustaa hyvin mahdollista vaikeampaa akuutin pankreatiitin taudinkuvaa.

Myös suuri hematokriitti- ja kreatiniinipitoisuus sekä huono diureesi ovat varoitusmerkkejä siitä, että potilaalle on kehittymässä vaikea akuutti pankreatiitti (AP). Hematokriitti kuvastaa third-spacingia hyvin (nestemenetys systeemisen tulehduksen takia -\textgreater{} hematokriitti nousee)

Vaikea AP = elinvaurio \textgreater48h

a,b,c: Amylaasi tai lipaasi ei korreloi vaikeusasteen kanssa

\pandocbounded{\includegraphics[keepaspectratio]{images/apvaikeusaste.png}}

\end{solution}

\section{Mikä leikkaus tehdään, kun potilaalla todetaan divertikkeliperforaatio ja fekaaliperitoniitti?}\label{mikuxe4-leikkaus-tehduxe4uxe4n-kun-potilaalla-todetaan-divertikkeliperforaatio-ja-fekaaliperitoniitti}

Ei vaihtoehtoja, mutta koita vastata ilman vinkkejä

\begin{solution}
\leavevmode

Vastaus

\begin{verbatim}
 Hartmannin toimenpide
\end{verbatim}

Divertikuliitin vaarallisin komplikaatio on peritoniitti, joka syntyy, kun suoli puhkeaa vatsaonteloon ja keho ei pysty rajaamaan sitä paiseeksi. Potilaat, joilla on peritoniitti, kuuluu hoitaa operatiivisesti; leikkauksessa vatsaontelo huuhdellaan puhtaaksi eritteestä ja poistetaan sairastunut ja puhjennut suolen osa, jonka jälkeen proksimaalinen paksusuolen pää tuodaan avanteeksi ihon pinnalle (Hartmanin leikkaus päivystyksenä).

Peräsuoli jää paikalleen ja suolen päiden yhdistäminen myöhemmin (tyypillisesti aikaisintaan puolen vuoden päästä) on usein mahdollista, kun potilas on toipunut päivystysleikkauksesta.

\pandocbounded{\includegraphics[keepaspectratio]{images/hartmann.png}}
\pandocbounded{\includegraphics[keepaspectratio]{images/divertikuliittiluokatjahoito.png}}

\end{solution}

\section{Raskaaksi haluava nainen ja hyperparatyreoosi}\label{raskaaksi-haluava-nainen-ja-hyperparatyreoosi}

Ei vaihtoehtoja, mutta todennäköisesti kysytty, että miten potilasta tulisi hoitaa:

\textbf{Paratyreoidektomia ei ole vasta-aiheinen primaarin hyperparatyreoosin (PHPT) hoitokeino, jos potilas suunnittelee raskautta; se on oikeastaan leikkausindikaatio!}

Yleisimmät hyperkalsemian syyt ovat primaarinen hyperparatyreoosi (PHPT) tai maligniteetti (ja siten paraneoplastinen PTHrP-välitteinen hyperkalsemia tai hyperkalsemia johtuen osteolyyttisistä metastaaseista tai myeloomasta), jotka yhdessä muodostavat noin 90 \% tapauksista. Hyperkalsemian yleisin yksittäinen syy on primaarinen hyperparatyreoosi (PHPT)

\begin{itemize}
\tightlist
\item
  PHPT:n yleisin syy (75-85\%) on yksittäinen hyvänlaatuinen lisäkilpirauhasen adenooma; muita syitä ovat mm. useamman lisäkilpirauhasen sporadinen hyperplasia (10-15\%) ja lisäkilpirauhasen karsinooma (\textgreater1\%).
\item
  Tärkein hyperkalsemian diagnostiikkaa ohjaava tutkimus on lisäkilpirauhashormonin pitoisuus (P-PTH). Jos hyperkalsemia ja P-PTH viitealueen yläosassa/koholla (n.~10-15\% tapauksista PTH on viitealueella; tässä tapauksessa kuitenkin liian korkealla, sillä hyperkalsemian yhteydessä PTH:n tulisi olla matala/viitealueen alaosassa) -\textgreater{} todnäk primäärinen hyperparatyreoosi (PHPT)
\end{itemize}

PHPT:n ainoa parantava hoito on leikkaus (paratyreoidektomia). Leikkaushoidolle on tietyt kriteerit; leikkaushoidon arviota varten potilas lähetetään alueen endokrinologian poliklinikkaan, josta käsin suunnitellaan mahdolliset paikantavat kuvantamistutkimukset.

\begin{itemize}
\tightlist
\item
  Jos primaarinen hyperparatyreoosi ei täytä leikkaushoidon aiheita tai leikkauksesta muusta syystä pidättäydytään, P-Ca-Ion -seurantaa suositellaan 1--2 kertaa vuodessa ja GFR mitataan kerran vuodessa; Jos P-Ca-Ion seurannassa suurenee, leikkausarvio on tehtävä uudelleen.
\end{itemize}

\pandocbounded{\includegraphics[keepaspectratio]{images/phptleikkauskriteerit.png}}
\pandocbounded{\includegraphics[keepaspectratio]{images/phptleikkauskriteerit4mk.png}}

\section{Mikä tyrä useimmiten on?}\label{mikuxe4-tyruxe4-useimmiten-on}

Vähän kryptinen kysymys wikissä, mutta mahdollisesti kysytty esim. joku näistä kysymyksistä:

\begin{itemize}
\tightlist
\item
  Mikä on yleisin tyrätyyppi?

  \begin{itemize}
  \tightlist
  \item
    Nivustyrä (ingvinaalihernia) on yleisin niin miehillä kuin naisilla. Miehillä nivustyrät ovat vain suhteellisesti paljon yleisempiä, kun taas esim. reisityrät ovat yleisempiä suhteellisesti naisilla (mutta nivustyrät ovat kuitenkin yleisempiä naisillakin). Miehistä noin neljännes saa nivustyrän, naisilla vain kolmelle sadasta kehittyy nivustyrä.
  \end{itemize}
\item
  Mikä on yleisin nivustyrätyyppi?

  \begin{itemize}
  \tightlist
  \item
    Epäsuora (mediaalinen) nivustyrä on yleisin.
  \item
    Nivustyrät jaetaan lateraalisiin (epäsuora) ja mediaalisiin (suora) sen mukaan, työntyykö tyräpussi epigastristen verisuonten (arteria ja vena epigastrica inferior) lateraali- vai mediaalipuolelta. Molemmat voivat olla myös yhtä aikaa (hernia inguinalis combinata).
  \item
    Lateraalinen tyrä (kongenitaalinen) on useimmiten lasten ja nuorten aikuisten sairaus. Käyttää tyräporttina heikosti sulkeutunutta aukkoa fascia transversaliksessa (nivuskanavan sisäaukko)
  \item
    Mediaalinen tyrä (hankinnainen) on pääasiassa vanhempien miesten ongelma. Työntyy ulos vatsaontelosta käyttämällä hankittua heikkoa kohtaa fascia transversaliksessa ja mahdollisesti työntyy vain nivuskanavan ulomman eikä sisemmän suuaukon läpi.
  \end{itemize}
\item
  Mikä tyrä itse asiassa on?

  \begin{itemize}
  \tightlist
  \item
    Tyrä tarkoittaa vatsaontelon sisällön purkautumista vatsaontelon ulkopuolelle vatsaontelon seinämän pitävässä rakenteessa olevan synnynnäisen tai hankitun aukon, ns. tyräportin, kautta. Useimmiten tyrä on vatsaontelon ulkopuolella oleva peritoneumpussi ja aukko on faskia-aukko.
  \end{itemize}
\end{itemize}

\pandocbounded{\includegraphics[keepaspectratio]{images/nivustyrävsreisityrä.png}}
\pandocbounded{\includegraphics[keepaspectratio]{images/typesofhernias.png}}

\section{Komplisoitunut divertikuliitti. Aloitetaanko ab ja myöhemmin kolonoskopia?}\label{komplisoitunut-divertikuliitti.-aloitetaanko-ab-ja-myuxf6hemmin-kolonoskopia}

\begin{itemize}
\tightlist
\item
  \begin{enumerate}
  \def\labelenumi{\alph{enumi}.}
  \tightlist
  \item
    Ab kyllä, kolonoskopia ei
  \end{enumerate}
\item
  \begin{enumerate}
  \def\labelenumi{\alph{enumi}.}
  \setcounter{enumi}{1}
  \tightlist
  \item
    Ab ei, kolonoskopia kyllä
  \end{enumerate}
\item
  \begin{enumerate}
  \def\labelenumi{\alph{enumi}.}
  \setcounter{enumi}{2}
  \tightlist
  \item
    Ab kyllä, kolonoskopia kyllä
  \end{enumerate}
\item
  \begin{enumerate}
  \def\labelenumi{\alph{enumi}.}
  \setcounter{enumi}{3}
  \tightlist
  \item
    Kumpaakaan ei tarvita
  \end{enumerate}
\end{itemize}

\begin{solution}
\leavevmode

Vastaus

\begin{verbatim}
 c
\end{verbatim}

Komplisoitunut divertikuliitti vaatii hoidoksi aina vähintään i.v. antibiootin (yleensä kefuroksiimi ja metronidatsoli) ja joskus myös jopa leikkauksen tai joidenkin paiseiden tapauksessa dreneerauksen. Komplisoituneen divertikuliitin jälkeen kontrollitutkimuksena tulee suorittaa kolonoskopia rauhallisessa vaiheessa (n.~1kk jälkeen akuutista tulehduksesta), jotta voidaan varmistaa, ettei kyseessä ole paksusuolisyöpä. Komplisoitumattomassa divertikuliitissa mitään kontrolleja ei tarvita.

Komplisoitumattomassa divertikuliitissa taas ei tarvitse ab-hoitoa rutiinisti (riskiryhmille kyllä) ja kontrollikolonoskopiaa ei myöskään tarvitse järjestää (ei ole merkittävää maligniteettiriskiä). Voidaan hoitaa oireenmukaisesti (NSAID + parasetamoli). Jos kyseessä on potilaan ensimmäinen divertikuliitti, niin diagnoosi tulee varmistaa TT:llä. Jos taas uusiutunut lievällä tyypillisellä taudinkuvalla, niin ei tarvitse TT:tä ja voidaan hoitaa oireenmukaisesti ilman lisädiagnostiikkaa. Paastoa tai ruokarajoituksia ei tarvita, vaan potilas voi syödä vapaasti. Mikäli oireet eivät helpota parissa päivässä tai vointi heikkenee, tulee diagnoosia tarkentaa TT:lla ja tarvittaessa aloittaa antibiootti.

\pandocbounded{\includegraphics[keepaspectratio]{images/divertikuliittiluokatjahoito.png}}

\end{solution}

\section{Lap.cholen jälkeen 3pop huonovointinen mies, uä:llä sapen lähellä nestettä, radiologin mukaan sappea. Miten hoidetaan ja mikä todnäk aiheuttaja?}\label{lap.cholen-juxe4lkeen-3pop-huonovointinen-mies-uuxe4lluxe4-sapen-luxe4helluxe4-nestettuxe4-radiologin-mukaan-sappea.-miten-hoidetaan-ja-mikuxe4-todnuxe4k-aiheuttaja}

Ei vaihtoehtoja, tässä tärkeimmät laparoskooppisen kolekystektomian (lap.chole) jälkeisestä (3POP tarkoittaa kolmas post-operatiivinen päivä) huonovointisuudesta

\begin{itemize}
\tightlist
\item
  Jos potilas voi huonosti laparoskooppista sappileikkausta seuraavina päivinä, niin tulisi epäillä sappitievauriota (n.~1\% riski; yleensä sappilekaasi, harvemmin sappitiestriktuura). Toki myös esim. sappitiekivi tai post.op. verenvuoto mahdollisia vaivan aiheuttajia.

  \begin{itemize}
  \tightlist
  \item
    Tutkitaan Pvk, CRP, maksa-arvot ja tärkeänä vatsan UÄ. Herkästi myös konsultoidaan kirurgia.
  \end{itemize}
\item
  \textbf{Lekaasin hoitona antibiootti, UÄ-ohjattu dreneeraus ja ERC-teitse sappitie-endoproteesi}.

  \begin{itemize}
  \tightlist
  \item
    Vaikeimmat sappitievauriot (vaikeat Amsterdam B, C ja D; Strasberg E1--5) hoidetaan yleensä päivystyksellisellä avoleikkauksella. Yhteisen sappitiehyen katkeaminen (Amsterdam D) hoidetaan aina avoleikkauksella. Avoleikkausta harkitaan myös, mikäli sappitievaurioon liittyy merkittävä verisuonivaurio.
  \end{itemize}
\end{itemize}

Iatrogeeniset sappitievauriot tulisi pyrkiä ennaltaehkäisemään ja tässä on tärkeintä itse leikkauksen taidokas suorittaminen. Sappitievaurioille altistavia tekijöitä ovat muun muassa potilaan korkea ikä ja ylipaino, poikkeamat sappiteiden anatomiassa (noin kolmasosalla potilaista), akuutti ja krooninen kolekystiitti sekä runsas verenvuoto leikkauksen aikana

\begin{itemize}
\tightlist
\item
  Kaikkein keskeisintä sappitievaurioiden ennaltaehkäisyssä on Calot'n kolmion (sappirakon tiehyen, sappirakon arterian ja yhteisen sappitiehyen muodostama kolmio) preparointi huolellisesti ja oikeaoppisesti. Jos tätä anatomiaa ei tunnista leikatessaan, niin voi helposti vaurioittaa sappiteitä.
\end{itemize}

\pandocbounded{\includegraphics[keepaspectratio]{images/calot.png}}

\section{Potilas laihtunut 10kg 3kk aikana ja ikterus. Tt:ssä haimasyöpä. Mikä syöpä todennäköisimmin?}\label{potilas-laihtunut-10kg-3kk-aikana-ja-ikterus.-ttssuxe4-haimasyuxf6puxe4.-mikuxe4-syuxf6puxe4-todennuxe4kuxf6isimmin}

Ei vaihtoehtoja

Vastaus mahdollisesti ollut tyyppiä: \textbf{Haiman pään (caput) duktaalinen adenokarsinooma}

\begin{itemize}
\tightlist
\item
  Todennäköisesti kyseessä on haiman pään syöpä, koska potilaalla on ikterus ja haiman pään syöpä voi usein painaa sappiteitä ja aiheuttaa obstruktiivisen ikteruksen (sappistaasi)
\item
  Haimasyövistä 85--90 \% on tyypiltään duktaalisia adenokarsinoomia, joten se on myös todennäköisesti tässäkin kyseessä

  \begin{itemize}
  \tightlist
  \item
    Erotusdiagnostiikassa tulee huomioida muut kasvaimet ja kasvainten etäpesäkkeet, krooninen haimatulehdus ja autoimmuunihaimatulehdus. Hyvistä kuvantamismenetelmistä huolimatta pahanlaatuisen kasvaimen ja tulehduksellisen muutoksen erottaminen toisistaan voi välillä olla hyvin haastavaa.
  \end{itemize}
\end{itemize}

Mikäli kuvausten perusteella epäillään vahvasti haimasyöpää ja kasvain vaikuttaa paikalliselta, kudosnäytteitä ei tarvita, vaan voidaan edetä suoraan leikkaukseen

\begin{itemize}
\tightlist
\item
  Ikterus ei ole este leikkaukselle, mikäli leikkausaika saadaan järjestymään nopeasti, eikä potilaalla ole kolangiittia tai elinhäiriöitä
\item
  Sappitiestentin asetus ennen leikkausta suurentaa infektiokomplikaatioiden riskiä, joten jos leikkaukseen on mahdollista edetä suoraan, ei sappiteiden stenttausta suositella
\item
  Onkologisia hoitoja ei voida antaa ikteeriselle potilaalle
\item
  Mikäli sappitietukos tarvitsee laukaista, näytteet voidaan saada ERCP:n yhteydessä (harjairtosolunäyte)
\end{itemize}

\section{Potilas 40v nainen oksentanut punaista verta kahvikupillisen. Sama toistuu päivystyksessä. Hb 144 ja tilanne nyt stabiili. Mitä teet?}\label{potilas-40v-nainen-oksentanut-punaista-verta-kahvikupillisen.-sama-toistuu-puxe4ivystyksessuxe4.-hb-144-ja-tilanne-nyt-stabiili.-mituxe4-teet}

\begin{itemize}
\tightlist
\item
  \begin{enumerate}
  \def\labelenumi{\alph{enumi}.}
  \tightlist
  \item
    Kotiin
  \end{enumerate}
\item
  \begin{enumerate}
  \def\labelenumi{\alph{enumi}.}
  \setcounter{enumi}{1}
  \tightlist
  \item
    Sairaalaan + PPI + ruoka sallittu
  \end{enumerate}
\item
  \begin{enumerate}
  \def\labelenumi{\alph{enumi}.}
  \setcounter{enumi}{2}
  \tightlist
  \item
    Sairaalaan + PPI x2 + ravinnotta
  \end{enumerate}
\item
  \begin{enumerate}
  \def\labelenumi{\alph{enumi}.}
  \setcounter{enumi}{3}
  \tightlist
  \item
    Kotiin + lähipäivinä gastroskopia
  \end{enumerate}
\end{itemize}

\begin{solution}
\leavevmode

Vastaus

\begin{verbatim}
 c
\end{verbatim}

Ensihoito ylä-GI-vuodossa on suoniyhteyden avaaminen ja punasolusiirtorajan asettaminen (usein 70g/l stabiileilla potilailla, 90ish g/l instabiileilla tai jos krooninen sydänsairaus), korkea-annoksisen protonipumpun estäjä (PPI)-hoidon aloittaminen (esim. pantopratsoli 80 mg i.v. jatkuen infuusiona 8 mg/t; annetaan siis normaalia isompi annos, johon PPI x2 viittaa) ja jos potilaalla maksakirroosi tai epäillään vuodon syyksi ruokatorven suonikohjuja -\textgreater{} ensiavussa vasopressiini- (Glypressin®) tai somatostatiinianalogilääkitys (Sandostatin®) + antibiootti tähystystä odotellessa.

a: Potilas on oksentanut verta (hematemeesi) kahdesti. Kyseessä on merkittävä ylä-GI-vuoto (aina vaarallinen), vaikka tilanne olisi juuri nyt stabiili eikä potilasta voi kotiuttaa, vaan vaatii sairaalahoitoa ja pikaisen gastroskopian. Skopiassa etsitään vuotopaikka eli selvitetään diagnoosi, saadaan käsitys verenvuodon merkeistä ja hoidetaan vuoto endoskooppisesti. Suurin osa vuodon syynä olevista löydöksistä ei tarvitse mitään muuta hoitoa. Jos potilaalla ei katsota olevan uusintavuodon riskiä, hänet voidaan turvallisesti kotiuttaa

b: Potilaan tulee olla ravinnotta, koska kohta tullaan tekemään gastroskopia.

c: Gastroskopia tulee tehdä päivystysluonteisesti saman vuorokauden aikana tai viimeistään seuraavana aamuna potilaan voinnin ja sairaalan palvelujen mukaan. Mikäli nesteytys ja verensiirrot eivät riitä korjaamaan hemodynamiikkaa ja vuoto jatkuu, skopia pitää tehdä heti.

\end{solution}

\section{Potilaalla haiman adenokarsinooma. Nyt päivystykseen keltaisena ja todetaan sappistaasi. Mikä palliatiivinen tmp?}\label{potilaalla-haiman-adenokarsinooma.-nyt-puxe4ivystykseen-keltaisena-ja-todetaan-sappistaasi.-mikuxe4-palliatiivinen-tmp}

Ei vaihtoehtoja, mutta koita vastata ilman vinkkejä

\begin{solution}
\leavevmode

Vastaus

\begin{verbatim}
 ERCP-teitse asetettava stentti
\end{verbatim}

Haimakasvaimen aiheuttaman sappitietukoksen ensisijainen hoito, jos tuumorin leikkaushoito ei ole indikoitua (palliatiivinen hoito) on siis stenttaus. Jos potilas leikataan kasvaimen vuoksi nopeasti, turhaa stenttausta pyritään välttämään, koska se lisää leikkauksen jälkeisten komplikaatioiden riskiä.

Tukoksen toissijainen laukaisumenetelmä on ihon ja maksan läpi sappiteihin radiologisesti läpivalaisussa asetettava dreeni (PTD).

\pandocbounded{\includegraphics[keepaspectratio]{images/haimasyöpäsappistentti.png}}

\end{solution}

\section{Mikä on virtsarakkosyövän tärkein riskitekijä?}\label{mikuxe4-on-virtsarakkosyuxf6vuxe4n-tuxe4rkein-riskitekijuxe4}

Ei vaihtoehtoja, mutta koita vastata ilman vinkkejä

\begin{solution}
\leavevmode

Vastaus

\begin{verbatim}
 Tupakointi
\end{verbatim}

Munuaissyövän tärkeimmät riskitekijät ovat tupakointi (tärkein), lihavuus ja kohonnut verenpaine. Tupakoinnin lopettaminen ja ylipainon vähentäminen ovat tärkeimmät keinot vaikuttaa munuaissyöpäriskiin.

\end{solution}

\section{Makrohematuria alkanut potilaalla 4h sitten. Mitä teet?}\label{makrohematuria-alkanut-potilaalla-4h-sitten.-mituxe4-teet}

Ei vaihtoehtoja, tässä mahdollinen toimintamalli:

\begin{enumerate}
\def\labelenumi{\arabic{enumi})}
\item
  Anamneesi + status (tuseeraus myös)
\item
  Labrat: U-KemSeul, U-Solut x2, U-BaktVi
\item
  Makrohematurian aiheuttaja on löydettävissä yli 90 \%:ssa tapauksista. Tärkeimpiä poissuljettavia sairauksia ovat virtsateiden kasvaimet sekä glomerulussairaudet. Tulee aina tehdä lähete urologisiin jatkotutkimuksiin, ellei syy ole virtsatietulehdus tai verikontaminaatio. Edeltävästi on suositeltavaa ohjelmoida virtsan irtosolututkimus ja munuaisten kaikututkimus. Jos epäillään IgA-nefropatiaa (samanaikainen valkuaisvirtsaisuus, alentunut eGFR, korkea verenpaine), laaditaan lähete nefrologille
\end{enumerate}

Ei siis ole tarvetta todennäköisesti päivystyksellisille toimille (kystoskopia kiireellisenä on seuraava), koska verenvuoto hematuriassa ei usein ole massiivista. Makroskooppiseen hematuriaan vaaditaan vain 1 ml verta / 1 000 ml virtsaa. Jos hyytymiä tai rakko ei pysty tyhjentymään, niin vaatii päivystyksellistä hoitoa ja asetetaan huuhtelukatetri.

\section{Reumaatikolla sääressä laskimoverta vuotava haava, antikoagulaatio. Mitä teet?}\label{reumaatikolla-suxe4uxe4ressuxe4-laskimoverta-vuotava-haava-antikoagulaatio.-mituxe4-teet}

Ei vaihtoehtoja, mutta tässä mahdollinen toimintamalli:

Vuoto tyrehdytetään alaraajan kohoasennolla, kompressiolla ja kylmäpussilla. Vuotaneen laskimon voi ensihoidon jälkeen sulkea ompeleella tai laapiksella. Kun vuoto on tyrehdytetty, potilas lähetetään elektiivisesti erikoissairaanhoitoon laskimovajaatoiminnan hoidon suunnittelemiseksi. Ei tarvitse tauottaa AK-hoitoa rutiinisti, ellei vuoto ole massiivista.

\begin{itemize}
\tightlist
\item
  Mikäli laskimopaine on jatkuvasti koholla esimerkiksi sydämen vajaatoiminnan tai maksakirroosin vuoksi, pullistunut pintalaskimo saattaa vuotaa. Vuotoalttiutta lisäävät potilaan ikä, yleinen hauraus ja ihon heikko kunto, sekä glukokortikoidien ja verenohennuslääkkeiden käyttö. Vaikka kohjuvuoto on potilaan näkökulmasta tilanteena pelottava, sen ennuste on hyvä.
\end{itemize}

\section{25v nainen, hypotyreoosi, rinnasta vaaleaa erittävää nestettä.}\label{v-nainen-hypotyreoosi-rinnasta-vaaleaa-erittuxe4vuxe4uxe4-nestettuxe4.}

Ei vaihtoehtoja tai kysymyksenasettelua. Todennäköisesti kysytty, mitä tehdään seuraavaksi. Tässä toimintamalli: Palpoi rinnat, tutki TSH ja prolaktiini. Jos erite on molemminpuolista ja labrat viittaa selittävään syyhyn (prolaktiini koholla, TSH koholla), niin syynä on todennäköisesti hypotyreoosin stimuloima maidoneritys ja se hoituu ilman lisäkuvantamisia perussairauden hoidon tehostamisella.

Periaatteita rinnan erityksestä:

Suurin osa rinnan erityksen taustalla olevista syistä on hyvänlaatuisia.

Erityksen konsistenssi ja väri voivat vaihdella ja on tärkeää selvittää: se voi olla paksua tahnaa tai ohutta ja vetistä; vihreää, ruskeaa, selkeän veristä tai keltaista ja kirkasta

\begin{itemize}
\tightlist
\item
  Patologinen, lisätutkimuksia aiheuttava erite on yksipuoleista, seroosista tai veristä. Patologinen erite tulee siis spontaanisti vain toisesta rinnasta ja yhdestä tiehyestä. Syövän riski on suurempi verisen kuin seroosisen eritteen taustalla.

  \begin{itemize}
  \tightlist
  \item
    Erittämisen perusselvittelyihin avoterveydenhuollossa kuuluu kliinisen tutkimuksen lisäksi mammografia (yli 35-vuotiailla) ja rintojen kaikukuvaus. Mikäli näissä kuvissa näkyy pesäke, tulee tästä ottaa paksuneulanäyte rintasyövän poissulkemiseksi. Jos eritys on kirkasta, selvästi veristä tai oluen väristä, erittävä tiehyt tutkitaan rintatiehyen varjoainekuvauksella eli duktografialla, jolloin erittävään rintatiehyeen ruiskutetaan varjoainetta, jonka jälkeen rinta kuvataan mammografialla. Näin mahdollinen rintatiehyen patologinen tukkeuma kuvautuu puutosvarjona varjoaineella täyttyneessä tiehyessä

    \begin{itemize}
    \tightlist
    \item
      Kirurgin konsultaatiota tarvitaan, mikäli kuvantamalla todetaan rintatiehyen sisäinen kasvain tai rintasyöpä tai jos tilanne jää epäselväksi.
    \end{itemize}
  \end{itemize}
\end{itemize}

\textbf{Galaktorrea eli maidoneritys (tai sen kaltaisen nesteen eli vaalea/vihertä/kellertävä/rusehtava/sinertävä/harmaa erite) miehillä, murrosikäisillä ja lapsilla on selvitettävä pikaisesti ja tarvittaessa endokrinologia konsultoiden, mutta naisilla se on hyvänlaatuinen ilmiö eikä ole aihe kuvantamiselle}

\begin{itemize}
\tightlist
\item
  Kaikilta galaktorreapotilailta on tutkittava prolaktiini ja kilpirauhasen toimintakokeet, miehiltä myös estradiolipitoisuus

  \begin{itemize}
  \tightlist
  \item
    Useat, erityisesti prolaktiinipitoisuutta suurentavat lääkkeet kuten trisykliset masennuslääkkeet, monet psykoosilääkkeet, yhdistelmäehkäisypillerit, metoklopramidi, antihistamiinit, verapamiili ja isoniatsidi voivat aiheuttaa maidoneritystä

    \begin{itemize}
    \tightlist
    \item
      Suurentuneen prolaktiinipitoisuuden yhteydessä otetaan lääkkeiden lisäksi huomioon prolaktinooman tai muun sellan alueen ekspansion mahdollisuus
    \end{itemize}
  \item
    \textbf{Hypotyreoosi voi aiheuttaa hyperprolaktinemian (primaarinen hypotyreeosi -\textgreater{} TRH nousee -\textgreater{} stimuloi prolaktiinin tuotantoa), joka korjautuu hoidon myötä lääkityksellä.}
  \end{itemize}
\end{itemize}

Ydinasiat rinnan erityksestä:

\begin{itemize}
\tightlist
\item
  Maitomainen, molemminpuolinen eritys rinnasta naisilla on hyvänlaatuinen ilmiö. Kaikilta galaktorreapotilailta on tutkittava prolaktiini ja kilpirauhasen toimintakokeet, miehiltä myös estradiolipitoisuus.
\item
  Syöpäepäilyn aiheuttava rintaerite on seroosinen tai verinen.

  \begin{itemize}
  \tightlist
  \item
    Perustutkimuksiin kuuluvat mammografia, rinnan kaikukuvaus ja duktografia. Maailmalla monessa paikassa tehdään eritteen vuoksi suoraan magneettikuvaus, mutta ei Suomessa alkututkimuksena.
  \end{itemize}
\item
  Rinnan eritykseen liittyvät konsultaatiot osoitetaan rintakirurgille, ei gynekologille.
\end{itemize}

\pandocbounded{\includegraphics[keepaspectratio]{images/rintaeritys.png}}

\section{Mitä tubulaarinen rinta tarkoittaa?}\label{mituxe4-tubulaarinen-rinta-tarkoittaa}

Ei vaihtoehtoja, mutta koita vastata ilman vinkkejä

\begin{solution}
\leavevmode

Vastaus

\begin{verbatim}
 Rinnan kehityspoikkeamma
\end{verbatim}

Rinnan sidekudoksinen juoste estää normaalin kasvun. Rinnat sijaitsevat kaukana toisistaan, rinta on tyvestään kapea, rinnanaluspoimu on usein alikehittynyt tai korkea. Nänninpiha voi olla voimakkaasti laajentunut, jolloin rauhaskudosta työntyy sen alla olevan rengasmaisen sidekudosvanteen läpi aiheuttaen nännin pullotusta (nänni työntyy tyrämäisen renkaan läpi). Kehityspoikkeama on usein molemminpuolinen, mutta sitä tavataan myös yksipuoleisena tai esimerkiksi Polandin oireyhtymän (synnynnäinen toispuoleinen rintakehän kehityspoikkeama, jossa iso rintalihas (musculus pectoralis major) puuttuu kokonaan (aplasia) tai osittain (hypoplasia)) yhteydessä.

Selvää toiminnallista ja psykososiaalista haittaa aiheuttavat poikkeamat ovat julkisen terveydenhuollon leikkaushoidon piirissä. Lievimpiä muotoja tubulaarisista rinnoista on toisinaan vaikea erottaa normaalista rinnasta, eivätkä ne vaadi leikkaushoitoa.

Tubulaarisen rinnan hoidossa rasvansiirto antaa luonnollisen ja pysyvän tuloksen. Rasvansiirtoja tarvitaan yleensä 1--3 kertaa, mutta tämän jälkeen lopputulos on pysyvä. Tarvittaessa rasvansiirron yhteydessä tai joskus ainoana toimenpiteenä laajaa nänninpihaa pienennetään sekä tarvittaessa nänninpiha ja nänni kohotetaan oikealle paikalleen. Tubulaarisen kehityspoikkeaman korjaamisessa käytetään myös silikoni-implantteja esteettistä rintojen suurentamista vastaavalla tekniikalla.

\pandocbounded{\includegraphics[keepaspectratio]{images/tubulaarinen.png}}
\pandocbounded{\includegraphics[keepaspectratio]{images/tubulaarinenkorjattu.png}}

\end{solution}

\section{Rintojen pienennysleikkaus, mitä laitat lähetteeseen?}\label{rintojen-pienennysleikkaus-mituxe4-laitat-luxe4hetteeseen}

Ei vaihtoehtoja, mutta tässä alla olevassa kuvassa lähetteeseen kuuluvat asiat.

\begin{itemize}
\tightlist
\item
  Erityisesti \textbf{kookkaista rinnoista aiheutuvat oireet, BMI (leikkauskriteeri on Tyksissä \textless28) ja tupakoimattomuus (kriteeri on vähintään 6kk tupakoimatta ennen leikkausta).} Monessa julkisessa yksikössä tupakoimattomuus on ehto leikkausjonoon asettamiselle.

  \begin{itemize}
  \tightlist
  \item
    Monesti potilaat vielä testataan ennen leikkausta näiden toteutumisen varmistamiseksi; pari päivää ennen leikkausta potilaalta voidaan mitata nikotiinin metaboliitti (kotiniini esim. virtsasta) ja varmistaa tupakoimattomuus. Samoin potilas punnitaan ja jos leikkauskriteeri eli BMI \textless28 ei täyty, niin leikkausta ei ehkä tehdä, vaan se siirretään myöhemmäksi, kunnes potilas on laihtunut tarpeeksi.
  \end{itemize}
\end{itemize}

\pandocbounded{\includegraphics[keepaspectratio]{images/pienennysleikkauslähete.png}}

\section{Brickerin avanne on:}\label{brickerin-avanne-on}

\begin{itemize}
\tightlist
\item
  \begin{enumerate}
  \def\labelenumi{\alph{enumi}.}
  \tightlist
  \item
    suolirakko
  \end{enumerate}
\item
  \begin{enumerate}
  \def\labelenumi{\alph{enumi}.}
  \setcounter{enumi}{1}
  \tightlist
  \item
    kontinentti avanne
  \end{enumerate}
\item
  \begin{enumerate}
  \def\labelenumi{\alph{enumi}.}
  \setcounter{enumi}{2}
  \tightlist
  \item
    valuva avanne
  \end{enumerate}
\item
  \begin{enumerate}
  \def\labelenumi{\alph{enumi}.}
  \setcounter{enumi}{3}
  \tightlist
  \item
    ureterokutaneostomia
  \end{enumerate}
\end{itemize}

\begin{solution}
\leavevmode

Vastaus

\begin{verbatim}
 c
\end{verbatim}

Virtsateiden rekonstruktio voi olla tarpeen esim. rakkosyövän hoidon yhteydessä, jos rakko poistetaan tai jopa silloin, kun yliaktiivinen rakko ei vastaa muulle hoidolle. Virtsateiden rekonstruktio voidaan toteuttaa usein eri tavoin. Tyypillisimpinä vaihtoehtoina ovat pidätyskyvytön avanne (``Bricker''), normaalia kautta tyhjenevä ortotooppinen suolirakko (``Studer'') tai pidätyskykyinen avanne (``Kock'', ``Mitrofanoff'', ``Indiana'').

Eugen Brickerin vuonna 1950 julkaisema ureteroileokutaneostomia on edelleen käytetyin virtsadiversion leikkausmenetelmä ja tavallisin pidätyskyvytön virtsa-avanne. Tässä niin sanotussa Brickerin leikkauksessa virtsanjohtimet yhdistetään sykkyräsuolen loppuosasta (terminaalinen ileum) eristetyn 15--20 cm:n pituisen osan vatsaontelon puoleiseen päähän. Loppupää tuodaan vatsanpeitteiden läpi tyypillisesti oikealle alavatsalle avanteeksi. Jos sykkyräsuolen loppuosaa ei voida käyttää, voidaan käyttää paksusuolen osaa.

a: Vaihtoehtona virtsa-avanteelle on valikoiduissa tapauksissa ortotooppinen suolirakko, joka on normaalia kautta tyhjenevä virtsarakon korvike (``Studer''-rekonstruktio). Poistettu rakko voidaan korvata suolesta tehdyllä säiliöllä (suolirakko) ja yhdistetään paikoilleen jätettyyn virtsaputkeen. Suolirakkorekonstruktion jälkihoito on potilaalle avannetta työläämpi. Virtsaamisen opettelu vie aikaa ja suolirakkoon liittyy varsinkin yöaikainen virtsankarkailu ja ajoittainen tarve katetroida virtsarakko ja huuhdella siihen kertynyttä suolilimaa. Myös suolirakkopotilaiden metabolisen asidoosin kehittymisen riski on otettava huomioon leikkauksen jälkeen.

b: Kontinentti (ei-inkontinenssia) avanne tarkoittaa, että pystyt kontrolloimaan, milloin päästät virtsaa valumaan eli kyseessä on pidätyskykyinen virtsa-avanne. Jotkut rekonstruktiot (kuten ortotooppinen suolirakko) voivat mahdollistaa sen, että voi virtsata samalla tavalla kuin ennen leikkausta, mutta tietyt avannevaihtoehdot taas vaativat sen, että virtsa katetroidaan ulos (katetroitava virtsa-avanne; näistä yleisimpiä ovat Kockin, Mitrofanoffin tai Indianan pussi).

d: Yksinkertaisimmassa pidätyskyvyttömässä avanteessa virtsajohtimet nostetaan leikkauksessa suoraan iholle. Tätä kutsutaan ureterokutaneostomiaksi. Ureterokutaneostomia on toimenpiteenä suhteellisen harvinainen ja tehdään yleensä tilanteissa, joissa suolta ei voida käyttää avanteen rakentamiseksi tai halutaan minimoida leikkauksenjälkeisen toipumisen riskit. Virtsanjohtimet tuodaan kumpikin iholle erillisinä avanteina. Tietyissä tilanteissa virtsanjohtimet voidaan yhdistää ja tuoda iholle samasta reiästä.

\pandocbounded{\includegraphics[keepaspectratio]{images/bricker.png}}
\pandocbounded{\includegraphics[keepaspectratio]{images/studer.png}}
\pandocbounded{\includegraphics[keepaspectratio]{images/katetroitavakontinentti.png}}

\end{solution}

\section{Martorellin ulkus}\label{martorellin-ulkus}

Ei vaihtoehtoja, mutta tässä tärkeimmät:

\textbf{Hypertensiivinen säärihaava} eli Martorellin haava on epätavallinen, mutta ei harvinainen säärihaavan muoto. Martorellin haava on pitkään huonossa hoitotasapainossa olleen verenpainetaudin komplikaatio 40-80-vuotiailla potilailla, joista osa sairastaa myös diabetesta.

\begin{itemize}
\tightlist
\item
  Kliininen kuva on selkeä ja tunnusomainen: kyseessä on erittäin kivulias pinnallinen nekroottinen haava, jossa on purppuran punainen reunus. Paraneminen on yleensä hidasta.
\end{itemize}

Martorellin haavan taustalla on mikroangiopatia. Verenpainetautia sairastavilla on suuri ääreisvaltimoiden vastus huolimatta normaalista nilkka-olkavarsipainesuhteesta. Tämä johtuu pienten valtimoiden ahtautumisesta tai tukkeutumisesta. Seurauksena on kudosperfuusion vähentyminen, paikallinen iskemia sekä haavoja ja kipua. Pienet valtimot eivät reagoi normaalisti laajenemalla, mikä johtaa jopa ihon infarkteihin.

Martorellin haava on yleensä resistentti tavanomaisille paikallishoidoille. Potilaalle tulisi kertoa erikoisesta taudinkulusta ja huonosta hoitovasteesta.

\begin{itemize}
\tightlist
\item
  Paikallishoidon onnistumisen kannalta kivun lievittäminen on ensiarvoisen tärkeää. Parasetamolin ja tulehduskipulääkkeiden käyttöä on suositeltu potilaan ikä ja perussairaudet huomioiden. Opioidien (laastarimuodossa tai suun kautta annettuna) tai niiden johdoksien käyttö voi olla tarpeellista ennen haavan hoitoa. Euforisoivien analgeettien, pregabaliinin tai trisyklisten masennuslääkkeiden yhdistäminen lääkehoitoon on suotavaa neuropaattisen kivun vuoksi.
\item
  Kortikosteroideja on käytetty kipua aiheuttavaan tulehdusreaktioon. Paikallisen vahvan kortikosteroidin (klobetasoli-17-propionaatti) on todettu lievittävän kipua, mutta tällaista lääkitystä käytetään vain muutama päivä kerrallaan.
\item
  Verenpaineen hoito on tärkeää, sillä se vähentää kipua ja pienentää haavojen kokoa. Epäselektiivisten beetasalpaajien käyttöä tulisi välttää, koska ne aiheuttavat paikallista vasokonstriktiota ja vähentävät ihon kudosperfuusiota, mikä heikentää haavan paranemista. Tehokas verenpainetaudin hoito ei kuitenkaan yksin paranna haavoja
\item
  Ihonsiirto on nykyisin lupaavin ja suositelluin hoitomuoto Martorellin haavassa, mutta tutkimukset sen vaikuttavuudesta puuttuvat
\end{itemize}

\pandocbounded{\includegraphics[keepaspectratio]{images/martorelli.png}}
\pandocbounded{\includegraphics[keepaspectratio]{images/martorellileo.png}}

\section{Akuutti alaraajaiskemia mikä tärkein aiheuttaja?}\label{akuutti-alaraajaiskemia-mikuxe4-tuxe4rkein-aiheuttaja}

Ei vaihtoehtoja, mutta koita vastata ilman vinkkejä

\begin{solution}
\leavevmode

Vastaus

\begin{verbatim}
 Tromboosi (n. 50%)
\end{verbatim}

Ateroskleroottisen plakin ruptuura ja tromboosi. Usein potilaalla on ollut aikaisemmin alaraajaiskemiaoireita ASO-taudista johtuen ja akuutti alaraajaiskemia voi siis syntyä kroonisen iskemian (ja siten kriittisenkin iskemian) päälle, jolloin on ns. acute on chronic -alaraajaiskemia. Monesti aikaisempia oireita ei kuitenkaan ole johtuen siitä, että ASO-tautipotilaiden toiminta-aste on usein niin matala, että he eivät koe iskeemisiä oireita (ei sohvalla makaaminen paljoa rasita jalkoja).

Aiemmin tavallisin syy oli embolia (pääasiassa eteisvärinästä), mutta AK-hoidon myötä ei enää ole nro 1 (on 2. yleisin syy nykyään eli n.~40\% tapauksista). Muita syitä ovat muun muassa polvinivelen sijoiltaanmeno, puukotusvamma ja aortan dissekoituma.

Acute on chronic -alaraajaiskemian oireet ovat yleensä muita reittejä syntyneen akuutin iskemian (oireet alle 2vk) oireisiin verrattuna lievempiä. Tämä johtuu siitä, että krooninen iskemia on johtanut kollateraalisuonien muodostumiseen -\textgreater{} iskemia on vähäisempää tromboosissa. Esim. motoriikan heikkenemisen ja tunnottomuuden vaikeusaste riippuvat hapenpuutteen vaikeudesta ja akutisoituneessa kroonisessa iskemiassa motoriikka sekä sensoriikka voivat olla hyvinkin normaalit juuri kollateraalisuonista johtuen.

\pandocbounded{\includegraphics[keepaspectratio]{images/acuteonchronic.png}}

\end{solution}

\section{Akuutti alaraajaiskemia hoito}\label{akuutti-alaraajaiskemia-hoito}

Ei vaihtoehtoja, tässä tärkeimmät:

Tyypilliset oireet, jotka tulee osata tunnistaa voi muistaa 6P-muistisäännöstä: pain, parestesia, pallor, pulselessness, poikilothermia (viileys), paralysis

Oirekuvan toteamisen jälkeen sopiva seuraava temppu on lähete päivystykselliseen verisuonikirurgian arvioon. Akuutin alaraajaiskemian tarkempi hoito ja sen kiireellisyys riippuu iskemian vaikeusasteen arviosta. Luokittelussa käytetään Rutherfordin luokitusta, jossa vaikeusaste jaetaan kolmeen luokkaan alaraajan elinkelpoisuuden perusteella (taulukko alla olevassa kuvassa).

\begin{itemize}
\tightlist
\item
  Ensihoitona annetaan ASA 250mg (jos potilaalla ei ole säännöllistä ASA-lääkitystä). Jos raajan tilanne sallii revaskularisaatiotoimenpiteen odotuksen esimerkiksi seuraavan aamuun (Rutherford I--IIa), annetaan pienimolekyylista hepariinia (esim. enoksapariini 1 mg/kg ihon alle) lisähyytymisen estämiseksi ennen toimenpidettä. Akuutin alaraajaiskemian alkuvaiheen hoidossa on myös tärkeä huolehtia riittävästä nesteytyksestä ja kivun hoidosta. Lisähapesta voi myös olla hyötyä.
\end{itemize}

Rutherfordin luokista erityisesti luokka II on tärkeä erottaa, sillä tällöin nopealla toiminnalla raaja pystytään vielä pelastamaan suurella todennäköisyydellä.

\begin{itemize}
\tightlist
\item
  Mikäli raajassa ei ole sensorista eikä motorista alenemaa ja valtimosignaalit ovat kuultavissa dopplerkaikukuvauksella, ei ole tarvetta päivystyksellisiin toimenpiteisiin (on siis Rutherford luokka 1)
\item
  Mikäli akuutin alaraajaiskemian oireena on ainoastaan lievä sensorinen alenema (Rutherford IIa), hoito pitää toteuttaa päivystyksellisesti mutta ei välttämättä välittömästi (alle 24 tuntia)
\item
  Välitön (0--6 tuntia) revaskularisaatio on aiheellinen, kun akuutissa alaraajaiskemiassa oireena on motorinen puutos, mutta arvioidaan, että raaja on revaskularisaatiolla säästettävissä (Rutherford IIb); eli siis ei ole täydellistä motoriikan ja sensoriikan puuttumista
\item
  Alaraajan ollessa peruuttamattomasti iskeeminen (Rutherford 3) on tehtävä amputaatio päivystyksellisesti (alle 24 tuntia). Kun kyseessä on elämän loppuvaiheessa oleva hauras, monisairas vanhus, voidaan myös valita palliatiivinen hoitolinja eli hyvä kivun hoito ja saattohoito. Tällöin hoitolinjasta on aina keskusteltava sekä potilaan että omaisten kanssa.
\end{itemize}

Akuutissa alaraajaiskemiassa välitön kuvantaminen on tarpeellista, jos potilaalla on Rutherfordin IIa/IIb -luokan iskemia.

\begin{itemize}
\tightlist
\item
  Mahdollisia kuvantamiskeinoja ovat TT-angiografia (CTA), magneettiangiografia (MRA) ja ultraääni. Ensisijaisena kuvantamistutkimuksena akuutissa alaraajaiskemiassa tulisi pitää TT-angiografiaa. Kuvantaminen ei kuitenkaan saa viivästyttää kiireellisissä tapauksissa revaskularisaatiotoimenpiteitä!

  \begin{itemize}
  \tightlist
  \item
    Luokan 1 iskemiassa kuvantaminen viimeistään seuraavana arkiaamuna; ei kuitenkaan siis ole tarvetta tehdä ihan heti (24h sisällä)
  \end{itemize}
\end{itemize}

Revaskularisaatio:

\begin{itemize}
\tightlist
\item
  Rutherfordin IIa -luokan akuutin alaraajaiskemian ensisijainen revaskularisaatiometodi on paikallinen trombolyysi. Trombolyysissä valtimon tukkiva tromboosi liuotetaan injektoimalla katetriteitse tPA-aktivaattoria (esim. alteplaasia eli Actilyse) suoraan trombiin, minkä jälkeen kuvauksella paljastetaan tukkeutumisen syy, usein paikallinen ahtauma tai valtimovaurio, ja hoidetaan se suonensisäisesti. Trombolyysi avaa myös kollateraaleja, joita ei voida hoitaa kirurgisesti. Trombolyysipotilaille jatketaan lisäksi pienimolekyylisen hepariinin antoa vähintään viikon ajan trombolyysin aloittamisesta (tromboosipotilaille määrätään myös yleisesti pysyväislääkkeeksi ASA 100 mg × 1 ja jos trombolyysihoidettu niin tämän päälle LMWH 1vk).

  \begin{itemize}
  \tightlist
  \item
    Akuutissa syvässä iskemiassa trombolyysi (IIb-luokitus) ei tule kyseeseen, koska hoitotulos saavutetaan vasta tuntien kuluttua.
  \end{itemize}
\item
  Rutherfordin IIb -luokan akuutin alaraajaiskemian ensisijainen revaskularisaatiometodi on embolektomia embolian tilanteessa tai tromboosissa angiografian perusteella valittu jokin trombin hoitokeino (kirurginen trombektomia/endarterektomia, aspiraatiotrombektoimia tai mekaaninen trombektomia)
\end{itemize}

Akuutissa alaraajaiskemiassa revaskularisaation jälkeen raajan tilannetta on seurattava säännöllisesti, koska voi tapahtua reperfuusiovaurio ja sen ilmentymänä aitiopaineoireyhtymä ja rabdomyolyysi. Mikäli kehittyy aitiopaineoireyhtymän oireita, on faskiotomiat tehtävä herkästi ja ilman viiveitä (\textless6h).

\begin{itemize}
\tightlist
\item
  Mekaanisen trombektomian jälkeen tehdään tyypillisesti aina faskiotomiat. Jos jo lähtötilanteessa todetaan sensomotorisia puutoksia, palpoiden kireät ja aristavat lihakset tai plasman suuri myoglobiinipitoisuus (yli 5 000 µg/l), kannattaa faskiotomiat tehdä profylaktisesti revaskularisaation yhteydessä
\end{itemize}

\pandocbounded{\includegraphics[keepaspectratio]{images/alaraajaiskemiaalgoritmi.png}}
\pandocbounded{\includegraphics[keepaspectratio]{images/6p.png}}
\pandocbounded{\includegraphics[keepaspectratio]{images/revask.png}}

\section{Kriittisen kroonisen alaraajaiskemian oire on?}\label{kriittisen-kroonisen-alaraajaiskemian-oire-on}

\begin{itemize}
\tightlist
\item
  \begin{enumerate}
  \def\labelenumi{\alph{enumi})}
  \tightlist
  \item
    leposärky
  \end{enumerate}
\item
  \begin{enumerate}
  \def\labelenumi{\alph{enumi})}
  \setcounter{enumi}{1}
  \tightlist
  \item
    katkokävely
  \end{enumerate}
\item
  \begin{enumerate}
  \def\labelenumi{\alph{enumi})}
  \setcounter{enumi}{2}
  \tightlist
  \item
    lateraalisesti indusoituva kipu erityisesti lonkan fleksiossa
  \end{enumerate}
\item
  \begin{enumerate}
  \def\labelenumi{\alph{enumi})}
  \setcounter{enumi}{3}
  \tightlist
  \item
    joku (ei wikissä)
  \end{enumerate}
\end{itemize}

\begin{solution}
\leavevmode

Vastaus

\begin{verbatim}
 a
\end{verbatim}

Kriittinen iskemia (CLI/CLTI) viittaa krooniseen raajaa uhkaavaan iskemiaan. Keskeisimmät oireet ja löydökset ovat lepokipu ja/tai kudosvaurio (haava tai kuolio), jonka on todettu johtuvan alentuneesta verenkierrosta (eli siis \textgreater2vk parantumaton valtimohaava tai lepokipu + ABI \textless{} 0,4 tai nilkkapaine alle 50 mmHg tai varvaspaine alle 30 mmHg).

Lepokipu ilmenee tyypillisesti yöllä vaaka-asennossa. Kipua lievittää jalan riiputtaminen alaspäin, istuva asento (potilas usein nukkuu istuen) tai pystyasentoon nousu (potilas usein tulkitsee kävelyn auttavan).

b: Katkokävely eli klaudikaatio on kohtalaisen kroonisen alaraajaiskemian merkki; se siis kehittyy jo ennen kriittistä iskemiaa ja leposärkyä. Tyypillistä on kipu kävellessä, joskus jopa seisominen voi provosoida oireet; kivun lokaatio vaihtelee ahtauman tason mukaan. Johtuu siitä, että lihakset menevät ''maitohapolle'' iskemiasta johtuen, mikä pakottaa pysähtymään tai hidastamaan vauhtia -\textgreater{} Oire helpottaa minuuteissa (tyypillisesti max 5-15 min) levossa.

Klaudikaation vaikeusaste ei täysin korreloi taudin vaikeusasteeseen. Kriittistä iskemiaa esiintyy 50 \%:lla ilman edeltävää katkokävelyoiretta, sillä monet vanhukset liikkuvat hyvin rajallisesti. Erityisesti myös diabeetikoilla alaraajaiskemia on pitkään oireeton ja jopa krooninen raajaa uhkaava iskemia (CLTI) voi olla kivuton neuropatian vuoksi.

c: Ei ole tyypillinen alaraajaiskemian oirekuva

\end{solution}

\section{Peräaukon fissuuran ensilinjan hoito suolen toiminnan normaalistamisen jälkeen?}\label{peruxe4aukon-fissuuran-ensilinjan-hoito-suolen-toiminnan-normaalistamisen-juxe4lkeen}

Ei vaihtoehtoja, mutta tässä anaalifissuuroista:

Anaalifissuura (fissura ani) on peräaukon haavauma, jonka pääoireina ovat kipu ja yleensä vähäinen ulostamiseen liittyvä kirkas verenvuoto.

\begin{itemize}
\tightlist
\item
  Todennäköisimmin saa alkunsa mekaanisesta vauriosta (primaarinen fissuura), kuten kova ulostus, anaaliyhdyntä, peräaukon puhdistus ulostamisen jälkeen pyyhkimällä tai ripuli ja sen vaatimat tiheät ulostamiskerrat. Voi myös olla sekundaarinen anaalifissuura (esim. Crohnin taudin, granulomatoottisten sairauksien, maligniteettien tai infektion aiheuttamia).
\end{itemize}

Diagnosoidaan inspektiolla (ja varovaisella tuseerauksella, mutta ei aina kivun takia onnistu). Anaalifissuura useimmiten (n.~80-85\%) muodostuu proktologisella kellotaululla kello kuuteen (keskiviivaan posteriorisesti eli selkäpuolelle). Noin 10 \% todetaan edessä keskiviivassa ja jos haava sijaitsee muualla (lateraalisesti), sen syyksi on epäiltävä sekundaarista syytä, esimerkiksi Crohnin tautia.

\begin{itemize}
\tightlist
\item
  Alkuvaiheessa fissuura on punoittava ja pehmeäreunainen. Sormen kärjellä varovaisesti tunnustellen arkuus on voimakasta ja sisäsulkija on kireä
\item
  Jos sulkijalihasten väliseen paiseeseen viittaavia oireita (kuume, voimakas kipu myös ulostamisten välillä, pullotus haavan kohdalla) ei ole, aloitetaan haavauman hoito näiden löydösten perusteella
\end{itemize}

\textbf{Hoito:}

Akuutilla, päivistä viikkoihin kestäneellä haavaumalla on hyvä taipumus parantua itsestään. Tärkeää on spontaanin paranemisen odottaminen taustasyyn hoitaminen: ulosteen pehmentäminen tai ripulin hoitaminen. Kipua helpottamaan voi käyttää käsikauppalääkkeenä myytäviä puuduttavia voiteita muutaman viikon ajan. Myös lämpimät (n.~40C) istumakylvyt voivat helpottaa kipua ja estää kroonistumista.

\begin{itemize}
\tightlist
\item
  Fissuura voi kuitenkin kroonistua (\textgreater8vk kesto), koska akuutti anaalifissuura on kipeä ja aiheuttaa sisemmän sulkijalihaksen spasmin, joka heikentää peräaukon verenkiertoa ja kudoksen happeutumista. Tämän takia fissuura tyypillisesti tulee klo 6:een, koska posteriorinen osa on heikosti perfusoitu -\textgreater{} trauma paranee jo normaalistikin hitaasti.

  \begin{itemize}
  \tightlist
  \item
    6-8 viikon kuluessa haavauman kroonistuessa sen reunat kiinteytyvät ja haavan pohjalla on nähtävissä sisäsulkijan säikeet. Tyyppimuutoksia ovat yläpuolella oleva kookkaampi anaalipapilla ja alapuolella vartijapoimuksi kutsuttu ihopoimu.
  \end{itemize}
\end{itemize}

Kroonista haavaumaa hoidetaan vähentämällä sulkijalihaksen tonusta ensisijaisesti voiteilla (nitraattivoiteet tai kalsiuminestäjävoiteet) tai mahdollisesti botuliinitoksiinipistoksella tai tarpeen mukaan kirurgisella hoidolla (sfinkterotomia)

\begin{itemize}
\tightlist
\item
  Ex tempore -kalsiuminestäjävoiteiden (diltiatseemi tai nifedipiini) etu on, että ne eivät aiheuta päänsärkyä kuten nitraattivoide (esim. Rectogesic; on myös kalliimpi).

  \begin{itemize}
  \tightlist
  \item
    N. 50 \% potilaista paranee voidehoidolla
  \item
    Jos potilas ei vastaa voidehoitoon parissa kuukaudessa, tulee hänet lähettää ESH leikkausarvioon (kirurgisessa hoidossa katkaistaan sisemmän sulkijalihaksen säikeitä (sfinkterotomia). Itse fissuuraan ei tarvitse kajota.)

    \begin{itemize}
    \tightlist
    \item
      Lateraalinen sfinkterotomia on peräaukon haavauman tehokkain hoito - yli 90 \% haavoista paranee. Toimenpiteen jälkeen 3-30 \%:lle potilaista kehittyy kuitenkin kaasunkarkailua tai tuhrimista. Tämän vuoksi haavauman kirurginen hoito yleensä aloitetaan revisiolla eli fibroottisten reunojen, vartijapoimun ja anaalipapillan poistamisella yhdistettynä botuliinipistokseen; Paranemisen todennäköisyys ei ole yhtä hyvä kuin sfinkterotomiassa, mutta pidätyskyvyttömyyden riski on vain 1-2 \%:n luokkaa.
    \end{itemize}
  \end{itemize}
\item
  Voidehoito voidaan myös aloittaa heti akuutissakin fissuurassa, jos oireet ovat voimakkaat. Peräaukon haavaumista 60--80 \% kuitenkin paranee spontaanisti eikä voidehoitoa usein tarvita akuutissa vaiheessa.
\end{itemize}

\textbf{Tiivistettynä hoito ja aiheet ESH-lähetteelle:}

\begin{enumerate}
\def\labelenumi{\arabic{enumi})}
\tightlist
\item
  Akuutissa vaiheessa (6-8vk) hoida taustatekijät ja kipu (puuduttavat voiteet). Istumakylpyjä voi ehdottaa. Vaikeassa tapauksessa voi aloittaa diltiatseemivoiteen heti.
\item
  Haavan kroonistuessa ensisijaisesti aloitetaan voidehoito (ensisijaisesti diltiatseemi). Jos potilas ei vastaa voidehoitoon 6-8 viikossa, tulee hänet lähettää erikoissairaanhoitoon leikkausarvioon. Usein tehdään ensiksi botuliinipistoshoitokokeilu, mutta tarvittaessa edetään lateraaliseen sfinkterotomiaan.
\item
  Potilas tulee mahdollisesti lähettää ESH myös, jos todetaan epätyypillisesti sijaitseva (lateraalinen) anaalifissuura epäiltäessä Crohnin tautia tai todetaan poikkeuksellisen näköinen haavauma, joka herättää epäilyn maligniteetista.
\end{enumerate}

\section{Mikä voi aiheuttaa PSA:n nousua?}\label{mikuxe4-voi-aiheuttaa-psan-nousua}

Ei vaihtoehtoja, tässä tärkeimmät aiheesta:

PSA on elin-, mutta ei syöpäspesifinen.

\begin{itemize}
\tightlist
\item
  Siksi se voi olla kohonnut \textbf{hyvänlaatuisessa eturauhasen liikakasvussa (BPH), eturauhastulehduksessa} ja muissa hyvänlaatuisissa tiloissa.
\item
  PSA:ta saattavat nostaa myös muun muassa \textbf{ejakulaatio, pyöräily, virtsatietulehdus, eturauhasen tulehdus, virtsaumpi ja virtsarakon katetrointi.} Nämä tulee ottaa huomioon näytettä otettaessa.
\end{itemize}

Eturauhassyöpä on myös mahdollinen, vaikka PSA-arvo olisi pieni. PSA:n viitearvot vaihtelevat iän mukaan (nousee iän myötä) ja laboratoriokohtaisesti.

\section{Potilaalla ASO-tauti. Oireena laihtuminen ja postbrandiaalinen vatsakipu. Ei halua syödä. Gastroskopia ja kolonoskopia normaalit. Tt:ssä ei tuumoria. Millä tutkimuksella pääset todennäköisimmin diagnoosiin?}\label{potilaalla-aso-tauti.-oireena-laihtuminen-ja-postbrandiaalinen-vatsakipu.-ei-halua-syuxf6duxe4.-gastroskopia-ja-kolonoskopia-normaalit.-ttssuxe4-ei-tuumoria.-milluxe4-tutkimuksella-puxe4uxe4set-todennuxe4kuxf6isimmin-diagnoosiin}

Ei vaihtoehtoja, mutta koita vastata ilman vinkkejä

\begin{solution}
\leavevmode

Vastaus

\begin{verbatim}
 TT-angiografia
\end{verbatim}

Potilaalla on todennäköisesti krooninen mesenteriaali-iskemia.

Kroonista mesenteriaali-iskemiaa kutsutaan usein ``suoliston angiinaksi''. Sen taustalla on lähes aina valtimokovettumataudin aiheuttamat tukokset tai tiukat ahtaumat suolilievevaltimoissa (yleensä vähintään 2/3 suoliston valtimopäähaaroista (CA, SMA, IMA) tulee olla ahtautuneet, jotta oireita ilmenee; yhden suonen tauti on siis useimmiten oireeton). Ilmenee vatsakipuna, mikä alkaa pian ruokailun jälkeen ja kestää 1--2 tuntia (johtuu lisääntyneestä hapentarpeesta suolistossa ruokaa liikuteltaessa; sama periaate kuin urheilun aiheuttamassa rintakivussa jos potilaalla on sepelvaltimotauti; myös itse urheilu voi pahentaa mesenteriaali-iskemiaa). Ruokailuun liittyvät kivut johtavat ruokahalun heikkenemiseen, syömisen välttelyyn ja lopulta tahattomaan laihtumiseen.

Vaikka oirekuvan, statuksen (mahdollisesti vatsan alueelta kuuluva suhahtava sivuääni (bruit)) ja angiografian perusteella diagnoosi olisi selvä, niin muut vatsakivun syyt on kuitenkin suljettava pois, ja ennen hoitopäätöstä tehdään yleensä gastroskopia (löytyy varsin usein atrofiaa tai haavaumia, jotka eivät liity helikobakteeriin tai tulehduskipulääkkeiden käyttöön. Normaali endoskopialöydös ei kuitenkaan sulje pois mesenteriaali-iskemiaa. Keliakian mahdollisuus on syytä selvittää potilailta, joilla esiintyy painon laskua). Tällekin potilaalle oltiin siis jo tehty gastroskopia, kolonoskopia ja vatsan TT. Tavallista on, että kroonisen vatsakivun vuoksi tehdään laajasti muita tutkimuksia, mitkä eivät johda diagnoosiin. Siksi on tärkeää muistaa epäillä kroonista mesenteriaali-iskemiaa. Diagnoosi absoluuttisesti varmistuu vasta, kun kivut helpottavat onnistuneen revaskularisaation jälkeen.

Kroonisen mesenteriaali-iskemian ensisijainen hoito on endovaskulaariset hoidot: pallolaajennus ja stentit. Kirurginen hoito on vaihtoehto potilaille, joilla suonensisäinen hoito ei ole mahdollista tai järkevää (esim. pitkä totaalitukos, vaskuliitti tai muu tukoksen syy kuin valtimokovettumatauti) tai jos suonensisäinen hoito ei ole onnistunut. Tällöin tehdään ohitus yleensä ylemmän suolilievevaltimon ja sisusvaltimorungon alueelle joko antegradisesti (ohituksen lähtökohta aortassa sisusvaltimorungon yläpuolella) tai retrogradisesti (ohitus munuaisvaltimotason alapuolisesta aortasta tai lonkkavaltimosta).

Revaskularisaation lisäksi on tärkeää huolehtia erityisesti kahdesta lääkityksestä sekundaaripreventiomielessä antitromboottinen (ASA 100mg 1x1) lääkitys ja statiinilääkitys. Stenttauksen jälkeen suositellaan normaalin valtimotaudin sekundaariprevention (100 mg asetyylisalisyylihappoa päivässä ja statiinilääkitys) lisäksi klopidogreelilääkitystä (DAPT) vähintään kuukauden ajaksi stenttitromboosin ehkäisemiseksi. Ylemmän suolilievevaltimon akuutti stenttitromboosi on kuitenkin harvinainen, joten jos potilaalla esiintyy vuoto-ongelmia toimenpiteen jälkeen, klopidogreelilääkitys voidaan lopettaa.

\end{solution}

\section{Potilaalla n.~1,5cm palpoituva muutos kilpirauhasessa. Ensilinjan tutkimus?}\label{potilaalla-n.-15cm-palpoituva-muutos-kilpirauhasessa.-ensilinjan-tutkimus}

Ei vaihtoehtoja, mutta koita vastata ilman vinkkejä

\begin{solution}
\leavevmode

Vastaus

\begin{verbatim}
 UÄ
\end{verbatim}

Yleisesti kilpirauhasen ensisijainen ja tärkein kuvantamiskeino on UÄ. Kliinisesti yksittäisenä havaitun kyhmyn ultraäänitutkimus löytää usein monia erillisiä kyhmyjä, ja joskus palpaatiolöydös ei saa vahvistusta ultraäänitutkimuksesta.

Jos UÄ:ssä todetaan suspekti muutos, siitä otetaan ohutneulabiopsia UÄ-ohjatusti. Alle 1 cm suspektia kyhmyä ei yleensä biopsoida (biopsioidaan, jos potilaalla on
riskitekijöitä tai ikää on alle 35 v)

\pandocbounded{\includegraphics[keepaspectratio]{images/uäonb.png}}

\end{solution}

\section{Laparotomia vs.~laparostomia}\label{laparotomia-vs.-laparostomia}

Ei tarkempaa kysymyksenasettelua tai vinkkejä, mutta todennäköisesti kysytty, että mikä on näiden termien ero. Koita vastata ilman vinkkejä.

\begin{solution}
\leavevmode

Vastaus

\begin{verbatim}
 Laparostomia = jätetty auki
\end{verbatim}

Laparotomia tarkoittaa vatsaontelon avoleikkausta. Yleisin vatsaontelon avausviilto on keskiviiltolaparotomia, jossa vatsaontelo avataan vatsalihasten välissä pystysuunnassa linea alban kohdalta sopivan mittaiselta pituudelta.

Laparostomia tarkoittaa kirurgista hoitokeinoa, jossa vatsaontelo avataan anteriorisesti, mutta jätetään leikkauksen jälkeen tarkoituksellisesti auki (ns. open abdomen). Käytetään intra-abdominaalisen paineen vapauttamiseen, kun potilaalla on vatsaontelon ylipaineoireyhtymä (esim. akuutissa pankreatiitissa).

Laparoskopia taas on tähystysleikkaus, jossa vatsaonteloon tehdään pieniä viiltoja, joiden kautta vatsaonteloon työnnetään instrumentit ja kamera.

Yleisesti toimenpiteiden nimeämisestä kannattaa muistaa ainakin seuraavat:

\begin{enumerate}
\def\labelenumi{\arabic{enumi}.}
\tightlist
\item
  -tomia = kirurginen avoleikkaus johonkin elimeen sisälle
\item
  -stomia = kirurgisesti uuden reiän luominen (esim. kolostomia, joka on iholle tuotu = tyypillinen paksusuoliavanne)
\item
  -ektomia = rakenteen poistaminen
\end{enumerate}

\pandocbounded{\includegraphics[keepaspectratio]{images/laparotomia.png}}
\pandocbounded{\includegraphics[keepaspectratio]{images/laparostomia.png}}

\end{solution}

\section{Polandin syndrooma: mitä tarkoittaa?}\label{polandin-syndrooma-mituxe4-tarkoittaa}

Ei vaihtoehtoja, mutta koita vastata ilman vinkkejä

\begin{solution}
\leavevmode

Vastaus

\begin{verbatim}
 Iso rintalihas puuttuu
\end{verbatim}

Polandin oireyhtymä on synnynnäinen \textbf{toispuoleinen} rintakehän kehityspoikkeama, jossa iso rintalihas (musculus pectoralis major) puuttuu kokonaan (aplasia) tai osittain (hypoplasia). Rintarauhasen tai nännin anomalioita tai hartialihasten puuttumista voi myös olla. Rintarauhaskudos -- jos sitä on -- on usein fibroottista, koska puuttuvan lihaksen tilalla on kovia sidekudosjuosteita. Polandin oireyhtymä on miehillä kaksi kertaa tavallisempi kuin naisilla.

Liitännäiskehityspoikkeamia ovat muun muassa pienen rintalihaksen (m. pectoralis minor) puuttuminen, kylkiluu- tai rustopuutokset, samanpuoleisen leveän selkälihaksen (m. latissimus dorsi) puuttuminen tai hypoplasia, ihon ja subkutiksen atrofia, sternumin rotaatiovirhe, saman kehonpuoliskon käden, kyynärvarren tai koko yläraajan kehityspoikkeama, kehityspoikkeamia voi myös olla esimerkiksi ruoansulatuskanavassa, maksassa tai sydämessä

Kudospuutoksen korjaus vaihtelee kehityspoikkeaman vaikeusasteen mukaan. Korjauksessa voidaan käyttää omakudossiirteitä, kuten rasvansiirtoa, kielekkeitä, implanttia tai näiden yhdistelmiä. Kehityspoikkeamia korjattaessa potilaat ovat usein nuoria. Menetelmää valittaessa implantti- tai omakudoskorjauksen välillä tulisi muistaa, että implantti ei laskeudu normaalisti ikääntyvän rinnan tavoin. Implantin ympärille voi myös kehittyä kapselikontraktuura, joka edellyttää uusintaleikkausta.

\pandocbounded{\includegraphics[keepaspectratio]{images/poland.png}}
\pandocbounded{\includegraphics[keepaspectratio]{images/polandosmosis.png}}

\end{solution}

\section{Prostatahyperplasian lääkkeet ja ortostaattinen hypotensio}\label{prostatahyperplasian-luxe4uxe4kkeet-ja-ortostaattinen-hypotensio}

Ei vaihtoehtoja, mutta tässä aiheesta:

Lievä- ja keskivaikeaoireisten komplisoitumattomien BPH-potilaiden virtsaamisoireita hoidetaan ensisijaisesti lääkehoidolla, joka on tyypillisesti alfasalpaaja ja/tai 5-alfareduktaasin estäjä.

Alfasalpaajat eli α1-reseptorin salpaajat (esim. tamsulosiini tai alfutsosiini) vaikuttavat rentouttamalla prostaattisen virtsaputken ja virtsarakon kaulan sileää lihaksistoa. Ne lievittävät nopeasti (vrt. 5-ARI:t jotka hitaammin) oireita, lisäävät virtsasuihkun huippuvirtaamaa ja vähentävät jäännösvirtsan tilavuutta.

\begin{itemize}
\tightlist
\item
  Ne kuitenkin myös salpaavat verisuonien alfareseptoreita ja siten laskevat verenpainetta ja erityisesti heikentävät verisuonien supistumista vasteena seisomaan nousemiselle, mikä pahentaa ortostaattista hypotensiota.
\end{itemize}

5-ARI:t (esim. finasteridi tai dutasteridi) taas vaikuttavat eturauhasen liikakasvuun estämällä testosteronin metaboloitumista dihydrotestosteroniksi (DHT), jolloin seerumin DHT:n pitoisuus pienenee. Tämä vaikutus ilmenee hitaasti ja tämän takia 5-ARI:t lievittävät oireita hitaammin kuin alfasalpaajat.

\begin{itemize}
\tightlist
\item
  Ne kuitenkaan eivät vaikuta suoraan verisuoniin, jonka takia ortostaattisen hypotension riski ei käytännössä nouse.
\end{itemize}

\section{Spigelin tyrän sijainti}\label{spigelin-tyruxe4n-sijainti}

\begin{itemize}
\tightlist
\item
  \begin{enumerate}
  \def\labelenumi{\alph{enumi}.}
  \tightlist
  \item
    molemminpuolinen ylävatsa
  \end{enumerate}
\item
  \begin{enumerate}
  \def\labelenumi{\alph{enumi}.}
  \setcounter{enumi}{1}
  \tightlist
  \item
    molemminpuolinen alavatsa
  \end{enumerate}
\item
  \begin{enumerate}
  \def\labelenumi{\alph{enumi}.}
  \setcounter{enumi}{2}
  \tightlist
  \item
    oikea ylävatsa
  \end{enumerate}
\item
  \begin{enumerate}
  \def\labelenumi{\alph{enumi}.}
  \setcounter{enumi}{3}
  \tightlist
  \item
    vasen alavatsa
  \end{enumerate}
\end{itemize}

\begin{solution}
\leavevmode

Vastaus

\begin{verbatim}
 d
\end{verbatim}

Spigelin tyrä työntyy vinojen vatsalihasten väliin rektuslihaksen lateraalireunasta linea arcuatan tasolla eli siis tyypillisesti noin 3--5 cm navan alapuolella ja siitä lateraalisesti rectuslihasten lateraalireunaan. Pientä Spigelin herniaa voi olla vaikea tuntea palpoiden, koska se jää uloimman vinon vatsalihaksen aponeuroosin alle eikä pullistu subkutikseen saakka kuten muut tyrät.

Spigelin tyrä on tyypillisesti unilateraalinen ja todetaan hieman useammin vasemmalla puolella. Yleisin toteamisikä n.~50-60 vuotta.

Kuvantamistutkimukset auttavat diagnostiikassa. Spigelin tyrä korjataan samoin periaattein kuin muut primaariset vatsanpeitteiden tyrät (hoito jos oireita ja pelkin ompelein jos pieni ja verkolla jos isompi tyräportti). Spigelin tyrällä on suuri kureutumisriski -\textgreater{} herkästi leikkaus.

\pandocbounded{\includegraphics[keepaspectratio]{images/spigel.png}}

\end{solution}

\section{55v nainen, rinnan punoitus ja kuumotus, ei palpoituvaa resistenssiä. Mikä jatkohoito?}\label{v-nainen-rinnan-punoitus-ja-kuumotus-ei-palpoituvaa-resistenssiuxe4.-mikuxe4-jatkohoito}

Ei vaihtoehtoja, tässä mahdollinen jatkohoito:

Potilaan oirekuva viittaa mastiittiin, mutta rintasyöpä voi muistuttaa mastiittia, jonka takia etenkin ei-imettävillä naisilla tulee sulkea pois rintasyöpä, jos rintaan tulee mastiitti → mammografia infektion rauhoituttua

\begin{itemize}
\tightlist
\item
  Tulehdus rauhoitetaan antibiooteilla, ja tulee valita stafylokokkiin tehoava mikrobilääke. Ensisijaisesti flukloksasilliini 750mgx3 tai kefaleksiini 500mgx3.
\item
  Tulehdukseen voi liittyä absesseja, jotka vaativat aina dreneerauksen.
\end{itemize}

\ldots{}

Periduktaalimastiitti on yleisin ei-imettävän naisen (potilas 55v) rintatulehdus, tyypillisimmin tupakoivalla (90\% polttaa) naisella. Tavallisimmat oireet ovat kipu ja märkäerite areolan reunasta. Areolan ympäristössä todetaan tulehdus, absessi, kyhmy, fisteli tai sisäänvetäytynyt nänni.

\begin{itemize}
\tightlist
\item
  Periduktaalimastiitin uusiutumisen ehkäisemiseksi on välttämätöntä lopettaa tupakointi.
\item
  Uusiutuvat tai kroonisiksi muuttuneet tulehdukset edellyttävät laajojakin kirurgisia toimenpiteitä, kuten mamillan saneerausta, jossa kyseinen duktusalue poistetaan mamillan kärkeen saakka. Se voidaan tehdä polikliinisesti paikallispuudutuksessa.
\end{itemize}

Imetysajan mastiitti johtuu imettämisestä ja haavan kautta päässystä vauvan suubakteerista -\textgreater{} ei tarvita mammografiaa, jos tila paranee normaalisti imetystä jatkamalla ja ab-hoidolla

\section{Rintasyövän kolmoisdiagnostiikka}\label{rintasyuxf6vuxe4n-kolmoisdiagnostiikka}

Ei vaihtoehtoja, mutta selitä kolmoisdiagnostiikan periaate ilman vinkkejä

\begin{solution}
\leavevmode

Vastaus

\begin{verbatim}
 Palpaatio+mgr+PNB
\end{verbatim}

Kolmoisdiagnostiikka siis perustuu kliiniseen tutkimukseen (palpaatio, inspektio, anamneesi), kuvantaminen (ensisijaisesti mammografia ja jatkotutkimuksena UÄ) ja näytteisiin (paksuneulabiopsia ensisijaisesti). Jos yksikin diagnostiikan osa viittaa pahanlaatuisuuteen, muutosta ei saa jäädä seuraamaan.

Toisin sanoen kolmoisdiagnostiikan periaate: Jokaisen yllä mainitun löydöksen oltava ilman ristiriitaa hyvänlaatuisia, jotta rinnan muutos voidaan jättää poistamatta. Epäselvissä tapauksissa muutos poistetaan.

Kolmoisdiagnostiikka on ns. perusdiagnostiikka ja toteutetaan perusterveydenhuollossa.

Mammografiaseulonnat ovat parantaneet varhaisvaiheen diagnostiikkaa; 50-69-vuotiaat naiset kutsutaan tutkimukseen kahden vuoden välein.

\end{solution}

\section{Mihin käytetään Sengstaken-Blakemoren tuubia?}\label{mihin-kuxe4ytetuxe4uxe4n-sengstaken-blakemoren-tuubia}

Ei vaihtoehtoja, mutta koita vastata ilman vinkkejä

\begin{solution}
\leavevmode

Vastaus

\begin{verbatim}
 Esofagusvariksien tyrehdyttämiseen
\end{verbatim}

Jos ruokatorven variksvuotoa ei saada tyrehtymään lääke- tai endoskooppisin hoidoin (ligaatio/skleroterapia), voidaan vuoto tyrehdyttää tilapäisesti tamponaatioputkella tai metallistentillä. Voidaan asettaa Sengstaken--Blakemoren tai Lintonin putki, jotka ovat tilapäisiä hoitoja, kunnes voidaan tehdä TIPS (transjugulaarinen intrahepaattinen portosysteemisuntti; ei hoitona sovi liian huonokuntoiselle eikä vaikeaa maksan vajaatoimintaa sairastavalle) tai sunttikirurgia.

\pandocbounded{\includegraphics[keepaspectratio]{images/sengstaken.png}}
\pandocbounded{\includegraphics[keepaspectratio]{images/varixalgoritmi.png}}

\end{solution}

\section{Rectuslihasten diastaasi: leikataanko?}\label{rectuslihasten-diastaasi-leikataanko}

Ei vaihtoehtoja, mutta koita vastata ilman vinkkejä

\begin{solution}
\leavevmode

Vastaus

\begin{verbatim}
 Ei leikata
\end{verbatim}

Rektuslihasten välissä oleva valkea jännesauma (linea alba) saattaa venyä iän ja painonnousun myötä; yleistä myös raskauden jälkeen. Lihaskalvoalue eli aponeuroosi antaa periksi, kun vatsaontelon tilavuus kasvaa. Pystyasennossa tämä näkyy niin sanottuna tynnyrivatsana. Makuulta noustessa tai vatsalihaksia jännittäessä rektuslihasten väliin tulee harjumainen pullistuma. Usein tätä luullaan tyräksi ja vatsakipujen syyksi, mutta tila on viaton eikä vaadi hoitoa. Kyseessä ei siis ole tyrä, koska linea alba on ehjä. \textbf{Jos potilaalla on vatsakipuja niin rektusdiastaasi ei ole niiden taustalla.}

Vatsalihasten venyminen hoidetaan pääsääntöisesti lihaksia vahvistavilla harjoituksilla. Jos lihasharjoituksista huolimatta on jatkuvia vartalon hallinnan ongelmia, selän kipuja ja väsymistä, vatsanpeitteiden löysyyden korjaaminen kirurgisesti voi auttaa. Myös raskauden aikana kehittyneen napatyrän korjauksen yhteydessä on joskus aiheellista korjata diastaasi.

\pandocbounded{\includegraphics[keepaspectratio]{images/rectusdiastaasianatomia.png}}
\pandocbounded{\includegraphics[keepaspectratio]{images/rectusdiastaasipotilas.png}}

\end{solution}

\section{Reduktioplastian yleisin komplikaatio}\label{reduktioplastian-yleisin-komplikaatio}

Ei vaihtoehtoja, koita vastata ilman vinkkejä

\begin{solution}
\leavevmode

Vastaus

\begin{verbatim}
 Haavan paranemisongelmat
\end{verbatim}

Reduktioplastialla tyypillisesti tarkoitetaan rintojen pienennysleikkausta. Sen yleisimpiä komplikaatioita ovat:

Haavan paranemisongelmat, joita tulee noin viidesosalle potilaista. Haava-aukileet paranevat useimmiten paikallishoidolla. Ihonsisäiset sulavat ompeleet aiheuttavat usein lankafisteleitä, jotka voivat pitkittää haavan eritystä. Hoitona on fistelöivän ommelmateriaalin poisto, eikä hoitoon tarvita mikrobilääkekuuria. On muistettava, että haavojen paraneminen on yksilöllistä. Reduktioplastian jälkeinen leikkaustulos muotoutuu 6--12 kuukauden aikana.

Leikkauksen jälkeinen verenvuoto ilmenee pian ja voi vaatia päivystysleikkauksen

Harvinainen komplikaatio, nänninpihan verenkiertohäiriö, voi johtaa nännin kuolioon, jolloin nänni joudutaan poistamaan

Reduktioplastiaan voi liittyä myös ihon ja nännin tuntohäiriöitä, joista osa palautuu

Reduktioplastian jälkeen imettäminen ei välttämättä onnistu

\end{solution}

\section{Kroonisen pankreatiitin diagnostiikka}\label{kroonisen-pankreatiitin-diagnostiikka}

Ei vaihtoehtoja, tässä tärkeimmät aiheesta:

Kroonisen pankreatiitin diagnostiikkaan on useita kriteeristöjä, mutta \textbf{kroonisen haimatulehduksen diagnostiikka perustuu ensisijaisesti TT-tutkimukseen}

\begin{itemize}
\tightlist
\item
  TT-kuvassa voidaan nähdä mm. haiman kalkkeutumia, jotka viittaavat krooniseen pankreatiittiin
\item
  Muita mahdollisia diagnoosiin viittaavia löydöksiä ovat eksokriininen vajaatoiminta (matala F-Elast1; ei ole suoraan diagnostinen krooniselle haimatulehdukselle) ja tietysti kroonisen pankreatiitin histologia
\end{itemize}

Normaali kaikututkimuslöydös ei ole poissulkeva, mutta haiman kalkkeumia tai muita kroonisen haimatulehduksen viitteitä voi joskus näkyä.

Amylaasi voi nousta kipukohtausten aikana, mutta pääasiassa ei tarvitse tutkia kroonista pankreatiittia epäiltäessä, koska on usein normaali (Haima on fibrotisoitunut ja ei enää pysty tuottaa entsyymejä normaalisti; vrt. akuuttiin pankreatiittiin, jossa solujen vaurioituminen vapauttaa runsaasti entsyymejä ja näitä voidaan käyttää diagnostiikassa).

\pandocbounded{\includegraphics[keepaspectratio]{images/krooninenpankreatiittitt.png}}

\section{Priapismin hoito}\label{priapismin-hoito}

Ei vaihtoehtoja, tässä tärkeimmät priapismista:

Priapismi = Kestokangistus; kivulias, pitkään (\textgreater4h) jatkuva erektio, joka ei yleensä liity kiihottumistilaan eikä laukea ejakulaatiosta

Priapismi voidaan periaatteessa jakaa kolmeen tyyppiin, mutta ainoa, josta tullaan kurssilla todennäköisesti kysymään on iskeeminen (low-flow) priapismi. Se on myös vakavin tyyppi. Low-flow priapismissa saadaan heikkoa arteriavirtausta penikseen ja verta ei pääse pois peniksestä -\textgreater{} vähähappinen veri kertyy ja johtaa iskeemiseen kudosvaurioon. Aiheuttajia on esimerkiksi idiopaattinen (yleisin), erektiolääkkeet, sirppisoluanemia (sirppiytyneet punasolut tukkivat peniksen laskimot), leukemia ja spinaalitrauma.

Vähävirtauksinen iskeeminen priapismi tulee tunnistaa ja hoitaa päivysluonteisesti. Yli kuusi tuntia kestänyt jäykkä ja kivulias erektio on riski kudosvaurioille ja voi johtaa paisuvaisen fibrotisoitumiseen tai jopa peniksen kuolioon.

\begin{itemize}
\tightlist
\item
  Mahdollisesti reipas liikunta ja kylmä suihku voivat auttaa pitkittyneessä erektiossa, mutta oikea \textgreater4h kestänyt priapismi tulee jo hoitaa lääkkeellisesti\\
\item
  Lääkäri saa priapismin laukeamaan punktoimalla toisen paisuvaiskudoksen paksulla, esim. 21G:n neulalla, aspiroimalla siitä 100--200 ml tummaa laskimoverta, puristamalla samalla siittimen vartta ja ruiskuttamalla paisuvaiseen adrenergista lääkettä, tavallisimmin etilefriinihydrokloridia, jonka kerta-annos on 5--10 mg. Vaihtoehtoisina lääkkeinä paisuvaiseen voidaan injisoida adrenaliinia 0,5--1,0 ml (1:1 000) tai noradrenaliinia 10--20 µg.

  \begin{itemize}
  \tightlist
  \item
    Lääkeinjektio voidaan tarvittaessa uusia puolen tunnin kuluttua edellisestä injektiosta. Lääke tulee ruiskuttaa hitaasti paisuvaiskudokseen. Potilaan pulssia ja verenpainetta tulee seurata injektion aikana ja sen jälkeen noin puolen tunnin ajan. On huomioitava, että siitin ei veltostu heti lääkeinjektion jälkeen, vaan yleensä kestää useita tunteja, ennen kuin siitin saavuttaa normaalin pehmeyden.
  \item
    Jos priapismi ei laukea kahden injektion jälkeen, potilas tulisi lähettää urologiseen yksikköön hoitoon. Priapismin kirurgisessa hoidossa corpus cavernosumin ylipaineinen veri pyritään ohjaamaan joko corpus spongiosumiin tai suoraan suureen laskimoon. Winterin sunttileikkauksessa pistoveitsellä tai prostatabiopsianeulalla tehdään yhteys corpus cavernosumin ja terskan paisuvaiskudoksen välille niin, että veri pääsee poistumaan corpus spongiosumin kautta. Distaalisen suntin osoittautuessa riittämättömäksi joudutaan tekemään avoleikkaus, jossa tehdään fisteli paisuvaisten välille niiden proksimaaliosaan tai yhdistetään paisuvainen suoraan laskimoon.
  \end{itemize}
\end{itemize}

\textbf{Lyhyesti siis hoito aspiraatio+etilefriini ad x2. Jos ei auta niin urologiseen yksikköön päivystysleikkaukseen.}

\pandocbounded{\includegraphics[keepaspectratio]{images/priapismihoitomahdollisuudet.png}}

\section{Mitä teet ensimmäisenä, jos kaveri saa palovamman kiukaasta?}\label{mituxe4-teet-ensimmuxe4isenuxe4-jos-kaveri-saa-palovamman-kiukaasta}

Ei vaihtoehtoja, mutta tässä mahdollinen vastaus:

\begin{itemize}
\tightlist
\item
  cABCDE tietysti aina toimintamallina, mutta palovamman suhteen kentällä kaverin irrottaminen kiukaasta ja palovamman viilentäminen. Viilennä palovamma-alue huuhtelemalla sitä n.~+20-asteisen juoksevan veden alla n.~10−15 min:n ajan; älä käytä viilennykseen jäätä tai jäävettä. Viilennys on hyödyllistä ensimmäisten 3 t:n ajan vammasta. Itse potilas ei saa viilentyä ja lämmön pitämisestä tulee pitää huolta, mutta palovammakohtaa tulee viilentää.
\item
  Vie kaveri päivystykseen. Kuljetusta varten palohaavat peitetään puhtaalla kuivalla sidoksella. Kuljetuksen ajaksi palovamma-alueet tulee pitää kohoasennossa.
\end{itemize}

\section{Melanooma, basaliooma ja levyepiteelikarsinooma (millaiset esiasteet, mitkä näistä metastasoi?)}\label{melanooma-basaliooma-ja-levyepiteelikarsinooma-millaiset-esiasteet-mitkuxe4-nuxe4istuxe4-metastasoi}

Ei vaihtoehtoja, mutta tässä suluissa spesifoitujen asioiden vastaukset:

\begin{itemize}
\tightlist
\item
  Melanooma: Ihon pigmenttisoluista alkunsa saava syöpä. Voi olla lähtöisin esimerkiksi dysplastisesta neevuksesta, mutta useimmiten melanooma kehittyy ilman esiastetta suoraan kliinisesti terveelle iholle. Ennen invasiivista melanoomaa voi olla in situ -melanooma (pintamelanooma), jossa kasvu on radiaalista pitkin epidermistä. Joissain lähteissä (esim. kirran dioissa) melanoomat jaetaan sen esiasteisiin (melanoma in situ ja lentigo maligna) ja invasiiviseen melanoomaan, mutta in situ melanoomat eivät sinänsä ole melanooman esiasteita, vaan jo ihan oikeaa melanoomaa, mutta vain ennen invaasiota. Lentigo maligna on myös in situ -melanooman alaluokka, joten tämä diojen jako ei ole kovinkaan järkevä.

  \begin{itemize}
  \tightlist
  \item
    Invasiiviset melanoomat metastasoivat. In situ -vaiheessa ei ole metastasointiriskiä. Ei ole vielä tapahtunut invaasiota dermikseen, jossa imusuonet sijaitsevat -\textgreater{} metastasointi ei onnistu epidermiksestä käsin.
  \end{itemize}
\item
  Basaliooma: Tyvisolusyövälle ei tyypillisesti ole todettavissa perkursorileesiota, vaan se syntyy de novo.

  \begin{itemize}
  \tightlist
  \item
    Ei käytännössä metastasoi koskaan. Kasvaa kuitenkin invasiivisesti, jonka takia se kuuluu hoitaa (voi esim. kehitysmaissa syövyttää invasiivisen kasvun kautta potilaan kasvot täysin)
  \end{itemize}
\item
  Levyepiteelikarsinooma: Okasolusyöpä voi kehittyä suoraan terveen näköiselle iholle, mutta useammin se kehittyy esiasteeseen, joita ovat aktiininen keratoosi eli solaarikeratoosi ja carcinoma in situ eli Bowenin tauti. Bowenin tauti itse asiassa on jo okasolusyöpää, mutta ei vain ole invasoinut epidermiksen alle; aktiinen keratoosi ei vielä ole okasolusyöpää

  \begin{itemize}
  \tightlist
  \item
    Metastasointi suhteellisen harvinaista (n.~0-16\% riippuen tuumorin syvyydestä), mutta tyvisolusyöpää yleisempää. Mikäli kasvaimen invaasiosyvyys on \textless2 mm, riski metastasoinnille on 0 \%, 2,01--6 mm:n kasvaimissa 4 \% ja \textgreater6 mm:n kasvaimissa 16 \% -\textgreater{} Paksuus \textgreater{} 6 mm tai kasvu verinahkan alaosiin tai ihonalaiskudokseen on täten okasolukarsinooman suuren riskin kriteeri.
  \end{itemize}
\end{itemize}

\section{Sarkoomien etiologia}\label{sarkoomien-etiologia}

Ei vaihtoehtoja, mutta tarjotusta vastauksesta on voinut päätellä, että on kysytty jotain luokkkaa: \textbf{``Onko suurin osa sarkoomista perinnöllisiä vai sporadisia?''}

\begin{solution}
\leavevmode

Vastaus

\begin{verbatim}
 Sporadisia
\end{verbatim}

Vain pieni osa (noin 5-10 \%) pehmytkudossarkoomista liittyy perinnöllisiin syndroomiin (esim. Li-Fraumenin oireyhtymä tai tyypin 1 neurofibromatoosi) ja valtaosa on sporadisia

\end{solution}

\section{Mihin sarkoomat tavallisesti metastasoivat?}\label{mihin-sarkoomat-tavallisesti-metastasoivat}

Ei vaihtoehtoja, mutta koita miettiä, mikä on pehmytkudossarkoomien yleisin metastasointilokaatio.

\begin{solution}
\leavevmode

Vastaus

\begin{verbatim}
 Keuhkot
\end{verbatim}

Sarkoomat metastasoivat hematogeenisesti (vrt. karsinoomat, jotka pääasiassa lymfateitse) ja metastasoivat tämän takia usein keuhkoihin, koska siellä on laajat ja rikkaat verisuoniverkostot.

Pehmytkudossarkoomien etäpesäkkeet, erityisesti keuhkometastaasit pyritään poistamaan kirurgisesti, jos se on mahdollista. Etäpesäkkeiden poisto voi joskus olla kuratiivinen toimenpide.

\end{solution}

\section{Potilaalla suhteettoman kova vatsakipu, vatsa pehmeä ja myötäävä, potilas tuskainen, mitä kuvannat?}\label{potilaalla-suhteettoman-kova-vatsakipu-vatsa-pehmeuxe4-ja-myuxf6tuxe4uxe4vuxe4-potilas-tuskainen-mituxe4-kuvannat}

Ei vaihtoehtoja, mutta koita vastata ilman vinkkejä

\begin{solution}
\leavevmode

Vastaus

\begin{verbatim}
 Vatsan TT ja viskeraalisuonten tilanne
\end{verbatim}

Varsinkin tällaisessa akuutissa vatsassa, jossa ei ole defancea, niin tulee muistaa mesenteriaali-iskemian mahdollisuus ja kuvata myös arteriavaihe.

\end{solution}

\section{Laskimovajaatoiminnan patofysiologia}\label{laskimovajaatoiminnan-patofysiologia}

Ei vaihtoehtoja, tässä patofysiologiasta tärkeimmät:

Perimmäinen syy tuntematon, mukana inflammaatio, jonka pohjalta laskimoiden seinämärakenne degeneroituu, laskimoiden seinämärakenne tuhoutuu ja lopulta laskimoläpät eivät ole enää pitävät.

\begin{itemize}
\tightlist
\item
  Normaalilaskimopaluu alaraajasta perustuu toimivaan pohjelihaspumppujärjestelmään. Lihassupistus puristaa syvät laskimot kasaan, jolloin laskimoveri siirtyy sydäntä kohti. Pinnallisista laskimoista veri virtaa syvään laskimojärjestelmään yhdyslaskimoiden ja pinnallisten päärunkojen kautta lihassupistusten välisenä aikana matalamman paineen suuntaan. Normaali läppätoiminta estää takaisinvirtauksen (refluksi) lihastyön päätyttyä.
\item
  Sulkeutuneet läpät katkaisevat laskimon sisällä olevan veripilarin muutaman sentin välein ja tämä vähentää hydrostaattista painetta.
\end{itemize}

Läppien ollessa ei-pitävät, niin laskimoiden hydrostaattinen paine nousee ja tämä johtaa suonikohjuihin ja alaraajaturvotukseen sekä lopulta laskimohaavoihin.

\pandocbounded{\includegraphics[keepaspectratio]{images/laskimopaine.png}}

\section{Aortta-aneurysman riskitekijät. Mikä ei ole riskitekijä?}\label{aortta-aneurysman-riskitekijuxe4t.-mikuxe4-ei-ole-riskitekijuxe4}

\begin{itemize}
\tightlist
\item
  \begin{enumerate}
  \def\labelenumi{\alph{enumi}.}
  \tightlist
  \item
    naissukupuoli
  \end{enumerate}
\item
  \begin{enumerate}
  \def\labelenumi{\alph{enumi}.}
  \setcounter{enumi}{1}
  \tightlist
  \item
    tupakointi
  \end{enumerate}
\item
  \begin{enumerate}
  \def\labelenumi{\alph{enumi}.}
  \setcounter{enumi}{2}
  \tightlist
  \item
    Ensimmäisen asteen sukulaisuus
  \end{enumerate}
\item
  \begin{enumerate}
  \def\labelenumi{\alph{enumi}.}
  \setcounter{enumi}{3}
  \tightlist
  \item
    ikä
  \end{enumerate}
\end{itemize}

\begin{solution}
\leavevmode

Vastaus

\begin{verbatim}
 a
\end{verbatim}

Aortan aneurysma ja varsinkin vatsa-aortan aneurysma (AAA) on selvästi yleisempi miehillä.

b: Tupakointi on voimakkain AAA-tautiin assosioituva riskitekijä, ja sen vähenemistä pidetään syynä esiintyvyyden vähenemiseen.

c: Aneurysmissa on selkeä perinnöllinen riski.

d: Riski kasvaa iän myötä.

\end{solution}

\section{Okkluusion auskultaatiolöydöksissä suoliäänet}\label{okkluusion-auskultaatioluxf6yduxf6ksissuxe4-suoliuxe4uxe4net}

\begin{itemize}
\tightlist
\item
  \begin{enumerate}
  \def\labelenumi{\alph{enumi}.}
  \tightlist
  \item
    poissa
  \end{enumerate}
\item
  \begin{enumerate}
  \def\labelenumi{\alph{enumi}.}
  \setcounter{enumi}{1}
  \tightlist
  \item
    vaimeat
  \end{enumerate}
\item
  \begin{enumerate}
  \def\labelenumi{\alph{enumi}.}
  \setcounter{enumi}{2}
  \tightlist
  \item
    normaalit
  \end{enumerate}
\item
  \begin{enumerate}
  \def\labelenumi{\alph{enumi}.}
  \setcounter{enumi}{3}
  \tightlist
  \item
    vilkkaat
  \end{enumerate}
\end{itemize}

\begin{solution}
\leavevmode

Vastaus

\begin{verbatim}
 d
\end{verbatim}

Huonot vaihtoehdot, koska sekä a, (b) ja d ovat sinänsä oikein. Suolitukoksen (okkluusion) auskultaatiolöydökset riippuvat taudin vaiheesta. Alkuvaiheessa suoli yrittää työntää läpi → vilkkaat, metalliset, kireästi kurahtelevat tai lorisevat suoliäänet viittaavat mekaaniseen suolitukokseen. Gastroenteriitissä suoliäänet ovat vilkkaita, mutta eivät ole kireän metallisia.

Myöhäisvaiheessa / paralyyttisessä vaiheessa: peristaltiikka väsyy → äänet vaimenevat tai puuttuvat. Potilaan heilauttaminen lantiosta voi aiheuttaa ns. loiskivat suoliäänet.

Kysymyksessä varmaan haettiin tyypillistä alkuvaiheen okkluusiota, joten oikea vastaus on todennäköisesti d.~vilkkaat

\end{solution}

\section{Akuutin vatsan diagnosoimisessa käytettävät kuvantamiskeinot}\label{akuutin-vatsan-diagnosoimisessa-kuxe4ytettuxe4vuxe4t-kuvantamiskeinot}

Ei vaihtoehtoja, mutta tässä tärkeimmät kuvantamiskeinot:

\begin{itemize}
\tightlist
\item
  Useimmiten varjoainetehosteinen TT ja/tai UÄ, mutta myös MRI ja natiivi-TT:llä on roolinsa.

  \begin{itemize}
  \tightlist
  \item
    UÄ-tutkimus on ensisijainen tutkimusmenetelmä akuutisti vatsakipuisella potilaalla silloin, kun epäillään sappiperäistä etiologiaa. Etenkin sappirakon kivet ja sappirakkotulehdus erottuvat ultraäänellä hyvin.
  \item
    Vatsan alueen TT-tutkimuksista ainoastaan virtsatiekivitutkimus tehdään rutiinisti ilman laskimonsisäisesti annettua tehosteainetta natiivitutkimuksena, muut vatsan alueen natiiviröntgenkuvaussarjat kuvataan TT:ssa yleensä vain tietyissä indikaatioissa ja osana laajempaa tehostettua tutkimusta.
  \item
    MRI:n merkitys akuutin vatsakivun osalta painottuu raskaana olevan potilaan akuutin vatsakivun selvittelyyn. Esimerkiksi umpilisäkkeen tulehdus voidaan poissulkea MRI:lla ilman tehosteaineen käyttöä. Käytännössä tavallisempi päivystyksellinen MRI:n tarve liittyy sappitiehytkivien poissulkemiseen (MRCP). Näillä potilailla päivystysaikainen kuvantaminen aloitetaan usein selvittämällä kokonaistilanne ultraäänellä, tarvittaessa täydentäen TT:lla ja siten varsinainen sappiteiden MRI voidaan tehdä myöhemmin, vasta muutaman päivän sisällä virka-aikaan.
  \end{itemize}
\end{itemize}

\pandocbounded{\includegraphics[keepaspectratio]{images/akuvatsakuvat.png}}

\section{Barronisaation komplikaatio ei ole}\label{barronisaation-komplikaatio-ei-ole}

\begin{itemize}
\tightlist
\item
  \begin{enumerate}
  \def\labelenumi{\alph{enumi}.}
  \tightlist
  \item
    inkontinenssi
  \end{enumerate}
\item
  \begin{enumerate}
  \def\labelenumi{\alph{enumi}.}
  \setcounter{enumi}{1}
  \tightlist
  \item
    verenvuoto
  \end{enumerate}
\item
  \begin{enumerate}
  \def\labelenumi{\alph{enumi}.}
  \setcounter{enumi}{2}
  \tightlist
  \item
    infektio
  \end{enumerate}
\item
  \begin{enumerate}
  \def\labelenumi{\alph{enumi}.}
  \setcounter{enumi}{3}
  \tightlist
  \item
    joku (ei wikissä)
  \end{enumerate}
\end{itemize}

\begin{solution}
\leavevmode

Vastaus

\begin{verbatim}
 a
\end{verbatim}

a: Barronisaatiossa vain laitetaan kumilenkki peräpukamien tyveen. Se ei vaurioita anaalikanavan lihasseinämää eikä hermoja eikä täten aiheuta inkontinenssia.

b: On pieni riski verenvuodolle erityisesti jos käytössä on antikoagulantti, mutta AK-hoitoa ei pääsääntöisesti tarvitse tauottaa. Lenkin irrotessa voi myös tulla vähän verenvuotoa, mutta runsas vuoto on harvinaista.

c: Infektio nyt kuuluu käytännössä kaikkiin toimenpiteisiin komplikaatioriskiksi.

d: Mahdollisesti ollut esim. kipu komplikaationa.

\pandocbounded{\includegraphics[keepaspectratio]{images/barronisaatio.png}}

\end{solution}

\section{Anteriorinen resektio tarkoittaa}\label{anteriorinen-resektio-tarkoittaa}

\begin{itemize}
\tightlist
\item
  \begin{enumerate}
  \def\labelenumi{\alph{enumi}.}
  \tightlist
  \item
    haiman kaudan poisto
  \end{enumerate}
\item
  \begin{enumerate}
  \def\labelenumi{\alph{enumi}.}
  \setcounter{enumi}{1}
  \tightlist
  \item
    ventrikkeliresektio
  \end{enumerate}
\item
  \begin{enumerate}
  \def\labelenumi{\alph{enumi}.}
  \setcounter{enumi}{2}
  \tightlist
  \item
    peräsuolen typistys
  \end{enumerate}
\item
  \begin{enumerate}
  \def\labelenumi{\alph{enumi}.}
  \setcounter{enumi}{3}
  \tightlist
  \item
    paksusuolen poisto
  \end{enumerate}
\end{itemize}

\begin{solution}
\leavevmode

Vastaus

\begin{verbatim}
 c
\end{verbatim}

Anteriorinen resektio on tyyppileikkaus peräsuolisyövässä. Peräsuolisyövän leikkaustavan määrittävät kasvaimen sijainti, sen paikallinen levinneisyys ja potilaan kunto. Jos kasvain ei kasva liian lähelle sulkijalihaksia, potilaalle tehdään peräsuolen anteriorinen resektio (poistetaan peräsuoli suoliliepeineen ja tehdään suoliliitos katkaistun paksusuolen pään ja peräsuolityngän välille). Anteriorinen resektio voidaan tehdä avoleikkauksena, laparoskooppisesti tai robottiavusteisesti.

Toinen tyypillinen leikkaus olisi abdominoperineaalinen resektio (rectumamputaatio). Rectumamputaatiossa tehdään pysyvä avanne, koska peräsuoli ja peräaukko poistetaan täysin.

Merkittäviä eroja komplikaatioluvuissa ja onkologisissa tuloksissa ei tekniikkojen välillä ole. Jos kasvain kasvaa peräaukon sulkijalihaksiin tai hyvin lähelle niitä, leikkauksessa on poistettava peräsuolen ja suoliliepeen lisäksi peräaukkokanava ja sulkijalihakset ja potilaalle tehdään pysyvä paksusuoliavanne.

Ns. LARS-oireet (Low Anterior Resection Syndrome) ovat yleisiä peräsuolen anteriorisen resektion jälkeen ja näistä yleisimpiä ovat mm. ulostamisfrekvenssin nousu, ulostamispakko (urge), pidätyskyvyn ongelmat (ulosteinkontinenssi), virtsaamisen ongelmat ja erektio-ongelmat.

\pandocbounded{\includegraphics[keepaspectratio]{images/anteriorinenresektio.png}}

\end{solution}

\section{Infektoituneessa haavassa kuolio, tärkein hoito ensimmäisenä?}\label{infektoituneessa-haavassa-kuolio-tuxe4rkein-hoito-ensimmuxe4isenuxe4}

Ei vaihtoehtoja, tässä mahdollinen hoitolinja:

\begin{itemize}
\tightlist
\item
  Nekroottinen, infektoitunut ja fibrinoottinen (puumainen) kudos poistetaan (kirurginen debridement). Jäljelle jää terve ja verestävä haavapohja.

  \begin{itemize}
  \tightlist
  \item
    Pelkkä antibioottihoito ei auta, koska se ei imeydy kuolioon -\textgreater{} hoito ei tehoa ilman debridementia. Debridement vähentää bakteerikuormaa ja mahdollistaa paranemisen.
  \end{itemize}
\end{itemize}

Jos infektio on selkeästi rajautunut eikä potilaalla ole nekroosia, luuhun yltävää haavaa, iskemiaa -\textgreater{} konservatiivinen hoito on mahdollinen

\section{Kuinka monta prosenttia rektumsyövistä voidaan havaita tuseeraamalla?}\label{kuinka-monta-prosenttia-rektumsyuxf6vistuxe4-voidaan-havaita-tuseeraamalla}

\begin{itemize}
\tightlist
\item
  \begin{enumerate}
  \def\labelenumi{\alph{enumi}.}
  \tightlist
  \item
    10 \%
  \end{enumerate}
\item
  \begin{enumerate}
  \def\labelenumi{\alph{enumi}.}
  \setcounter{enumi}{1}
  \tightlist
  \item
    20 \%
  \end{enumerate}
\item
  \begin{enumerate}
  \def\labelenumi{\alph{enumi}.}
  \setcounter{enumi}{2}
  \tightlist
  \item
    30 \%
  \end{enumerate}
\end{itemize}

\begin{solution}
\leavevmode

Vastaus

\begin{verbatim}
 c
\end{verbatim}

Koska noin 40\% paksu- ja peräsuolen syövistä sijaitsee nimenomaan peräsuolen alueella, on tuseeraus nopea ja helppo tapa varmistaa peräsuoli syöpäkasvaimen tai verenvuodon osalta. Tuseerauksella voidaan kuitenkin todeta vain hyvin selvät tapaukset ja suolen loppuosan syövät, joten diagnostisointimenetelmänä uuden suolistosyövän löytymisen suhteen, se on kovin rajallinen. Kaikista paksu- ja peräsuolen syövistä pelkästään tuseeraamalla pystytään löytämään 20 -- 25 \%. Peräsuolen kasvaimista arviolta n.~30\% on löydettävissä tuseeraamalla.

\pandocbounded{\includegraphics[keepaspectratio]{images/30pinnaarektum.png}}

\end{solution}

\section{Sappilekaasin diagnoosin varmistaminen kolekystektomian jälkeen}\label{sappilekaasin-diagnoosin-varmistaminen-kolekystektomian-juxe4lkeen}

\begin{itemize}
\tightlist
\item
  \begin{enumerate}
  \def\labelenumi{\alph{enumi}.}
  \tightlist
  \item
    laparotomia
  \end{enumerate}
\item
  \begin{enumerate}
  \def\labelenumi{\alph{enumi}.}
  \setcounter{enumi}{1}
  \tightlist
  \item
    laparoskopia
  \end{enumerate}
\item
  \begin{enumerate}
  \def\labelenumi{\alph{enumi}.}
  \setcounter{enumi}{2}
  \tightlist
  \item
    MRCP
  \end{enumerate}
\item
  \begin{enumerate}
  \def\labelenumi{\alph{enumi}.}
  \setcounter{enumi}{3}
  \tightlist
  \item
    endoskopia
  \end{enumerate}
\end{itemize}

\begin{solution}
\leavevmode

Vastaus

\begin{verbatim}
 c
\end{verbatim}

Huonohkot vaihtoehdot, mutta näistä MRCP on paras valinta.

Jos epäillään iatrogeenista sappitievauriota tai sappilekaasia, niin potilaalle tehdään kliininen tutkimus, minkä lisäksi häneltä mitataan lämpö ja tutkitaan perusverenkuva, C-reaktiivinen proteiini (CRP), maksa-arvot, bilirubiini ja amylaasi. Kuvantamistutkimuksista tehdään ylävatsan dupleksiultraäänitutkimus (onko sappea vapaassa vatsaontelossa), varjoainetehosteinen tietokonekerroskuvaus sekä magneettikuvaus ja magneettikolangiografia (MRCP). Tavoitteena on kuvantaa vaurio tarkkaan ja poissulkea mahdolliset oheisvauriot ennen hoidollisia toimenpiteitä. Mahdollisten nestekertymien diagnostinen punktio ja dreneeraus tehdään myös tässä vaiheessa. Laskuputkesta tuleva sappineste on vahva merkki iatrogeenisesta sappitievauriosta. Vaurion vaikeusaste on usein mahdollista määritellä jo tässä vaiheessa.

Diagnostiikan varmistamiseksi on syytä tehdä pikaisesti myös endoskooppinen retrogradinen kolangiografia (ERC). Samalla arvioidaan, voiko vaurion hoitaa ERC-teitse.

\end{solution}

\chapter{2023 (Vividus)}\label{vividus}

Taas puuttuu usein vaihtoehdot ja monesti on vain mainittu kysymyksen aihe eikä kysymyksenasettelua ole kirjoitettu tarkemmin. Jonkin verran myös aikaisempia tärppejä, joita ei nyt ole tähän taaskaan laitettu, koska ne on käyty jo edellisissä kappaleissa läpi.

\section{Epäilet potilaalla kroonista pankreatiittia. Mikä laboratoriokokeen tulos viittaisi tähän?}\label{epuxe4ilet-potilaalla-kroonista-pankreatiittia.-mikuxe4-laboratoriokokeen-tulos-viittaisi-tuxe4huxe4n}

\begin{itemize}
\tightlist
\item
  \begin{enumerate}
  \def\labelenumi{\alph{enumi}.}
  \tightlist
  \item
    Ulosteen matala elastaasi
  \end{enumerate}
\item
  \begin{enumerate}
  \def\labelenumi{\alph{enumi}.}
  \setcounter{enumi}{1}
  \tightlist
  \item
    Ulosteen korkea elastaasi
  \end{enumerate}
\item
  \begin{enumerate}
  \def\labelenumi{\alph{enumi}.}
  \setcounter{enumi}{2}
  \tightlist
  \item
    Seerumin matala amylaasi
  \end{enumerate}
\item
  \begin{enumerate}
  \def\labelenumi{\alph{enumi}.}
  \setcounter{enumi}{3}
  \tightlist
  \item
    Seerumin korkea amylaasi
  \end{enumerate}
\end{itemize}

\begin{solution}
\leavevmode

Vastaus

\begin{verbatim}
 a
\end{verbatim}

Elastaasi on haiman erittämä sidekudosta hajottava entsyymi, joka poistuu suolesta muuttumattomana ulosteessa. Voidaan käyttää tutkimaan haiman eksokriinista toimintakykyä.

Kroonisessa pankreatiitissa haiman eksokriiniset osat (ja myös endokriinisetkin -\textgreater{} sekundaarinen diabetes) tuhoutuvat, mikä johtaa siihen, että ruoansulatusentsyymejä ei enää tuoteta normaaleja määriä -\textgreater{} F-Elast1 laskee. Ruuassa ja eri entsyymivalmisteissa ei esiinny elastaasi 1:stä, joten ne eivät vaikuta analyysin tuloksiin.

Voimakas vesiripuli voi laimentaa ulosteen elastaasi 1 -pitoisuuden ja antaa vääriä positiivisia tuloksia eksokriinisen vajaatoiminnan suhteen

\pandocbounded{\includegraphics[keepaspectratio]{images/elast1.png}}

\end{solution}

\section{Mitä kautta kroonisen pankreatiitin operatiivinen hoito yleensä suoritetaan?}\label{mituxe4-kautta-kroonisen-pankreatiitin-operatiivinen-hoito-yleensuxe4-suoritetaan}

\begin{itemize}
\tightlist
\item
  \begin{enumerate}
  \def\labelenumi{\alph{enumi}.}
  \tightlist
  \item
    Endoskopia
  \end{enumerate}
\item
  \begin{enumerate}
  \def\labelenumi{\alph{enumi}.}
  \setcounter{enumi}{1}
  \tightlist
  \item
    Laparoskopia
  \end{enumerate}
\item
  \begin{enumerate}
  \def\labelenumi{\alph{enumi}.}
  \setcounter{enumi}{2}
  \tightlist
  \item
    Avoleikkaus
  \end{enumerate}
\item
  \begin{enumerate}
  \def\labelenumi{\alph{enumi}.}
  \setcounter{enumi}{3}
  \tightlist
  \item
    Kroonisen pankun hoidossa ei toimenpiteillä ole sijaa
  \end{enumerate}
\end{itemize}

\begin{solution}
\leavevmode

Vastaus

\begin{verbatim}
 a
\end{verbatim}

Invasiiviseen hoitoon päädytään, jos potilaalla on hallitsemattomia ylävatsakipuja siitä huolimatta, että hän ei käytä alkoholia. Nykyisin suurin osa kroonisen haimatulehduksen aiheuttamista haimatiehyen striktuuroista tai komplikaatioista, kuten pseudokystista tai haimafisteleistä, hoidetaan endoskooppisella retrogradisella kolangiopankreatografialla ja tarvittaessa esim. stenttauksella.

b, c: Kirurginen hoito, jos endoskooppinen hoito epäonnistuu tai on riittämätön

d: Kroonisen pankreatiitin hoito voidaan jakaa konservatiiviseen hoitoon, johon kuuluvat muun muassa haiman vajaatoiminnan hoito ja kivun lääkehoito, sekä kajoavaan hoitoon. Kroonisen pankreatiitin hoito on ensisijaisesti konservatiivinen. Tärkeintä on lopettaa alkoholin käyttö ja tupakointi, jotta kroonisen tulehdusreaktion eteneminen hidastuu. Ateriat kannattaa jakaa useaan osaan sekä rajoittaa rasvojen ja haimaentsyymejä estävien kuitujen määrää. Eksokriinista vajaatoimintaa hoidetaan aterioiden yhteydessä kapseleina otettavilla haimaentsyymeillä (esim. creon). Endokriininen vajaatoiminta aiheuttaa usein aluksi vain vähän oireita. Sitä tulee epäillä herkästi seurannassa, jotta diabeteksen hoito voidaan aloittaa ajoissa.

Kroonisen pankreatiitin vaikein oire on usein hankalahoitoinen kipu, jota voidaan joutua lääkitsemään opioideilla. Myös pregabaliini, antioksidantit, vitamiinivalmisteet ja haimaentsyymivalmisteet voivat lievittää kipua. Viimeksi mainitut vähentävät haiman omaa entsyymituotantoa ja haimanesteen eritystä, mikä voi laskea tiehytpainetta.

Mutta myös operatiivisella hoidolla on sijansa, joten vaihtoehto on väärin.

\end{solution}

\section{Spyglass -operaatio, miten tehdään?}\label{spyglass--operaatio-miten-tehduxe4uxe4n}

\begin{itemize}
\tightlist
\item
  \begin{enumerate}
  \def\labelenumi{\alph{enumi}.}
  \tightlist
  \item
    Endoskopia
  \end{enumerate}
\item
  \begin{enumerate}
  \def\labelenumi{\alph{enumi}.}
  \setcounter{enumi}{1}
  \tightlist
  \item
    Laparoskopia
  \end{enumerate}
\item
  \begin{enumerate}
  \def\labelenumi{\alph{enumi}.}
  \setcounter{enumi}{2}
  \tightlist
  \item
    Laparotomia
  \end{enumerate}
\item
  \begin{enumerate}
  \def\labelenumi{\alph{enumi}.}
  \setcounter{enumi}{3}
  \tightlist
  \item
    Peräsuolen kautta näkökontrollissa
  \end{enumerate}
\end{itemize}

\begin{solution}
\leavevmode

Vastaus

\begin{verbatim}
 a
\end{verbatim}

Sappitietähystys (kolangioskopia) on osoitettu hyödylliseksi epäselvien sappiteiden ahtaumien diagnostiikassa ja hankalasti poistettavien sappitiekivien hoidossa.

Uuden sukupolven sappitietähystin (SpyGlass, Boston Scientific) tuli markkinoille 2008. Tämä tähystin viedään sappiteihin pohjukaissuolen tähystimen kautta endoskooppisen retrogradisen kolangio-pankreatografian (ERCP) yhteydessä. Tähystimessä on neljään suuntaan kääntyvä kärki, erilliset kanavat optiikalle, ohjainvaijerille ja huuhtelulle sekä 1,2 mm:n läpimittainen työskentelykanava, jonka kautta voidaan viedä näkökontrollissa sappi- tai haimatiehyeen kudosnäytepihdit (SpyBite) ja sähköhydraulinen kivenmurskain (EHL).

\pandocbounded{\includegraphics[keepaspectratio]{images/spyglass.png}}
\pandocbounded{\includegraphics[keepaspectratio]{images/spyglasskuva.png}}

\end{solution}

\section{Mikä tuumori maksassa yleisin?}\label{mikuxe4-tuumori-maksassa-yleisin}

Ei vaihtoehtoja, mutta tässä yleisimmät maksamuutokset:

Maksan pesäkemuutokset ovat useimmiten hyvänlaatuisia. Parantuneiden kuvantamismenetelmien ansiosta maksasta löydetään paikallismuutoksia jopa 20 \%:lta tutkituista. Heistä viidesosalla muutos on pahanlaatuinen.

\begin{itemize}
\tightlist
\item
  Yleisiä hyvänlaatuisia pesäkemuutoksia ovat fokaalinen rasvoittuminen (ei usein lasketa), yksittäiset kystat, hemangioomat (yleisin solidi tuumori) sekä fokaalinen nodulaarinen hyperplasia (FNH). Riippuen lähteestä hemangioomat voivat olla suorastaan yleisin maksan pesäkemuutos, joissain taas kystat ovat hieman yleisempiä kuin hemangioomat.
\item
  Maksan yleisin pahanlaatuinen muutos on metastaasi (n.~90\% maligniteeteista on metastaaseja). Yleisin maksan primaarinen syöpä on hepatosellulaarinen karsinooma (HCC) ja toiseksi yleisin on maksan sisäisistä sappiteistä lähtenyt kolangiokarsinooma.
\end{itemize}

\section{ED-potilastapaus}\label{ed-potilastapaus}

Potilas tulee muun syyn takia tk-vastaanotolle, ja lopuksi ottaa puheeksi erektiovaikeudet. Potilaalla reilusti ylipainoa, tupakoi jne. Kertoo ottaneensa kaverilleen määrättyä sildenafiilia ja saanut tästä apua. Pyytää reseptiä tästä, mitä teet?

Ei vaihtoehtoja, tässä mahdollinen toimintatapa:

\begin{itemize}
\tightlist
\item
  Arvioi vasta-aiheet, kartoita anamneesi/taustasairaudet/riskitekijät tarkemmin (potilaalla ainakin ylipainoa ja tupakoi). Jos ei vasta-aiheita, niin kirjoita resepti ja ohjaa elämäntapamuutokset (tupakoinnin lopettaminen, laihdutus\ldots) sekä tarvittavat jatkotutkimukset (HbA1c, gluk, lipidit, pvkt, testo).

  \begin{itemize}
  \tightlist
  \item
    Tyypillisin vasta-aihe ensisijaiselle lääkehoidolle eli sildenafiilille on samanaikainen nitraattien käyttö (sildenafiilin on todettu lisäävän nitraattien verenpainetta alentavaa vaikutusta; sen vuoksi sen samanaikainen käyttö typpioksidien luovuttajien (kuten amyylinitriitti) tai nitraattien kanssa on kontraindisoitu).
  \end{itemize}
\end{itemize}

Suositusannos on 50 mg otettuna tarvittaessa noin tunti ennen aiottua seksuaalista toimintaa.

\section{Akuutti vatsa -- mikä on huolestuttavin löydös suoliäänten auskultaatiossa?}\label{akuutti-vatsa-mikuxe4-on-huolestuttavin-luxf6yduxf6s-suoliuxe4uxe4nten-auskultaatiossa}

\begin{itemize}
\tightlist
\item
  \begin{enumerate}
  \def\labelenumi{\alph{enumi}.}
  \tightlist
  \item
    Hiljaiset suoliäänet
  \end{enumerate}
\item
  \begin{enumerate}
  \def\labelenumi{\alph{enumi}.}
  \setcounter{enumi}{1}
  \tightlist
  \item
    Vilkkaat suoliäänet
  \end{enumerate}
\item
  \begin{enumerate}
  \def\labelenumi{\alph{enumi}.}
  \setcounter{enumi}{2}
  \tightlist
  \item
    Kilahtelevat suoliäänet
  \end{enumerate}
\item
  \begin{enumerate}
  \def\labelenumi{\alph{enumi}.}
  \setcounter{enumi}{3}
  \tightlist
  \item
    Suoliäänten auskultaatiolla ei ole merkitystä sillä vain palpaatiolla saadaan tietoa
  \end{enumerate}
\end{itemize}

\begin{solution}
\leavevmode

Vastaus

\begin{verbatim}
 a
\end{verbatim}

Akuutissa vatsassa hiljaiset tai puuttuvat suoliäänet ovat huolestuttavin löydös, koska ne viittaavat mahdollisesti peritoniittiin ja paralyyttiseen ileukseen tai vaikeaan iskemiaan.

b ja c: Vilkkaat tai kilahtelevat äänet voivat esiintyä (aikaisessa) mekaanisessa obstruktiossa. Myöhemmässä vaiheessa suolen väsyttyä voi obstruktiossakin esiintyä olemattomat suoliäänet.

d: Suoliäänillä on tilanteen arviossa ja päättelyssä merkitystä, vaikka diagnostiikka ei todellakaan perustu pelkästään niihin.

\end{solution}

\section{Potilaalla tulee virtsatessa ilmaa virtsaputkesta. Mitä epäilet?}\label{potilaalla-tulee-virtsatessa-ilmaa-virtsaputkesta.-mituxe4-epuxe4ilet}

Ei vaihtoehtoja, mutta koita vastata ilman vinkkejä

\begin{solution}
\leavevmode

Vastaus

\begin{verbatim}
 Kolovesikaalifisteli
 
\end{verbatim}

Pneumaturia eli ilma virtsassa viittaa vahvasti kolovesikaalifisteliin eli yhteyteen koolonista (tai muusta suolen osastakin mahdollisesti) rakkoon. Suolessa on ilmaa, joka pääsee fisteliä pitkin rakkoon ja tämä havaitaan ilmana virtsaputken kautta virtsatessa. Fisteli ilmenee myös toistuvina virtsatieinfektioina (bakteereita suolesta rakkoon) ja mahdollisesti jopa ulosteina virtsassa (tulee kysyä suoraan potilaalta).

Yleensä tämän aiheuttajana on reikä sigman huipusta virtsarakkoon \textbf{divertikkeliperforaation} takia. Diagnostinen tutkimus on kolonoskopia tai koolonin TT-kuvantaminen.

\end{solution}

\section{Yleisin rintasyöpätyyppi}\label{yleisin-rintasyuxf6puxe4tyyppi}

Ei vaihtoehtoja, mutta tässä rintasyöpien luokittelusta tärkeimmät:

Rintasyövät jaetaan invasiivisiin (yleisempiä) ja ei-invasiivisiin:

\begin{itemize}
\tightlist
\item
  Invasiivisista yleisimmät ovat \textbf{duktaalinen (tiehytperäinen)} (n.~70\%) ja lobulaarinen (rintarauhasperäinen) (n.~20\%)
\item
  Muita on mm. tubulaarinen, papillaarinen, musinoottinen
\end{itemize}

Ei-invasiivisia (in situ karsinooma) on duktaalinen karsinooma in situ (DCIS) (ja lobulaarinen intraepiteliaalinen neoplasia (LIN))

\begin{itemize}
\tightlist
\item
  DCIS on invasiivisen duktaalisen karsinooman esiaste, lähtenyt atyyppisestä duktaalisesta hyperplasiasta (ADH) -\textgreater{} DCIS -\textgreater{} invasiivinen duktaalinen karsinooma
\item
  LIN kuvailee atyyppistä lobulaarista hyperplasiaa (ALH) ja lobulaarista karsinoomaa in situ LCIS), jotka eivät ole pahanlaatuisia muutoksia, vaan osoitus suurentuneesta riskistä sairastua jommankumman rinnan duktaaliseen tai lobulaariseen karsinoomaan
\end{itemize}

\textbf{Bonustietoa:}

Rintasyövät voidaan myös jaotella eri tavalla kuin histologisesti alla mainittujen kolmen tekijän perusteella:

\begin{itemize}
\tightlist
\item
  syöpäkasvaimen ilmentämät hormonireseptorit (ER ja/tai PR)
\item
  epidermaalisen kasvutekijän reseptori 2:n (HER2) onkogeenimonistuma
\item
  jakautumisaste (Ki67 korkea vai matala)
\end{itemize}

Rintasyövät voidaan jakaa näitä käyttämällä neljään molekylaariseen ja geneettiseen alatyyppiin: luminaalinen A (yleisin), luminaalinen B, kolmoisnegatiivinen (triplanegatiivinen/basal like) ja HER2-positiivinen

\begin{itemize}
\tightlist
\item
  Hoitokäytännössä yleensä puhutaan hormonireseptoripositiivisista (luminaalinen A ja luminaalinen B), HER2-positiivisista ja kolmois- eli triplanegatiivisista
\item
  Luminaalisissa A karsinoomissa ER ja/tai PR on positiivinen, HER2 on negatiivinen ja Ki67 on matala
\item
  Luminaalisissa B karsinoomissa ER ja/tai PR on positiivinen, HER2 on positiivinen tai negatiivinen, mutta Ki67 on korkea.
\item
  Kolmoisnegatiivisissa karsinoomissa ER ja PR ovat molemmat negatiivisia, HER2 on negatiivinen ja Ki67 on ei-merkityksellinen
\item
  HER2-positiivisissa/-rikastuneissa karsinoomissa ER ja PR ovat molemmat negatiivisia, HER2 on positiivinen ja Ki67 on ei-merkityksellinen
\item
  Duktaalisia ja lobulaarisia karsinoomia voi olla kaikissa molekulaarisissa alatyypeissä
\end{itemize}

Alatyyppi ohjaa syövän onkologista hoitoa

\begin{itemize}
\tightlist
\item
  Hormonireseptoripositiivisilla hyvä vaste antiestrogeenisille lääkkeille, kuten tamoksifeenille
\item
  HER2-positiivisilla hyvä vaste HER2-proteiinin monoklonaaliselle vasta-aineelle (trastutsumabi eli Herceptin)
\item
  Kolmoisnegatiivisilla hoitona tavallisesti immunologinen lääkehoito (immuunivasteen vapauttajat, mm. PD-L1-estäjälääke
\end{itemize}

Ennuste vaihtelee alatyyppien välillä:

\begin{itemize}
\tightlist
\item
  Ennusteeltaan paras on luminaalinen A-tyyppi, johon kuuluvat kasvaimet ovat yleensä hyvin erilaistuneita, vahvasti estrogeenireseptoripositiivisia ja HER2-onkogeeninegatiivisia. Luminaalinen B-tyyppi on ennusteeltaan A-tyyppiä huonompi ja nopeakasvuisempi. HER2-positiivinen rintasyöpä on aiemmin ollut ennusteeltaan huono, mutta nykyisten täsmälääkkeiden ansiosta sen ennuste on parantunut. Huonoin ennuste on basaalisella tyypillä, jonka kasvaimet ovat huonosti erilaistuneita ja niin sanottuja kolmoisnegatiivisia, eli hormonireseptori- sekä HER2-negatiivisia.
\end{itemize}

\section{Nekroottinen varvas: miten toimit}\label{nekroottinen-varvas-miten-toimit}

Ei vaihtoehtoja.

Vastaus riippuu siitä, mistä syystä varvas on nekrotisoitunut ja mikä on potilaan vointi. Todennäköisesti kysymyksessä on kuitenkin ollut krooninen raajaa uhkaavaa iskemia (jos potilaalla nekroottinen varvas tai haava yli 2vk), jossa kudosvaurio (haava tai kuolio) sijaitsee tyypillisesti varpaiden kärjissä tai luu-ulokkeen kohdalla.

\begin{itemize}
\tightlist
\item
  Jos potilas on muuten hyvävointinen (ei sepsikseen viittaavaa todettavissa) eikä raajassa ole infektion merkkejä niin mikrobilääkettä eikä päivystyksellistä hoitoa rutiinisti tarvita.

  \begin{itemize}
  \tightlist
  \item
    Tutki siis potilas huolella (varsinkin alaraajapulssit, ABI, alaraajavoimat/-sensoriikka yms.) ja tee kiireinen lähete verisuonikirurgialle (Jos haavauman ja jo nyt nekroosin syyksi on jo todettu iskemia, niin lähete tulee tehdä välittömästi ilman hoitokokeilua). Optimaalinen kardiovaskulaaristen riskitekijöiden hoito aloitetaan välittömästi lähetteen teon yhteydessä kokonaisennusteen parantamiseksi.
  \end{itemize}
\end{itemize}

Kriittinen iskemia vaatii aina revaskularisaation (ehkä ei jos on vuodepotilas ja päätetään palliatiivinen linjaus). ASO-taudin kajoava hoito on joko kirurgista (endarterektomia tai ohitusleikkaus) tai suonensisäistä (pallolaajennus (percutaneous transluminal angioplasty, PTA) ja stenttaus (metalliverkon asentaminen laajennettuun suonisegmentin sisälle) ja trombolyysihoidot) ja samassa toimenpiteessä voidaan käyttää molempia hoitomuotoja (ns. hybriditoimenpide).

\section{Palovammojen hoito}\label{palovammojen-hoito}

Ei vaihtoehtoja, tässä palovammoista tiivistelmä:

\textbf{Ensihoito:}

\begin{itemize}
\tightlist
\item
  Tapahtumapaikalla tärkein ensimmäinen toimenpide on palovammaa aiheuttavan altistuksen lopettaminen. Tämän jälkeen tulee viipymättä hälyttää lisäapua hätäkeskuksesta, jos kyseessä on iso tapaturma tai vakava loukkaantuminen.
\item
  Palovamma-aluetta viilennetään mahdollisuuksien mukaan. Paras vaihtoehto on suihkuttaa haava-aluetta juoksevalla 15--20-asteisella vedellä 20--30 minuutin ajan. Tärkeää on kuitenkin välttää potilaan jäähtymistä, joten lumihankeen menoa tai jääkylmää vettä tulee välttää, ennen kaikkea lapsipotilailla.
\end{itemize}

\textbf{Lääkärinä primääritilanteessa:}

\begin{itemize}
\tightlist
\item
  Poista rakkulat --- pohja on arvioitavissa paremmin
\item
  Jos kyseessä on lievä palovamma ja oletat, että se ei tarvitse leikkaushoitoa -\textgreater{} aloita hopeahoito (mepilex ag tmv.) ja kontrolloi tilanne PTH:ssa 2-5vrk kohdalla syvenemisen toteamiseksi

  \begin{itemize}
  \tightlist
  \item
    Palovamma syvenee 48--72 tuntia vamman jälkeen. Tämän takia vamman syvyysarvio tulee tehdä uudelleen lopullisen syvyysarvion ja hoitosuunnitelman tekoa varten.
  \end{itemize}
\item
  Jos kyseessä on pieni mutta syvä (leikkaushoito) --- aloita hopea ja tee plastiikkakirurgialle lähete 1-7 vrk
\item
  Jos kyseessä on laaja palovamma joka vaatii välittömiä toimenpiteitä/sijaitsee kriittiseillä anatomisilla alueilla tai vaatii tehohoitoa/nesteresuskitaatiota ---\textgreater{} Plastiikkakirurgian päivystäjä ja osastohoito
\item
  Jos kyseessä on lievempi palovamma mutta potilas on kohtuuttoman kipeä --- konsultoi plastiikkakirurgian päivystäjää ---\textgreater{} osastohoito mahdollisesti
\end{itemize}

\textbf{Lyhyesti palovammojen laajuuden ja syvyyden arvioinnista:}

\begin{itemize}
\tightlist
\item
  Palovamman laajuus arvioidaan käyttäen 9 \%:n sääntöä (rule of nines). Pienten läiskäisten vamma-alueiden arviointia varten voidaan käyttää niin sanottua ``kämmensääntöä'', jossa potilaan oma käsi sormet yhdessä vastaa noin 1 \% potilaan kehon pinta-alasta.

  \begin{itemize}
  \tightlist
  \item
    Eroaa lapsipotilailla
  \item
    Yli 20\% palovamma (lapsilla 10\%) on systeemisairaus ja vaatii erityisosaamista sekä tehohoitoa = laaja palovamma
  \end{itemize}
\item
  Syvyyden mukaan palovammat on perinteisesti jaoteltu kolmeen eri vaikeusasteeseen. Toisen asteen dermaalinen palovamma voidaan kuitenkin vielä jakaa kolmeen eri syvyysluokkaan.

  \begin{itemize}
  \tightlist
  \item
    Ensimmäisen asteen palovammoissa (esim. tyypillinen auringonpolttama iho) vaurio rajoittuu epidermikseen. Iho on pinnaltaan kuiva, punoittava ja kosketusarka. Ihossa on turvotusta muttei rakkuloita. Kapillaarireaktio nopea. Nämä vammat paranevat 3--7 vuorokauden sisällä arpia jättämättä. Ensimmäisen asteen palovammaa ei lasketa mukaan potilaan palovammaprosenttiin palovamman laajuutta arvioitaessa.
  \item
    2A-asteen palovamma (pinnallinen dermaalinen): tyypillisiä ihon pintaan (epidermis ja pinnallinen kerros dermiksestä) syntyvät rakkulat, jotka usein syntyvät muutamien tuntien kuluessa vammautumisesta; Haavan pohja on vaaleanpunainen, kiiltäväpintainen, painettaessa siinä on nähtävissä nopea vitaalireaktio (capillary refill) ja vamma on hyvin kivulias. Nämä vammat paranevat paikallishoidolla 10--14 päivän kuluessa arpia jättämättä.
  \item
    2B ja 2C (keskisyvä ja syvä dermaalinen): pohja on tumma ja läiskittäin kalpea. Kapillaarireaktio ja ihotunto puuttuvat -\textgreater{} kipu on vähäisempää kuin pinnallisessa dermaalisessa vammassa. Paranemistaipumus konservatiivisesti hoidettuna huono (keskisyvät voivat ehkä parantua spontaanisti).
  \item
    3 asteen vamma: Haavapinnat ovat vaaleita tai ruskeita, kuivia, tunnottomia ja nahkamaisia. Tunnoton, kivuton ja kapillaarireaktiota ei ole.
  \end{itemize}
\end{itemize}

Syvyyden arvioimisesta tärkeintä muistaa on se, että se tulee kontrolloida (2-5vrk ish), koska palovamma syvenee ensiarvion jälkeen. Alkuvaiheessa pinnallinen voi siis syventyä sen verran, että spontaani paraneminen ei enää ole mahdollista.

\pandocbounded{\includegraphics[keepaspectratio]{images/ruleofnines.png}}
\pandocbounded{\includegraphics[keepaspectratio]{images/palovammaluokatleokuvat.png}}
\pandocbounded{\includegraphics[keepaspectratio]{images/palovammaluokatleo.png}}
\pandocbounded{\includegraphics[keepaspectratio]{images/palovammahoitopaikat.png}}

\textbf{Syvän palovamman kirurginen hoito}

Palovamman kirurgisen hoidon tavoitteena on tunnistaa leikkaushoitoa vaativat palovammat ja leikata ne riittävän ajoissa. Syvissä toisen ja kolmannen asteen vammoissa päätöksenteko on usein helppoa, mutta keskisyvissä dermaalisissa toisen asteen vammoissa tarvitaan joskus 2--3 viikkoa seuranta-aikaa, jolloin odotetaan vamman lopullista rajautumista.

\begin{itemize}
\tightlist
\item
  Pienemmissä palovammoissa on usein mahdollista tehdä palovamman poistoleikkaus, eksisio, ja palovamman peittäminen samassa istunnossa. Laajemmissa palovammoissa (\textgreater{} 15 \%) leikkaus tehdään usein useammassa vaiheessa, jossa ensin tehdään palovamman eksisio ja tilapäinen peitto. Lopullinen haavojen peittäminen omaihosiirteillä tehdään sitten vaiheittain riippuen palovamman laajuudesta.
\item
  Leikatut palovamma-alueet tulee aina peittää. Pienet palovammat voidaan leikata pois ja haava sulkea suoraan tai sitten käyttäen esimerkiksi paikallista iho-subkutiskielekettä. Tämä kuitenkin soveltuu käytettäväksi vain harvoin. Yleisin palovammahaavan peittoon käytetty menetelmä onkin autografti eli potilaan oma iho. Ihosiirre otetaan sieltä, mistä se on parhaiten saatavilla, kuitenkin mahdollisuuksien mukaan huomioiden esteettiset ja toiminnalliset ottokohtaan liittyvät asiat, kuten siirteen väri (kasvojen iho on eriväristä kuin reiden iho) ja ottokohdan jättämä arpi.

  \begin{itemize}
  \tightlist
  \item
    Allografteja eli elinluovuttajalta saatua ihoa käytetään laajoissa palovammoissa väliaikaisena peittona.
  \end{itemize}
\end{itemize}

\pandocbounded{\includegraphics[keepaspectratio]{images/palovammaluokat.png}}

\textbf{Vakava palovamma:}

Aikuisella yli 20 \% kehon pinta-alasta käsittävässä vammassa permeabiliteetihäiriö leviää koko elimistöön johtaen lisääntyneeseen nestetarpeeseen ja yleistyneisiin turvotuksiin ja hoitamattomana palovammasokkiin. Lapsipotilailla laskimonsisäinen nestehoito on aiheellinen jo palovammoissa, jotka käsittävät kehon pinta-alasta 10 \% tai enemmän.

\begin{itemize}
\tightlist
\item
  Tämän vuoksi laajan palovamman hoidon kulmakivenä on riittävä nesteytys.
\item
  Palovammapotilaan nestehoidossa seurataan useimmiten modifoitua Parklandin kaavaa eli 3 ml × paino (kg) × palovamman laajuus (\%). Puolet kaavasta laskettavasta nestemäärästä annetaan ensimmäisen 8 tunnin kuluessa vamman sattumishetkestä ja loput siitä seuraavan 16 tunnin aikana.

  \begin{itemize}
  \tightlist
  \item
    Ennen oli alkuperäinen Parklandin kaava, jossa kaavassa oli 3ml tilalla 4ml ja tämä johti helposti potilaan ylinesteytykseen
  \end{itemize}
\end{itemize}

Kivunhoito on tärkeää ja alkuvaiheessa kivunhoitoon soveltuu parhaiten suonensisäisesti annostellut opioidit kuten oksikodoni tai fentanyyli.

Syvän palovamman alueelle muodostuu panssarimainen, joustamaton kudos, eskar. Sirkulaarinen eskar raajojen alueella voi johtaa alla olevan kudoksen verenkierron salpaantumiseen ja iskemiaan, jolloin raajan elinkelpoisuus voi olla uhattuna. Rintakehän alueella eskar voi johtaa hengitystiepaineiden nousuun ja potilaan ventilaation huononemiseen. Vatsan alueella eskar voi aiheuttaa vatsaontelon sisäisen paineen nousun ja johtaa suoli-iskemiaan ja munuaistoiminnan huononemiseen.

\begin{itemize}
\tightlist
\item
  Jos nesteytys, kivunhoito, raajojen kohoasento yms ei auta, niin tarvitaan eskarotomia, jossa syvä palanut kudos halkaistaan
\item
  Eskarotomiat voidaan tehdä vuodeosasto-olosuhteissa tai teho-osastolla, kunhan on varauduttu verenvuodon hoitamiseen esimerkiksi diatermialla. Syvissä lihaskalvoon asti ulottuvissa palovammoissa ja sähköpalovammoissa tarvitaan usein myös faskiotomiat, ja nämä on syytä tehdä leikkaussalissa yleisanestesiassa.
\end{itemize}

Laajojen palovammojen hoito on keskitetty Suomessa HUSin palovammakeskukseen

\begin{itemize}
\tightlist
\item
  Jokaisessa laajan päivystyksen sairaalassa tulee kuitenkin olla valmius vaikeastikin loukkaantuneen palovammapotilaan hoitamiseen ensimmäisen 24--72 tunnin ajan vamman sattumisesta
\end{itemize}

\begin{figure}
\centering
\pandocbounded{\includegraphics[keepaspectratio]{images/eskarotomia.png}}
\caption{Eskarotomia}
\end{figure}

\section{Mitä abdominoplastia tarkoittaa?}\label{mituxe4-abdominoplastia-tarkoittaa}

Ei vaihtoehtoja, mutta tässä vastaus:

Abdominoplastiassa eli \textbf{riippuvatsan korjausleikkauksessa} (``tummy tuck'') poistetaan alavatsan roikkuva ihopoimu (jää usein merkittävän painonlaskun merkiksi). Tavoitteena on palauttaa vatsanpeitteiden ihon venyttymistä edeltänyt anatomia.

\begin{itemize}
\tightlist
\item
  Leikkaustekniikoita on useita, yleisimmin käytetty on kyljestä kylkeen viilto pubiksen yläpuolella ja ylimääräinen iho-ja pehmytkudos poistetaan. Napa jätetään kiinni faskiaan ja nostetaan iholle erillisestä viillosta (napaplastia). Suorien vatsalihasten erkauma (rektusdiastaasi) voidaan korjata abdominoplastian yhteydessä lihaskalvon ompelemisella (rektusplikaatio) tai käyttämällä verkkoa.

  \begin{itemize}
  \tightlist
  \item
    Abdominoplastia voidaan tehdä myös fleur de lis -tyyppisesti, jolloin perinteisen abdominoplastian lisäksi poistetaan vatsan keskilinjan ihoylimäärä erillisen pitkittäisen leikkausviillon avulla. Lopputuloksena on tällöin alavatsalla kulkevan horisontaalisen arven lisäksi keskiviiltoarpi.
  \item
    Alavartalon kohotusleikkauksessa (bodylift/ belt lipektomia, circumferential abdominoplasty) poistetaan alavatsan ihopoimun lisäksi kylkien ja alaselän alueilla olevat roikkuvat poimut (voivat pahimmillaan haitata istumista tai ulostamista).
  \end{itemize}
\end{itemize}

Jos potilaalla on kookas ja painava alavatsapoimu, joka aiheuttaa toistuvia selluliitteja, kroonista haavautumista ja imunestekiertohäiriötä tai vaikeuttaa liikkumista, voidaan potilaalle harkita \textbf{pannikulektomiaa.} \emph{Toimenpide soveltuu potilaille, jotka ovat monisairaita tai joilla on merkittävä obesiteetti, jolloin abdominoplastian leikkauskriteerit eivät painoindeksin suhteen täyty.}

\begin{itemize}
\tightlist
\item
  Toimenpiteessä riippuvatsa poistetaan amputaation tapaan, jolloin napa dislokoituu kaudaalisesti tai se poistetaan kokonaan. Tällaisessa toimenpiteessä ei huomioida vartalon muodon esteettisiä tarpeita eli kosmeettinen lopputulos on huonompi.
\item
  Harvinainen toimenpide ja tapauskohtainen arvio
\end{itemize}

\pandocbounded{\includegraphics[keepaspectratio]{images/vartalonmuovauskriteerit.png}}
\pandocbounded{\includegraphics[keepaspectratio]{images/abdominoplastiatyypit.png}}
\pandocbounded{\includegraphics[keepaspectratio]{images/pannikulektomia.png}}

\section{Abdominoplastian tavallisimmat komplikaatiot}\label{abdominoplastian-tavallisimmat-komplikaatiot}

Ei vaihtoehtoja, mutta tässä tärkeimmät:

Vartalonmuovausleikkausten jälkeiset komplikaatiot ovat yleisiä laajoista leikkausalueista, leikkauksen kestosta ja mahdollisista liitännäissairauksista johtuen. Myös venyttyneen ihon paranemiskyky on heikentynyt verrattuna ei-venyttyneeseen ihoon. \textbf{Komplikaatioita esiintyy jopa 50 \%:lla leikatuista.} Suurin osa komplikaatioista on kuitenkin pieniä ja paranee paikallishoidoilla. Yleisimpiä ovat:

\begin{itemize}
\tightlist
\item
  Serooma
\item
  Aukileet
\item
  Infektiot
\item
  Akuutti vuoto
\item
  Hematooma
\item
  Navan nekroosi
\item
  Kivuliaisuus tai tunnottomuus
\item
  Tromboemboliset komplikaatiot (harvinaisia)
\end{itemize}

\section{Lippaluomien leikkausindikaatio}\label{lippaluomien-leikkausindikaatio}

Dermatochalasis eli lippaluomi tarkoittaa yläluomen ihoylimäärää +/- rasvaa. Laskeutuu silmäluomen päälle ja voi peittää näkökenttää, aiheuttaen näköhaittaa tai väsyneen ilmeen

\begin{itemize}
\tightlist
\item
  Hyvänlaatuinen muutos
\item
  Erotettava ptoosista eli riippuluomesta, jossa yläluomi laskeutuu ja peittää näkökenttää. Nopeasti edennyt toispuolinen ptoosi vaatii neurologin kiireellisen arvion (oculomotorius pareesi, myasthenia gravis, carotisdissekaatio\ldots). Seniili riippuluomi johtuu yleensä levator-lihaksen toiminnan vajeesta tai venymisestä.

  \begin{itemize}
  \tightlist
  \item
    Riippu- ja lippaluomi voivat esiintyä yhdessä
  \end{itemize}
\end{itemize}

Hoitona on leikkaus. Lippaluomen yläluomileikkauksessa poistetaan yläluomen alueelle muodostunut ylimääräinen ihopoimu ja tarvittaessa muotoillaan alla olevaa rasvaa. \textbf{Blefaroplastia = silmäluomileikkaus.} Riippuluomen leikkauksessa erilaisia tekniikoita.

\begin{itemize}
\tightlist
\item
  Raskas yläluomi aiheuttaa väsyneen ilmeen, mutta näkökentän eteen laskeutuessaan lippaluomipoimu aiheuttaa myös toiminnallisen ongelman. Tällöin tilanne rinnastetaan sairaudeksi, jolloin leikkaus voidaan toteuttaa julkisessa sairaanhoidossa.
\item
  Julkisella yleisinä yläluomen kirurgian kriteereinä voidaan pitää:

  \begin{itemize}
  \tightlist
  \item
    Toiminnallinen näkökykyhaitta (kompensatorinen otsajännitys voi esim aiheuttaa estolääkitystä vaativan migreenin)
  \item
    MRD-1 mitta \textless2mm (MRD = Margin-reflex distance. Iho peittää näkökenttää tulemalla alle 2 mm:n päähän mustuaisen valoheijasteesta/optiselta akselilta)
  \end{itemize}
\end{itemize}

\pandocbounded{\includegraphics[keepaspectratio]{images/lippaluomi.png}}

\section{Potilaalla selkeä virtsatiekivikohtaus + CRP 85 ja lämpö 38,5. → Miten toimit?}\label{potilaalla-selkeuxe4-virtsatiekivikohtaus-crp-85-ja-luxe4mpuxf6-385.-miten-toimit}

Ei vaihtoehtoja, mutta tässä tärkeimpiä pointteja:

Diagnostiikka:

\begin{itemize}
\tightlist
\item
  Virtsatiekivikohtauksessa U-BaktVi, U-KemSeul ja P-Krea tarkistetaan; myös CRP, PVKT, Ca-Ion, Uraat, U-solut jos käytettävissä
\item
  Kaikututkimus näyttää mm. hydronefroosin ja on ensisijainen tutkimus avohoidossa. Jos hydronefroosia ei todeta, kreatiniinipitoisuus on normaali ja virtsatieinfektiota ei ole, voidaan tilannetta jäädä seuraamaan.
\end{itemize}

\textbf{Potilaalla on kuumetta ja selvästi voidaan epäillä virtsatieinfektiota -\textgreater{} kuumeinen virtsatieinfektio -\textgreater{} ei voida seurata perusterveydenhuollossa vaan lähetetään erikoissairaanhoitoon}

\begin{itemize}
\tightlist
\item
  Erikoissairaanhoidossa päivystystutkimuksena tehdään usein natiivi-TT
\end{itemize}

Hoito:

\begin{itemize}
\tightlist
\item
  Hoito on aiheellista päivystyksellisesti, jos tukokseen liittyy virtsatieinfektio. Infektiossa aloitetaan suonen sisäinen mikrobilääkehoito (kefuroksiimi 1.5 g × 3 i.v.).
\item
  Infektiotukoksen laukaisu radiologin asettamalla punktiopyelostomialetkulla tai tähystyksen yhteydessä asetettavalla ureterkatetrilla
\item
  Jos tukoksen aiheuttama kivi sijaitsee alaureterin alueella, kiven voi poistaa ureteroskooppisesti myös infektiotilanteessa
\end{itemize}

\section{Ruokatorviakalasia}\label{ruokatorviakalasia}

Ei vaihtoehtoja tai kysymyksenasettelua, mutta wikissä tarjotusta vastauksesta voi päätellä, mitä on kysytty: \textbf{``Mikä on akalasian diagnostiikassa tärkein tutkimus?''}

\begin{solution}
\leavevmode

Vastaus

\begin{verbatim}
 Manometria
\end{verbatim}

Akalasia on ruokatorven motiliteettihäiriö, johon liittyy ruokatorven alasulkijan epätäydellinen relaksaatio sekä ruokatorven runko-osan puuttuva peristaltiikka. Tavallisin akalasian oire on kiinteän ja usein myös nestemäisen ruoan nielemisvaikeus (dysfagia), joka on useimmiten pahentunut hitaasti jo vuosia ennen diagnoosia. Sen seurauksena potilaat usein laihtuvat. Ruokatorveen retentoituu ruokaa ja nestettä, ja niiden regurgitoituminen on yleistä. Joskus regurgitaatio on pääoire, ja siihen voi liittyä kroonista aspiraatiota sekä yskää ja toistuvia keuhkokuumeita. Regurgitaatio voidaan sekoittaa refluksitautiin, koska molempiin voi liittyä närästystä.

Akalasian diagnoosi perustuu tyypilliseen manometrialöydökseen ja tyypilliseen taudinkuvaan. Manometriassa siis mitataan ruokatorveen paineen muutoksia katetrin avulla, joka ohjataan ruokatorveen nenän kautta. Lisäksi tarvitaan endoskopia kasvainten ja muiden sekundaarisen akalasian syiden sulkemiseksi pois.

Tutkimuksessa todetaan mm. korkea LES:n (alemman ruokatorven sfinkterin) lepopaine, epätäydellinen LES:n relaksaatio ja epäkoordinoitu/olematon peristaltiikka (nielemisen seurauksena ei joko synny minkäänlaisia supistuksia ruokatorven runko-osassa tai voi toisinaan näkyä samanaikaisia matalapaineisia, ei-peristalttisia supistuksia). Normaalisti nielemisen jälkeen alasulkija on täydellisen relaksoitunut eli paine on alle 8 mmHg yli mahalaukun vallitsevan paineen. Akalasiassa sitä vastoin relaksaatiota ei ilmene tai se on vajavainen.

\end{solution}

\section{Potilaalla toistuvia sappikivikohtauksia. Ultrassa todettu kiviä sappirakossa. ALAT, AFOS ja Bil lievästi koholla. Mitä seuraavaksi?}\label{potilaalla-toistuvia-sappikivikohtauksia.-ultrassa-todettu-kiviuxe4-sappirakossa.-alat-afos-ja-bil-lievuxe4sti-koholla.-mituxe4-seuraavaksi}

\begin{itemize}
\tightlist
\item
  \begin{enumerate}
  \def\labelenumi{\alph{enumi}.}
  \tightlist
  \item
    MRCP
  \end{enumerate}
\item
  \begin{enumerate}
  \def\labelenumi{\alph{enumi}.}
  \setcounter{enumi}{1}
  \tightlist
  \item
    Päivystyksellinen sappirakon poisto
  \end{enumerate}
\item
  \begin{enumerate}
  \def\labelenumi{\alph{enumi}.}
  \setcounter{enumi}{2}
  \tightlist
  \item
    Elektiivinen sappirakon poisto
  \end{enumerate}
\item
  \begin{enumerate}
  \def\labelenumi{\alph{enumi}.}
  \setcounter{enumi}{3}
  \tightlist
  \item
    Joku (ei wikissä)
  \end{enumerate}
\end{itemize}

\begin{solution}
\leavevmode

Vastaus

\begin{verbatim}
 c
\end{verbatim}

Sappikivikoliikkipotilaat (komplisoitumaton sappikivitauti, joka oireilee ajoittaisina sappikivikohtauksina) hoidetaan elektiivisesti. Tulisi leikata muutaman kuukauden kuluessa, vaikeaoireiset vielä nopeammin. Jo ensimmäisen sappikoliikin jälkeen potilaasta kannattaa tehdä elektiivinen lähete kolekystektomian harkintaan. Leikkausta odotettaessa on vältettävä sappikipuja provosoivia ruokia. Koliikkien hoidoksi annetaan tulehduskipulääkkeitä.

a: Nyt ei epäillä sappitiekiviä, joiden tutkimisessa MRCP olisi paras. Maksa-arvot ja Bilirubiini kyllä ovat lievästi koholla, mutta sappikoliikki kohtauksen aikana voidaan usein todeta lievää arvojen nousua. Jos arvot nousevat enemmän tai potilaan vointi on merkittävän huono, niin silloin kyllä tulisi tehdä MRCP.

b: Ei ole kyseessä kolekystiitti (ei todettu ultrassa), jonka hoito olisi päivystyksellinen (tai vähintään kiireellinen) poisto

\end{solution}

\section{Leikkaushaava märkii muutama päivä leikkauksen jälkeen. Potilas muuten hyväkuntoinen. Mitä tehdään}\label{leikkaushaava-muxe4rkii-muutama-puxe4ivuxe4-leikkauksen-juxe4lkeen.-potilas-muuten-hyvuxe4kuntoinen.-mituxe4-tehduxe4uxe4n}

Ei vaihtoehtoja, mutta koita vastata ilman vinkkejä

\begin{solution}
\leavevmode

Vastaus

\begin{verbatim}
 Klaffaus
\end{verbatim}

Jos leikkaushaava märkii (ihan märkää, haava-alueelle rajoittuva tai haavan reunan kapea punoitus ja vähäinen kirkas tai kellertävä tuoksuton kudosnestevuoto eivät ole merkkejä tulehduksesta), niin kyseessä on todennäköisesti pinnallinen haavainfektio. Poista ompeleet, klaffaa (avaa haava) ja jätä haava auki. Ota bakteeriviljelyväyte. Ohjaa haavan suihkutukset ja antibioottihoito.

\end{solution}

\section{Epäily uretervauriosta. Mikä tutkimus?}\label{epuxe4ily-uretervauriosta.-mikuxe4-tutkimus}

Ei vaihtoehtoja, mutta koita vastata ilman vinkkejä

\begin{solution}
\leavevmode

Vastaus

\begin{verbatim}
 UÄ ja pyelografia/urografia
\end{verbatim}

Virtsan ekstravasaatio on uretervaurion komplikaatio. Diagnoosi varmistuu kaikututkimuksella. Mikäli rakon ulkopuolella nähdään urinoomaksi sopiva nestekollektio, se on syytä kanalisoida. Kreatiniinin määritys punktaatista varmistaa virtsan ekstravasaation. Tilanne pyritään rauhoittamaan kanalisaation avulla.

Vuotokohta paikannetaan perkutaanisen antegradisen pyelografian avulla. Hoito onnistuu useimmissa tapauksissa virtsan diversiolla perkutaanisen pyelostooman kautta. Jos ureterin korjausleikkaukseen joudutaan, se on syytä lykätä muutaman kuukauden päähän, jotta virtsavuodon vaurioittamat kudokset saavat kunnolla parantua.

Pyelografia = Virtsateitä voidaan kuvantaa niihin suoraan viedyn varjoaineen avulla. Antegradisessa pyelografiassa varjoaine ruiskutetaan suoraan altaaseen ohuen punktioneulan kautta. Altaaseen voidaan virtausesteen tultua todetuksi asentaa väliaikainen tai pysyvä nefrostoomaputki. Retrogradinen pyelografia suoritetaan ruiskuttamalla varjoaine virtsanjohtimen distaaliosaan kystoskopiassa viedyn katetrin kautta.

Urografiassa varjoaine annetaan suonensisäisesti ja otetaan urografiasarjat. Pyelografia on tarkempi, mutta urografia helpompi suorittaa.

\end{solution}

\section{Akuutti pankku. Tarvitaanko antibiootteja?}\label{akuutti-pankku.-tarvitaanko-antibiootteja}

Ei vaihtoehtoja, mutta koita vastata ilman vinkkejä

\begin{solution}
\leavevmode

Vastaus

\begin{verbatim}
 Ei rutiinisti
\end{verbatim}

Akuutin pankreatiitin ensisijainen hoito on yleensä konservatiivinen ja tärkeintä on alkuvaiheen nesteytys, elektrolyyttihäiriöiden korjaaminen, varhaisessa vaiheessa p.o. ravitsemukseen siirtyminen ja kipulääkitys; vaikeissa taudeissa ulkusprofylaksia (PPI).

Edes ab-hoitoa ei yleensä tarvita eikä suositella pankreatiitin infektiokomplikaatioiden ehkäisyyn (aloitetaan kuitenkin herkemmin sappipankreatiitissa ja varsinkin jos myös kolangiitti). Mikrobilääkkeistä ei ole hyötyä infektioiden ehkäisemisessä, eikä niitä tule käyttää, jos potilaalla ei ole toista infektiopesäkettä, kuten sepsistä tai kolangiittia.

\end{solution}

\section{Peräaukon abskessi avattu. Alkanut parantua, mutta välillä aukeaa uudestaan ja vuotaa märkää. Mikä juttu?}\label{peruxe4aukon-abskessi-avattu.-alkanut-parantua-mutta-vuxe4lilluxe4-aukeaa-uudestaan-ja-vuotaa-muxe4rkuxe4uxe4.-mikuxe4-juttu}

Ei vaihtoehtoja, mutta koita vastata ilman vinkkejä

\begin{solution}
\leavevmode

Vastaus

\begin{verbatim}
 Fisteli
\end{verbatim}

Anaaliabsessin ensisijainen hoito on insisio päivystysluontoisesti (leikkaussalissa anestesiassa). Absessi avataan perianaaliselta iholta fluktuaation tai resistenssin kohdalta. Pelkkä mikrobilääkehoito ei koskaan riitä yksin hoidoksi, koska absessi ei parane ja tila voi vain entisestään komplisoitua

Absessin avauksen jälkeen hoitona on suihkuttelu. On tärkeä huolehtia siitä, että ihon reunat eivät sulkeudu ennenaikaisesti. \textbf{Absessin uusiutuminen hoidon jälkeen viittaa hoitamattomaan fisteliin.}

\end{solution}

\section{Kenelle paksusuolisyövän kansallinen seulonta}\label{kenelle-paksusuolisyuxf6vuxe4n-kansallinen-seulonta}

Ei vaihtoehtoja, mutta koita vastata ilman vinkkejä

\begin{solution}
\leavevmode

Vastaus

\begin{verbatim}
 Lopulta 56-74v 
\end{verbatim}

Suomessa aloitettiin uusi suolistosyöpäseulonta vuonna 2022. Seulonta koskee tällä hetkellä 60--70-vuotiaita, ja se laajenee ikäryhmittäin siten, että ohjelma ulottuu kaikkiin 56--74-vuotiaisiin vuodesta 2031 alkaen. Seulontaan kutsutaan 2 v:n välein niinä vuosina, jolloin kutsuttava täyttää parillisia vuosia.

Seulontatesti on immunokemiallinen ulosteen veritesti (FIT). Seulontaa varten tarvitaan vain yksi ulostenäyte, joka otetaan itse kotona ja lähetetään postitse laboratorioon tutkittavaksi. Veritestin ollessa positiivinen tehdään kolonoskopia.

Seulonta-aikataulua muutetaan syöpävaaran mukaan (esim. jos 1-2 lähisukulaisella on kolorektaalisyöpä, niin on perusteltua harkita seulontakolonoskopiaa 5 vuotta nuorempana kuin nuorin sairastunut sukulainen).

\pandocbounded{\includegraphics[keepaspectratio]{images/paksusuoliseulonta.png}}

\end{solution}

\section{Epäily, että askitesdreeni vaurioittanut virtsateitä ja dreeni erittää virtsaa. Miten selvität?}\label{epuxe4ily-ettuxe4-askitesdreeni-vaurioittanut-virtsateituxe4-ja-dreeni-erittuxe4uxe4-virtsaa.-miten-selvituxe4t}

\begin{itemize}
\tightlist
\item
  \begin{enumerate}
  \def\labelenumi{\alph{enumi}.}
  \tightlist
  \item
    Krea-mittaus
  \end{enumerate}
\item
  \begin{enumerate}
  \def\labelenumi{\alph{enumi}.}
  \setcounter{enumi}{1}
  \tightlist
  \item
    Virtsateiden kuvantamisia
  \end{enumerate}
\end{itemize}

\begin{solution}
\leavevmode

Vastaus

\begin{verbatim}
  Virtsateiden kuvantaminen
\end{verbatim}

Vähän huonot vaihtoehdot, koska molemmat ovat sinänsä oikein. Kreatiniinin määritys dreenin nesteestä varmistaa virtsan ekstravasaation. Tarkempi diagnoosi kuitenkin tehdään kuvantamisella.

Varjoainetehosteinen tietokonekerroskuvaus urografiasarjoin osoittaa lähes aina varjoaineen kulkeutumisen virtsanjohtimen ulkopuolelle. Pyelografia (varjoaine suoraan virtsateihin) on vielä tarkempi.

Hoito riippuu vaurion sijainnista ja laajuudesta.

\end{solution}

\section{Mikä osa suolesta yleisimmin poistetaan Crohnin taudin leikkaushoidossa?}\label{mikuxe4-osa-suolesta-yleisimmin-poistetaan-crohnin-taudin-leikkaushoidossa}

Ei vaihtoehtoja, mutta koita vastata ilman vinkkejä (käytännössä kysymys ollut jo).

\begin{solution}
\leavevmode

Vastaus

\begin{verbatim}
  Terminaalinen ileum
\end{verbatim}

Tavallisin Crohnin taudin ilmentymä on aivan distaalisen sykkyräsuolen (ileum) usein lyhyehkö ahtauma. Niitä esiintyy 75 \%:lla sairastuneista. Jos ahtauma aiheuttaa oireita, kuten oksentelua, kouristavia vatsakipuja ja laihtumista, tehdään ileosekaalinen suolentypistys. Suurimmalle osalle potilaista toimenpiteen voi tehdä tähystysleikkauksena.

Noin kolmasosalla potilaista on tulehdus paksusuolen alueella. Tyypillisimmillään tulehdus on paksusuolen oikealla puolella, mutta se voi olla missä tahansa paksusuolen osassa. Jos Crohnin tauti rajoittuu paksusuolen oikeaan puoleen, voidaan tehdä oikeanpuoleinen paksusuolen typistys. Mikäli tulehtunut alue on laajempi eikä tautia ole peräsuolen alueella, tehdään subtotaali kolektomia ja ileosigmoidaalinen tai ileorektaalinen liitos. Jos myös peräsuolen alueella on vaikea tulehdus, anaalikanava on ahtautunut tai potilaalla on anaalifisteleitä, poistetaan peräsuoli ja peräaukko sekä tehdään pysyvä pääteavanne.

Periaate: Crohnin taudissa leikkaushoito keskittyy vain affisioituneeseen suolen osaan (ei hoida koko tautia, vaan vain poistaa sen hetken pahimmin affisioituneen suolen osan), kun taas colitis ulcerosassa voidaan poistaa koko paksusuoli ja saada siten kuratiivinen hoitotulos.

\end{solution}

\section{Transversostooma}\label{transversostooma}

Ei vaihtoehtoja tai tarkempaa kysymyksenasettelua, mutta tässä edes jotain avattu tästä termistä:

Suoliavanteet voidaan jaotella muutamalla tavalla (tässä ei kaikkia):

\begin{itemize}
\tightlist
\item
  Niiden pysyvyyden kautta: tilapäinen (suljetaan myöhemmin) tai pysyvä
\item
  Ulkonäön ja funktionaalisuuden kautta: pääteavanteet, lenkkiavanteet, kaksipiippuiset avanteet
\item
  Sijainnin mukaan: jejunostooma, ileostooma, ascendostooma, transversostooma, sigmoideostooma yms yms.
\end{itemize}

Transversostooma tarkoittaa, että on kyseessä transversumin eli poikittaisen paksusuolen avanne

\begin{itemize}
\tightlist
\item
  \textbf{Käytetään usein tilapäisesti, kun distaalinen koolon paranee jostakin (esim. kaksipiippuinen transversostooma anteriorisen resektion jälkeen).} Voi myös olla pysyvä, jos esim. koko distaalinen paksusuoli poistetaan.
\item
  Sijoitetaan tyypillisesti ylävatsalle oikealle navan yläpuolelle
\item
  Sigmoideostooman ja transversostooman uloste on aluksi löysää, mutta normaaliin ruokavalioon palattaessa uloste kiinteytyy
\end{itemize}

\pandocbounded{\includegraphics[keepaspectratio]{images/avannelokaatiot.png}}

\section{Pintalaskimoiden termoablaation jälkeen}\label{pintalaskimoiden-termoablaation-juxe4lkeen}

\begin{itemize}
\tightlist
\item
  \begin{enumerate}
  \def\labelenumi{\alph{enumi}.}
  \tightlist
  \item
    Sairaslomaa 4 vko
  \end{enumerate}
\item
  \begin{enumerate}
  \def\labelenumi{\alph{enumi}.}
  \setcounter{enumi}{1}
  \tightlist
  \item
    Potilas sairaalahoidossa vain 1-2h
  \end{enumerate}
\item
  \begin{enumerate}
  \def\labelenumi{\alph{enumi}.}
  \setcounter{enumi}{2}
  \tightlist
  \item
    Usein tulee runsasta verenvuotoa
  \end{enumerate}
\item
  \begin{enumerate}
  \def\labelenumi{\alph{enumi}.}
  \setcounter{enumi}{3}
  \tightlist
  \item
    Joudutaan usein uusimaan
  \end{enumerate}
\end{itemize}

\begin{solution}
\leavevmode

Vastaus

\begin{verbatim}
  b
  
\end{verbatim}

Polikliinisen vaahtoskleroterapian jälkeen potilas voidaan kotiuttaa käytännössä heti. Polikliinisen termoablaation jälkeen tarvitaan yleensä lyhyt 30-60 minuutin seuranta, jonka jälkeen potilas voidaan kotiuttaa. Keskeistä jälkihoidossa on välitön mobilisaatio ja hoitosukan käyttö laskimotukosriskin vähentämiseksi. Jälkihoitoon suositellaan painepuristusluokan II hoitosukkaa joko reisipituisena tai polvipituisena sen mukaan, mille alueille vaahtoskleroterapia on kohdentunut. Näyttöä pitkäkestoisen kompressiohoidon hyödyistä ei ole, joten jälkihoidon suositeltu pituus on 3-7 vrk.

a: Sairausloman tarve on yleensä vähäinen, mutta työn luonne tulee ottaa huomioon. 4vk on joka tapauksessa liian pitkään.

c: Verenvuoto on yleensä lievää.

d: Laser- ja radiotaajuusablaation jälkeen laskimovajaatoiminta on uusiutunut vuoden seurannassa 2--6 \%:lla ja 5 vuoden seurannassa 10--23 \%:lla. Taudin uusiutumisen perussyy on yleensä ollut anteriorisen aksessorisen magnarungon vajaatoiminta tai hoidetun pinnallisen päärungon rekanalisaatio. Vaahtoskleroterapiaan liittyy huomattavasti enemmän sekä ultraäänellä todettavan refluksin että kliinisesti merkittävän vajaatoiminnan uusiutumista kuin termoablaatioon.

\end{solution}

\section{Kuinka paljon terveydenhuollon kustannuksista menee painehaavoihin?}\label{kuinka-paljon-terveydenhuollon-kustannuksista-menee-painehaavoihin}

Ei vaihtoehtoja, koita vastata ilman vinkkejä.

\begin{solution}
\leavevmode

Vastaus

\begin{verbatim}
  2-4%
\end{verbatim}

Eli 460--920 milj. €.

Painehaavoja esiintyy 5--25 \%:lla potilaista eri terveydenhuollon yksiköissä riippuen hoidettavista potilasryhmistä. Suomessa hoidetaan vuosittain arviolta 55 000--80 000 potilasta, jolla on yksi tai useampi painehaava.

Painehaavojen ehkäisyn kustannukset ovat vain n.~10 \% niiden hoidon kustannuksista.

Painehaavan kehittymiseen liittyy lisääntynyt kuolemanriski mekanismilla, joka on tuntematon. Suomessa arviolta n.~500--1 000 ihmistä vuodessa kuolee painehaavojen aiheuttamiin komplikaatioihin.

\end{solution}

\section{Mikä pitää paikkaansa kiinnikkeistä}\label{mikuxe4-pituxe4uxe4-paikkaansa-kiinnikkeistuxe4}

\begin{itemize}
\tightlist
\item
  \begin{enumerate}
  \def\labelenumi{\alph{enumi}.}
  \tightlist
  \item
    Voidaan leikkauksella vähentää
  \end{enumerate}
\item
  \begin{enumerate}
  \def\labelenumi{\alph{enumi}.}
  \setcounter{enumi}{1}
  \tightlist
  \item
    Arpien muodostuminen korreloi kiinnikkeiden muodostumisen kanssa
  \end{enumerate}
\item
  \begin{enumerate}
  \def\labelenumi{\alph{enumi}.}
  \setcounter{enumi}{2}
  \tightlist
  \item
    Kiinnikkeillä menee 6-12 kk muotoutua
  \end{enumerate}
\item
  \begin{enumerate}
  \def\labelenumi{\alph{enumi}.}
  \setcounter{enumi}{3}
  \tightlist
  \item
    Kiinnikkeet muotoutuu heti
  \end{enumerate}
\end{itemize}

\begin{solution}
\leavevmode

Vastaus

\begin{verbatim}
  c
\end{verbatim}

Kiinnikkeiden muodostus vaatii peritoneumin vaurion. Vaurion jälkeen muodostuu vastaava reaktio kuin kudoksen paranemisessa -\textgreater{} arven muodostus. Kuukausien kuluessa kiinnikkeet muokkautuvat, voivat olla kalvomaisia tai paksua arpea.

a: Leikkauksella ei voida estää kiinnikkeitä; ne syntyvät usein juuri leikkauksen seurauksena. Joskus kiinnikkeitä tulee irrotella uudella leikkauksella, mutta tätä vältetään mahdollisuuksien mukaan, koska uusintaleikkaus lisää kiinnikkeiden syntyä.

b: Arpien tai haavojen näkyvyys iholla ei korreloi vatsan sisäisten kiinnikkeiden kanssa. Keloiditaipumus (alkuperäisen haavan rajat ylittävä arven liikakasvu) tosin saattaa lisätä kiinnikkeiden muodostumista.

\pandocbounded{\includegraphics[keepaspectratio]{images/kiinnikkeet.png}}

\end{solution}

\chapter{2024 (Apricus)}\label{apricus}

Taas paljon vajavaisia kysymyksenasetteluita ja vaihtoehtoja. Jonkin verran myös aikaisempia tärppejä, joita ei nyt tässä yhteydessä käsitellä uudestaan.

\section{Sappistaasin labrat}\label{sappistaasin-labrat}

Ei vaihtoehtoja, mutta tässä sappistaasin labroista tärkeimmät:

\textbf{Sappikivitautia epäiltäessä kannattaa määrittää tulehdus-, bilirubiini-, maksa- ja amylaasiarvot.}

\begin{itemize}
\tightlist
\item
  Sappikoliikissa yleensä viitealueella
\item
  Akuutissa kolekystiitissä tulehdusarvot yleensä selvästi suurentuneet, joskus myös bilirubiini- ja maksa-arvot suurenevat lievästi
\item
  \textbf{Sappitiekiviin (koledokolitiaasi) viittaavat selvemmin koholla olevat bilirubiini- ja maksa-arvot (erityisesti AFOS, Bil eli kolestaattinen (sappistaasi) kuva)}
\item
  Kolangiitissa sekä tulehdus-, bilirubiini- että maksa-arvot (erityisesti AFOS, Bil eli kolestaattinen kuva) ovat selvästi suurentuneet
\end{itemize}

Sappikoliikissa ei siis ole vielä mitään merkittävää tulehdusta -\textgreater{} labrat normaalit

\begin{itemize}
\tightlist
\item
  Akuutissa kolekystiitissä sappikoliikkia aiheuttava kivi ei poistu ductus cysticuksesta ja johtaa lopulta sappirakon tulehdukseen -\textgreater{} tulehdusarvot nousevat
\item
  Koledokolitiaasissa bilirubiini- ja maksa-arvot (ALAT, AFOS, GT) nousevat kolestaasin takia. Jos koledokolitiaasi ei väisty ja tila etenee kolangiitiksi, niin tulehdusarvot nousevat myös.
\end{itemize}

Amyl nousisi pankreatiitissa

\begin{itemize}
\tightlist
\item
  Sappipankreatiittia esiintyy noin 5 \%:lla oireisista sappikivipotilaista
\item
  Akuutti sappipankreatiitti syntyy, kun sappitiekivi kiilautuu Oddin sulkijalihakseen
\end{itemize}

\section{Mikä avanne muodostetaan Hartmannin leikkauksessa?}\label{mikuxe4-avanne-muodostetaan-hartmannin-leikkauksessa}

Ei vaihtoehtoja, mutta koita vastata ilman vinkkejä.

\begin{solution}
\leavevmode

Vastaus

\begin{verbatim}
 Laskevan koolonin pääteavanne
\end{verbatim}

Hartmannin leikkaus on usein käytetty päivystyksellisesti esim. perforoituneen divertikuliitin hoidossa. Leikkauksessa poistetaan sairastunut ja puhjennut suolen osa, jonka jälkeen proksimaalinen paksusuolen pää tuodaan avanteeksi ihon pinnalle.

Peräsuoli jää paikalleen ja suolen päiden yhdistäminen myöhemmin (tyypillisesti aikaisintaan puolen vuoden päästä) on usein mahdollista, kun potilas on toipunut päivystysleikkauksesta.

\pandocbounded{\includegraphics[keepaspectratio]{images/hartmann.png}}

\end{solution}

\section{Potilaalla colonca ja suoli poistettu ja tehty suojaava transversostooma. Suunnitteilla vielä sädehoito imusolmukemetastaasien takia. Milloin avanne suljetaan?}\label{potilaalla-colonca-ja-suoli-poistettu-ja-tehty-suojaava-transversostooma.-suunnitteilla-vieluxe4-suxe4dehoito-imusolmukemetastaasien-takia.-milloin-avanne-suljetaan}

Ei vaihtoehtoja, mutta koita vastata ilman vinkkejä.

\begin{solution}
\leavevmode

Vastaus

\begin{verbatim}
 Vasta adjuvanttiterapian jälkeen
\end{verbatim}

Paksusuolen distaalisen osan poiston ja päiden liittämisen jälkeen usein tarvitaan liitoksesta proksimaalisesti rakennettu avanne (yleensä transversostooma) suojaamaan liitosta sen paranemiseen saakka -\textgreater{} yleensä poistetaan 2-3kk päästä sauman parannuttua. Jos onkologisissa tapauksissa suunnitellaan jatkohoitoja (sädehoito, adjuvanttikemoterapia), niin avanteen poisto tehdään tyypillisesti vasta hoitojen jälkeen.

Paksusuolen sekä peräsuolen poistaminen ei siis aina tarkoita pysyvää avannetta ja paksusuolisyövän leikkauksessa ei yleensä edes tarvita avannetta. Pysyvään paksusuoliavanteeseen (kolostomia, kolostooma) päädytään, kun peräaukko ja sen lihakset joudutaan poistamaan (esim. rectumsyövässä abdominoperineaalinen resektio eli rectumamputaatio)

\end{solution}

\section{Mikä tyypillisin leikkaus ei-pienisoluisessa keuhkosyövässä?}\label{mikuxe4-tyypillisin-leikkaus-ei-pienisoluisessa-keuhkosyuxf6vuxe4ssuxe4}

\begin{itemize}
\tightlist
\item
  \begin{enumerate}
  \def\labelenumi{\alph{enumi}.}
  \tightlist
  \item
    Keuhkon poisto
  \end{enumerate}
\item
  \begin{enumerate}
  \def\labelenumi{\alph{enumi}.}
  \setcounter{enumi}{1}
  \tightlist
  \item
    Lohkon poisto
  \end{enumerate}
\item
  \begin{enumerate}
  \def\labelenumi{\alph{enumi}.}
  \setcounter{enumi}{2}
  \tightlist
  \item
    Tuumorin poisto marginaalein
  \end{enumerate}
\item
  \begin{enumerate}
  \def\labelenumi{\alph{enumi}.}
  \setcounter{enumi}{3}
  \tightlist
  \item
    Keuhkonsiirto
  \end{enumerate}
\end{itemize}

\begin{solution}
\leavevmode

Vastaus

\begin{verbatim}
 b
\end{verbatim}

Modernin kirurgian suuntaus on mini-invasiivisuus sekä keuhkokudoksen säästäminen. Valtaosa leikattavista keuhkosyövistä rajoittuu yhden keuhkolohkon alueelle, ja lohkonpoisto onkin edelleen keuhkosyöpäkirurgian perusleikkaus. Kansainvälisessä suosituksessa lohkonpoisto on edelleen ensisijainen vaihtoehto levinneisyysluokan I ja II taudissa normaalin leikkausriskin potilailla.

a: Joskus keuhkonpoistokin on paikallaan. Perinteisesti pääkeuhkoputkeen, keuhkovaltimoon tai lohkorajan yli laajalti kasvavissa syövissä leikkaus on ollut keuhkonpoisto. Nykysuosituksena keuhkovaltimoon, keuhkoputkeen tai näihin molempiin kasvavissa syövissä on ensisijaisesti hiharesektio. Lohkonpoistoon liitetään pääkeuhkoputken tai valtimon resektio, ja jäljelle jäävän lohkon rakenteet liitetään proksimaalisesti katkaistuun keuhkoputkeen tai keuhkovaltimoon. Leikkaus on teknisesti monimutkaisempi, mutta leikkauskuolleisuus on pienempi ja pitkäaikaisennuste jopa parempi kuin keuhkonpoistossa. Lohkorajan ylittävissä syövissä keuhkonpoisto on usein vältettävissä tekemällä lohkonpoisto ja viereisen lohkon jaokkeenpoisto tai vaihtoehtoisesti lohkorajan molemmin puolin keuhkojaokkeiden poistot.

c: Lohkoa säästävää kirurgiaa ja tällöin etenkin keuhkojaokkeen poistoa suositellaan suuremman leikkausriskin potilailla (keuhkojen heikentynyt toiminta, huono suorituskyky, muut sairaudet) levinneisyysluokan I syövissä ei-kirurgisten hoitojen sijasta sekä alle 2 cm:n mattalasimuutoksissa. Kahden tuoreen ison satunnaistetun tutkimuksen mukaan alle 2 cm:n syövissä pitkäaikaistulosten on osoitettu olevan jaokkeen poistolla vähintäänkin yhtä hyvät kuin lohkon poistossa. Nämä tutkimukset tulevat muuttamaan nykyisiä suosituksia alle 2 cm:n syöpien kirurgisesta hoidosta hyväkuntoisillakin potilailla.

d: Keuhkonsiirron aiheena on syövän sijasta pikemminkin loppuvaiheen keuhkosairaus, joka etenee muista hoidoista huolimatta ja jonka ennuste on huono. Yleisimpiä indikaatioita keuhkonsiirrolle ovat keuhkofibroosi, keuhkoahtaumatauti (COPD, chronic obstructive pulmonary disease) ja muut keuhkolaajentumataudit (mm. alfa1-antitrypsiinin puute), idiopaattinen keuhkoverenpainetauti ja kystinen fibroosi.

\end{solution}

\section{Potilaalla tänään tehty laparoskooppinen kolekystektomia. Nyt paineet putoaa ja takykardiaa. Vatsa aristaa painellen. Mikä todennäköisin komplikaatio?}\label{potilaalla-tuxe4nuxe4uxe4n-tehty-laparoskooppinen-kolekystektomia.-nyt-paineet-putoaa-ja-takykardiaa.-vatsa-aristaa-painellen.-mikuxe4-todennuxe4kuxf6isin-komplikaatio}

Ei vaihtoehtoja, mutta huomioiden nopea kehittyminen (leikattu tänään) on todennäköisesti kyseessä intra-abdominaalinen verenvuoto johtuen leikkaustraumasta johonkin suoneen. Tietysti voisi olla mikä tahansa muukin, kuten sappilekaasi tai suolitrauma ja näiden aiheuttama sepsis.

\section{Tyrä leikattu 2 viikkoa sitten. Nyt samalla kohdalla pullotus, joka ei reponoidu. Ei aristava. Mikä vaivaa?}\label{tyruxe4-leikattu-2-viikkoa-sitten.-nyt-samalla-kohdalla-pullotus-joka-ei-reponoidu.-ei-aristava.-mikuxe4-vaivaa}

\begin{itemize}
\tightlist
\item
  \begin{enumerate}
  \def\labelenumi{\alph{enumi}.}
  \tightlist
  \item
    Kureutunut tyrä
  \end{enumerate}
\item
  \begin{enumerate}
  \def\labelenumi{\alph{enumi}.}
  \setcounter{enumi}{1}
  \tightlist
  \item
    Uusiutunut tyrä
  \end{enumerate}
\item
  \begin{enumerate}
  \def\labelenumi{\alph{enumi}.}
  \setcounter{enumi}{2}
  \tightlist
  \item
    Infektio
  \end{enumerate}
\item
  \begin{enumerate}
  \def\labelenumi{\alph{enumi}.}
  \setcounter{enumi}{3}
  \tightlist
  \item
    Serooma
  \end{enumerate}
\end{itemize}

\begin{solution}
\leavevmode

Vastaus

\begin{verbatim}
 d
\end{verbatim}

Sopii seroomaan, joka on yleinen leikkauskomplikaatio, jossa leikkausalueelle kertyy kudosnestettä; yleistä 1--3 viikkoa postoperatiivisesti. Pullotus tuntuu, mutta ei kipua eikä punoitusta. Voi balloteerata palpoidessa.

a: Kureutunut tyrä on palpaatioarka, eikä potilas saa sitä itse reponoiduksi kuten ennen

b: Tyrä ei tyypillisesti uusiudu näin nopeasti

c: Infektio olisi myös aristava yleensä

\end{solution}

\section{Mitä tehdään uusiville paineoireita aiheuttaville kilpirauhaskystille?}\label{mituxe4-tehduxe4uxe4n-uusiville-paineoireita-aiheuttaville-kilpirauhaskystille}

Ei vaihtoehtoja, koita vastata ilman vinkkejä

\begin{solution}
\leavevmode

Vastaus

\begin{verbatim}
 Rf-hoito/Leikkaus
\end{verbatim}

Paineoireita aiheuttava kysta voidaan tyhjentää aspiraatiolla. Toimenpideradiologi tyhjentää kystan ja täyttää ontelon joko etanolilla/polidokanolilla (pelkän tyhjennyksen jälkeen tilanne uusii todella usein ja nopeasti). Skleroterapia ei tosin poista uusiutumisriskiä. Tarkoitus on hoitaa oireettomaksi, ei hävittää kystaa kokonaan.

\textbf{Leikkaus voi joskus olla tarpeen, jos kysta oireilee (toistetusta) skleroterapiasta riippumatta jatkuvasti täyttymällä uudestaan. Voidaan koittaa aluksi toimenpiteenä radiofrekvenssiohoitoa.} Teho paras yksilokeroisissa puhtaasti kystisissä muutoksissa: suuret kystat voivat tarvita useampia käsittelyitä.

Kystat ovat hyvin yleisiä. Erilaisia kilpirauhaskyhmyjä on uä:llä tutkittuna ad 75\% ihmisistä, naisilla enemmän. Näistä kystisiä on 15-25\% ja benignejä 95\%.

\end{solution}

\section{Paksu- ja peräsuolisyövän seurannassa käytettävä markkeri?}\label{paksu--ja-peruxe4suolisyuxf6vuxe4n-seurannassa-kuxe4ytettuxe4vuxe4-markkeri}

\begin{itemize}
\tightlist
\item
  \begin{enumerate}
  \def\labelenumi{\alph{enumi}.}
  \tightlist
  \item
    CEA
  \end{enumerate}
\item
  \begin{enumerate}
  \def\labelenumi{\alph{enumi}.}
  \setcounter{enumi}{1}
  \tightlist
  \item
    CA19-9
  \end{enumerate}
\item
  \begin{enumerate}
  \def\labelenumi{\alph{enumi}.}
  \setcounter{enumi}{2}
  \tightlist
  \item
    AFP
  \end{enumerate}
\item
  \begin{enumerate}
  \def\labelenumi{\alph{enumi}.}
  \setcounter{enumi}{3}
  \tightlist
  \item
    CA-125
  \end{enumerate}
\end{itemize}

\begin{solution}
\leavevmode

Vastaus

\begin{verbatim}
 a
\end{verbatim}

CEA = Karsinoembryonaalinen antigeeni; osoitettiin ensin sikiön kolonin limakalvosta ja kolonkarsinoomakudoksista -\textgreater{} CEA-pitoisuus on korkeimillaan raskaana olevalla naisella noin 22. raskausviikolla. Pieniä määriä antigeenia syntetisoituu myös aikuisen paksusuolen limakalvossa, joten terveelläkin aikuisella on pieniä pitoisuuksia CEA:ta. Useimmiten CEA on koholla vasta kun syöpä on levinnyt, ei niinkään syövän varhaisvaiheissa. CEA-pitoisuus laskee nopeasti, mikäli primaaristi hoitotulos on ollut hyvä.

CEA:n spesifisyys ja sensitivisyys syövän diagnostiikassa ja seulonnassa on alhainen, joten sitä voidaan käyttää vain syövän hoidon ja taudin kulun seurannassa. Syövän seurannassa pienenevä pitoisuus viittaa hyvään hoitovasteeseen, kun taas kahdessa perättäisessä näytteessä toistuva pitoisuuden suureneminen viittaa taudin uusiutumiseen. Jos pitoisuus toistuvissa määrityksissä suurenee edelleen, se on merkitsevä, vaikka CEA-arvo on vielä viitealueella

b: CA19-9 käytetään eniten haima-, maha-, maksa- ja sappitiesyöpien seurannassa.

c: AFP (alfafetoproteiinin) paras käyttöaihe on primaarin maksakarsinooman tai itusolusyövän epäily sekä hoidon seuranta

d: CA-125 eli musiini 16 on ensisijaisesti munasarjasyövän merkkiaine

\end{solution}

\section{Nissenin fundoplikaatio - mikä vaiva hoidettu?}\label{nissenin-fundoplikaatio---mikuxe4-vaiva-hoidettu}

\begin{itemize}
\tightlist
\item
  \begin{enumerate}
  \def\labelenumi{\alph{enumi}.}
  \tightlist
  \item
    Refluksitauti
  \end{enumerate}
\item
  \begin{enumerate}
  \def\labelenumi{\alph{enumi}.}
  \setcounter{enumi}{1}
  \tightlist
  \item
    Ruokatorven akalasia
  \end{enumerate}
\item
  \begin{enumerate}
  \def\labelenumi{\alph{enumi}.}
  \setcounter{enumi}{2}
  \tightlist
  \item
    Zenkerin divertikkeli
  \end{enumerate}
\item
  \begin{enumerate}
  \def\labelenumi{\alph{enumi}.}
  \setcounter{enumi}{3}
  \tightlist
  \item
    Esofagusvarikset
  \end{enumerate}
\end{itemize}

\begin{solution}
\leavevmode

Vastaus

\begin{verbatim}
 a
\end{verbatim}

Refluksitaudin kirurgisessa hoidossa yleisimmin käytetty fundoplikaatio-tyyppi on ns. Nissenin 360° mansetti, mutta jos ruokatorven motorinen toiminta on vahvasti heikentynyt, voidaan rakentaa osittainen posteriorinen (Toupet) tai anteriorinen (Dor) mansetti. Tehdään laparoskooppisesti. Fundoplikaation yleisimmät pitkäaikaisongelmat liittyvät vatsan turvottamiseen, ilmavaivoihin sekä kyvyttömyyteen röyhtäillä tai oksentaa.

Fundoplikaatio on indikoitu, jos adekvaatti PPI-hoito ei auta (6kk hoito), potilas ei pysty/halua käyttää PPI/H2-salpaajia tai jos potilaalla on komplisoitunut erosiivinen esofagiitti, vaikea refluksitauti, suuri hiatushernia tai regurgitaatio vallitsevana oireena

b: Akalasiassa ruokatorven normaalia peristaltiikkaa ei millään hoidolla pystytä palauttamaan. Hoidossa pyritään parantamaan ruokatorven tyhjenemistä alasulkijaa löysentämällä. Dilataatio eli laajennus on akalasian perinteinen hoitokeino, minkä lisäksi nykyään voidaan käyttää myös botuliini-injektioita tai leikkausta. Akalasian tehokkain hoito on alasulkijalihaksen katkaisu eli Hellerin myotomia. Tähystystekniikalla laparoskopiateitse tehtävässä leikkauksessa lihaskerrokset avataan ruokatorven alaosan ja alasulkijan sekä mahalaukun yläosan kohdalta anteriorisesti limakalvoa avaamatta. Leikkaukseen yhdistetään aina myös osittainen fundoplikaatio, joka estää liiallisen refluksin.

c: Zenkerin divertikkeli eli faryngoesofageaalinen divertikkeli on ruokatorven umpipusseista yleisin, ja se syntynee nielun lihasten toimintahäiriön seurauksena. Nielemisen aikana yläsulkija ei relaksoidu tarpeeksi tai oikea-aikaisesti. Paine venyttää krikofaryngeuslihaksen yläpuolella ja takana keskiviivassa olevaa lihaksistoltaan heikompaa kohtaa. Aluksi oireet johtuvat sulkijan toimintahäiriöstä ja myöhemmin divertikkeliin retentoituvan ruoan regurgitaatiosta. Tärkein tutkimus on ruokatorven röntgenkuvaus.

Oireisen divertikkelin hoitona on leikkaus. Pienessä divertikkelissä riittää krikofaryngeuslihaksen myotomia, isommissa poistetaan myös divertikkeli. Toimenpide on mahdollista tehdä myös suun kautta tähystystekniikalla.

d: Vuotavia ruokatorven suonikohjuja hoidetaan kumilenkkiligatuurilla, joilla saadaan 80--90 \% vuodoista hallintaan. Endoskooppinen hoito on tarpeen, jos potilaalla on akuutti suonikohjurungon vuoto tai jos ruokatorven suonikohjuista kärsivän potilaan mahalaukussa on verta, eikä muuta vuotopaikkaa löydetä.

Mikäli vuoto jatkuu lääkityksestä huolimatta ja tähystys viivästyy tai endoskooppinen hoito ei onnistu, hoitona voidaan käyttää kardiaa komprimoivaa ballonkitamponaatiota (Sengstaken--Blakemoren tai Lintonin putki, ventrikkeliballongin paikka tarkistetaan röntgenkuvauksella) tai ruokatorven suonikohjun vuodon hoitoon tarkoitettua päällystettyä metalliverkkostenttiä, joka tamponoi vuodon.

\pandocbounded{\includegraphics[keepaspectratio]{images/fundoplikaatio.png}}

\end{solution}

\section{Akuutti alaraajaiskemia - mikä oire viittaa raajan elinkelpoisuuden menettämiseen}\label{akuutti-alaraajaiskemia---mikuxe4-oire-viittaa-raajan-elinkelpoisuuden-menettuxe4miseen}

\begin{itemize}
\tightlist
\item
  \begin{enumerate}
  \def\labelenumi{\alph{enumi}.}
  \tightlist
  \item
    raajan tunnottomuus ja plegia
  \end{enumerate}
\item
  \begin{enumerate}
  \def\labelenumi{\alph{enumi}.}
  \setcounter{enumi}{1}
  \tightlist
  \item
    iskeeminen kipu
  \end{enumerate}
\item
  \begin{enumerate}
  \def\labelenumi{\alph{enumi}.}
  \setcounter{enumi}{2}
  \tightlist
  \item
    alaraajan kutina
  \end{enumerate}
\item
  \begin{enumerate}
  \def\labelenumi{\alph{enumi}.}
  \setcounter{enumi}{3}
  \tightlist
  \item
    alaraaja tuntuu lämpimältä
  \end{enumerate}
\end{itemize}

\begin{solution}
\leavevmode

Vastaus

\begin{verbatim}
 a
\end{verbatim}

Akuutin alaraajaiskemian hoidon kiireellisyys perustuu Rutherfordin luokitukseen (iskemian asteeseen). Erityisen tärkeää on tunnistaa Rutherford II, jossa raajan elinkelpoisuus on uhattu, mutta se voidaan vielä palauttaa.

b: Iskeeminen kipu ei vielä viittaa elinkelpoisuuden menetykseen

c: Kutina ei liity asiaan.

d: Akuutissa alaraajaiskemiassa alaraaja on yleensä kylmä (6P-säännön poikilothermia)

\pandocbounded{\includegraphics[keepaspectratio]{images/alaraajaiskemiaalgoritmi.png}}
\pandocbounded{\includegraphics[keepaspectratio]{images/6p.png}}

\end{solution}

\section{Mikä ei aiheuta akuuttia alaraajaiskemiaa?}\label{mikuxe4-ei-aiheuta-akuuttia-alaraajaiskemiaa}

\begin{itemize}
\tightlist
\item
  \begin{enumerate}
  \def\labelenumi{\alph{enumi}.}
  \tightlist
  \item
    alaraajan embolia
  \end{enumerate}
\item
  \begin{enumerate}
  \def\labelenumi{\alph{enumi}.}
  \setcounter{enumi}{1}
  \tightlist
  \item
    tromboosi
  \end{enumerate}
\item
  \begin{enumerate}
  \def\labelenumi{\alph{enumi}.}
  \setcounter{enumi}{2}
  \tightlist
  \item
    keuhkoembolia
  \end{enumerate}
\item
  \begin{enumerate}
  \def\labelenumi{\alph{enumi}.}
  \setcounter{enumi}{3}
  \tightlist
  \item
    alaraajan ohitteen tukos
  \end{enumerate}
\end{itemize}

\begin{solution}
\leavevmode

Vastaus

\begin{verbatim}
 c
\end{verbatim}

Akuutin alaraajaiskemian yleisimmät aiheuttajat ovat tromboosi ja alaraajan embolia. Myös ohitteet ovat alttiita tukoksille, jopa enemmän kuin normaalit suonet.

Keuhkoembolia ei suoraan aiheuta alaraajaiskemiaa, mutta voi tietysti pahentaa aikaisempaa kroonista iskemiaa, jos embolian aiheuttama hypoksemia on merkittävää.

\end{solution}

\section{Pankreatiitissa antibiootin käyttö?}\label{pankreatiitissa-antibiootin-kuxe4yttuxf6}

\begin{itemize}
\tightlist
\item
  \begin{enumerate}
  \def\labelenumi{\alph{enumi}.}
  \tightlist
  \item
    herkemmin alkoholipankreatiitissa kuin sappipankreatiitissa
  \end{enumerate}
\item
  \begin{enumerate}
  \def\labelenumi{\alph{enumi}.}
  \setcounter{enumi}{1}
  \tightlist
  \item
    herkemmin sappipankreatiitissa kuin alkoholipankreatiitissa pankussa
  \end{enumerate}
\item
  \begin{enumerate}
  \def\labelenumi{\alph{enumi}.}
  \setcounter{enumi}{2}
  \tightlist
  \item
    ei juuri koskaan käytetä
  \end{enumerate}
\item
  \begin{enumerate}
  \def\labelenumi{\alph{enumi}.}
  \setcounter{enumi}{3}
  \tightlist
  \item
    käytetään aina
  \end{enumerate}
\end{itemize}

\begin{solution}
\leavevmode

Vastaus

\begin{verbatim}
 b
\end{verbatim}

Akuutin pankreatiitin ensisijainen hoito on yleensä konservatiivinen ja tärkeintä on alkuvaiheen nesteytys, elektrolyyttihäiriöiden korjaaminen, varhaisessa vaiheessa p.o. ravitsemukseen siirtyminen ja kipulääkitys; vaikeissa taudeissa ulkusprofylaksia (PPI).

Edes ab-hoitoa ei yleensä tarvita eikä suositella pankreatiitin infektiokomplikaatioiden ehkäisyyn (aloitetaan kuitenkin herkemmin sappipankreatiitissa ja varsinkin jos myös kolangiitti). Mikrobilääkkeistä ei ole hyötyä infektioiden ehkäisemisessä, eikä niitä tule käyttää, jos potilaalla ei ole toista infektiopesäkettä, kuten sepsistä tai kolangiittia.

Akuutti pankreatiitti on tyypillisesti steriili tulehdus, joka johtuu haimaentsyymien aktivoitumisesta haiman sisällä ja siitä aiheutuvasta autodigestiosta -\textgreater{} tämän takia antibiootit eivät ole erityisenkään tärkeitä rutiinisti.

\end{solution}

\section{Femoraalityrä, mikä väittämä oikein?}\label{femoraalityruxe4-mikuxe4-vuxe4ittuxe4muxe4-oikein}

\begin{itemize}
\tightlist
\item
  \begin{enumerate}
  \def\labelenumi{\alph{enumi}.}
  \tightlist
  \item
    vanhoilla naisilla tyypillinen
  \end{enumerate}
\item
  \begin{enumerate}
  \def\labelenumi{\alph{enumi}.}
  \setcounter{enumi}{1}
  \tightlist
  \item
    leikataan usein
  \end{enumerate}
\item
  \begin{enumerate}
  \def\labelenumi{\alph{enumi}.}
  \setcounter{enumi}{2}
  \tightlist
  \item
    kaikki oikein
  \end{enumerate}
\item
  \begin{enumerate}
  \def\labelenumi{\alph{enumi}.}
  \setcounter{enumi}{3}
  \tightlist
  \item
    sijaitsee nivusligamentin kaudaalipuolella
  \end{enumerate}
\end{itemize}

\begin{solution}
\leavevmode

Vastaus

\begin{verbatim}
 c
\end{verbatim}

a: Reisityrät iäkkäillä naisilla, nivustyrät miehillä (lateraaliset nuorilla ja mediaaliset iäkkäämmillä)

b: Koska tyräportti on pieni, niin kureutumisriski on suuri ja tämän takia lähes aina leikataan; voidaan korjata avoimesti tai tähystysteitse. Jälkimmäinen on suositeltavampi etenkin päivystystilanteessa, jotta kureutuneen kudoksen vitaliteetti voidaan varmistaa. Leikkauksessa tyräpussin sisältö palautetaan vatsaonteloon ja femoraalikanava suljetaan verkon avulla, tai jos kudoskuoliosta on seurannut infektiotilanne, ompelemalla.

d: Reisityrä työntyy nivustaipeeseen femoraalikanavaa pitkin. Femoraalikanava on nivussiteen (ligamentum inguinale) alla reisilaskimon (vena femoralis) ja häpyluun välissä oleva ahdas tila. Reisityrä onkin lähes aina pieni, enintään luumun kokoinen pullistuma nivustaipeen alaosassa tai mediaalisesti reiden tyvessä ja jää kliinisessä tutkimuksessa helposti huomaamatta.

\pandocbounded{\includegraphics[keepaspectratio]{images/femoraalityrä.png}}
\pandocbounded{\includegraphics[keepaspectratio]{images/nivustyrävsreisityrä.png}}
\pandocbounded{\includegraphics[keepaspectratio]{images/nivustyrä2.png}}

\end{solution}

\section{Hydronefroosin hoito}\label{hydronefroosin-hoito}

Ei vaihtoehtoja, mutta tässä tärkeimmät:

Hydronefroosilla tarkoitetaan munuaisen kollektiosysteemin (munuaisallas ja kaliksit) dilataatiota. Jos dilataatiota todetaan myös virtsanjohtimessa, voidaan käyttää termiä hydroureteronefroosi.

\begin{itemize}
\tightlist
\item
  Tyypillisiä kliinisiä löydöksiä ja oireita kivun ohella ovat koputus- ja tärinäarkuus selässä, virtsamäärän väheneminen, plasman kreatiniinipitoisuuden suureneminen, elektrolyyttihäiriöt ja virtsan korkea pH (heikentynyt H+-eritys virtsaan). Pahoinvointia ja oksentelua voi ilmetä etiologiasta riippuen.
\item
  Hydronefroosin aiheuttajat voidaan jakaa virtsankulun esteen sijainnin perusteella rakon yläpuolisiin, rakkotason ja rakon alapuolisiin aiheuttajiin. Lisäksi virtsateiden ulkopuoliset syyt voivat aiheuttaa hydronefroosia.
\end{itemize}

\textbf{Hydronefroosi voidaan todeta ultraäänitutkimuksella, mutta jos tilan syy ei ole muuten ilmeinen (esim. virtsaumpi), syyn selvittämiseksi tehdään yleensä tietokonekerroskuvaus (TT), johon voidaan tarvittaessa liittää urografiavaihe.}

\begin{itemize}
\tightlist
\item
  Hydronefroosin syytä voidaan selvittää ruiskuttamalla varjoainetta esimerkiksi nefrostooman kautta tai alateitse virtsanjohtimeen kuvauskatetrilla (anterogradinen/retrogradinen pyelografia).
\end{itemize}

Hydronefroosi voidaan laukaista asettamalla polikliinisesti tai leikkaussalissa \textbf{virtsanjohdinstentti} (ns. JJ-stentti).

\begin{itemize}
\tightlist
\item
  Radiologi voi laukaista hydronefroosin myös asettamalla kyljestä tai selästä nefrostooman tai potilaalle voidaan asettaa alatiekatetri, jos kyseessä on virtsaummen aiheuttama tilanne
\item
  Päivystyksellisiä tilanteita, joissa hydronefroosi pitää laukaista välittömästi yleensä nefrostoomalla tai JJ-stentillä, ovat pyonefroosi eli märkäinen (septinen) munuaisallastulehdus sekä vakava munuaisten vajaatoiminta elektrolyyttihäiriöineen.
\item
  Oleellista hoidossa on hydronefroosin altistavan perussyyn selvittäminen ja sen hoito. Mikäli hydronefroosin syy on kirurgisesti hoidettavissa esimerkiksi pyeloplastia pyeloureteraalisen junktion stenoosissa tai eturauhasen höyläys liikakasvun aiheuttamassa molemminpuolisessa hydronefroosissa, tulisi toimenpide tehdä elektiivisesti akuutin tilanteen rauhoituttua. Joskus on kuitenkin tyydyttävä pysyvään JJ-stentti- tai nefrostoomahoitoon parantumattomissa sairauksissa, hyvin iäkkäillä tai huonokuntoisilla potilailla.
\item
  Jos munuaisfunktio ei hydronefroosin laukaisun jälkeen normalisoidu (obstruktiivinen uropatia), potilas tulee lähettää nefrologille.

  \begin{itemize}
  \tightlist
  \item
    Hydronefroosin kesto ja aste vaikuttaa munuaisvaurion kehittymiseen. Alle 7 vuorokautta kestävä hydronefroosi ei yleensä aiheuta pysyvää munuaisvauriota aiemmin terveellä potilaalla. Kaksi viikkoa kestävässä hydronefroosissa 70 \% munuaisfunktiosta yleensä palautuu 3--6 kuukauden aikana ja neljä viikkoa kestävässä taas 30 \%. Kuusi viikkoa kestäneessä täydellisessä hydronefroosissa munuaisen toiminta ei enää käytännössä palaudu.
  \end{itemize}
\end{itemize}

\section{Yleisin ERCP:n komplikaatio on}\label{yleisin-ercpn-komplikaatio-on}

\begin{itemize}
\tightlist
\item
  \begin{enumerate}
  \def\labelenumi{\alph{enumi}.}
  \tightlist
  \item
    perforaatio
  \end{enumerate}
\item
  \begin{enumerate}
  \def\labelenumi{\alph{enumi}.}
  \setcounter{enumi}{1}
  \tightlist
  \item
    verenvuoto
  \end{enumerate}
\item
  \begin{enumerate}
  \def\labelenumi{\alph{enumi}.}
  \setcounter{enumi}{2}
  \tightlist
  \item
    pankreatiitti
  \end{enumerate}
\item
  \begin{enumerate}
  \def\labelenumi{\alph{enumi}.}
  \setcounter{enumi}{3}
  \tightlist
  \item
    kolangiitti
  \end{enumerate}
\end{itemize}

\begin{solution}
\leavevmode

Vastaus

\begin{verbatim}
 c
\end{verbatim}

ERCP:hen (endoskooppinen retrogradinen kolangiopankreatografia; toimenpide, jossa sivulle katsovalla duodenoskoopilla voidaan tutkia ja hoitaa sappi- ja haimatiehyen sairauksia läpivalaisulaitteen avulla) liittyviä komplikaatioita ovat

haimatulehdus (2-5\%, joissain lähteissä 3,5--9,7 \%)

kolangiitti (1 \%)

sfinkterotomian jälkeinen verenvuoto (1 \%)

tiehyen tai suolen puhkeama (alle 1 \%)

\pandocbounded{\includegraphics[keepaspectratio]{images/ercp.png}}

\end{solution}

\section{Todennäköisin maksan pahanlaatuinen tuumori}\label{todennuxe4kuxf6isin-maksan-pahanlaatuinen-tuumori}

\begin{itemize}
\tightlist
\item
  \begin{enumerate}
  \def\labelenumi{\alph{enumi}.}
  \tightlist
  \item
    HCC
  \end{enumerate}
\item
  \begin{enumerate}
  \def\labelenumi{\alph{enumi}.}
  \setcounter{enumi}{1}
  \tightlist
  \item
    kolangioca
  \end{enumerate}
\item
  \begin{enumerate}
  \def\labelenumi{\alph{enumi}.}
  \setcounter{enumi}{2}
  \tightlist
  \item
    metastaasi
  \end{enumerate}
\item
  \begin{enumerate}
  \def\labelenumi{\alph{enumi}.}
  \setcounter{enumi}{3}
  \tightlist
  \item
    joku (ei wikissä)
  \end{enumerate}
\end{itemize}

\begin{solution}
\leavevmode

Vastaus

\begin{verbatim}
 c
\end{verbatim}

Maksan yleisin pahanlaatuinen muutos on metastaasi (n.~90\% maligniteeteista on metastaaseja). Yleisin maksan primaarinen syöpä on hepatosellulaarinen karsinooma (HCC) ja toiseksi yleisin on maksan sisäisistä sappiteistä lähtenyt kolangiokarsinooma.

Maksan maksan pesäkemuutokset ovat useimmiten hyvänlaatuisia. Parantuneiden kuvantamismenetelmien ansiosta maksasta löydetään paikallismuutoksia jopa 20 \%:lta tutkituista. Heistä viidesosalla muutos on pahanlaatuinen.

Yleisiä hyvänlaatuisia pesäkemuutoksia ovat fokaalinen rasvoittuminen (ei usein lasketa), yksittäiset kystat, hemangioomat (yleisin solidi tuumori) sekä fokaalinen nodulaarinen hyperplasia (FNH). Riippuen lähteestä hemangioomat voivat olla suorastaan yleisin maksan pesäkemuutos, joissain taas kystat ovat hieman yleisempiä kuin hemangioomat.

\end{solution}

\section{Virtsaamistiheyden ja sen etiologian selvittäminen, mikä paras}\label{virtsaamistiheyden-ja-sen-etiologian-selvittuxe4minen-mikuxe4-paras}

\begin{itemize}
\tightlist
\item
  \begin{enumerate}
  \def\labelenumi{\alph{enumi}.}
  \tightlist
  \item
    DAN-PSS
  \end{enumerate}
\item
  \begin{enumerate}
  \def\labelenumi{\alph{enumi}.}
  \setcounter{enumi}{1}
  \tightlist
  \item
    IPSS
  \end{enumerate}
\item
  \begin{enumerate}
  \def\labelenumi{\alph{enumi}.}
  \setcounter{enumi}{2}
  \tightlist
  \item
    IIEF
  \end{enumerate}
\item
  \begin{enumerate}
  \def\labelenumi{\alph{enumi}.}
  \setcounter{enumi}{3}
  \tightlist
  \item
    virtaamispäiväkirja
  \end{enumerate}
\end{itemize}

\begin{solution}
\leavevmode

Vastaus

\begin{verbatim}
 d
\end{verbatim}

Virtsaamislista on objektiivinen (kunhan vain se täytetään oikein) ja lääkäri näkee heti jatkotutkimusten tarpeellisuuden. Virtsaamispäiväkirjaa potilas täyttää 2--3 vuorokauden ajalta, kirjaa kellonajan, jolloin käy virtsalla, ja mittaa virtsan tilavuuden yksinkertaisesti esimerkiksi tavaratalosta ostamallaan jauhokannulla. Kaikki ``noin'' -mittalaitteet (kahvikupit, lasit ja jogurttipurkit) saisi unohtaa, että saataisiin objektiivinen luku. Kannun kyljestä voidaan suoraan mitta-asteikolta lukea tilavuus millilitroina.

Yövirtsauskerrat on syytä merkitä selkeästi, esim. rengastamalla mainitut kellonlyömät, sillä lääkäri ei voi tietää, millainen on kunkin potilaan yö. Leipomotyöntekijä aloittaa varmasti työnsä siihen aikaan, kun tiskijukka alkaa nukkua. Virtsauslistoista on helppo nähdä, mikä on rakon maksimaalinen kertakapasiteetti.

Jos potilas käy virtsalla enemmän kuin 8 kertaa vuorokaudessa, puhutaan pollakisuriasta

a: DAN-PSS-1-kyselyllä (Danish Prostatic Symptom Score 1) pääasiassa selvitetään eturauhasen liikakasvun oireiden vaikeusastetta. Mittaa paitsi oireiden vaikeutta, myös niiden aiheuttamaa haittaa potilaalle.

b: Toinen yleisesti käytetty BPH:n oireiden vaikeusastetta mittaava kysely on on IPSS (International Prostate Symptom Score). DAN-PSS-1 vaikuttaa olevan ekstensiivisempi ja herkempi kuin IPSS.

c: Erektiohäiriöpotilaan anamneesissa on hyvänä apuna kansainvälinen kyselylomake (International Index of Erectile Function).

\end{solution}

\section{Mikä on Marjolinin ulkus?}\label{mikuxe4-on-marjolinin-ulkus}

Ei vaihtoehtoja, mutta koita vastata ilman vinkkejä

\begin{solution}
\leavevmode

Vastaus

\begin{verbatim}
 SCC krooniseen haavaan
\end{verbatim}

Okasolusyöpä voi joskus tulla aikaisemmin vaurioituneeseen ihoon, kuten vanhaan arpeen, haavaan joka ei parannu kunnolla tai krooniseen tulehdukselliseen ihotautiin. Tällöin sitä kutsutaan Marjolinin haavaksi.

Kasvaa hitaasti, mutta prognoosi on huono. Diagnosoidaan koepalalla. Hoitona on kirurgia, mutta uusiutuu usein (20-30\%).

\pandocbounded{\includegraphics[keepaspectratio]{images/marjolininhaava.png}}

\end{solution}

\section{Puukon viillosta 5 cm lihakseen ylettyvä haava, 7h vanha, mitä teet?}\label{puukon-viillosta-5-cm-lihakseen-ylettyvuxe4-haava-7h-vanha-mituxe4-teet}

Ei vaihtoehtoja. Haavan sulkeminen ja antibioottiprofylaksian tarve riippu siitä onko haava arvioitava puhtaaksi vai likaiseksi ja millä sijainnilla se on.

\begin{itemize}
\tightlist
\item
  Hermo-, jänne- ja lihasvammat tulee hoitaa lähipäivinä tai kirurgin arvion mukaan, viimeistään 3 viikon kuluessa. Jänteiden tutkiminen kämmenvammoissa.
\item
  Esim. illalla tapahtunut sormen viiltohaava ja koukistajajänteen katkeaminen: heti haavan puhdistus ja ihon atraumaattinen sulku 5--0-langalla sekä potilaan ohjaaminen seuraavaksi aamuksi käsikirurgiseen yksikköön
\end{itemize}

\pandocbounded{\includegraphics[keepaspectratio]{images/sulkeminen.png}}
\pandocbounded{\includegraphics[keepaspectratio]{images/profylaksi.png}}

\section{Mitä Kehrin oire tarkoittaa?}\label{mituxe4-kehrin-oire-tarkoittaa}

Ei vaihtoehtoja, mutta koita vastata ilman vinkkejä

\begin{solution}
\leavevmode

Vastaus

\begin{verbatim}
 Palleaärsytys -> hartiasärky
 
\end{verbatim}

Ulkusperforaatiossa kipu tuntuu aluksi epigastriumissa tai oikean kylkikaaren alla, mutta leviää nopeasti koko vatsan alueelle. Myös hartiakipua saattaa esiintyä merkkinä palleaärsytyksestä (Kehrin oire)

Muita tilanteita:

Laparoskopian jälkeinen hartiakipu

Pernan repeäminen -\textgreater{} kipu vasempaan hartiaan (joskus klassisesti Kehrin merkki yhdistetty vain tähän tilanteeseen)

Sappikoliikki -\textgreater{} kipu oikeaan hartiaan

\pandocbounded{\includegraphics[keepaspectratio]{images/kehr.png}}

\end{solution}

\section{Mitä blefaroplastia tarkoittaa?}\label{mituxe4-blefaroplastia-tarkoittaa}

Ei vaihtoehtoja, mutta koita vastata ilman vinkkejä

\begin{solution}
\leavevmode

Vastaus

\begin{verbatim}
 Silmäluomileikkausta
 
\end{verbatim}

Tyypillisesti tarkoitetaan yläluomen leikkausta. Ensisijaisesti voidaan \textbf{korjata lippaluomi} eli dermatochalasis, joka tarkoittaa yläluomen ihoylimäärää +/- rasvaa. Laskeutuu silmäluomen päälle ja voi peittää näkökenttää, aiheuttaen näköhaittaa tai väsyneen ilmeen. Leikkauksessa poistetaan yläluomen alueelle muodostunut ylimääräinen ihopoimu ja tarvittaessa muotoillaan alla olevaa rasvaa.

Raskas yläluomi aiheuttaa väsyneen ilmeen, mutta näkökentän eteen laskeutuessaan lippaluomipoimu aiheuttaa myös toiminnallisen ongelman. Tällöin tilanne rinnastetaan sairaudeksi, jolloin leikkaus voidaan toteuttaa julkisessa sairaanhoidossa. Julkisella yleisinä yläluomen kirurgian kriteereinä voidaan pitää:

Toiminnallinen näkökykyhaitta (kompensatorinen otsajännitys voi esim aiheuttaa estolääkitystä vaativan migreenin)

MRD-1 mitta \textless2mm (MRD = Margin-reflex distance. Iho peittää näkökenttää tulemalla alle 2 mm:n päähän mustuaisen valoheijasteesta/optiselta akselilta)

\pandocbounded{\includegraphics[keepaspectratio]{images/lippaluomi.png}}

\end{solution}

\section{Mikä on tärkein prognostinen tekijä, joka ennustaa melanooman metastasointia ja kertoo siten eniten ennusteesta?}\label{mikuxe4-on-tuxe4rkein-prognostinen-tekijuxe4-joka-ennustaa-melanooman-metastasointia-ja-kertoo-siten-eniten-ennusteesta}

Ei vaihtoehtoja, mutta koita vastata ilman vinkkejä.

\begin{solution}
\leavevmode

Vastaus

\begin{verbatim}
 Breslow'n mitta
 
\end{verbatim}

Mitta epidermiksen granulaarisolukerroksen (stratum granulosum) pinnasta syvimpiin melanoomasoluihin. Jos ei invaasiota, niin ei myöskään ilmoiteta Breslow'n mittaa. In situ-vaiheessa ei siis vielä ole riskiä metastasoinnille.

Breslow'n mitta määrittää melanooman hoidollisessa inkiisiossa vaadittavat marginaalit.

\pandocbounded{\includegraphics[keepaspectratio]{images/breslow.png}}
\pandocbounded{\includegraphics[keepaspectratio]{images/melanoomamarginaali.png}}

\end{solution}

\section{Akuutti kolekystiitti, mikä leikkaustapa}\label{akuutti-kolekystiitti-mikuxe4-leikkaustapa}

\begin{itemize}
\tightlist
\item
  \begin{enumerate}
  \def\labelenumi{\alph{enumi}.}
  \tightlist
  \item
    endoskopia
  \end{enumerate}
\item
  \begin{enumerate}
  \def\labelenumi{\alph{enumi}.}
  \setcounter{enumi}{1}
  \tightlist
  \item
    laparoskopia
  \end{enumerate}
\item
  \begin{enumerate}
  \def\labelenumi{\alph{enumi}.}
  \setcounter{enumi}{2}
  \tightlist
  \item
    avoleikkaus
  \end{enumerate}
\item
  \begin{enumerate}
  \def\labelenumi{\alph{enumi}.}
  \setcounter{enumi}{3}
  \tightlist
  \item
    ei leikkausta
  \end{enumerate}
\end{itemize}

\begin{solution}
\leavevmode

Vastaus

\begin{verbatim}
 b
 
\end{verbatim}

Laparoskooppinen kolekystektomia (lap.chole) on akuutin kolekystiitin ensisijainen hoitokeino. Avoleikkaukseen joudutaan joskus konvertoimaan, mutta ei tyypillisesti.

\end{solution}

\section{Kuinka kauan laparoskopian jälkeen ilma näkyy vapaassa vatsaontelossa}\label{kuinka-kauan-laparoskopian-juxe4lkeen-ilma-nuxe4kyy-vapaassa-vatsaontelossa}

Ei vaihtoehtoja, mutta koita vastata ilman vinkkejä

\begin{solution}
\leavevmode

Vastaus

\begin{verbatim}
 Muutaman päivän
\end{verbatim}

Laparoskopian jälkeen intra-abdominaalinen kaasu häviää n.~3 vrk:ssa

Taas laparotomian jälkeen ilmaa usein ad 1vk operaatiosta, tämänkin jälkeen voi olla pieniä määriä nähtävissä. Vapaata ilmaa voi olla näkyvissä jopa 3 vk kohdalla laparotomiasta.

\end{solution}

\section{Potilaalla peniksen tyvessä laattamainen kova alue, mitä epäilet?}\label{potilaalla-peniksen-tyvessuxe4-laattamainen-kova-alue-mituxe4-epuxe4ilet}

\begin{itemize}
\tightlist
\item
  \begin{enumerate}
  \def\labelenumi{\alph{enumi}.}
  \tightlist
  \item
    penissyöpä
  \end{enumerate}
\item
  \begin{enumerate}
  \def\labelenumi{\alph{enumi}.}
  \setcounter{enumi}{1}
  \tightlist
  \item
    kysyt käyristyykö penis erektiossa
  \end{enumerate}
\item
  \begin{enumerate}
  \def\labelenumi{\alph{enumi}.}
  \setcounter{enumi}{2}
  \tightlist
  \item
    ahdas esinahka
  \end{enumerate}
\end{itemize}

\begin{solution}
\leavevmode

Vastaus

\begin{verbatim}
 b
\end{verbatim}

Peyronien taudissa voidaan tyypillisesti tuntea siittimessä kovettuman.

Peyronien tauti etenee vaiheittain:

Alkuvaihe eli akuutti inflammatorinen vaihe: Arpikudosta muodostuu ja oireena esiintyy usein erektionaikainen kipu ja tunnusteltavissa oleva patti tai kovettuma jo ennen käyristymisen ilmaantumista. Aiheuttaa kipua erityisesti erektioissa, ajan kuluessa kipu helpottaa (kipuvaihe eli alkuvaihe saattaa kestää useita kuukausia)

Kroonisessa fibroottisessa vaiheessa siittimessä selkeä kovettuma, joka kalkkeutuu ja aiheuttaa siittimen käyristymisen (käyristyy kovettuman suuntaan; useimmiten plakki on dorsumissa -\textgreater{} käyristyy ylöspäin, mutta plakki voi muodostua minne vain)

Ei ole tehokasta lääkehoitoa (injisoitava kollagenaasi poistunut markkinoilta), akuutissa vaiheessa tulehduskipulääkkeitä kivunhoidossa ja PDE5:n estäjät, jos potilaalla samanaikaisesti erektiohäiriö. Ensisijainen hoitokeino on operatiivinen, mutta mahdollinen oikaisuleikkaus tehdään vasta, kun kipuvaihe on ohi ja tilanne on ollut stabiili vähintään 3 kk. Useimmat potilaat pystyvät yhdyntään, jolloin kirurginen hoito ei ole tarpeen

\end{solution}

\section{Potilaalla ortostaattista hypotensiota, mikä lääke eturauhasen liikakasvuun?}\label{potilaalla-ortostaattista-hypotensiota-mikuxe4-luxe4uxe4ke-eturauhasen-liikakasvuun}

Ei vaihtoehtoja, mutta koita vastata ilman vinkkejä

\begin{solution}
\leavevmode

Vastaus

\begin{verbatim}
 5-ARI (esim. dutasteridi)
\end{verbatim}

Lievä- ja keskivaikeaoireisten komplisoitumattomien BPH-potilaiden virtsaamisoireita hoidetaan ensisijaisesti lääkehoidolla, joka on tyypillisesti alfasalpaaja ja/tai 5-alfareduktaasin estäjä (5-ARI).

Alfasalpaajat eli α1-reseptorin salpaajat (esim. tamsulosiini tai alfutsosiini) vaikuttavat rentouttamalla prostaattisen virtsaputken ja virtsarakon kaulan sileää lihaksistoa. Ne lievittävät nopeasti (vrt. 5-ARI:t jotka hitaammin) oireita, lisäävät virtsasuihkun huippuvirtaamaa ja vähentävät jäännösvirtsan tilavuutta. Ne kuitenkin myös salpaavat verisuonien alfareseptoreita ja siten laskevat verenpainetta ja erityisesti heikentävät verisuonien supistumista vasteena seisomaan nousemiselle, mikä pahentaa ortostaattista hypotensiota.

5-ARI:t (esim. finasteridi tai dutasteridi) taas vaikuttavat eturauhasen liikakasvuun estämällä testosteronin metaboloitumista dihydrotestosteroniksi (DHT), jolloin seerumin DHT:n pitoisuus pienenee. Tämä vaikutus ilmenee hitaasti ja tämän takia 5-ARI:t lievittävät oireita hitaammin kuin alfasalpaajat. Ne taas kuitenkaan eivät vaikuta suoraan verisuoniin, jonka takia ortostaattisen hypotension riski ei käytännössä nouse.

\end{solution}

\section{10-vuotta sitten operoitu sitten rektumsyöpä, nyt keuhkoissa joku tarkkarajainen tiivistymä.}\label{vuotta-sitten-operoitu-sitten-rektumsyuxf6puxe4-nyt-keuhkoissa-joku-tarkkarajainen-tiivistymuxe4.}

Ei vaihtoehtoja, mutta todennäköisesti kysytty keuhkotiivistymän etiologiasta (esim. \textbf{onko primaari keuhkokasvain vai rektumsyövän metastaasi}).

\begin{solution}
\leavevmode

Vastaus

\begin{verbatim}
 Primaari syöpä
\end{verbatim}

Yksittäinen keuhkotuumori on usein primääri keuhkosyöpä vaikka potilaalla olisi aiempi syöpä. Metastaasit ilmenevät tyypillisesti multippeleina ja ilmenevät tyypillisesti periferiassa.

\pandocbounded{\includegraphics[keepaspectratio]{images/metareista.png}}

\end{solution}

\section{Radikaalisti sädehoidettu prostataca. Nadir-PSA 0.25 ja nyt PSA 2.5. Mitä teet?}\label{radikaalisti-suxe4dehoidettu-prostataca.-nadir-psa-0.25-ja-nyt-psa-2.5.-mituxe4-teet}

Ei vaihtoehtoja, mutta koita vastata ilman vinkkejä.

\begin{solution}
\leavevmode

Vastaus

\begin{verbatim}
 Lähete ESH
\end{verbatim}

Jos PSA-arvo nousee 2 ug/l alimman sädehoidon jälkeisen arvon (nadir-psa; nadir = alin piste (zenithin vastakohta)) yli, niin eturauhassyöpä katsotaan uusineeksi (määritelmä koskee myös potilaita, jotka ovat saaneet hormonaalista hoitoa). Kun potilas siirtyy perusterveydenhuollon seurantaan, tulee potilasasiakirjoihin kirjata erikoissairaanhoitoon lähettämisen raja-arvo.

Vrt. radikaaliin prostatektomiaan, jossa tauti katsotaan uusineeksi, jos PSA arvo nousee ollenkaan mitattavaksi leikkauksen jälkeisessä seurannassa. Sädehoidossa PSA ei yleensä katoa mittaamattoman matalaksi ja se saattaa vaihdella mittausten välillä, toisin kuin radikaalissa prostatektomiassa.

\end{solution}

\section{Palovamma kiulusta, tulee pari pv vamman jälkeen näyttämään, mitä teet?}\label{palovamma-kiulusta-tulee-pari-pv-vamman-juxe4lkeen-nuxe4yttuxe4muxe4uxe4n-mituxe4-teet}

Ei vaihtoehtoja, mutta koita vastata ilman vinkkejä

\begin{solution}
\leavevmode

Vastaus

\begin{verbatim}
 Ag-sidos+ktrl
\end{verbatim}

Jos kyseessä on lievä palovamma ja oletat, että se ei tarvitse leikkaushoitoa -\textgreater{} aloita hopeahoito (mepilex ag tmv.) ja kontrolloi tilanne PTH:ssa 2-5vrk kohdalla syvenemisen totamiseksi. Jos kontrollissa toteat merkittävän syvän vamman, joka ei todennäköisesti tule paranemaan itsestään, niin ESH-lähete.

\end{solution}

\section{Just eläköitynyt nainen, käynyt plastikon arviossa reduktioplastiaan, plastikko kieltäytynyt, miksi?}\label{just-eluxe4kuxf6itynyt-nainen-kuxe4ynyt-plastikon-arviossa-reduktioplastiaan-plastikko-kieltuxe4ytynyt-miksi}

Ei vaihtoehtoja, joten olen tähän keksinyt mahdolliset:

\begin{itemize}
\tightlist
\item
  \begin{enumerate}
  \def\labelenumi{\alph{enumi}.}
  \tightlist
  \item
    potilaan ikä liian korkea
  \end{enumerate}
\item
  \begin{enumerate}
  \def\labelenumi{\alph{enumi}.}
  \setcounter{enumi}{1}
  \tightlist
  \item
    potilaan bmi 27
  \end{enumerate}
\item
  \begin{enumerate}
  \def\labelenumi{\alph{enumi}.}
  \setcounter{enumi}{2}
  \tightlist
  \item
    potilas tupakoi puoli askia per pv
  \end{enumerate}
\item
  \begin{enumerate}
  \def\labelenumi{\alph{enumi}.}
  \setcounter{enumi}{3}
  \tightlist
  \item
    jugulum-mamillamitta 40cm
  \end{enumerate}
\end{itemize}

\begin{solution}
\leavevmode

Vastaus

\begin{verbatim}
 c
\end{verbatim}

Leikkaushoidon aiheellisuus arvioidaan yksilöllisesti. Jos leikkaus arvioidaan aiheelliseksi (oireet ja objektiivisesti todettu suuririntaisuus suhteessa muuhun kehoon), niin leikkaukseen pääsyn ehtoina ovat \textbf{tupakoimattomuus 6kk ja painoindeksi alle 28} (Ainakin Tyksissä; joissain paikoissa, varsinkin muissa Pohjoismaissa alle 30). Ylipainoisia potilaita kehotetaan ensin laihduttamaan, jolloin leikkauskomplikaatioiden ja anestesian riskit ovat vähäisemmät ja rintojen lopullinen leikkaustulos on pysyvämpi. Tupakointi heikentää leikkaushaavan paranemista ja altistaa infektioille. Jos potilas tupakoi, lähettävän lääkärin tulisi ohjata hänet lopettamaan tupakointi ja tarjota tukea (ryhmä, lääkkeet) ennen lähetteen tekemistä.

a: Yläikärajaa rintojen pienennysleikkaukselle ei ole. Hyvin nuorten naisten kohdalla on suositeltavaa odottaa murrosiän kehityksen päättymistä. Pääsääntöisesti pienennysleikkauksia tehdään siis yli 18-vuotiaille, mutta rintojen liikakasvun äärimuodossa voidaan leikkausta harkita alle 18-vuotiaalle.

d: Jugulum-mamillamitta tarkoittaa kaulakuopan ja nännin välistä mittaa rinnankokoa mitattaessa. On kuitenkin syytä muistaa jugulum-mamillamittaa käytettäessä, että se kertoo ainoastaan rintojen laskeutumisen asteen ja etenkin voimakkaan laihtumisen jälkeen laskeutuneet rinnat voivat olla volyymiltaan hyvinkin pienet. Yli 31cm jugulum-mamillamitta alkaa olemaan jo todella suuri ja antaa jo vahvaa aihetta reduktioplastialle.

Taulukko 1 rintojen reduktioplastiassa käytettävää Sosiaali- jaterveysministeriön laatimaa pisteytysjärjestelmää kiireettömän leikkaushoidon perusteisiin. Myöskään potilasta, joka ei saa vaadittua 50 pistettä täyteen, ei välttämättä suoraan käännytetä pois. Tässä tilanteessa poikkeavasta hoitopäätöksestä lääkärin on tehtävä kirjallinen perustelu potilaan saamista hyödyistä.

\pandocbounded{\includegraphics[keepaspectratio]{images/pisteytysrinnat.png}}
\pandocbounded{\includegraphics[keepaspectratio]{images/reduktioplastia.png}}

\end{solution}

\section{Mikä oikein tyriä koskien?}\label{mikuxe4-oikein-tyriuxe4-koskien}

Ei vaihtoehtoja, mutta vastauksen arveltu olevan: ``Vatsaontelon sisältö pääsee faskia-aukosta subkutaanitilaan'', joka on kylläkin oikein.

Tyrällä tarkoitetaan vatsaontelon sisällön purkautumista vatsaontelon ulkopuolelle vatsaontelon seinämän pitävässä rakenteessa olevan synnynnäisen tai hankitun aukon, ns. tyräportin, kautta. Useimmiten tyrä on vatsaontelon ulkopuolella oleva peritoneumpussi ja aukko on faskia-aukko.

\pandocbounded{\includegraphics[keepaspectratio]{images/tyräanatomia.png}}

\section{Potilaalla lipoomaksi sopiva muutos, jonka koko on arviolta n.~4x2x3 cm. Mitä teet?}\label{potilaalla-lipoomaksi-sopiva-muutos-jonka-koko-on-arviolta-n.-4x2x3-cm.-mituxe4-teet}

\begin{itemize}
\tightlist
\item
  \begin{enumerate}
  \def\labelenumi{\alph{enumi}.}
  \tightlist
  \item
    poistan itse
  \end{enumerate}
\item
  \begin{enumerate}
  \def\labelenumi{\alph{enumi}.}
  \setcounter{enumi}{1}
  \tightlist
  \item
    lähetän esh
  \end{enumerate}
\item
  \begin{enumerate}
  \def\labelenumi{\alph{enumi}.}
  \setcounter{enumi}{2}
  \tightlist
  \item
    otat näytteen
  \end{enumerate}
\item
  \begin{enumerate}
  \def\labelenumi{\alph{enumi}.}
  \setcounter{enumi}{3}
  \tightlist
  \item
    joku muu
  \end{enumerate}
\end{itemize}

\begin{solution}
\leavevmode

Vastaus

\begin{verbatim}
 a
\end{verbatim}

Mikäli pehmytkydoskasvain on mobiili, pehmeä, hitaasti kasvanut, ihonalainen ja alle 5 cm:n läpimittainen, se on hyvin todennäköisesti hyvänlaatuinen. Sen voi tällöin poistaa itse TK-lääkärinä tai potilaan niin halutessa jäädä seurantalinjalle.

Jos muutos olisi kliinisesti poikkeava eli \textless5cm, mutta ei mobiili tai nopeakasvuinen, niin voitaisiin tehdä kaikukuvaus ja sen perusteella jatkaa. Jos UÄ:ssä muutos olisi ihonalainen ja hyvänlaatuinen, niin se voitaisiin poistaa TK:ssa, jos se vaikuttaisi helposti poistettavalta. Pahanlaatuisuuden merkkien ollessa läsnä UÄ:ssä tehtäisiin lähete plastiikkakirurgille.

b: Jos hyvänlaatuisen muutoksen kriteerit eivät täyty ja muutos on kliinisesti pahanlaatuinen, niin ei voi jäädä pelkälle seurantakannalle, vaan tulee tehdä lähete plastiikkakirurgille.

c: Jos epäilisit pehmytkudostuumorin olevan maligni (eli sarkooma), niin niistä ei saa ottaa biopsioita, vaan lähetetään ESH.

\pandocbounded{\includegraphics[keepaspectratio]{images/pehmytkudoskasvainalgoritmi.png}}

\end{solution}

\section{Potilas kertoo peniksensä käyristyvän erektiossa, mistä kyse?}\label{potilas-kertoo-peniksensuxe4-kuxe4yristyvuxe4n-erektiossa-mistuxe4-kyse}

Ei vaihtoehtoja, mutta koita vastata ilman vinkkejä

\begin{solution}
\leavevmode

Vastaus

\begin{verbatim}
 Peyronien tauti
\end{verbatim}

Peyronien tauti etenee vaiheittain:

Alkuvaihe eli akuutti inflammatorinen vaihe: Arpikudosta muodostuu ja oireena esiintyy usein erektionaikainen kipu ja tunnusteltavissa oleva patti tai kovettuma jo ennen käyristymisen ilmaantumista
- Aiheuttaa kipua erityisesti erektioissa, ajan kuluessa kipu helpottaa (kipuvaihe eli alkuvaihe saattaa kestää useita kuukausia)

Kroonisessa fibroottisessa vaiheessa siittimessä selkeä kovettuma, joka kalkkeutuu ja aiheuttaa siittimen käyristymisen (käyristyy kovettuman suuntaan; useimmiten plakki on dorsumissa -\textgreater{} käyristyy ylöspäin, mutta plakki voi muodostua minne vain). Peyronien taudissa voidaan tyypillisesti tuntea siittimessä kovettuman.

Ei ole tehokasta lääkehoitoa (injisoitava kollagenaasi poistunut markkinoilta), akuutissa vaiheessa tulehduskipulääkkeitä kivunhoidossa ja PDE5:n estäjät, jos potilaalla samanaikaisesti erektiohäiriö. Ensisijainen hoitokeino on operatiivinen, mutta mahdollinen oikaisuleikkaus tehdään vasta, kun kipuvaihe on ohi ja tilanne on ollut stabiili vähintään 3 kk. Useimmat potilaat pystyvät yhdyntään, jolloin kirurginen hoito ei ole tarpeen

\end{solution}

\section{Mikä on basaliooman poiston tavoiteltu marginaali?}\label{mikuxe4-on-basaliooman-poiston-tavoiteltu-marginaali}

\begin{solution}
\leavevmode

Vastaus

\begin{verbatim}
 5 mm (makroskooppinen)
\end{verbatim}

Pienen basaliooman voi poistaa terveyskeskuksessa ja hoitaa alusta loppuun itse. Hoito on ensisijaisesti 5mm makroskooppisilla marginaaleilla poisto veneviillosta. Tulee muistaa varmistaa PAD-vastaus ja puhtaat marginaalit. Jos sait koko muutoksen poistettua PAD:n mukaan ja leikkeessä on terveet marginaalit, niin hoito oli siinä. Seurantaakaan ei tarvitse, jos marginaalit olivat riittävät. Jos kyseessä on ollut suuren uusiutumisriskin (high grade) basaliooma tai ei-kirurgisesti hoidettu basaliooma, seuranta suunnitellaan harkinnan mukaan hoitavassa yksikössä. Jos basaliooma on leikkauksessa poistettu epätäydellisesti, pitää uusiutumisen ehkäisemiseksi suorittaa joko arven poistoleikkaus tai tilanteen mukaan ei-kirurginen hoito. Pelkkä seuranta ei ole näissä tapauksissa hyväksyttävää.

Basalioomasta ei oteta ohutneulabiopsiaa. Muutoksen poisto kokonaisuudessaan on paras. Jos poisto ei onnistu, niin koepalaksi riittää 3--5 mm:n stanssipala, mutta jos kyseessä on pieni, kliinisesti selvä basaliooma esimerkiksi vartalolla, kannattaa tuumori poistaa kerralla kokonaan riittävin marginaalein.

\end{solution}

\section{Basaliooma, voiko metastasoida?}\label{basaliooma-voiko-metastasoida}

Ei vaihtoehtoja, mutta koita vastata ilman vinkkejä

\begin{solution}
\leavevmode

Vastaus

\begin{verbatim}
 Kyllä, mutta on todella harvinaista
\end{verbatim}

Insidenssi vaihtelee 0.0028\%-0.55\%. Tyvisolusyöpä on siis hyvin mukava syöpä.

\end{solution}

\section{Lapsen appendisiitti, mikä on totta}\label{lapsen-appendisiitti-mikuxe4-on-totta}

\begin{itemize}
\tightlist
\item
  \begin{enumerate}
  \def\labelenumi{\alph{enumi}.}
  \tightlist
  \item
    ei ole yleisin lapsen akuutin vatsan aiheuttaja
  \end{enumerate}
\item
  \begin{enumerate}
  \def\labelenumi{\alph{enumi}.}
  \setcounter{enumi}{1}
  \tightlist
  \item
    lapsen CRP käyttäytyy taudissa eri tavalla kuin aikuisilla
  \end{enumerate}
\item
  \begin{enumerate}
  \def\labelenumi{\alph{enumi}.}
  \setcounter{enumi}{2}
  \tightlist
  \item
    diagnosoidaan usein TT:llä
  \end{enumerate}
\item
  \begin{enumerate}
  \def\labelenumi{\alph{enumi}.}
  \setcounter{enumi}{3}
  \tightlist
  \item
    hoidetaan useammin konservatiivisesti kuin aikuisilla
  \end{enumerate}
\end{itemize}

\begin{solution}
\leavevmode

Vastaus

\begin{verbatim}
 b
 
\end{verbatim}

Tulehdusarvoja (leukosyyttiluku ja CRP-pitoisuus) voidaan käyttää appendisiitin todennäköisyyden arvioimisessa. Normaali leukosyyttiluku ja CRP-pitoisuus ei täysin poissulje appendisiittia, mutta molempien ollessa normaalialueella on appendisiitti hyvin epätodennäköinen (varsinkin aikuisilla). N. 1/100 aikuispotilaasta on leuk ja CRP normaalit vaikka appendisiitti ja n.~6/100 lapsista on leuk ja CRP normaalit, koska lasten immunologia hieman erilainen kuin aikuisilla.

a: Appendisiitti on yleisin kirurgisen akuutin vatsan syy lapsilla

c: Lapsilla TT:tä vältetään säteilyn takia. Ensisijainen on UÄ, ja tarvittaessa MRI, ei TT

d: Konservatiivinen hoito (antibiootit) ei ole lapsilla ensisijaista. Lapset pääsääntöisesti leikataan. Joitakin aikuispotilaita on viime aikoina hoidettu konservatiivisesti ab-hoidolla, jos komplisoitunut appendisiitti on poissuljettu TT:llä. On kuitenkin jo näyttöä siitä, että lastenkin komplisoitumattomia tauteja (komplisoitumista ei tosin voi poissulkea UÄ:llä vaan vaatii vähintään MRI:n tai TT:n) voitaisiin hoitaa konservatiivisesti, mutta käytäntö ei ole yleisessä käytössä.

Komplisoitumatonta appendisiittia (ja jos ei ole sen riskitekijöitä) ei tarvitse leikata välittömästi, vaan voidaan odottaa virka-aikaan pääsääntöisesti

\end{solution}

\section{Milloin lippaluomileikkaus julkisella puolella}\label{milloin-lippaluomileikkaus-julkisella-puolella}

\begin{itemize}
\tightlist
\item
  \begin{enumerate}
  \def\labelenumi{\alph{enumi}.}
  \tightlist
  \item
    potilaalla oireita, BMI 30 tai alle, ei tupakoi
  \end{enumerate}
\item
  \begin{enumerate}
  \def\labelenumi{\alph{enumi}.}
  \setcounter{enumi}{1}
  \tightlist
  \item
    oireina näköhäiriöt, yläluomi painaa ripsien tyveen
  \end{enumerate}
\item
  \begin{enumerate}
  \def\labelenumi{\alph{enumi}.}
  \setcounter{enumi}{2}
  \tightlist
  \item
    kosmeettinen haitta
  \end{enumerate}
\item
  \begin{enumerate}
  \def\labelenumi{\alph{enumi}.}
  \setcounter{enumi}{3}
  \tightlist
  \item
    joku (ei wikissä)
  \end{enumerate}
\end{itemize}

\begin{solution}
\leavevmode

Vastaus

\begin{verbatim}
 b
\end{verbatim}

Dermatochalasis eli lippaluomi tarkoittaa yläluomen ihoylimäärää +/- rasvaa. Laskeutuu silmäluomen päälle ja voi peittää näkökenttää, aiheuttaen näköhaittaa tai väsyneen ilmeen. Hoitona on leikkaus (blefaroplastia). Lippaluomen yläluomileikkauksessa poistetaan yläluomen alueelle muodostunut ylimääräinen ihopoimu ja tarvittaessa muotoillaan alla olevaa rasvaa.

Raskas yläluomi aiheuttaa väsyneen ilmeen, mutta näkökentän eteen laskeutuessaan lippaluomipoimu aiheuttaa myös toiminnallisen ongelman. Tällöin tilanne rinnastetaan sairaudeksi, jolloin leikkaus voidaan toteuttaa julkisessa sairaanhoidossa. Julkisella yleisinä yläluomen kirurgian kriteereinä voidaan pitää:

Toiminnallinen näkökykyhaitta (kompensatorinen otsajännitys voi esim aiheuttaa estolääkitystä vaativan migreenin)

MRD-1 mitta \textless2mm (MRD = Margin-reflex distance. Iho peittää näkökenttää tulemalla alle 2 mm:n päähän mustuaisen valoheijasteesta/optiselta akselilta)

a: Enemmänkin esim. rintojenpienennysleikkauksen kriteerit (BMI-rajana joissain paikoissa 30, joissain 28)

c: Pelkkä kosmeettinen haitta ei ole aihe julkisen puolen leikkaukselle. Tarvitsee olla oireita.

\pandocbounded{\includegraphics[keepaspectratio]{images/lippaluomi.png}}

\end{solution}

\section{54-vuotiaan miehen PSA 0,25, ei mitään oireita, mutta tuseeraten eturauhasen vasemmalla puolella kova resistenssi, miten jatkot?}\label{vuotiaan-miehen-psa-025-ei-mituxe4uxe4n-oireita-mutta-tuseeraten-eturauhasen-vasemmalla-puolella-kova-resistenssi-miten-jatkot}

\begin{itemize}
\tightlist
\item
  \begin{enumerate}
  \def\labelenumi{\alph{enumi}.}
  \tightlist
  \item
    pyydät myös vapaan PSA:n määritystä ja, jos se on matala lähetät urolle jatkoselvityksiin
  \end{enumerate}
\item
  \begin{enumerate}
  \def\labelenumi{\alph{enumi}.}
  \setcounter{enumi}{1}
  \tightlist
  \item
    lähetät suoraan jatkoselvityksiin
  \end{enumerate}
\item
  \begin{enumerate}
  \def\labelenumi{\alph{enumi}.}
  \setcounter{enumi}{2}
  \tightlist
  \item
    ei tarvetta jatkotoimille
  \end{enumerate}
\item
  \begin{enumerate}
  \def\labelenumi{\alph{enumi}.}
  \setcounter{enumi}{3}
  \tightlist
  \item
    joku
  \end{enumerate}
\end{itemize}

\begin{solution}
\leavevmode

Vastaus

\begin{verbatim}
 b
\end{verbatim}

Eturauhassyöpää voidaan epäillä toistetusti poikkeavan PSA-pitoisuuden tai poikkeavan eturauhasen tunnustelulöydöksen perusteella. Eturauhassyöpä todetaan n.~kolmasosalla potilaista, joilla eturauhasessa tuntuu kyhmy. \textbf{Eturauhasessa tuntuva uusi kyhmy on urologin konsultaation aihe riippumatta PSA-pitoisuudesta.} Normaali eturauhasen tunnustelulöydös ei poissulje eturauhassyöpää.

a: Näin voisi toimia, jos palpaatiolöydös olisi normaali ja PSA olisi lievästi koholla. Jos PSA on 2-15µg/l, niin lasketaan automaattisesti vapaa-PSA/kokonais-PSA-suhde, joka auttaa eturauhassyövän diagnostiikassa. Syöpäsolut tuottavat sitoutunutta PSA:a -\textgreater{} suhde laskee (Jos suhde \textless10\%, on riski syövälle yli 50\%; Jos suhde on \textgreater25\%, niin riski alle 10\%).

c: Epäilyttävä tuseerauslöydös -\textgreater{} aihe jatkoselvityksille

\end{solution}

\section{Mikä/Mitkä suonet hoidetaan useimmiten alaraajojen laskimoiden vajaatoiminnassa termoablaatiolla}\label{mikuxe4mitkuxe4-suonet-hoidetaan-useimmiten-alaraajojen-laskimoiden-vajaatoiminnassa-termoablaatiolla}

\begin{itemize}
\tightlist
\item
  \begin{enumerate}
  \def\labelenumi{\alph{enumi}.}
  \tightlist
  \item
    pinnalliset laskimot
  \end{enumerate}
\item
  \begin{enumerate}
  \def\labelenumi{\alph{enumi}.}
  \setcounter{enumi}{1}
  \tightlist
  \item
    syvät laskimot
  \end{enumerate}
\item
  \begin{enumerate}
  \def\labelenumi{\alph{enumi}.}
  \setcounter{enumi}{2}
  \tightlist
  \item
    laskimohaarat
  \end{enumerate}
\item
  \begin{enumerate}
  \def\labelenumi{\alph{enumi}.}
  \setcounter{enumi}{3}
  \tightlist
  \item
    vena saphena magna \& parva
  \end{enumerate}
\end{itemize}

\begin{solution}
\leavevmode

Vastaus

\begin{verbatim}
 d
\end{verbatim}

Termoablaatiossa kohteena oleva pinnallinen päärunko (yleensä vena saphena magna tai parva) kanyloidaan ultraääniohjauksessa. Hoitokatetri viedään ultraääniohjauksessa 1--2 cm safenofemoraalisen tai safenopopliteaalisen junktion distaalipuolelle (turvamarginaali syvään laskimoon).

Hoidettavan laskimon ympärille injisoidaan tumesenssi-puudutus.Tumesenssi muodostaa lisäksi suojan laskimon ja ihon sekä muiden kudosten välille, jolloin termoablaation vaikutusten leviäminen ympäröiviin kudoksiin estyy.

Peruutellessa suonta takaisinpäin se suljetaan joko laserablaatiolla tai radiofrekvenssioablaatiolla.

\pandocbounded{\includegraphics[keepaspectratio]{images/saphenamagnalaser.png}}

\end{solution}

\section{Kolarissa ollut nuori mies, jolla syke 120, RR 90/50, pieni ilmarinta ja nestettä/verta perikardiumissa, miten toimit päivystäjänä?}\label{kolarissa-ollut-nuori-mies-jolla-syke-120-rr-9050-pieni-ilmarinta-ja-nestettuxe4verta-perikardiumissa-miten-toimit-puxe4ivystuxe4juxe4nuxe4}

\begin{itemize}
\tightlist
\item
  \begin{enumerate}
  \def\labelenumi{\alph{enumi}.}
  \tightlist
  \item
    soitan thx-kirurgille, pleuradreeni ja valmistelu leikkaukseen
  \end{enumerate}
\item
  \begin{enumerate}
  \def\labelenumi{\alph{enumi}.}
  \setcounter{enumi}{1}
  \tightlist
  \item
    adrenaliinia ja tehoseuranta
  \end{enumerate}
\item
  \begin{enumerate}
  \def\labelenumi{\alph{enumi}.}
  \setcounter{enumi}{2}
  \tightlist
  \item
    soitan thx-kirurgille ja aloitan itse hätätoimenpiteen
  \end{enumerate}
\item
  \begin{enumerate}
  \def\labelenumi{\alph{enumi}.}
  \setcounter{enumi}{3}
  \tightlist
  \item
    joku
  \end{enumerate}
\end{itemize}

\begin{solution}
\leavevmode

Vastaus

\begin{verbatim}
 c
\end{verbatim}

Potilaalla on todennäköisesti tamponaatio (hemoperikardium estää sydämen normaalin pumppaustoiminnan). Hoito riippuu tilanteesta: potilaan voinnista ja erityisesti siitä, missä ollaan töissä ja mitä resursseja on käytössä. Optimitilanteessa kokematon ei joutuisi tekemään hätätoimenpiteitä eikä ainakaan ilman välitöntä tukea.

Henkeä uhkaavassa, hemodynaamiseen sokkiin johtaneessa tamponaatiossa on viipymättä tehtävä sydänpussinesteen tyhjennys joko perkutaanisesti kaikukuvaus- tai läpivalaisukontrollissa tai kirurgisesti. Vaaran merkkejä ovat matala verenpainetaso, viileä periferia, pullottavat kaulalaskimot, tajunnantason heikkeneminen ja levottomuus. Ennen kirurgista hoitoa sokkinen potilas voidaan yrittää stabiloida perikardiumpunktiolla eli -senteesillä.

Punktiota ei suositella, jos sydämen edessä oleva nestekertymä on läpimitaltaan alle 10 mm tai jos kertymä on lokeroitunut perikardiumin adheesioiden takia. Perdikardiumpunktion komplikaationa voi sattua harvoin sepelvaltimon (1 \%) tai rintavaltimon perforaatio, punktio sydämen lokeroihin (1 \%) tai ilma- tai veririnta (0,5 \%). Vielä harvinaisempia komplikaatioita ovat maksan tai keuhkon perforaatio ja purulentti perikardiitti (0,3 \%). Kaikukuvauksen rutiininomainen käyttö vähentää perikardiosenteesin komplikaatioita.

a: Pieni ilmarinta ei todennäköisesti ole potilaan epästabiilin hemodynamiikan taustalla. Jos kyseessä taas olisi paineilmarinta (näkyisi kyllä radiologisesti), niin silloin sen hoito olisi hätätoimenpiteellisesti aiheellista.

b: Adrenaliini ja seuranta ei tule hoitamaan potilaan tamponaatiota, vaan tämä hoitolinja johtaisi kuolemaan.

\pandocbounded{\includegraphics[keepaspectratio]{images/senteesi.png}}
\pandocbounded{\includegraphics[keepaspectratio]{images/subcostaltamponation.png}}

\end{solution}

\section{25-v raskautta suunnittelevalla naisella yli viitealueen kalsium (1,50) ja vasemman puolen kilpirauhasadenooma, mikä ensisijainen hoito}\label{v-raskautta-suunnittelevalla-naisella-yli-viitealueen-kalsium-150-ja-vasemman-puolen-kilpirauhasadenooma-mikuxe4-ensisijainen-hoito}

\begin{itemize}
\tightlist
\item
  \begin{enumerate}
  \def\labelenumi{\alph{enumi}.}
  \tightlist
  \item
    leikkaus ennen raskautumista
  \end{enumerate}
\item
  \begin{enumerate}
  \def\labelenumi{\alph{enumi}.}
  \setcounter{enumi}{1}
  \tightlist
  \item
    lääkehoito
  \end{enumerate}
\item
  \begin{enumerate}
  \def\labelenumi{\alph{enumi}.}
  \setcounter{enumi}{2}
  \tightlist
  \item
    leikkaus vasta raskauden jälkeen
  \end{enumerate}
\end{itemize}

\begin{solution}
\leavevmode

Vastaus

\begin{verbatim}
 a
\end{verbatim}

\textbf{Paratyreoidektomia ei ole vasta-aiheinen primaarin hyperparatyreoosin (PHPT) hoitokeino, jos potilas suunnittelee raskautta; se on oikeastaan leikkausindikaatio!}

\pandocbounded{\includegraphics[keepaspectratio]{images/phptleikkauskriteerit.png}}
\pandocbounded{\includegraphics[keepaspectratio]{images/phptleikkauskriteerit4mk.png}}

\end{solution}

\section{Potilastapaus}\label{potilastapaus-38}

Nuorella miehellä kipeytyneet kivekset ja penis, ollut aikaisemminkin, mutta usein helpottanut ab-kuurilla. Ei ole aikaisemmin otettu virtsanäytettä. Mikä todennäköinen diagnoosi ja hoito?

\begin{itemize}
\tightlist
\item
  \begin{enumerate}
  \def\labelenumi{\alph{enumi}.}
  \tightlist
  \item
    krooninen prostatiitti akutisoitunut, aloitetaan fluorokinoloni
  \end{enumerate}
\item
  \begin{enumerate}
  \def\labelenumi{\alph{enumi}.}
  \setcounter{enumi}{1}
  \tightlist
  \item
    krooninen prostatiitti akutisoitunut, otetaan virtsanäyte, aloitetaan NSAID ja tamsulosiini, jos virtsanäyte negatiivinen
  \end{enumerate}
\item
  \begin{enumerate}
  \def\labelenumi{\alph{enumi}.}
  \setcounter{enumi}{2}
  \tightlist
  \item
    krooninen prostatiitti akutisoitunut, pahenee helposti, joten lähete päivystykseen
  \end{enumerate}
\end{itemize}

\begin{solution}
\leavevmode

Vastaus

\begin{verbatim}
 b
\end{verbatim}

Kroonisessa prostatiitissa oireet ovat samat kuin akuutissa prostatiitissa, mutta lievempiä ja toistuvia (erityisesti talviaikaan).

Kroonisessa prostatiitissa U-Kemseul ja BaktVi ovat yleensä negatiiviset. Krooninen prostatiitti on siis yleensä nonbakterielli etiologialtaan (vain. 10\% virtsanäytteissä bakteereita). Kroonisesta nonbakteriellistä eturauhastulehduksesta käytetään usein nimitystä miehen krooninen lantion kiputila (chronic pelvic pain syndrome, CPPS, tyypin 3 prostatiitti). Syy ei aina selvä ja usein jää idiopaattiseksi; tärkeää sulkea pois ainakin eturauhassyöpä (voidaan tarvittaessa ottaa eturauhasbiopsia), vaikka potilas olisikin alle 50v.

Tulee siis ottaa virtsanäyte ja varmistaa, että kyseessä on nonbakterielli tulehdus. Sen parantavaa hoitoa ei tunneta, mutta ensilinjan lääkkeitä ovat: tulehduskipulääkkeet (NSAID), alfasalpaajat (esim. tamsulosiini), 5-alfareduktaasin estäjät (finasteridi ja dutasteridi; erityisesti jos suurentunut eturauhanen), tarvittaessa trisykliset lääkkeet (esim. nortriptyliini; hyviä monissa kroonisissa kipuoireyhtymissä).

a: Mikrobilääkkeistä voi olla hyötyä ja joillakin antibiooteilla on osoitettu esiintyvän sytokiinivälitteistä anti-inflammatorista vaikutusta, joka saattaa selittää niiden oireita lievittävän vaikutuksen; kannattaa harkita, jos \emph{oireet esiintyvät ensimmäistä kertaa} (varsinkin jos todetaan pyuriaa), jos \textbf{aikaisemmista on ollut hyötyä} tai tietysti jos todetaan bakteereita virtsanäytteessä.

Potilaalle kylläkin voisi aloittaa antibiootin sen perusteella, että hän on kokenut niistä hyötyä aikaisemminkin. Kuitenkaan uusien ohjeiden mukaan fluorokinoloneja ei tulisi käyttää ei-bakteriellin kroonisen eturauhastulehduksen hoidossa.

c: Ei ole aihetta päivystyslähetteelle.

Kuvan hierontatesti viitaaa siihen, että otetaan kaksi virtsanäytettä (PPMT, pre and post massage test); ns. fraktioitu virtsanäyte. Ensimmäisestä tutkitaan sakka ja tehdään bakteeriviljely. Tämän jälkeen eturauhasta hierotaan usean minuutin ajan ja otetaan heti uusi alkuvirtsanäyte (jälkihierontanäyte, 10--20 ml virtsasuihkun alusta). Kroonisessa eturauhastulehduksessa ilman bakteerietiologiaa (tyyppi 3) hierontanäytteet ovat steriilejä, mutta tyypin 3a eturauhastulehduksessa esiintyy valkosoluja jälkihierontanäytteessä (\textgreater{} 10/nk), kun taas tyypissä 3b ne puuttuvat. Pyuria, johon ei liity bakteerikasvua, hoidetaan kerran 1--2 kk:n pituisella mikrobilääkekuurilla. Jos tästä ei ole apua, mikrobilääkekuureja ei kannata toistaa.

\pandocbounded{\includegraphics[keepaspectratio]{images/cpps.png}}
\pandocbounded{\includegraphics[keepaspectratio]{images/krooninenprostatiittiterveysportti.png}}

\end{solution}

\section{Pehmytkudosinfektion hoidon kulmakivi}\label{pehmytkudosinfektion-hoidon-kulmakivi}

\begin{itemize}
\tightlist
\item
  \begin{enumerate}
  \def\labelenumi{\alph{enumi}.}
  \tightlist
  \item
    ylipainehappihoito
  \end{enumerate}
\item
  \begin{enumerate}
  \def\labelenumi{\alph{enumi}.}
  \setcounter{enumi}{1}
  \tightlist
  \item
    nekroottisen kudoksen laaja poisto kirurgisesti
  \end{enumerate}
\item
  \begin{enumerate}
  \def\labelenumi{\alph{enumi}.}
  \setcounter{enumi}{2}
  \tightlist
  \item
    laajakirjoinen ab esim. meropeneemi + klindamysiini
  \end{enumerate}
\end{itemize}

\begin{solution}
\leavevmode

Vastaus

\begin{verbatim}
 b
\end{verbatim}

Kuoliota aiheuttavien pehmytkudostulehdusten hoidon tärkeimmät osiot:

\begin{enumerate}
\def\labelenumi{\arabic{enumi}.}
\tightlist
\item
  Eksploratiivinen leikkaushoito (tärkein) ja tietysti sepsiksen ja mahdollisen septisen sokin tukihoito
\item
  Laajakirjoinen iv-antibioottihoito
\item
  Ylipainehappihoito
\end{enumerate}

Leikkaushoitoa ei voi korvata mikrobilääke-, ylipainehappi- tai muulla hoidolla. Kirurginen hoito on suoraviivaista: ihoavausta jatketaan, kunnes iho muuttuu terveeksi ja ihonalainen rasva ei enää irtoa kevyesti lihaskalvosta; kaikki kuollut kudos ja märkäkertymät poistetaan huolellisesti. Kaikki huono kudos 5 - 10 mm terveyskudosmarginaalilla infektion rajaamiseksi. Päälle kompressisidokset ja seuraava haavan tarkistus \textless24 tunnin toimenpiteestä; jos todetaan uusia kuolioalueita tai märkäkertymiä on tehtävä viipymättä uusintaleikkaus.

a: Ylipainehappihoidon (HBO): lisää leukosyyttien toimintaa, tukahduttaa bakteerien kasvua, tehostaa ab:n vaikutuksia ja parantaa kudosparantumista. Ei kuitenkaan ole läheskään tärkein hoito.

c: Ilman kirurgista hoitoa ei ab tule auttamaan.

\end{solution}

\section{Mihin melanooma metastasoi ensimmäisenä?}\label{mihin-melanooma-metastasoi-ensimmuxe4isenuxe4}

Ei vaihtoehtoja, mutta tässä tärkeimmät melanooman metastasoimisesta:

\begin{itemize}
\tightlist
\item
  Leviää aluksi alueellisiin imusolmukkeisiin. Niitä imusolmukkeita, jotka vastaanottavat ensimmäisenä suoraan kasvainalueelta tulevan imunesteen ja sen mukana kulkevat mahdolliset kasvainsolut, kutsutaan vartijasolmukkeiksi. Noin joka viidennellä melanoomapotilaalla voidaan todeta vartijasolmukkeen mikrometastaasi, mikä on jatkohoidon ja seurannan suunnittelussa huomioitava epäedullinen ennustetekijä.

  \begin{itemize}
  \tightlist
  \item
    Ennen imusolmukkeisiin pääsyä primaarituumorista irroneita solukertymiä eli satelliittileesioita kutsutaan eri nimillä:

    \begin{itemize}
    \tightlist
    \item
      Mikrosatelliittimetastaasit = syöpäsolut nähdään primäärituumorin vieressä (eivät koske siihen) mikroskoopilla
    \item
      Satelliittimetastaasit = syöpäsoluja on levinnyt alle 2cm päähän primäärituumorista
    \item
      In-transit-metastaasit = syöpäsoluja on levinnyt yli 2cm päähän mutta ei vielä imusolmukkeisiin
    \end{itemize}
  \end{itemize}
\end{itemize}

Ihon ja imusolmukkeiden ulkopuolisista rakenteista melanoomien yleisimpiä etäisiä metastasointipaikkoja ovat keuhkot, maksa, luut ja aivot

\section{Mitä tarkoittaa gradus 4 sarkooma}\label{mituxe4-tarkoittaa-gradus-4-sarkooma}

Ei vaihtoehtoja, mutta ehkä haettu vain simppeliä määritelmää:

\begin{itemize}
\tightlist
\item
  \textbf{Huonosti erilaistunut eli korkean pahanlaatuisuusasteen sarkooma}

  \begin{itemize}
  \tightlist
  \item
    Hyvin erilaistuneet kasvaimet (gradus 1 ja 2) ovat käytännössä matalan pahanlaatuisuusasteen kasvaimia, huonosti erilaistuneet kasvaimet (gradus 3 ja 4) puolestaan korkean pahanlaatuisuusasteen kasvaimia
  \end{itemize}
\end{itemize}

Tuumorien gradus siis tyypillisesti kuvailee tuumorin erilaistumisastetta. Graduksen noustessa tuumorin erilaistuminen huononee.

\section{Poistettu 3cm adenooma nousevasta kolonista, jossa lievää dysplasiaa. Miten seurataan?}\label{poistettu-3cm-adenooma-nousevasta-kolonista-jossa-lievuxe4uxe4-dysplasiaa.-miten-seurataan}

\begin{itemize}
\tightlist
\item
  \begin{enumerate}
  \def\labelenumi{\alph{enumi}.}
  \tightlist
  \item
    Ei seurantaa
  \end{enumerate}
\item
  \begin{enumerate}
  \def\labelenumi{\alph{enumi}.}
  \setcounter{enumi}{1}
  \tightlist
  \item
    Kontrollikolonoskopia, mutta ei parin vuoden sisään
  \end{enumerate}
\item
  \begin{enumerate}
  \def\labelenumi{\alph{enumi}.}
  \setcounter{enumi}{2}
  \tightlist
  \item
    Seurataan tiheästi mahdollisen syövän muodostumisen vuoksi
  \end{enumerate}
\item
  \begin{enumerate}
  \def\labelenumi{\alph{enumi}.}
  \setcounter{enumi}{3}
  \tightlist
  \item
    Joku
  \end{enumerate}
\end{itemize}

\begin{solution}
\leavevmode

Vastaus

\begin{verbatim}
 b
\end{verbatim}

Adenoomien seurannasta on olemassa tarkat suositukset, joiden perusteellista osaamista ei vaadita. Pääsääntöisesti voi muistaa, että seuranta on aiheellista, jos todetaan adenooma \textgreater1cm, high grade dysplasia tai muu riskiä kohottava histologia (esim. sahalaitaisuus tai villööttisuus), saatiin vain osittainen poisto tai potilaalla on polypoosi.

Edenneen adenooman löytyminen ennustaa suurentunutta riskiä sairastua kolorektaalisyöpään, minkä vuoksi poiston täydellisyyden varmistamisen jälkeen tehdään 3 vuoden kuluttua vielä uusintatähystys.

a: Seurantaa tarvitaan, koska muutos on \textgreater1cm. Pienen riskin muutoksissa potilas voidaan palauttaa takaisin paksusuolisyövän seulontaprosessiin 6v kuluttua.

c: Ei tarvitse tiheää seurantaa, kolmen vuoden päästä riittää. Mikäli tähystyksessä todetaan polyyppi, mutta poisto jää puutteelliseksi, tulee kolonoskopia uusia kolmen kuukauden kuluessa.

\pandocbounded{\includegraphics[keepaspectratio]{images/polyypitseuranta.png}}

\end{solution}

\section{Potilaalla meleenaa ja Hb 84, mikä todennäköisin vuodon alkuperä?}\label{potilaalla-meleenaa-ja-hb-84-mikuxe4-todennuxe4kuxf6isin-vuodon-alkuperuxe4}

\begin{itemize}
\tightlist
\item
  \begin{enumerate}
  \def\labelenumi{\alph{enumi}.}
  \tightlist
  \item
    yläGI
  \end{enumerate}
\item
  \begin{enumerate}
  \def\labelenumi{\alph{enumi}.}
  \setcounter{enumi}{1}
  \tightlist
  \item
    alaGI
  \end{enumerate}
\item
  \begin{enumerate}
  \def\labelenumi{\alph{enumi}.}
  \setcounter{enumi}{2}
  \tightlist
  \item
    50-50
  \end{enumerate}
\item
  \begin{enumerate}
  \def\labelenumi{\alph{enumi}.}
  \setcounter{enumi}{3}
  \tightlist
  \item
    joku
  \end{enumerate}
\end{itemize}

\begin{solution}
\leavevmode

Vastaus

\begin{verbatim}
 a
\end{verbatim}

Meleenan (mustien ulosteiden) tulisi herättää epäily ylä-GI-vuodosta (veri ehtii hajoamaan ja ulosteen väri muuttuu mustaksi). Hemorrhagia ex ano (kirkas punainen vuoto) tulisi taas herättää epäilyn ala-GI-vuodosta. Meleena voi joskus johtua ala-GI-vuodostakin (esim. ohutsuolesta tai proksimaalisesta paksusuolesta), mutta yleensä aiheuttaja on ylä-GI-vuoto.

\end{solution}

\section{DM1, munuaisten vajaatoiminta, haava varpaassa. Reidessä 30x30 punainen ja kipeä alue.}\label{dm1-munuaisten-vajaatoiminta-haava-varpaassa.-reidessuxe4-30x30-punainen-ja-kipeuxe4-alue.}

\begin{itemize}
\tightlist
\item
  \begin{enumerate}
  \def\labelenumi{\alph{enumi}.}
  \tightlist
  \item
    ihotautilääkärille
  \end{enumerate}
\item
  \begin{enumerate}
  \def\labelenumi{\alph{enumi}.}
  \setcounter{enumi}{1}
  \tightlist
  \item
    erikoissairaanhoitoon (epäily faskian infektiosta)
  \end{enumerate}
\item
  \begin{enumerate}
  \def\labelenumi{\alph{enumi}.}
  \setcounter{enumi}{2}
  \tightlist
  \item
    seurantaan osastolle, antibiootti
  \end{enumerate}
\item
  \begin{enumerate}
  \def\labelenumi{\alph{enumi}.}
  \setcounter{enumi}{3}
  \tightlist
  \item
    joku
  \end{enumerate}
\end{itemize}

\begin{solution}
\leavevmode

Vastaus

\begin{verbatim}
 b
\end{verbatim}

Haava varpaassa on voinut toimia infektioporttina. Reiden laaja kivulias eryteema diabetesta sairastavalla tulisi herättää pelon siitä, että kyseessä voisi mahdollisesti olla nekrotisoiva faskiitti ja potilas vaatii päivystysluonteista arviota.

a: Ihotautilääkäri ei ole faskiitin hoitaja.

c: Perusterveydenhuollon vuodeosasto ei ole sovelias paikka mahdollisen nekrotisoivan pehmytkudosinfektion hoidolle.

\end{solution}

\chapter{2025 (Calidus)}\label{calidus}

Paljon vanhoja tärppejä taas; ei niitä tässä yhteydessä uudestaan. Plastiikan suhteen enemmän uusia.

\section{Potilas sairastaa metastasoitunutta ventrikkelikarsinoomaa. Ohutsuolessa metastaaseja, koolon siisti. Potilas hyötyisi todennäköisimmin}\label{potilas-sairastaa-metastasoitunutta-ventrikkelikarsinoomaa.-ohutsuolessa-metastaaseja-koolon-siisti.-potilas-hyuxf6tyisi-todennuxe4kuxf6isimmin}

\begin{itemize}
\tightlist
\item
  \begin{enumerate}
  \def\labelenumi{\alph{enumi}.}
  \tightlist
  \item
    Ohutsuoliavanteesta
  \end{enumerate}
\item
  \begin{enumerate}
  \def\labelenumi{\alph{enumi}.}
  \setcounter{enumi}{1}
  \tightlist
  \item
    Paksusuoliavanteesta
  \end{enumerate}
\item
  \begin{enumerate}
  \def\labelenumi{\alph{enumi}.}
  \setcounter{enumi}{2}
  \tightlist
  \item
    Ohutsuoliresektiosta ja ohutsuolen yhdistämisestä paksusuoleen
  \end{enumerate}
\end{itemize}

\begin{solution}
\leavevmode

Vastaus

\begin{verbatim}
 c
\end{verbatim}

Etäpesäkkeiset mahasyövät (M1) ovat useimmiten kuratiivisen hoidon ulkopuolella, kuten myös osa paikallisesti edenneistä syövistä. Lisäksi oheissairaudet ja yleiskunto voivat muodostua esteeksi radikaalikirurgialle.

Tukosoiretta voidaan hoitaa joko asettamalla stentti endoskooppisesti tukoskohtaan tai tekemällä palliatiivinen resektio tai operatiivinen ohitus (gastrojejunostomia). Stenttihoito tulee kyseeseen, jos potilas on huonokuntoinen ja hänellä on vain muutama kuukausi elinaikaa. \textbf{Hyväkuntoiselle potilaalle resektio on paras vaihtoehto, jos kasvain ei ole kiinnittynyt retroperitoneaalisesti.} Palliatiivisena toimenpiteenä totaaligastrektomia on huono, ja se voi jopa lyhentää elinaikaa.

\end{solution}

\section{Mikä ei ole tyypillinen kivessyövän oire/löydös?}\label{mikuxe4-ei-ole-tyypillinen-kivessyuxf6vuxe4n-oireluxf6yduxf6s}

\begin{itemize}
\tightlist
\item
  \begin{enumerate}
  \def\labelenumi{\alph{enumi}.}
  \tightlist
  \item
    kasvain kiveksessä
  \end{enumerate}
\item
  \begin{enumerate}
  \def\labelenumi{\alph{enumi}.}
  \setcounter{enumi}{1}
  \tightlist
  \item
    kasvain kiveksen yläpuolella
  \end{enumerate}
\item
  \begin{enumerate}
  \def\labelenumi{\alph{enumi}.}
  \setcounter{enumi}{2}
  \tightlist
  \item
    kipu
  \end{enumerate}
\item
  \begin{enumerate}
  \def\labelenumi{\alph{enumi}.}
  \setcounter{enumi}{3}
  \tightlist
  \item
    kiveksen suureneminen
  \end{enumerate}
\end{itemize}

\begin{solution}
\leavevmode

Vastaus

\begin{verbatim}
 b
\end{verbatim}

Kasvain tuntuu tyypillisesti kiinteänä tai kumimaisena resistenssinä kiveksen sisällä. Kiveksen päällä tuntuva massa voi olla spermatoseele, epididymiitti tai joskus varikoseele.

a: Palpaatiossa kiveksessä tunnettavaa kyhmyä tulee pitää pahanlaatuisena, kunnes se on osoitettu hyvänlaatuiseksi. Kivespussin ultraäänitutkimus on hyvä ja luotettava tutkimus. Mikäli ultraäänitutkimus viittaa vahvasti kiveksen maligniteettiin, on syytä tehdä levinneisyystutkimuksena vartalon tietokonekerroskuvaus (TT). Koepalan ottoa kiveskasvaimesta kivespussin läpi ei suositella tuumorisolujen leviämisen riskin vuoksi. Primaarihoitona niin seminoomissa kuin ei-seminoomissakin on radikaali orkiektomia, jossa nivusviillosta poistetaan kives ja funikkeli.

c: Kivessyöpä on yleensä kivuton, mutta kipuilu ei ole harvinaista.

d: Yleisin syy hoitoon hakeutumiseen on toisen kiveksen suureneminen. Vaikka potilas tavallisesti havaitsee itse kiveksen suurentumisen, hoitoon hakeutuminen viivästyy usein varsin paljon.

\end{solution}

\section{Mikä ei ole tyypillinen virtsarakkosyövän oire/löydös?}\label{mikuxe4-ei-ole-tyypillinen-virtsarakkosyuxf6vuxe4n-oireluxf6yduxf6s}

Ei vaihtoehtoja, mutta tässä virtsarakkosyövän tyypillisimmät oireet:

\begin{itemize}
\tightlist
\item
  Yleisin uroteelisyövän oire on verivirtsaisuus, jota on ainakin 85 \%:lla rakkosyöpäpotilaista. Verivirtsaisuus voi olla makroskooppista eli silminnähtävää tai mikroskooppista eli virtsakokeen paljastamaa.
\item
  Noin kolmanneksella rakkokasvainpotilaista on ärsytysoireita, kuten kivuliaisuutta virtsatessa, tiheä- ja yövirtsaisuutta sekä virtsapakon tunnetta.
\end{itemize}

\section{Mikä ei ole virtsatiekivitaudin löydös?}\label{mikuxe4-ei-ole-virtsatiekivitaudin-luxf6yduxf6s}

\begin{itemize}
\tightlist
\item
  \begin{enumerate}
  \def\labelenumi{\alph{enumi}.}
  \tightlist
  \item
    mikroskop. hematuria
  \end{enumerate}
\item
  \begin{enumerate}
  \def\labelenumi{\alph{enumi}.}
  \setcounter{enumi}{1}
  \tightlist
  \item
    makroskop. hematuria
  \end{enumerate}
\item
  \begin{enumerate}
  \def\labelenumi{\alph{enumi}.}
  \setcounter{enumi}{2}
  \tightlist
  \item
    krea koholla
  \end{enumerate}
\item
  \begin{enumerate}
  \def\labelenumi{\alph{enumi}.}
  \setcounter{enumi}{3}
  \tightlist
  \item
    joku
  \end{enumerate}
\end{itemize}

\begin{solution}
\leavevmode

Vastaus

\begin{verbatim}
 c
 
\end{verbatim}

Jos näistä pitäisi valita, niin c olisi sopivin, koska kreatiniini nousee harvoin yksittäisen kiven vuoksi, vaikka olisikin merkittävä tukos, koska toinen munuainen pystyy kompensoimaan hyvin. Jos molemmissa munuaisissa on obstruktio tai potilaalla on vain yksi toiminnallinen munuainen, niin silloin kreakin nousisi merkittävästi.

a-b: Mikroskooppinen (harvoin makroskooppinen) hematuria todetaan 90 \%:lla

d: Muita tyypillisiä löydöksiä: Kova, koliikkimainen kipu, joka säteilee kylkikaaresta vinosti alavatsalle, nivustaipeeseen ja sukuelimiin. Usein virtsaamiskipua, jos kivi sijaitsee alaureterissa. Usein pahoinvointia ja oksentelua. Munuaisten koputus- ja palpaatioarkuutta voi ilmetä. Potilaan on vaikea pysyä paikallaan (vrt. peritoniitti: potilas lepää mieluiten paikallaan).

\end{solution}

\section{Mikä ei ole tyypillinen munuaissyövän löydös?}\label{mikuxe4-ei-ole-tyypillinen-munuaissyuxf6vuxe4n-luxf6yduxf6s}

\begin{itemize}
\tightlist
\item
  \begin{enumerate}
  \def\labelenumi{\alph{enumi}.}
  \tightlist
  \item
    hematuria
  \end{enumerate}
\item
  \begin{enumerate}
  \def\labelenumi{\alph{enumi}.}
  \setcounter{enumi}{1}
  \tightlist
  \item
    tulehdusarvojen nousu
  \end{enumerate}
\item
  \begin{enumerate}
  \def\labelenumi{\alph{enumi}.}
  \setcounter{enumi}{2}
  \tightlist
  \item
    elektrolyyttihäiriö
  \end{enumerate}
\item
  \begin{enumerate}
  \def\labelenumi{\alph{enumi}.}
  \setcounter{enumi}{3}
  \tightlist
  \item
    kuume
  \end{enumerate}
\end{itemize}

\begin{solution}
\leavevmode

Vastaus

\begin{verbatim}
 c
\end{verbatim}

Munuaissyövät ovat kylläkin yleensä oireettomia ja suurin osa löytyy sattumalta vatsan kuvantamistutkimuksissa.

Mahdollisia oireita ovat kuitenkin selkä- ja kylkikipu, hematuria, metastaasien oireet, yleisoireet (tulehdusarvojen nousu ja kuume); hypersedimentaatio, anemia ja mikroskooppinen verivirtsaisuus ovat tavallisia löydöksiä virtsatutkimuksissa.

Kaikukuvaus on suositeltavin seulontatutkimus epäiltäessä munuaissyöpää. Laboratoriotutkimukset: La, PVKT, Krea, AFOS ja U-KemSeul. Kasvainlöydös varmistetaan tavallisesti vartalon varjoainetehosteisella tietokonetomografialla (TT). Histologinen varmistus biopsialla tulee ottaa aina ennen ablatiivisen (radiofrekvenssi- tai kryoablaatio) tai onkologisen hoidon aloitusta. Kudosnäytteiden ottaminen munuaiskasvaimista on yleistynyt, koska ennustetta sekä mahdollisen leikkauksen tai muun hoidon hyötyjä ja haittoja joudutaan punnitsemaan yhä tarkemmin.

Munuaissyöpä syntyy tubulusepiteelin solujen muuttuessa kumuloituvien geenimutaatioiden kautta pahanlaatuisiksi. Munuaissyövän tärkeimmät histologiset alatyypit ovat \textbf{kirkassoluinen karsinooma (75 \%), papillaarinen karsinooma (10 \%) ja kromofobinen karsinooma (5 \%)}

Tärkein riskitekijä on tupakointi. Muita riskitekijöitä ovat lihavuus ja korkea verenpaine. Myös perinnöllisillä tekijöillä on vaikutusta.

\end{solution}

\section{PSA 2, millä vapaan PSA:n arvolla on merkittävä eturauhassyövän riski?}\label{psa-2-milluxe4-vapaan-psan-arvolla-on-merkittuxe4vuxe4-eturauhassyuxf6vuxe4n-riski}

Ei vaihtoehtoja, mutta tässä pohdintaa:

Vapaan PSA:n osuuden pieneneminen auttaa diagnostiikassa (mitataan automaattisesti jos PSA 2-15; ei kuitenkaan oikein diagnostista merkitystä, jos kokonais-PSA \textgreater10). Jos suhde on välillä 4-10 \textless10\%, on riski syövälle yli 50\%. Jos suhde on \textgreater25\%, niin riski alle 10\%.

\begin{itemize}
\tightlist
\item
  Kokonais-PSA-pitoisuuden ollessa alle 3 µg/l (ns. seulontanegatiivinen) on vapaan PSA:n osuuden merkitys ollut epäselvä, mutta ilmeisesti vapaa PSA kertoo eturauhassyövän riskistä myös kokonais-PSA-arvon ollessa alle 3 µg/l.
\item
  Vapaan PSA:n osuuden kliininen merkitys on kuitenkin pienentynyt magneettikuvauksen ja muiden menetelmien käytön yleistyessä. Eturauhassyöpäriskiä arvioitaessa kokonais-PSA on selvästi tärkeämpi ja pelkästään pieni vapaan PSA:n osuus ei ole aihe lisätutkimuksiin, jos kokonais-PSA on normaali.
\end{itemize}

Yleisesti voi ajatella, että eturauhassyöpäriski on koholla, jos vapaa-psa/kokonais-psa-suhde on \textless15\% (viitearvo \textgreater15\%). Eli jos vapaa psa on yli 0.3, niin eturauhassyövän riskin voi ajatella tässä tapauksessa olevan koholla.

\section{Priapismi hoito:}\label{priapismi-hoito}

\begin{itemize}
\tightlist
\item
  \begin{enumerate}
  \def\labelenumi{\alph{enumi}.}
  \tightlist
  \item
    punktio + etilefriini
  \end{enumerate}
\item
  \begin{enumerate}
  \def\labelenumi{\alph{enumi}.}
  \setcounter{enumi}{1}
  \tightlist
  \item
    punktio + sildenafiili
  \end{enumerate}
\item
  \begin{enumerate}
  \def\labelenumi{\alph{enumi}.}
  \setcounter{enumi}{2}
  \tightlist
  \item
    joku
  \end{enumerate}
\item
  \begin{enumerate}
  \def\labelenumi{\alph{enumi}.}
  \setcounter{enumi}{3}
  \tightlist
  \item
    joku
  \end{enumerate}
\end{itemize}

\begin{solution}
\leavevmode

Vastaus

\begin{verbatim}
 a
\end{verbatim}

Lääkäri saa priapismin laukeamaan punktoimalla toisen paisuvaiskudoksen paksulla, esim. 21G:n neulalla, aspiroimalla siitä 100--200 ml tummaa laskimoverta, puristamalla samalla siittimen vartta ja ruiskuttamalla paisuvaiseen adrenergista lääkettä, tavallisimmin etilefriinihydrokloridia (Effortil), jonka kerta-annos on 5--10 mg. Vaihtoehtoisina lääkkeinä paisuvaiseen voidaan injisoida adrenaliinia 0,5--1,0 ml (1:1 000) tai noradrenaliinia 10--20 µg.

Lääkeinjektio voidaan tarvittaessa uusia puolen tunnin kuluttua edellisestä injektiosta. Lääke tulee ruiskuttaa hitaasti paisuvaiskudokseen. Potilaan pulssia ja verenpainetta tulee seurata injektion aikana ja sen jälkeen noin puolen tunnin ajan. On huomioitava, että siitin ei veltostu heti lääkeinjektion jälkeen, vaan yleensä kestää useita tunteja, ennen kuin siitin saavuttaa normaalin pehmeyden.

Jos priapismi ei laukea kahden injektion jälkeen, potilas tulisi lähettää urologiseen yksikköön hoitoon. Priapismin kirurgisessa hoidossa corpus cavernosumin ylipaineinen veri pyritään ohjaamaan joko corpus spongiosumiin tai suoraan suureen laskimoon. Winterin sunttileikkauksessa pistoveitsellä tai prostatabiopsianeulalla tehdään yhteys corpus cavernosumin ja terskan paisuvaiskudoksen välille niin, että veri pääsee poistumaan corpus spongiosumin kautta. Distaalisen suntin osoittautuessa riittämättömäksi joudutaan tekemään avoleikkaus, jossa tehdään fisteli paisuvaisten välille niiden proksimaaliosaan tai yhdistetään paisuvainen suoraan laskimoon.

\end{solution}

\section{Katetrointi, mikä pitää paikkansa?}\label{katetrointi-mikuxe4-pituxe4uxe4-paikkansa}

\begin{itemize}
\tightlist
\item
  \begin{enumerate}
  \def\labelenumi{\alph{enumi}.}
  \tightlist
  \item
    Katetri työnnetään kokonaan virtsarakkoon
  \end{enumerate}
\item
  \begin{enumerate}
  \def\labelenumi{\alph{enumi}.}
  \setcounter{enumi}{1}
  \tightlist
  \item
    Katetri on tarpeeksi syvällä kun virtsaa alkaa tulemaan katetrista
  \end{enumerate}
\item
  \begin{enumerate}
  \def\labelenumi{\alph{enumi}.}
  \setcounter{enumi}{2}
  \tightlist
  \item
    Balongin täytön kuuluu tehdä kipeää
  \end{enumerate}
\item
  \begin{enumerate}
  \def\labelenumi{\alph{enumi}.}
  \setcounter{enumi}{3}
  \tightlist
  \item
    Riittää että vain virtsaputken pään puuduttaa geelillä koska muuten virtsaputkessa ei ole tuntoa
  \end{enumerate}
\end{itemize}

\begin{solution}
\leavevmode

Vastaus

\begin{verbatim}
 a
\end{verbatim}

b: Kun virtsaa alkaa purkautua, katetria tulee työntää vielä n.~5 cm ennen kuin pallo täytetään, jotta tämä tapahtuu varmasti rakossa eikä virtsaputkessa

c: Balongin täyttö EI saa sattua. Mikäli pallo täytetään virtsaputken tai eturauhasen alueella, potilas tuntee yleensä kipua.

d: Mikäli ei käytetä hydrofiilipäällysteistä veteen kostutettavaa katetria, pitää virtsaputki täyttää ennen katetrointia puuduttavalla ja liukastavalla geelillä. Miehelle sitä tarvitaan 20 ml, kun taas naiselle riittää muutama millilitra. Erityisesti miehelle geeli pitää ruiskuttaa tasaisen hitaasti, jotta se leviää myös ulkoisen sulkijan läpi takavirtsaputken alueelle.

\end{solution}

\section{Eturauhasen hyvänlaatuinen liikakasvu ja hankalat oireet (suuri eturauhanen, hankalat virtsaamisoireet). Mistä lääkityksestä potilas hyötyisi eniten?}\label{eturauhasen-hyvuxe4nlaatuinen-liikakasvu-ja-hankalat-oireet-suuri-eturauhanen-hankalat-virtsaamisoireet.-mistuxe4-luxe4uxe4kityksestuxe4-potilas-hyuxf6tyisi-eniten}

Ei vaihtoehtoja, mutta koita vastata ilman vinkkejä

\begin{solution}
\leavevmode

Vastaus

\begin{verbatim}
 5-ARI (dutasteridi tai finasteridi)
\end{verbatim}

5-alfareduktaasin estäjät vaikuttavat eturauhasen liikakasvuun estämällä testosteronin metaboloitumista dihydrotestosteroniksi (DHT), jolloin seerumin DHT:n pitoisuus pienenee. Tämä johtaa eturauhasen koon pienenemiseen ja oireiden lievittymiseen. Myös rakon ulosvirtauskanavan ahtauma pienenee eturauhasen koon pienentyessä.

Myös alfasalpaajat (tamsulosiini tai alfutsosiini) ovat hyödyllisiä ja ne vaikuttavat selvästi nopeammin kuin 5-alfareduktaasin estäjät, mutta ne eivät pienennä eturauhasen kokoa.

Usein näitä lääkkeitä käytetään yhdessä, jolloin lääkkeiden teho paranee ja saadaan oirelievitystä 5-ARI:n pienentävää vaikutusta odotellessa.

\end{solution}

\section{Missä yhteydessä ei tarvita jäännösvirtsan mittausta?}\label{missuxe4-yhteydessuxe4-ei-tarvita-juxe4uxe4nnuxf6svirtsan-mittausta}

\begin{itemize}
\tightlist
\item
  \begin{enumerate}
  \def\labelenumi{\alph{enumi}.}
  \tightlist
  \item
    Miehen tiheävirtsaisuus
  \end{enumerate}
\item
  \begin{enumerate}
  \def\labelenumi{\alph{enumi}.}
  \setcounter{enumi}{1}
  \tightlist
  \item
    Miehen vti-epäily
  \end{enumerate}
\item
  c-d.~Jotain (ei muistettu tärppeihin)
\end{itemize}

\begin{solution}
\leavevmode

Vastaus

\begin{verbatim}
 C-D
\end{verbatim}

a: Hyvin tyypillinen syy mitata jäännösvirtsa (esim. eturauhasperäinen tyhjenemishäiriö mahdollinen -\textgreater{} kerääntymisoireet, kuten yövirtsaaminen ja pollakisuria)

b: Miehen VTI:n taustalla on usein joku syy, esim. juuri tyhjenemishäiriöt

Tilanteita, joissa ei tarvittaisi jäännösvirtsan mittausta on mm. nuorella naisella epäilty VTI ilman virtsaumpioireita (hoitona antibiootti ilman sen suurempia selvittelyjä).

\end{solution}

\section{Mikä ei ole tyypillinen rakon overflow-oire?}\label{mikuxe4-ei-ole-tyypillinen-rakon-overflow-oire}

Ei vaihtoehtoja, mutta tässä overflowsta:

Ylivuotovirtsankarkailu (overflow urinary incontinence) on tyypillisesti yöaikaan tapahtuvaa virtsankarkaamista, joka liittyy virtsarakon akuuttiin tai krooniseen tyhjenemishäiriöön eli virtsaumpeen (virtsa-retentio)

\begin{itemize}
\tightlist
\item
  On siis kroonista retentiota ja lopulta ylitäyttymistä -\textgreater{} ylitäyden rakon paine ylittää virtsaputken paineen ja virtsaa tihkuu ulos
\item
  Miehillä mekaaninen este voi esim. olla suurentunut eturauhanen, mutta usein miehilläkin taustalla on jostain syystä heikentynyt virtsaamisheijaste (esim. hermoston sairaus tai vamma, virtsarakon seinämälihaksen arpeutuminen, leikkauksen jälkitila)
\item
  Naisilla mekaaninen este on harvinainen (voi joskus olla esimerkiksi lantion alueen kasvain tai raskaus) ja naisella tilanne syntyy yleensä rakon supistushäiriöstä, jonka aiheuttaa esimerkiksi voimakas antikolinerginen lääkitys, ääreishermon sairaus tai vamma (esim. cauda equina oireyhtymä) tai sentraalinen neurologinen sairaus (esim. MS-tauti).
\end{itemize}

Tyypillisiä oireita ovat siis mm:

\begin{itemize}
\tightlist
\item
  jatkuva tiputteluvuoto
\item
  heikko virtsasuihku ja pienet virtsamäärät
\item
  tunne, ettei rakko tyhjene
\item
  vatsan/suprapubisen alueen täyteyden tunne
\item
  suuri jäännösvirtsa
\end{itemize}

\section{Vuorokauden nesteen tarve, kun palovamma rintakehällä, käsissä ja yläselässä 70kg henkilöllä}\label{vuorokauden-nesteen-tarve-kun-palovamma-rintakehuxe4lluxe4-kuxe4sissuxe4-ja-yluxe4seluxe4ssuxe4-70kg-henkiluxf6lluxe4}

Ei vaihtoehtoja, mutta koita vastata ilman vinkkejä

\begin{solution}
\leavevmode

Vastaus

\begin{verbatim}
 7500 (tai 4200) 
\end{verbatim}

Modifioitu parklandin kaava: 3ml x paino (kg) x TBSA (\%) = arvioitu vuorokauden nesteentarve

Arvioidaan, että potilaalla on vain rintakehän (eikä abdomenin eli puolikas yläkehon etuosasta) ja yläselän (ei alaselän eli puolikas selästä) palovamma, joten voidaan sanoa, että rule of nines perusteella on yhteensä näiden suhteen 18\% palovamma-alue. Jos molemmat yläraajat ovat palaneet kokonaan, niin siitäkin tulee 18\% eli yhteensä potilaalla on 36\% palanut (jos vain kädet toiselta puolelta, niin sitten olisi 2\% eli yhteensä 20\%).

3ml x 70 x 36 = 7560 ml (jos pelkät kädet eikä koko yläraajat niin sitten 3ml x 70 x 20 = 4200). Tästä 50\% annetaan ensimmäisen 8h aikana vamman aiheutumisesta ja loput seuraavan 16h aikana. Ringerillä jos ei ole hemodynaamisesti merkittävää vuotoa (lapsilla lisäksi G5-korjausneste rajallisempien maksan glykogeenivarastojen vuoksi)

\pandocbounded{\includegraphics[keepaspectratio]{images/ruleofnines.png}}

\end{solution}

\section{Potilastapaus}\label{potilastapaus-39}

Lääketieteen opiskelija kastellut lääkärintakkinsa bensalla ja sytyttänyt sen tuleen näyttäessään kurssikavereille pikku showta. Palovammoja rintakehällä, käsivarsissa ja yläselässä. Palovamma pohjalta vaalea, ei kapillaarireaktiota, ei tuntoa. Palovamman syvyyden arviointi:

\begin{itemize}
\tightlist
\item
  \begin{enumerate}
  \def\labelenumi{\alph{enumi}.}
  \tightlist
  \item
    1
  \end{enumerate}
\item
  \begin{enumerate}
  \def\labelenumi{\alph{enumi}.}
  \setcounter{enumi}{1}
  \tightlist
  \item
    2A
  \end{enumerate}
\item
  \begin{enumerate}
  \def\labelenumi{\alph{enumi}.}
  \setcounter{enumi}{2}
  \tightlist
  \item
    2B
  \end{enumerate}
\item
  \begin{enumerate}
  \def\labelenumi{\alph{enumi}.}
  \setcounter{enumi}{3}
  \tightlist
  \item
    2C
  \end{enumerate}
\item
  \begin{enumerate}
  \def\labelenumi{\alph{enumi}.}
  \setcounter{enumi}{4}
  \tightlist
  \item
    3
  \end{enumerate}
\end{itemize}

\begin{solution}
\leavevmode

Vastaus

\begin{verbatim}
 e
\end{verbatim}

\pandocbounded{\includegraphics[keepaspectratio]{images/palovammaluokatleo.png}}
\pandocbounded{\includegraphics[keepaspectratio]{images/palovammaluokatleokuvat.png}}

\end{solution}

\section{Potilaalla mobiili ihonalainen patti, n.~2cm. Puristaessa pattia sieltä tulee mönjää. Mitä teet?}\label{potilaalla-mobiili-ihonalainen-patti-n.-2cm.-puristaessa-pattia-sieltuxe4-tulee-muxf6njuxe4uxe4.-mituxe4-teet}

\begin{itemize}
\tightlist
\item
  \begin{enumerate}
  \def\labelenumi{\alph{enumi}.}
  \tightlist
  \item
    poistan itse
  \end{enumerate}
\item
  \begin{enumerate}
  \def\labelenumi{\alph{enumi}.}
  \setcounter{enumi}{1}
  \tightlist
  \item
    otan biopsian
  \end{enumerate}
\item
  \begin{enumerate}
  \def\labelenumi{\alph{enumi}.}
  \setcounter{enumi}{2}
  \tightlist
  \item
    lähete plastikkakirurgille
  \end{enumerate}
\item
  \begin{enumerate}
  \def\labelenumi{\alph{enumi}.}
  \setcounter{enumi}{3}
  \tightlist
  \item
    jotain
  \end{enumerate}
\end{itemize}

\begin{solution}
\leavevmode

Vastaus

\begin{verbatim}
 a
\end{verbatim}

Todennäköisesti kyseessä on aterooma (epidermaalikysta). Kyseessä on ihon taliretentiokysta, joka muodostuu laajentuneesta talirauhasesta tai karvatupesta, jonka tiehyt on mennyt umpeen.

Rauhallisen vaiheen aterooman hoito on poisto veneviillon kautta ja tämä toimenpide kuuluu yleislääkärin tehtäviin. Haava suljetaan, koska alueella ei rauhallisessa tilanteessa ole infektiota. Jos kyseessä on infektoitunut aterooma (abskessi), niin se poistetaan inkisiolla ja haava jätetään auki (harkinnan mukaan voi laittaa kumiliuskan tai Sorbact®- nauhaa); antibiootti aloitetaan myös.

\pandocbounded{\includegraphics[keepaspectratio]{images/aterooma.png}}

\end{solution}

\section{Eskarotomia}\label{eskarotomia}

Ei vaihtoehtoja, tässä aiheesta:

Syvän palovamman alueelle muodostuu panssarimainen, joustamaton kudos, eskar. Sirkulaarinen eskar raajojen alueella voi johtaa alla olevan kudoksen verenkierron salpaantumiseen ja iskemiaan, jolloin raajan elinkelpoisuus voi olla uhattuna. Rintakehän alueella eskar voi johtaa hengitystiepaineiden nousuun ja potilaan ventilaation huononemiseen. Vatsan alueella eskar voi aiheuttaa vatsaontelon sisäisen paineen nousun ja johtaa suoli-iskemiaan ja munuaistoiminnan huononemiseen.

\begin{itemize}
\tightlist
\item
  Jos nesteytys, kivunhoito, raajojen kohoasento yms ei auta, niin tarvitaan eskarotomia, jossa syvä palanut kudos halkaistaan
\item
  Eskarotomiat voidaan tehdä vuodeosasto-olosuhteissa tai teho-osastolla, kunhan on varauduttu verenvuodon hoitamiseen esimerkiksi diatermialla. Syvissä lihaskalvoon asti ulottuvissa palovammoissa ja sähköpalovammoissa tarvitaan usein myös faskiotomiat, ja nämä on syytä tehdä leikkaussalissa yleisanestesiassa.
\end{itemize}

\begin{figure}
\centering
\pandocbounded{\includegraphics[keepaspectratio]{images/eskarotomia.png}}
\caption{Eskarotomia}
\end{figure}

\section{Babysitter-proseduuri}\label{babysitter-proseduuri}

Ei vaihtoehtoja tai kysymyksenasettelua, mutta tässä aiheesta:

Kasvohermohalvaus (CN VII pareesi) voi johtua monesta syystä:

\begin{itemize}
\tightlist
\item
  Bellin pareesi (eli idiopaattinen)

  \begin{itemize}
  \tightlist
  \item
    Potilaat (yli 16--18 v) hyötyvät mahdollisimman nopeasti (mielellään alle 48--72 t:ssa) aloitetusta suun kautta annetusta glukokortikoidista (prednisoloni 60 mg × 1/vrk 5 vrk:n ajan, minkä jälkeen annosta pienennetään 10 mg/vrk; hoidon kesto yhteensä 10 vrk)
  \end{itemize}
\item
  Tapaturma
\item
  Kasvaimet ja niiden kirurgia
\item
  Infektiot (borrelioosi -\textgreater{} molemminpuolisessa halvauksessa epäile ensisijaisesti syyksi borrelioosia; varicella zoster -\textgreater{} aiheuttaa ns. Ramsay Huntin oireyhtymän (kasvohalvaus + vyöruusu)

  \begin{itemize}
  \tightlist
  \item
    Borrelioosin hoitona mikrobilääkitys myöhäisoireiden ehkäisemiseksi
  \item
    Varicellan hoitona glukokortikoidi (kuten Bellin pareesissa, 10 vrk) ja valasikloviiri suun kautta 1 g × 3/vrk 7 vrk:n ajan
  \end{itemize}
\item
  Synnynnäinen (Möbiuksen oireyhtymä)
\item
  Perinnöllinen (Meretojan tauti)
\end{itemize}

Kasvohermohalvauksen hoidon tavoitteena on auttaa potilasta toiminnallisissa ja esteettisissä ongelmissa sekä estää halvauksesta aiheutuvia haittoja. Kasvohermohalvauksen seurauksena silmäluomia ei pysty sulkemaan ja silmä jää auki aiheuttaen silmän kuivumista, mikä voi pysyvästi vaurioittaa silmää. Ongelmat syömisessä, juomisessa ja puhumisessa vaikeuttavat elämää, ja kasvojen ulkonäön muuttuminen heikentää elämänlaatua ja aiheuttaa sosiaalista ja henkistä haittaa.

Kirurgiset hoidot voidaan jakaa staattisiin ja toiminnallisiin korjausleikkauksiin

\begin{itemize}
\tightlist
\item
  Staattisilla toimenpiteillä pyritään vähentämään halvauksesta johtuvia haittoja, mutta ne eivät palauta toimintaa kuten toiminnalliset korjausleikkaukset.
\item
  Staattisista toimenpiteistä yleisin on yläluomen kulta- tai platinapainon asettaminen, joka auttaa mekaanisesti yläluomea sulkeutumaan.
\end{itemize}

Toiminnalliset kirurgiset korjaukset voidaan jakaa hermokorjauksiin ja lihaskorjauksiin.

\begin{itemize}
\tightlist
\item
  Suoria hermokorjauksia voidaan käyttää silloin, kun kasvohermo on katkennut tai vaurioitunut vamman tai leikkauksen seurauksena.
\item
  Korjaus voi olla välitön (1-7pv) eli hermon katkenneet päät ommellaan mikroskooppisesti yhteen tai katkenneen hermon päiden väliin ommellaan muualta tuotu hermosiirre. Myös \textbf{hermotranspositiota eli toimivan hermon kytkentää vioittuneeseen hermoon voidaan käyttää (mm. masseterhermon kytkentä kasvohermoon).}
\item
  Lihaskorjauksia käytetään silloin, kun kasvojen lihakset ovat pysyvästi hermotuksen puuttumisen vuoksi vaurioituneet.
\end{itemize}

\textbf{Babysitter-toimenpiteessä jokin lähistöllä oleva toimiva hermo (yleensä masseter, voi olla myös hypoglossus) kiinnitetään distaalisiin kasvohermohaaroihin ja näin saadaan suhteellisen nopeasti takaisin hermotusta kasvolihaksiin}

\begin{itemize}
\tightlist
\item
  Jos odotettaisiin cross-face nerve graftin tai jonkun muun pidemmän korjauksen saapumista alueelle, niin sillä välillä lihakset olisivat jo ehtineet atrofioitua, joten tarvitaan jokin keino hermottaa lihaksia (baby sitting), kunnes pitkät graftit (vanhemmat) saapuvat paikalle
\end{itemize}

Hyvä selitys leikkauksesta: \url{https://youtu.be/IQaegIcds9o?si=LLAPDIp6AcXy2ZZL}

\pandocbounded{\includegraphics[keepaspectratio]{images/babysitter.png}}
\pandocbounded{\includegraphics[keepaspectratio]{images/babysitteryoutube.png}}

\section{60v potilas löytynyt ulkoa. Ruumiinlämpö 36°C. Ranteen puoliväliin asti paleltumaoireita, tyyliin tunto ja motoriikka alentunut, sormet vaaleat. Mikä hoito aloitetaan?}\label{v-potilas-luxf6ytynyt-ulkoa.-ruumiinluxe4mpuxf6-36c.-ranteen-puolivuxe4liin-asti-paleltumaoireita-tyyliin-tunto-ja-motoriikka-alentunut-sormet-vaaleat.-mikuxe4-hoito-aloitetaan}

\begin{itemize}
\tightlist
\item
  \begin{enumerate}
  \def\labelenumi{\alph{enumi}.}
  \tightlist
  \item
    iloprosti
  \end{enumerate}
\item
  \begin{enumerate}
  \def\labelenumi{\alph{enumi}.}
  \setcounter{enumi}{1}
  \tightlist
  \item
    liuotus ja jotain
  \end{enumerate}
\item
  \begin{enumerate}
  \def\labelenumi{\alph{enumi}.}
  \setcounter{enumi}{2}
  \tightlist
  \item
    kolme muuta vaihtoehtoa
  \end{enumerate}
\end{itemize}

\begin{solution}
\leavevmode

Vastaus

\begin{verbatim}
 a tai b 
\end{verbatim}

Riippu vähän miten tilanne arvioidaan tai se on kuvattu kysymyksessä (ehkä unohdettu jotain tärppiä kirjoitettaessa). Potilaalla on todennäköisesti jotain vuototaipumusta (ehkä alkoholisti, kallonsisäinen vamma hyvin mahdollinen yms.), jolloin iloprosti on liuotusta parempi.

Paleltumien hoidosta:

Hypotermia korjataan ennen paikallisten paleltumavammojen hoitoa. Hypotermisen potilaan perifeeristen ruumiinosien liikuttelua on vältettävä, sillä kylmä veri voi sydämeen päästessään aiheuttaa rytmihäiriöitä.

Paleltuman paras hoito on nopea sulatus 40--42 °C vedessä 15--30 min:n ajan tai kunnes paleltuneelle alueelle ilmaantuu verenkiertoa. Kipulääkityksenä ibuprofeeni 600mg x3 ja sulatuksen jälkeen voimakkaan kivun hoidon tarve todennäköistä (vahvat opioidit). Lämmityksen jälkeenkin jatkuva tunnottomuus, veriset rakkulat ja dopplersignaalin puuttuminen viittaavat vaikeaan paleltumavammaan ja vaatii sairaalahoitoa (2.--4. asteen paleltumavammat kuuluvat sairaalahoitoon). Sulattamisen jälkeen rakkuloiden poisto on tärkeää.

Alkuvaiheen hoitoon kuuluvat myös ASA 100 mg × 1 ja enoksapariini 40 mg × 1 noin kuukauden ajan.

Välitön angiografia on indisoitu, jos paleltumavamma on vaikea ja sen syntymisestä on kulunut alle 48 t (mieluummin alle 24 t) eikä tutkimukselle ole vasta-aiheita. Jos angiografiassa todetaan selvä tukos, on annettava valtimonsisäistä alteplaasi-liuotushoitoa (actilyse), jonka on osoitettu vähentävän amputaatioriskiä. Jos liuotushoito on vasta-aiheinen tai ei todeta liuotuskohteita, on vasodilataattori-infuusio (iloprosti 6h/vrk 3vrk ajan) hyvä vaihtoehto. Molemmat tarvitsevat valvonta/tehohoito tasoista hoitoa sairaalassa.

Nekroosin annetaan demarkoitua muutaman viikon-kuukauden ajan ennen operatiivista hoitoa (Paleltuma jouluna --- amputaatio juhannuksena'' --- oikeasti 1-2k). Kostea kuolio on kuitenkin revidoitava tulehdusvaaran takia.

\pandocbounded{\includegraphics[keepaspectratio]{images/paleltumavammaluokat.png}}
\pandocbounded{\includegraphics[keepaspectratio]{images/paleltumatleo.png}}

\end{solution}

\section{Taustalla diabetes, nyt punoittava ja turvonnut jalka. Tulehdusarvot matalat. Mistä kyse?}\label{taustalla-diabetes-nyt-punoittava-ja-turvonnut-jalka.-tulehdusarvot-matalat.-mistuxe4-kyse}

Ei vaihtoehtoja, koita vastata ilman vinkkejä

\begin{solution}
\leavevmode

Vastaus

\begin{verbatim}
 Charcot 
\end{verbatim}

Punoittavaa, kuumoittavaa ja turvonnutta jalkaa on diabetesta sairastavalla pidettävä Charcot'n jalkana, kunnes toisin osoitetaan.

Charcot'n jalka eli neuro-osteoartropatia (neuropaattinen osteoartropatia) syntyy diabeettisen neuropatian seurauksena. Akuutti Charcot muistuttaa septistä niveltulehdusta tai ruusua, ja prosessi johtaa nivelten tuhoutumiseen, jalkaholvin romahtamiseen ja epämuodostumiin. Charcot'n jalan perimmäiset syntymekanismit ovat toistaiseksi epäselvät. Luuta hajottavien solujen, osteoklastien, lisääntynyt aktiviteetti johtaa vähitellen luiden ja nivelten paikalliseen luhistumiseen. Laukaiseva tekijä voi olla nilkan nyrjähdys, haava, leikkaus tai infektio, jonka aiheuttama poikkeuksellisen voimakas tulehdusvaste saa aikaan luun ja nivelten tuhoutumisen.

CRP ja lasko ovat normaalit, AFOS voi olla suurentunut.

Huolimatta runsaasta niveltuhosta oireet ovat usein neuropatian vuoksi vähäisiä.

Röntgenmuutokset näkyvät myöhäisvaiheessa. Tyypillinen on TMT-nivelten tuhoutumisesta johtuva keinutuolijalka. Diagnoosi varmistetaan tarvittaessa magneettikuvauksella.

\pandocbounded{\includegraphics[keepaspectratio]{images/charcot.png}}
\pandocbounded{\includegraphics[keepaspectratio]{images/charcotkeinutuoli.png}}

\end{solution}

\section{Komplisoitumaton Charcot, ensilinjan hoito}\label{komplisoitumaton-charcot-ensilinjan-hoito}

Ei vaihtoehtoja, koita vastata ilman vinkkejä

\begin{solution}
\leavevmode

Vastaus

\begin{verbatim}
 Immobilisaatio 3-9kk
\end{verbatim}

Hoitona potilaan jalka immobilisoidaan joko kipsillä tai ortoosilla ja kyynärsauvojen avulla; alkuvaiheessa jalkaan kohdistuvan kuormituksen täydellinen poistaminen on ensiarvoisen tärkeää.

\end{solution}

\section{Potilas kaatunut joku aika sitten. Sen yhteydessä loukannut reittä ja huomannut siinä tyyliin 3cm mobiilin, alustaansa kiinnittymättömän patin. Mitä teet tk:ssa?}\label{potilas-kaatunut-joku-aika-sitten.-sen-yhteydessuxe4-loukannut-reittuxe4-ja-huomannut-siinuxe4-tyyliin-3cm-mobiilin-alustaansa-kiinnittymuxe4ttuxf6muxe4n-patin.-mituxe4-teet-tkssa}

\begin{itemize}
\tightlist
\item
  \begin{enumerate}
  \def\labelenumi{\alph{enumi}.}
  \tightlist
  \item
    Poistan itse
  \end{enumerate}
\item
  \begin{enumerate}
  \def\labelenumi{\alph{enumi}.}
  \setcounter{enumi}{1}
  \tightlist
  \item
    Plastiikkakirurgin konsultaation
  \end{enumerate}
\item
  \begin{enumerate}
  \def\labelenumi{\alph{enumi}.}
  \setcounter{enumi}{2}
  \tightlist
  \item
    Kiireellinen lähete
  \end{enumerate}
\item
  \begin{enumerate}
  \def\labelenumi{\alph{enumi}.}
  \setcounter{enumi}{3}
  \tightlist
  \item
    Poistan, mikäli haittaa potilasta
  \end{enumerate}
\end{itemize}

\begin{solution}
\leavevmode

Vastaus

\begin{verbatim}
 d
\end{verbatim}

Ei maligneja piirteitä, sopii lipoomaan. Voidaan poistaa terveyskeskuksessa, jos potilas niin toivoo.

\pandocbounded{\includegraphics[keepaspectratio]{images/pehmytkudoskasvainalgoritmi.png}}

\end{solution}

\section{Ihomuutos poistettu, paljastunut melanoomaksi, Breslow 1.2. PAD:ssa ei riittävät marginaalit. Erikoissairaanhoidossa}\label{ihomuutos-poistettu-paljastunut-melanoomaksi-breslow-1.2.-padssa-ei-riittuxe4vuxe4t-marginaalit.-erikoissairaanhoidossa}

\begin{itemize}
\tightlist
\item
  \begin{enumerate}
  \def\labelenumi{\alph{enumi}.}
  \tightlist
  \item
    Re-ekskisio
  \end{enumerate}
\item
  \begin{enumerate}
  \def\labelenumi{\alph{enumi}.}
  \setcounter{enumi}{1}
  \tightlist
  \item
    Re-ekskisio ja vartijaimusolmuketutkimus
  \end{enumerate}
\item
  \begin{enumerate}
  \def\labelenumi{\alph{enumi}.}
  \setcounter{enumi}{2}
  \tightlist
  \item
    Sytostaatit
  \end{enumerate}
\item
  d-e. jotain
\end{itemize}

\begin{solution}
\leavevmode

Vastaus

\begin{verbatim}
 b
\end{verbatim}

Jos patologi vastaa kyseessä olevan melanooma, potilas ohjataan kirurgiseen jatkohoitoon ja varmistetaan, että hoito toteutuu ilman viivytystä. Suurelle osalle primaarimelanoomapotilaista tehdään arpialueen resektio (poiston laajuus riippuu tuumorin sijainnista ja melanooman paksuudesta (Breslowin luokitus)) sekä vartijasolmuketutkimus. Muita rutiinimaisia kuvantamis- tai laboratoriotutkimuksia ei tehdä leikkausvaiheessa eikä myöhemmin seurannan aikana.

Hyvin pinnalliset melanoomat (Breslow ≤ 2 mm) poistetaan 1 cm:n terveen kudoksen marginaalilla. Syvemmissä melanoomissa poistetun melanooma-alueen arpi tai biopsoitu melanooma poistetaan 2 cm:n marginaalein ja ihonalainen rasva faskiatasolle saakka.

a: Melanooman vartijaimusolmuketutkimus tehdään, kun Breslow'n mitta on vähintään 1mm (tai 0.8-1mm, jos muita aktiivisuuden merkkejä esim. ulseraatiota tai mitooseja tai potilas on keskimääräistä nuorempi (\textless40-50v))

c: Ei vielä viitteitä leviämisestä. Jos melanooma on paikallisesti niin levinnyt, että arvioidaan, ettei sen poisto ole kirurgisesti mahdollinen, niin voidaan aloittaa lääkitys esim immuno-onkologisella lääkkeellä tai BRAF-mutatoituneessa melanoomassa BRAF/MEK estäjillä pienentämään melanoomaa leikkauskelpoiseksi. Tällöin puhutaan neoadjuvanttihoidosta. Yksittäisiä distaalisia etäpesäkkeitäkin voidaan tapauskohtaisesti hoitaa kirurgisesti (samoin jos melanooma on levinnyt vain paikallisiin imusolmukkeisiin)

\pandocbounded{\includegraphics[keepaspectratio]{images/melanoomaalgoritmi.png}}

\end{solution}

\section{Mihin Suomessa on keskitetty laajojen palovammojen hoito?}\label{mihin-suomessa-on-keskitetty-laajojen-palovammojen-hoito}

Ei vaihtoehtoja, mutta koita vastata ilman vinkkejä

\begin{solution}
\leavevmode

Vastaus

\begin{verbatim}
 HUS
\end{verbatim}

Laajojen palovammojen hoito on keskitetty Suomessa HUSin palovammakeskukseen

Jokaisessa laajan päivystyksen sairaalassa tulee kuitenkin olla valmius vaikeastikin loukkaantuneen palovammapotilaan hoitamiseen ensimmäisen 24--72 tunnin ajan vamman sattumisesta

\end{solution}

\section{Potilaalla oikean rinnan ylälateraalipuolella joku pieni möykkynen, mitäs seuraavaksi kuuluu tehdä?}\label{potilaalla-oikean-rinnan-yluxe4lateraalipuolella-joku-pieni-muxf6ykkynen-mituxe4s-seuraavaksi-kuuluu-tehduxe4}

Ei vaihtoehtoja, koita vastata ilman vinkkejä

\begin{solution}
\leavevmode

Vastaus

\begin{verbatim}
 MGR+PNB
\end{verbatim}

Rintasyövän kolmoisdiagnostiikka: kliininen tutkimus, mammografia (ja UÄ jatkotutkimuksena tai raskaana oleville/nuorille ensisijaisena) ja biopsia (ensisijaisesti paksuneulabiopsia). Jos yksikin diagnostiikan osa viittaa pahanlaatuisuuteen, rinnan muutosta ei saa jäädä seuraamaan, vaan se tulee poistaa.

\end{solution}

\section{Potilaalla dm2 ja vitusti kaikkea muuta, oiskohan ollut että nyt pari kk ollut haava jalkapohjassa tai varpaassa tai jossain, abi korkea, mikä pitää paikkansa?}\label{potilaalla-dm2-ja-vitusti-kaikkea-muuta-oiskohan-ollut-ettuxe4-nyt-pari-kk-ollut-haava-jalkapohjassa-tai-varpaassa-tai-jossain-abi-korkea-mikuxe4-pituxe4uxe4-paikkansa}

Vastaukseksi arveltu: ``Mediaskleroosi nostaa ABI arvoja, tarve varvaspainemittaukselle'', joka on oikein.

Mediaskleroosi (Mönckebergin skleroosi, MAC) on vanhemmilla ihmisillä yleinen arterioskleroosin muoto, jolle on tyypillistä verisuonen seinämän median kalsifikaatio.

\begin{itemize}
\tightlist
\item
  Viime vuosina MAC:n rooli ASO-taudin (alaraajojen ahtauttava valtimotauti) taustalla on kuitenkin noussut esiin; ennen ajateltiin olevan käytännössä kliinisesti merkityksetön löydös, joka vain häiritsee ABI-mittausta (skleroosi estää normaalin puristumisen mittauksen aikana -\textgreater{} virheellisen korkeat ABI-arvot) ja saattoi tulla esille mammografioissa kuvantaessa

  \begin{itemize}
  \tightlist
  \item
    Häiritsee hemodynamiikkaa jäykistämällä suonen seinämää -\textgreater{} pulssipaine kasvaa ja lisää afterloadia -\textgreater{} vasemman kammion kuormitus lisääntyy. Häiritsee myös vasodilataatiokykyä ja myös altistaa trombooseille. Pitkälle edetessään voi myös ahtauttaa suonen luumenia yksinään/ voimistaa ateroskleroosin ahtauttavaa muutosta (inward remodelling)
  \end{itemize}
\end{itemize}

Tärkeimmät mediaskleroosin riskitekijät ovat ikääntyminen, diabetes ja munuaisten vajaatoiminta.

Mediaskleroosin takia diabeetikoiden ABI-mittauksiin ei voi luottaa liikaa ja tarvittaessa tulee mitata varvaspaineet (mediaskleroosi ei affisioi pieniä varpaiden suonia oikein).

\begin{itemize}
\tightlist
\item
  Lähteestä vaihdellen ABI \textgreater1.3-1.4 viittaa mediaskleroosiin (ehkä useimmiten ajatellaan \textgreater1.4)
\item
  Varvaspainemittauksessa kriittisen alaraajaiskemian (CLI) diagnoosin raja-arvo on 30 mmHg. Joissain lähteissä raja-arvo diabeetikoilla ja haavapotilailla on 50 mmHg. Varvaspaineet ilmoitetaan absoluuttisina arvoina eikä suhteena olkapaineisiin.
\end{itemize}

\pandocbounded{\includegraphics[keepaspectratio]{images/macvsath.png}}
\pandocbounded{\includegraphics[keepaspectratio]{images/machis.png}}

\section{Laskimovajaatoiminnan C-luokitus}\label{laskimovajaatoiminnan-c-luokitus}

Ei vaihtoehtoja, tässä laskimovajaatoiminnan luokittelusta:

Laskimovajaatoiminnan vaikeusaste voidaan kuvata tarkasti CEAP-luokitusta (CEAP = kliinis-etiologis-anatomis-patofysiologinen luokitus) käyttäen. Yleensä kuitenkin vain turvaudutaan C-osioon eli kliiniseen luokitukseen:

\begin{itemize}
\tightlist
\item
  C1-C3 komplisoitumaton (ei ihomuutoksia)
\item
  C4-C6 komplisoitunut (ihomuutoksia)
\end{itemize}

\pandocbounded{\includegraphics[keepaspectratio]{images/ceap.png}}
\pandocbounded{\includegraphics[keepaspectratio]{images/cluokitus.png}}
\pandocbounded{\includegraphics[keepaspectratio]{images/c4luokankuvat.png}}

\section{Potilastapaus}\label{potilastapaus-40}

Joku käynyt viikko sitten Turkissa kauneusleikkauksissa. Tehty ehkä kasvojuttuja, vatsajuttuja ja peppujuttuja. Nyt kuumetta ja CRP nousu, virtsanäytteessä Eryt+ ja leuk +. Kaikki muut leikkauskohat näytti ok mutta peppu fluktuoi ja punoittaa. Mikä on infektion lähde?

\begin{itemize}
\tightlist
\item
  \begin{enumerate}
  \def\labelenumi{\alph{enumi}.}
  \tightlist
  \item
    perse
  \end{enumerate}
\item
  \begin{enumerate}
  \def\labelenumi{\alph{enumi}.}
  \setcounter{enumi}{1}
  \tightlist
  \item
    vti
  \end{enumerate}
\item
  \begin{enumerate}
  \def\labelenumi{\alph{enumi}.}
  \setcounter{enumi}{2}
  \tightlist
  \item
    keskushermosto
  \end{enumerate}
\item
  \begin{enumerate}
  \def\labelenumi{\alph{enumi}.}
  \setcounter{enumi}{3}
  \tightlist
  \item
    joku
  \end{enumerate}
\end{itemize}

\begin{solution}
\leavevmode

Vastaus

\begin{verbatim}
 a
\end{verbatim}

Varmaan jäänyt jotain piirteitä muistamatta kysymyksen suhteen, mutta todennäköisesti kyseessä on postoperatiivinen haavainfektio gluteusalueella. Klaffataan haava auki ja otetaan bakteeriviljely. Vaatinee iv-ab hoitoa, jos CRP selvästi koholla.

\end{solution}

\section{}\label{section}

Gastrokirra:
miten diagnosoit sappilekaasin?

potilaalla sappileikkauksen jälkeen kurjaa, mikä komplikaatio? -\textgreater{} sappilekaasi

Mihin käytetään Sengstake-Blakemoore tuubia?
Laparostomia

Komplisoitunut divertikuliitti (perforaatio), mikä leikkaus? = Hartman

Potilas oksentanut kahvikupillisen verta kotona, vastaanotolla toinen kupillinen, tilanne nyt stabiili, mitä teet? (Kotiin, sairaalaan + PPI + ravinto, sairaalaan + PPI x2 + ravinnotta, kotiin + lähipäivinä gastroskopia)

Akuutti haimatulehdus on patologialtaan
-autodigestiivinen tauti
-autoinflammatorinen
-autoimmuuni
-infektio

Urologia:
Mikä ei ole tyypillinen kivessyövän oire/löydös?
-kasvain kiveksessä
-kasvain kiveksen yläpuolella
-kipu
-kiveksen suureneminen

Mikä ei ole tyypillinen virtsarakkosyövän oire/löydös?
Mikä ei ole virtsatiekivitaudin löydös?
mikroskop. hematuria
makroskop. hematuria
krea koholla
joku
Mikä ei ole tyypillinen munuaissyövän löydös? (hematuria, tulehdusarvojen nousu, elektrolyyttihäiriö, joku - tää joku oli kuume)
PSA 2, millä vapaan PSA:n arvolla on merkittävä eturauhassyövän riski?
Munuaissyövän luokituksesta kysyttiin, opettele suurpiirteisesti ne luokat.

Priapismi hoito:
-punktio + etilefriini
-punktio + sildenafiili
-joku
-joku

Potilaalla aloitettu alfasalpaaja, milloin kontrolli?

Katetrointi, mikä pitää paikkansa?
-katetri työnnetään kokonaan virtsarakkoon
-katetri on tarpeeksi syvällä kun virtsaa alkaa tulemaan katetrista
-Balongin täytön kuuluu tehdä kipeää
-riittää että vain virtsaputken pään puuduttaa geelillä koska muuten virtsaputkessa ei ole tuntoa

Plastiikka:
Vuorokauden nesteen tarve, kun palovamma rintakehällä, käsissä ja yläselässä 70kg henkilöllä

Lääketieteen opiskelija kastellut lääkärintakkinsa bensalla ja sytyttänyt sen tuleen näyttäessään kurssikavereille pikku showta. Palovammoja rintakehällä, käsivarsissa ja yläselässä. Palovamma pohjalta vaalea, ei kapillaarireaktiota, ei tuntoa. Palovamman syvyyden arviointi:
1, 2A, 2B, 2C, vai 3?

Joku käynyt viikko sitten Turkissa kauneusleikkauksissa. Tehty ehkä kasvojuttuja, vatsajuttuja ja peppujuttuja tai jotain en muista. Oiskohan ollut että nyt kuumetta ja CRP nousu, virtsanäytteessä Eryt+ ja joku muu + en muista xd Leuk tai Nitr. Kaikki muut leikkauskohat näytti ok mutta peppu fluktuoi ja punoittaa tai jtn. Mikä on infektion lähde?
-perse
-vti
-joku keskushermos
to tai joku emt
-joku muut--- täydentäkää joku tää paremmaks jos muistatte, ite vastasin vti ja aattelin et perseessä vaan seroomaa tai jotain mut en tiedä mistään mitään

Potilaalla mobiili ihonalainen patti, joku 2cm tai jotain. Puristaessa pattia sieltä tulee jotain mönjää. Mitä teet?
-poistan itse - taisin vastata tän kun vaikutti ateroomalta
-otan biopsian
-lähete pkir tai jotain
-jotain

Potilaalla pehmytkudosinfektio, mikä näistä toimenpiteistä on tappava jos sen jättää tekemättä?
-nekroottisen kudoksen poisto - tää
-antibiootti
-ylipainehappihoito
-joku

Eskarotomia
Abdominoplastia
Pari erilaista sarkooma kyssäriä-\textgreater{} muista kriteerit, cm määrät jolla lähetät eteenpäi
aika paljo terminologiasta hyötyä: esim joku englanninkielinen sana jota en muista mut liitty johonkin korjausoperaatioon, OIsko ollu baby jotain tai emt XD?

Thorax:
Mikä tyypillisin leikkaus ei-pienisoluisessa keuhkosyövässä? Keuhkon poisto, lohkon
poisto, tuumorin poisto marginaalein vai keuhkonsiirto?

30?v mies ollut kolarissa, RR 100/80 tai jotain, syk 120 tai jtn, rintakehä naksuu palpoidessa. Hengitysäänet kauttaaltaan vaimeat, mutta vaikea saada selvää metelissä. FAST ultrassa nestetä/verta perikardiumissa. Potilas reagoi huonosti, kipuun vain mumisee. Mitä teet? - en muista tätäkään täydellisesti, täydentäkää joku
-thx-kir soitto, pleuradreenin laitto, leikkaussalin valmistelu
-thorax-rtg, trauma-tt, pään-tt, pleuradreeni tarvittaessa
-jotain
-jotain tämmösiä komboja, en muista, noikaan ei ehkä oo oikeet vaihtoehdot

Potilaalla oikean rinnan ylälateraalipuolella joku pieni möykkynen, mitäs sitten?
-mammografia ja biopsia
-jotain
-jotain
-jotain

Vkir:
Akuutti alaraajaiskemia - mikä oire viittaa raajan elinkelpoisuuden menettämiseen
(raajan tunnottomuus ja plegia, iskeeminen kipu, kaks muuta vaihtoehtoo)

Mikä ei aiheuta akuuttia alaraajaiskemiaa? (alaraajan embolia, tromboosi,
keuhkoembolia vai alaraajan ohitteen tukos)

Potilaalla dm2 ja vitusti kaikkea muuta, oiskohan ollut että nyt pari kk ollut haava jalkapohjassa tai varpaassa tai jossain, abi korkea, mikä pitää paikkansa?
-jotain
-jotain
-jotain
-Mediaskleroosi nostaa ABI arvoja, tarve varvaspainemittaukselle

Potilaalla dm2 ja vitusti kaikkea muuta, nyt jalkapöydässä punoitusta ja turvotusta, ei kipua, CRP matala. Mikä vaivaa?
-Charcot
-Kihti
-joku
-joku

\chapter{Palaute}\label{palaute}

Jos jotain puuttuu, löydät korjattavaa tai muita ehdotuksia yms, niin voit ottaa yhteyttä mitä tahansa reittiä pitkin. Jos haluat toimia anonyymisti, niin \href{https://docs.google.com/forms/d/e/1FAIpQLSericnXGU2U_h7stCFVZ5X0-6Q9BLGdiDugun_Mex3kf_bTpg/viewform?usp=sharing&ouid=112689903880978617225}{C7-oppaan formsin} kautta voi laittaa koodia.

\bibliography{book.bib,packages.bib}

\end{document}
